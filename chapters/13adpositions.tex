\chapter{Adpositions and adverbs}\label{chapter:adverbs}


\section{Syntax of adverbs and adpositions in the clause}


\section{Syntax of adpositional phrases}

\section{Classes of adpositions}\label{sec:ch13:adpositions}


\subsection{-ánó `inside'}  %DON'T THINK ANO AND NANO SHOULD BE IN THIS SECTION

-\textsc{ánó} is a locative marker that can appear attached to nouns. Its general meaning is ‘inside’. The other ‘inside’ locative \textsc{é}- appears to be more widespread in usage. 

\begin{tabular}[t]{lllll}
ugi-ánó 	&	/i-á /&	[já]	&	[›ugjánó] &	‘inside the plank’\\
ome-ánó 	&	/i-á /&	[já]	&	[omjánó]  &	‘inside the fish’\\
umu-ánó 	&	/u-á/ &	[wá]	&	[umwánó]  &	‘inside the Arab (derog.)’\\
ŋombogó-ánó &	/o-á/ &	[wá]	&	[ŋombogwánó]& ‘inside the calf’\\
utɾə-ánó 	&	/ə-á/ &	[á] 	&	[utɾánó]  &	‘inside the pig’\\
ʌwíɾʌ-ánó 	&	/ʌ-á/ &	[á] 	&	[ʌwíɾánó] &	‘inside the (type of) tree’\\
aŋorá-ánó 	&	/á-á  &	[á] 	&	[aŋoránó] &	‘inside the elephant’\\
\end{tabular}

Unlike \textit{é} and \textit{n}-, -\textit{ánó} can be separated from the noun, and functions more like a post-position. It can also appear combined with verbs as a particle. 

% New stuff from microsoft word doc

Moro has postpositions, which appear after the noun phrase. However, there are two locative prefixes, \textit{é} and \textit{n}-, which affix to nouns, that could be analyzed as prepositions. Many postpositions are derived from nouns, and some appear to have incorporated one of the two locative prefixes.

\subsection{é-}
The locative prefix \textit{é}- has many allomorphs, and conditions or undergoes a number of phonological changes. See the section on nominal affixation for an overview. Here we present examples of its use:
\\
\gll ŋəndri ŋadáɾá ílədwóŋ\\
\\
\trans the ox is flat in the back (= the ox has a flat back)	\hfill June 28, 2013\\
\\
\gll kʌt̪ə́vʌ́tʃiə átʃə́váŋ íŋútʃʌ́\\
\\
\trans he is dipping bread in sauce	\hfill July 8, 2013\\
\\
\gll kʌpwʌ  é-ŋáw\\
\\
\trans he is swimming (= he is hitting in the water)	 \hfill	Jan. 15, 2010\\
\\
\gll liŋgwʌ lʌpwʌ é-ŋaw\\
\\
\trans the frog is swimming in the water\\
\\
\gll égadaŋó íkúgi\\
\\
\trans I sat in the tree\\
\\
\gll égadaŋó íkwːiə\\
\\
\trans I sat on the ground\\
\\
\gll kʌ́ssʌʧəðiə íŋ́ɾə́m\\
\\
\trans he can see in the dark	\hfill	6/23/2011\\
\\
\gll kaɾarreðó ékə́lá\\
\\
\trans he cleaned off/up the plate (w/multiple wipes)\hfill	4/18/2011\\
\\
\gll ógóvát̪ó ékansi net̪u\\
\\
\trans go back to the place you came from \hfill	3/18/2011\\
\\
\gll kúku gandró tərəbesa éŋə́mátár\\
\\
\trans Kuku slept under the table legs\\
\\
\subsection{n-}
The locative n- attaches to nouns, except those that begin with l. 
\\
\gll égadaŋó ndeté\\
\\
\trans I sat on the branch	\hfill	(branch = ðetea, n = syllabic)\\
\\\hfill
\gll égadaŋó loándra\\
\\
\trans I sat on the rock	\hfill(no n- because l-initial word)\\
\\
\gll égadaŋó notéleə\\
\\
\trans I sat on the mat	\hfill(mat = wətéliə)\\
\\
\gll loandra lʌmurkú najn\\
\\
\trans the rock rolled down the hill\\

The following postpositions all begin with \textit{é} and are high-toned, indicating a derivation from the locative preposition \textit{é}-

\subsection{éðə́pé – on, on top of}
This word is formed from the locative marker \textit{é}- and \textit{ðə́péə} ‘top of the head’
\\
\gll kúku gandró (n)tərəbésá éðə́pé\\
\\
\trans Kuku slept on the table		\hfill		6/6/2011\\

\subsection{ékáɾé – under}
This word also means ‘stomach’ and is formed from the locative marker \textit{ék}- and \textit{aɾa} ‘soul, stomach’. 
\\
\gll kaɾagó tɾambílí ékáré\\
\\
\trans he crawled under the car		\hfill 4/18/2011\\
\\
\gll kʌɾəgí tərəbésá ékáré\\
\\
\trans he passed it under the table’	\hfill	4/18/2011\\
\\
\gll égadaŋó tərəbésá ékáɾé\\
\\
\trans he sat under the table		\hfill	6/6/2011\\
\\
\gll kúku gʌninú rəmwɔ loandra ékáɾé\\
\\
\trans Kuku looked for a snake under the rock	\hfill	6/6/2011\\

\subsection{ékɾél – next to} 
This word is formed from the locative marker \textit{ék}- and \textit{eɾél}  ‘side’. The initial vowel deleted upon affixation. 
\\
\gll kúku gʌgú ŋeɾá ugi ékɾél\\
\\
\trans Kuku put the child next to the tree	\hfill 6/6/2011\\
\\					Kuku put the tree next to the child
% Why is this here....
\gll ɲagadaŋó yasir ékɾél\\
\\
\trans we sat around Elyasir	\hfill 	3/16/2010\\
\\
\gll ígʌdʌt̪rwʌ ege ékɾél ikʌ loándra gógəná\\
\\
\trans I am standing beside the house that is made of big rocks\\

\subsection{éɽo - around}
\gll kagəðat̪ó égé (éɽo)\\
\\
\trans	he went around the house		\hfill 6/16/2011\\
\\
\gll kavákːagəðat̪a ajen eɽo\\
\\
\trans he is going around the mountain	\hfill 3/9/2011\\
\\
\subsection{elo	‘above’}
\subsection{ilikano - in between, in middle}
\gll kúku kʌsʌʧú uɽi ilbʌ́mbəre ilikano\\
\\
\trans Kuku saw a mouse between/ in middle of stools		\hfill 6/6/2011\\

\subsection{ndʌwuɾ	-	out or outside}
This word is formed from the locative marker \textit{n}- and the word \textit{ʌwuɾ} ‘door’. \\
\\
\gll katéðét̪a áʧə́váŋ ndʌwuɾ\\
\\
\trans	he is scraping the food up out(side)	\hfill	6/23/2011\\
\\
\gll wáɾá ndʌwuɾ\\
\\
\trans	outside the animal pen		\hfill	6/23/2011\\

\subsection{(u)ndrt̪u	- ‘under, behind’}
\gll ugi gʌnnʌ́ʧəniə́u undrt̪u\\
\\
\trans	the tree was listened under	\\
\\
\gll mbú  ánoánəðe   ege ndúrtu\\
\\
\trans	go look at back of /behind the house	4/20/2012\\
\\
\subsection{nano}
This word means ‘near, around’, but it is also used to form particle verbs.
\\
\gll kʌpwʌ́t̪ʌ́ ðəloŋ uta nano\\
\\
\trans he is hammering a nail into the wall \hfill June 28, 2013\\
\\
\gll kadáŋád̪at̪a gí nano\\
\\
\trans	he is depending on the field (for his future)/ he is staying near the field\\
\\
\gll kadáŋáðat̪a gí nano\\
\\
\trans \hfill	July 8, 2013\\
\\
\gll kugʌtʃú ugə nano\\
\\
\trans	he fenced around the tree	\hfill July 9, 2013\\
\\
\gll égabəgá nano\\
\\
\trans	‘I’m industrious.’\\
\\
\gll kaɾéðia nano\\
\\
\trans	he feels nauseous\\
\\
\gll Kúku kʌməgí jusif ŋʌmiə nano\\
\\
\trans	Kuku placed a curse on Yusef’s spirit (caused Y to have a curse)\\
\\
\gll kawaðat̪ó ðərmbegea nano\\
\\
\trans	‘he fixed the harp’\\
\\
\gll gəla gawó ðərna nano\\
\\
\trans	‘the plate is near the animal skin’\\
\\
\gll kat̪avət̪ə́ŋó nano\\
\\
\trans	‘she spat on him’\\

\subsection{ánó}
This postposition means 
\\
\gll t̪ə́və́ðó ðəbarlánó\\
\\
\trans	cross the river!	 \hfill	June 28, 2013\\
\\
\gll kat̪ə́və́ða ajen ánó\\
\\
\trans	he is crossing the mountain\\
\\
\gll katə́və́ðeə gi ánó\\
\\
\trans	he is crossing the field		\hfill July 8, 2013  (why diff. vowel ending?)\\
\\
\gll udʒí kudɪɲit̪ú rəmwa ánó\\
\\
\trans man is kneeling before God	< kodəɲa	‘squat’\\

\subsection{ánoŋ}
\gll kadáŋá íkí\\
\\
\trans he is staying in the field	\hfill	July 8, 2013\\
\\
\gll kadáŋánoŋ íkí\\
\\
\trans he is sitting in the field\\

\subsection{áŋəno}
\gll ŋeɾá ŋaɾətiə áŋə́ná\\
\\
\trans	girl is shaking (i.e. from cold) = lit. she is dancing body \\
\\
\gll ʌ́ləgʌdə́ðiə́r áŋə́nó\\
\\
\trans	we are working hard		\hfill 6/23/2011\\
\\
\gll áləgʌməðíaráŋəno\\
\\
\trans	we are thin\\
\\
\gll gʌdə́ðíə áŋəno\\
\\
\trans	he is active\\

\subsection{ánoŋ	-	‘down’}
\gll kagákəlːat̪a reteə ánoŋ\\
\\
\trans	he is pulling branches down (on ground) \hfill	6/26/2013\\
(LOC and the ánoŋ must co-occur) \\
\\
\gll ʌ́kʌ́drí laŋ-ánoŋ\\
\\
\trans	‘make things accessible, available!’\\
\\
\gll lʌút̪ə́r-ánoŋ\\
\\
\trans	‘they are at peace’\\

% Words in Black \& Black:
\textit{alo} ‘beneath’ \textit{nʌiɲua} ‘before, in front of’, \textit{t̪waɲ} ‘near’, \textit{ŋenə ŋant̪a} ‘because of, for’, \textit{məldin} ‘until, still’ \textit{bəɽaŋ} ‘on ones own account’ \textit{garno} ‘like as’ \textit{iðurt̪u} ‘under, close behind’

\section{Adverbs}

Adverbs have free ordering, but typically occur after object in texts. The standard position for adverbs in Moro is at the end of the phrase, but they may also be placed in other positions. 

%Can these always be topicalized? can they all occur between subject and verb?

\subsection{Manner adverbs}

- Reduplication, various classes?\\

- Can these occur between subject and verb?\\

Manner adverbs conveying speed or value are as follows:\\

\begin{tabular}[t]{ll}
aten		&	‘quietly’\\
atenaten	&	‘slowly’\\
ŋgíljáŋa	&	‘loudly’\\
pə́léló		&	‘high (pitch), loudly’\\
ram			&	‘early’\\
ramram		&	‘quickly’\\
táltal		&	‘quickly’\\
ŋwəɲoŋ		&	‘fast’\\
ŋópeá		&	‘well’\\	
\end{tabular}
\vspace{8pt}
\\
\gll égaválə́ŋa ŋgíljáŋa\\
\\
\trans I sing loudly	\hfill	3/17/2010\\
\\
\gll égaválə́ŋa pə́léló	\\
\\
\trans	I sing high\\
\\
\gll kadoátá kaɲkaɲ	\\
\\
\trans he is talking loudly \hfill	10/6/2011\\

\subsection{Temporal adverbs}

The following represent a partial list of time adverbs, conveying the time at which an event occurs or occurred. Although these concepts translate as nouns in English, they are not nouns in Moro. The main argument for this position is that they do not show noun class agreement. They cannot head a clause ‘yesterday was a good day’ for example. \\
% The second sentence was highlighted. 

\begin{supertabular}[t]{lll}
ŋinɜŋi		&	now, today			&	check tones - high or low?\\
ŋinɜŋətəŋi	&	now exactly, right now\\
ilɜ́ki		&	today\\
úlɜlítú̪		&	tomorrow\\
etəkwɔ		&	day after tomorrow\\
úlɜlítá̪nó	&	morning\\
éréká		&	yesterday\\
érékánó		&	afternoon, evening\\
érékə́kɜ́i		&	day before yesterday\\
úlúŋgí		&	night\\
úlúŋguluŋ	&	all night\\
úləŋánó		&	midnight\\
bə́ŕnibərni	&	dawn\\
bəte̪		&	never\\
bəte̪bəte̪	&	never ever\\
bət̪ukəluŋ	&	long time ago\\
bətánoŋ		&	earlier in the day		&	check dentals\\
bətéréká	&	yesterday\\
bət̪érékɜ́ka / érékɜ́ka	&	the day before yesterday\\
ɜðəɲɜ́ðəɲí	&	all day				&	check tones and vowels\\
eto			&	every time\\
etoto		&	always\\
dʒatʃa		&	every day\\
ɜtəni		&	several days later\\
fərfər		&	never\\
lómanaŋ		&	once upon a time\\
pə́ndé		&	long time ago\\
t̪wánáŋ		&	long time\\
ikrəŋ		&	‘all the time’\\
ododo		&	‘always’\\
\end{supertabular}\\

\textit{In B\&B, not in Th: ananoŋ ‘before’, maijən ‘recently’, ɜðəɲinano ‘midday’, bətaŋəɽan ‘before (not today), agəloŋ ‘before (some time back)’, pənde ram ‘in the beginning’ aten t̪ia ‘after awhile’ ɜtindi ‘this morning (past)’, orn ‘later’ jaica ‘always’}

Some of these adverbs involve reduplication. The two words for ‘never’ \textit{fərfər} and \textit{bət̪ebət̪e} both do. \textit{etoto} ‘always’ and \textit{ɜ́ðə́ɲɜ́ðə́ɲí} ‘all day’ are probably derived from \textit{eto} ‘every time’ and \textit{ɜ́ðə́ɲí} ‘day’ repeated, with deletion of the last vowel of the first word, ex. /eto-eto/  [etoto]. The word \textit{úlúŋguluŋ} ‘all night’ also seems to involve some reduplication - it is full reduplication minus the final syllable. 

Many of these adverbs appear to be derived forms from a base word. The word \textit{bət̪e} ‘never’ forms the base for three words all conveying time in the past, the most transparent being \textit{bətéréká} from /bət̪e-éréká/, although its meaning is not compositional (never + yesterday). The word or suffix -\textit{ánó} is added to three words as follows:\\

\begin{tabular}[t]{llll}
úlɜlítú̪ &	‘tomorrow’ 	&	úlɜlít̪ánó 	&	‘morning’ \\
éréká 	&	‘yesterday’ 	&	érékánó 		&	‘afternoon, evening’ \\
úlúŋgí	&	‘night’		&	úləŋánó		&	‘midnight’\\
ŋínɜ́ŋí	&	'today'\\

\end{tabular}\\

\textit{éréká} is also the base for \textit{érékə́kɜ́i} ‘day before yesterday’, which could be derived from \textit{éréká} plus the demonstrative medial suffix -\textit{ikːɜi}. 
\\
\\
\gll á-g-oása  ndréð   eðá    ŋínɜ́ŋí?\\
2\textsc{sg.subj-cl-main}-wash-\textsc{ipfv}  \textsc{cl}.clothes  why  today\\
\trans Why are you washing clothes today?\\

\subsection{Locative adverbs and adjuncts}\label{section:locativeadverb}

- List adpositions/case markers. 
- Indicate how or whether these differ from locative objects, if at all

\subsection{Degree Adverbs}
\begin{tabular}[t]{ll}
káɲ			&	‘very’\\
pr			&	‘a lot’\\
dət̪əl/rət̪əl	&	‘more’\\
təllɜŋ		&	‘only all’\\
po			&	‘not filled, missing, little amount’\\
təb(ə́)tə́bu	&	‘all'\\
ʧáŋəʧáŋ		&	‘only’\\
tʃom		&	‘also, too’\\
t̪iə			&	‘like this, in this manner’\\
ikərəŋ		&	‘only’\\
d̪et̪əm		&	‘truly’\\
durri		&	‘exactly’\\
ekworəv		&	‘even more, additional’\\
iðurt̪u		&	‘after’\\
ŋwúlɜ́		&	‘only’\\
təmor		&	‘plainly, clearly’\\
t̪ɜge		&	‘enough’\\
\end{tabular}
\\
\\
\gll matʃó	g-a-tə́m-á 	ŋén	áŋəno	ododo	\\
\textsc{cl}g.man	\textsc{sm.cl}g-clause-detail-\textsc{ipfv}  \textsc{cl}ŋ.word	body 	all\\
\trans ‘the man is talking about everything’		\hfill	June 28, 2013\\
\\
\gll k-ið-ið-it̪-ú 	ŋə́mə́gə́niə 	t̪iə 	ɜðəɲɜ́ðəɲiɲ\\
\textsc{cl}g-do-??\textsc{appl-pfv}  	\textsc{cl}ŋ.work 	like.this 	all.day  	\\
\trans ‘she did work like this all day’			\hfill		July 2, 2013\\
\\
\gll álə́ŋ-ád̪át̪-ó 	t̪iə 	ilɜ́ki\\
sing-\textsc{loc.appl-ipfv}	like.this 	today	\\
\trans ‘sing this way today!’ (i.e. it’s your only opportunity)\\
\\
\gll k-a-r-at̪-ó 	ŋeniə	təmor\\
\textsc{sm.cl}g-\textsc{rtc-iter}-say-\textsc{pfv}  	\textsc{cl}ŋ.word   plainly\\
\trans ‘he said words plainly’\\
\\
\gll á-g-a-wət̪-ó 	ɲəwa 	tə́lːɜ́ŋ\\
2\textsc{sg.sm-cl-rtc}-choose-\textsc{pfv}  	\textsc{cl}ɲ.young.girls  only.all\\
\trans ‘you chose all grown girls only’\\
\\
\gll ʧom\\
\\
\trans too\\
\\
\gll ɲ́ɲí ʧom\\
\\
\trans me, too		\hfill	3/17/2010\\
\\
\gll é-g-a-v-álə́ŋ-a 	tʃáŋətʃáŋ	\\
1\textsc{sg.sm-cl-rtc-prog}-sing-\textsc{ipfv}	only	\\
\trans ‘I only sing (I don’t dance)’\\

\subsection{Other adverbs?}

'one by one', on purpose/by mistake, speaker-oriented particles (hopefully, unfortunately), any others?


