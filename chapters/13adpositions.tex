\chapter{Adpositions and adverbs}\label{chapter:adverbs}

Adpositional phrases, headed by adpositions, and adverbial phrases pattern as verbal modifiers or adjuncts in the clause. Evidence for this claim comes from the fact that they are optional. Adpositional phrases and adverbs typically occur at the right edge of the clause in texts, after objects. Adverbs typically follow adpositional phrases, but their order is basically free.

%todo EXAMPLES of optionality, recursion
%In the first two examples above, we see that an adpositional phrase must follow a nominal argument, particularly when both are inanimate. However, if the complement of the adposition is animate, it must precede inanimate objects, just as with locative objects (\secref{sec:ch12:locobj}).

%I put the bag on top of the camel.

While adpositions occur at the right edge of the verb phrase, their nominal complement can surface as a verbal clitic, when a pronoun, or be promoted to subject position, when the adpositional phrase is a locative object (\sectref{sec:ch12:locobj}). However, they at least clearly can be clausal adjuncts. One piece of evidence for this position is that, while they cannot occur preverbally, they can occur between aspectual verbs and their VP complements:

%. Manner applicatives can occur with full manner NPs, often in an oblique locative case such as the inessive.

%kúku gɜ́--gʌɲ-aðat-ó ðélaga gí ésá yento	‘Kuku finished cultivating the field in an hour’
%	kill-way-


Adverbs never surface as subjects or objects, and they order more freely than adpositional phrases. Some adverbs, such as temporal adverbs, can occur in essentially any position in the clause, including between the subject and verb.

\ea \textit{Free position of `yesterday'} \label{ex:ch13:yesterday}
\ea \gll kúku g-ɜgr-í ér:éká ádámá\\
Kuku \textsc{cl}g.\textsc{rtc}-read-\textsc{pfv} book yesterday\\
\ex \gll kúku g-ɜgr-í ádámá ér:éká\\
Kuku \textsc{cl}g.\textsc{rtc}-read-\textsc{pfv} yesterday book\\
\ex \gll kúku ér:éká g-ɜgr-í ádámá\\
Kuku yesterday \textsc{cl}g.\textsc{rtc}-read-\textsc{pfv} book\\
\ex \gll ér:éká kúku g-ɜgr-í ádámá\\
yesterday Kuku \textsc{cl}g.\textsc{rtc}-read-\textsc{pfv} book\\
\glt `Kuku read the book yesterday.' (all) \hfill EJ072518
\z
\z
However, many subclasses of adverbs, such as manner adverbs or focus adverbs, have a more restricted distribution.


Adpositional phrases and at least manner adverbs are alike in being able to alternate between adjunct and argument position with the addition of a locative applicative (\sectref{sec:ch12:locapplicative}) or manner applicative (\sectref{sec:ch12:manapplicative}) suffix on the verb. The addition of the appropriate applicative suffix enables pronominalization of their associated argument, cleft formation, and passivization, in the case of the nominal complements of adpositional phrases. 

%Some adverbs are more restricted in their distribution, including manner adverbs??? and argument-adverbs, many of which agree with a verbal argument. %%come back to this and add cites after checking adverbs with Elyasir.
%
%(20)	a.	[ N-P-DEM-NUM]
%kúkú	k-andr-ó     	[PP   n-tr̃bésá	  éðápé   íð-ɜtí-ðɜ	 ð-ɜgətʃin  ]
%		Kuku	CL-sleep-PRF	       AD-table	  on.top   CL-that-CL	 CL-three
%		‘Kuku slept on top of those three tables.’
%
%19.	aðám	kətíkə	kore	i-́ɡɜɡ-ú		tərəbesa	kare	íðətíð́ə		ðoɡəná
%	book	that	red	I-put-prf	table		under	that		big
%	‘I put that red book under that big table.’


%The position of the modifiers which follow adpositions are the same as those which occur noun-phrase internally \citep{Jenks:2014}.


%todo adjsection cite

%Aux-PP-VP
%
%girǝwa gwaña garno gwuji gǝfo neɽǝndr gǝɽiṯa đǝɽini,

\section{Adpositions}\label{sec:ch13:adpositions}

There are three ways of marking spacial relations and trajectories on nominals in Moro: adessive and inessive case markers \sectref{sec:ch6:case}, enclitic adpositions \sectref{sec:ch13:enclitic}, and lexical adpositions \sectref{sec:ch13:lexical}. These markers all interact in complex ways with different markers of verbal location, including venitive marking (\sectref{sec:ch11:venitivesem}) and locative applicatives (\sectref{sec:ch12:locapplicative}).

%More specifically, while case markers are transformed to verbal clitics (\secref{sec:ch11:clitic}), when their associated nominal is a pronoun or is extracted from the verb phrase. In contrast, adpositions occur at the right edge of the verb phrase regardless of whether they form a constituent with their associated NP, and their form is essentially invariant. 

The clearest evidence that adpositions are syntactically distinct from locative case markers in Moro is that both lexical and enclitic adpositions co-occur with case markers. Lexical adpositions typically co-occur with locative case marking on their associated nominal if the verb is one that assigns locative case marking

\ea \textit{Co-occurrence of lexical adposition and locative case}
\gll é-ɡ-a-wənd-at̪-ó	ŋə́ndiɾi		n-aín		éðápé	s-ətísa\\
	\textsc{1sg-cl}g-\textsc{rtc}-see-\textsc{loc.appl-pfv} bull		\textsc{loc.ad}-mountain	\textsc{on.top} 	\textsc{cl}j-that\\
\glt	`I saw a bull on top of that mountain.' \hfill EJ052509 
\z%check: I saw a fly next to Kuku/on top of
Enclitic adpositions, on the other hand, typically occur together with accusative case-marked nouns:
\ea \textit{Co-occurence of enclitic adposition and accusative case}
\gll í-g-ɜg-ú ádámá Kuku-ŋ nano\\
\textsc{1sg-cl}g.\textsc{rtc}-put-\textsc{pfv} book Kuku-\textsc{acc} at\\
\glt `I put the book next to Kuku.'\hfill EJ072518
\z  %on top of Kuku?
The ability of both of these adpositions to co-occur with case indicates that they are not case forms themselves.

Further evidence for this distinction comes from the fact that adpositions are syntactically independent while locative and instrumental case markers are not. For example, while case markers are always nominal affixes, adpositions can appear without their associated nouns or noun phrases, such as in cases where locative objects are passivized, stranding the adposition (\sectref{sec:ch12:locobj}). In such environments, locative case markers would be transformed into the locative enclitic \textit{=u} (\sectref{sec:ch11:locative}). 

Two criteria distinguish enclitic adpositions from lexical adpositions in Moro. First, while both classes of adpositions take noun phrase complements, enclitic adpositions or particles do so only optionally, as they also form complex predicates with verbs (\sectref{sec:ch12:particle}) or adjectives (\sectref{section:adjective}) without any accompanying nominal argument. The semantics of the particles in these constructions can be transparent or they can be lexicalized together with the verb, forming an irregular meaning. In contrast, lexical adpositions always take a nominal complement and do not form complex predicates with verbs.

Second, while enclitic adpositions are `pure' adpositions in the sense that they have no other use in the language, lexical adpositions are transparently related to body part terms, and are often locative case marked variants of them.
 

\subsection{Internal syntax of adpositional phrases}\label{sec:ch13:adsyntax}

The position of both lexical and enclitic adpositions relative to noun phrases in Moro is typologically unusual: they are \textit{intrapositions}, occurring in the middle of their noun phrase complement. Their position there is fixed; they must follow the head noun and precede all nominal modifiers:

\ea \ea \textit{Noun phrase order: 
		N-Dem-Num-Adj} (\sectref{sec:ch8:nporder})\\
\gll 	tərəbésá	íð:-ɜtíðɜ   ð-əgitʃin ð-oɡən-á \\
		table	\textsc{scl}ð-that \textsc{cl}ð-three \textsc{cl}ð-big-\textsc{adj} \\
\glt `those three big tables' 
	\ex \textit{Adposition position (lexical adposition): 
		N-\textbf{Ad}-Dem-Num-Adj}\\ \label{ex:ch13:under}
\gll 	tərəbésá	\textbf{ékáɾé}	íð:-ɜtíðɜ   ð-əgitʃin ð-oɡən-á \\
		table	under	\textsc{scl}ð-that \textsc{cl}ð-three \textsc{cl}ð-big-\textsc{adj} \\
\glt `under those three big tables' 
	\ex \textit{Adposition position (enclitic adposition): 
		N-\textbf{Ad}-Dem-Num-Adj}\\
\gll 	tərəbésá	\textbf{=nano}	íð:-ɜtíðɜ   ð-əgitʃin ð-oɡən-á \\
		table	near	\textsc{scl}ð-that \textsc{cl}ð-three \textsc{cl}ð-big-\textsc{adj} \\
\glt `near those three big tables' 
\z 
\z 

While the entire \textit{noun-adposition-modifier*} sequence forms a constituent, adpositions also form a subconstituent with following modifiers to the exclusion of the preceding noun. There are two pieces of evidence which support this constituency. 

First, adpositions and the following modifiers can occur discontinuously from their associated noun, which can be passivized (\sectref{sec:ch12:passive}) or extracted to form a focus cleft. %(\sectref{sec:chXX:cleft}}
Second, adpositions and following modifiers can be coordinated to the exclusion of a shared nominal:

\ea \textit{Coordination of AdP $+$ modifiers}\\
\gll	é-g-a-daŋ-ó   təɾəbésá  ékaɾe íð:-i  nə=  éðápé íð:-ɜtíðɜ\\
\textsc{1sg}-\textsc{cl}g-\textsc{rtc}-sit-\textsc{pfv} table under \textsc{cl}ð-this and on.top.of \textsc{scl}ð-that\\
\glt		`I sat under this and on top of that table.'
\hfill EJ110111
\z 
Thus, while they have a number of differences in their distribution, enclitic adpositions and lexical adpositions seem to be fundamentally similar in terms of their syntax relative to their associated noun phrase.

\subsection{Enclitic adpositions}\label{sec:ch13:enclitic}  

The following five enclitic adpositions have been identified in Moro. 

\ea	\textit{Enclitic adpositions}\\
	\begin{tabular}{ll}
	\lsptoprule
	{=ánó} & `inside, through' (`interior to')\\
	{=nano} & `near, at, by' (`exterior to')\\
	{=ánoŋ} & `down' \\
	{=éló} & `up' \\
	{=éɽo} & `around' \\
	{=áŋə́nó} & `bodily' \\
	\lspbottomrule
	\end{tabular}
\z 
The adpositions \textit{=ánó} and \textit{=nano} are common, but lack a clear translation into English as they entail general location in the interior to or exterior to a bounded space. 

The distinctive syntactic property of enclitic adpositions is that they can occur with or without nominal complements, while lexical adpositions typically must occur with a nominal complement.

\ea \textit{Enclitic adpositions with nominal complement}\\
\ea \gll g-a-t̪ə́və́ð-a gi ={ánó}\\
cross-\textsc{imper} field ={inside}\\
\glt	`S/he is crossing through the field.' \hfill EJ070813
\ex \gll g-ɜ-pw-ʌ́t̪-ʌ́ ðəloŋ uta =nano\\
\textsc{cl}g-\textsc{rtc}-beat-\textsc{loc}-\textsc{ipfv} nail wall {=at}\\
\glt `S/he is hammering a nail into the wall.' \hfill EJ062813
\ex \gll k-agəðat̪-ó égé {=éɽo}\\
\textsc{cl}g-circle-\textsc{pfv} house =around\\
\glt	`S/he went around the house.'		\hfill EJ061611\\
\z 
\ex \textit{Enclitic adpositions with no nominal complement}\\
\ea \gll Kúku g-abort̪-o =éló ɲomon ɲigitʃin \\
Kuku \textsc{cl}g-ascend-\textsc{pfv} =up times \textsc{cl}ɲ-three \\
\glt `Kuku jumped up three times.' \hfill	EJ072618
\ex \gll é-g-a-daŋ-ó =ánoŋ \\
\textsc{1sg-cl}g-\textsc{rtc}-sit-\textsc{pfv} =down {}\\
\glt `I am sitting down.' \hfill	EJ072618
\z
\z

The one exception to this generalization is the semantically surprising adposition \textit{=áŋə́nó} `bodily', which never takes a nominal complement.
\ea \textit{Enclitic adposition with no nominal object}\\
 \gll ʌ́lə-g-ɜ-də́ð-iə́-r =áŋə́nó\\
\textsc{1pl.in-cl}g-\textsc{rtc-}work-\textsc{ipfv-pl} =bodil\\
\glt `We are working hard.'		\hfill 6/23/2011\\
\z
One might wonder what evidence there is to consider \textit{=áŋə́nó} an enclitic adposition at all, or whether it is better classified as an adverb. 

The evidence that it is an enclitic adpositions comes from particle verbs (\sectref{sec:ch12:particle}) and particle adjectives (\sectref{sec:ch10:admorph}). Enclitic adpositions are unique in their ability to form complex predicates, which consist of a verb or adjective along with an obligatory enclitic adposition, often corresponding in a somewhat irregular meaning. 
\ea \textit{Transitive and intransitive particle verbs}\\
\ea \gll g-a-dáŋ-ád̪-at̪-a gí =nano\\
 \textsc{cl}g-\textsc{rtc}-?-\textsc{loc.appl-ipfv} field =at\\
\glt	`S/he is depending on the field (for his future).' (\textit{literally}: `S/he is staying near the field.'\\
\ex \gll é-ga-bə́g-á =nano\\
\textsc{1sg-cl}g-\textsc{rtc}-hit-\textsc{ipfv} =at\\
\glt	`I’m industrious.' (\textit{literally}: `I'm hitting at/around.')
\z 
\z 

The fact that \textit{áŋə́nó} `bodily' participates in particle verb formation, both with verbs \REF{ex:ch13:bodyv} and with adjectives \REF{ex:ch13:bodya}, indicates that it is an enclitic adposition.

\ea \textit{Complex predicates with  \textit{áŋə́nó} `bodily'} 
\ea \gll ŋeɾá ŋ-a-ɾə́t-íə =áŋə́nó\\
			child \textsc{cl}g-\textsc{rtc-}dance-\textsc{ipfv} =bodily \\
\glt	`The child is shivering.' (i.e. from cold) \label{ex:ch13:bodyv}
\ex \gll g-a-ɲə́l-á =áŋə́nó\\ 
\textsc{cl}g-\textsc{rtc}-\textsc{sweet}-\textsc{adj} bodily\\
\glt `S/he is lazy.’ (\textit{literally}: `Sweet-bodied')\label{ex:ch13:bodya}
\z 
\z
The adposition \textit{=áŋə́nó} literally means `body', and as the examples above demonstrate typically describes actions involving the body. One way to make sense of the fact that it never takes a nominal complement, then, is with the assumption that it inherently includes `body' as the ground or location, and in that sense might best be translated `at x's body' or `by x's body.' where the body is typically that of the subject. This would render a literal translation of \REF{ex:ch13:bodyv}, for example, as `The child is dancing with her body,' which idiomatically means `shiver.'

While the nominal complement of enclitic adpositions are not locative case-marked, passivizing the nominal complement of a locative case-marked adposition consistently results in the presence of a locative clitic on the verb (\sectref{sec:ch11:locative}).

On the phonological side, enclitic adpositions are distinctive from lexical adpositions in that they fuse with the preceding word in normal-paced speech. This pattern resembles the fusion of objects with verb and the fusion of demonstratives and other modifiers with nouns. Encliticization is most transparently visible with the vowel-initial forms, which undergo hiatus with the final vowel  of the preceding word.  %TODO add section ref to HIATUS

\ea \gll é-g-a-daŋ-ó =ánoŋ {\ \ \ \ \ \ \ \ \ \ $\to$ [égadaŋánoŋ]}\\
\textsc{1sg-cl}g-\textsc{rtc}-sit-\textsc{pfv} =down {}\\
\glt `I am sitting down.' \hfill	EJ072618
\ex \gll t̪ə́və́ð-ó ðəbarəla =ánó {\ \ \ \ \ \ \ \ \ \ $\to$ [ðəbarlánó]}\\
cross-\textsc{imper} river =inside {} \\
\glt `Cross the river!'	 \hfill	EJ062813
\z
Vowel hiatus is illustrated in detail for each of the vowel possibilities for the adposition \textit{=ánó} `inside' below.

\ea \textit{Vowel hiatus with noun and enclitic adposition}\\
\begin{tabular}[t]{lllll}
\lsptoprule
ugi-ánó 	&	/i-á /&	[já]	&	[ugjánó] &	‘inside the plank’\\
ome-ánó 	&	/e-á /&	[já]	&	[omjánó]  &	‘inside the fish’\\
umu-ánó 	&	/u-á/ &	[wá]	&	[umwánó]  &	‘inside the Arab (derog.)’\\
ŋombogó-ánó &	/o-á/ &	[wá]	&	[ŋombogwánó]& ‘inside the calf’\\
utɾə-ánó 	&	/ə-á/ &	[á] 	&	[utɾánó]  &	‘inside the pig’\\
ɜwíɾɜ-ánó 	&	/ɜ-á/ &	[á] 	&	[ɜwíɾánó] &	‘inside the (type of) tree’\\
aŋorá-ánó 	&	/á-á  &	[á] 	&	[aŋoránó] &	‘inside the elephant’\\
\lspbottomrule
\end{tabular}
\z
While enclitic adpositions do fuse with the preceding word, they are independent for the purposes of vowel harmony and do not have any significant tonal interactions with that word.

Cliticization is syntactically conditioned: enclitic adpositions do not typically fuse with an intervening adverb or with noun that is not their complement. The following examples illustrate syntactic conditioning with the consonant-initial enclitic \textit{=nano}, as the following pair demonstrates.

\ea \ea í-g-ɜg-ú kúku-ŋ ádámá =nano {\ \ \ \ \ \ \ \ \ \ $\to$ [ádámnano]} \\
\textsc{1sg-cl}g-put-\textsc{pfv} Kuku-\textsc{acc} book =at {}\\
`I put Kuku next to the book' \label{ex:ch13:bookat} 
\ex ígɜgú=ŋó ádámá nano {\ \ \ \ \ \ \ \ \ \ \ \ \ \ \ \ \ \ \ \ \ \ $\to$ [ádámá nano]} 
\textsc{1sg-cl}g-put-\textsc{pfv}-\textsc{3sg.om} book book =at {}\\
\glt `I put the book next to him.'\label{ex:ch13:himat} \hfill EJ072518 
\z 
\z

In \REF{ex:ch13:bookat}, the enclitic adposition \textit{=nano} immediately follows its complement noun, with which it fuses to form a single phonological unit. In \REF{ex:ch13:himat}, the complement of this adposition is a pronoun, and as a result appears on the verb as a clitic. This results in the surface-adjacency of the adposition \textit{nano} with another noun, and these two elements surface as separate words. The generalization seems to be that the enclitic adposition must be in the same immediate constituent as the preceding word to fuse with it; no major phrase boundary can separate them.

%todo what are the case properties of 'book'? It would be nice to double-check this with more examples.


%\subsubsection{\textit{=ánó }`inside'}
%
%The adposition \textit{=ánó} means ‘inside’, and denotes containment in an enclosed space. This meaning is illustrated in the following 
%
%This postposition means `inside'
%%todo EXAMPLES from Peterson
%
%This adposition occurs with motion predicates, where it can indicate movement through a bounded space:
% 
%\ea \gll g-a-t̪ə́və́ð-a ajen ánó\\
%cross-\textsc{imper} river-\textsc{inside}\\
%\glt	he is crossing the mountain\\
%\ex \gll katə́və́ðeə gi ánó\\
%\\
%\glt	he is crossing the field		\hfill July 8, 2013  (why diff. vowel ending?)\\
%
%It can also form participles with certain verbs, for example with the verb \textit{-udəɲit̪-} `squat' to form the verb meaning `kneel.'


%\subsubsection{\textit{=nano} `near, at, by'}
%
%The adposition \textit{nano} denotes proximity. It has no direct English translation, but is translated variably as `near, at, by, on, and around', though we gloss it as `at.'
%
%In a simple locative copular clause, the :
%
%
%
%\ex \gll ka-ɾéð-ia nano\\
%\\
%\glt	he feels nauseous\\
%\ex \gll Kúku kɜməgí jusif ŋɜmiə nano\\
%\\
%\glt	Kuku placed a curse on Yusef’s spirit (caused Y to have a curse)\\
%\ex \gll kawaðat̪ó ðərmbegea nano\\
%\\
%\glt	‘he fixed the harp’\\
%\ex \gll kat̪avət̪ə́ŋó nano\\
%\\
%\glt	‘she spat on him’\\
%\z 

%The following postpositions all begin with \textit{é} and are high-toned, indicating a derivation from the locative preposition \textit{é}-

%\ex \gll áləgɜməðíaráŋəno\\
%\\
%\glt	we are thin\\
%\ex \gll gɜdə́ðíə áŋəno\\
%\\
%\glt	he is active\\
%\z 

%\textit{ánoŋ} ‘down’ \textit{alo} ‘beneath’ in Wërria.
%
%\ea \gll kagákəlːat̪a reteə ánoŋ\\
%\\
%\glt	he is pulling branches down (on ground) \hfill	6/26/2013\\
%(LOC and the ánoŋ must co-occur) \\
%\ex \gll ʌ́kʌ́drí laŋ-ánoŋ\\
%\\
%\glt	‘make things accessible, available!’\\
%
%\textit{ánoŋ} `down'
%
%\ea \gll kadáŋá íkí\\
%\\
%\glt he is staying in the field	\hfill	July 8, 2013\\
%\ex \gll kadáŋánoŋ íkí\\
%\\
%\glt he is sitting in the field\\
%\z 

% Words in Black \& Black:
%\textit{nɜiɲua} ‘before, in front of’, \textit{t̪waɲ} ‘near’, \textit{ŋenə ŋant̪a} ‘because of, for’, \textit{məldin} ‘until, still’ \textit{bəɽaŋ} ‘on ones own account’ \textit{garno} ‘like as’ \textit{iðurt̪u} ‘under, close behind’  

%
%\subsubsection{\textit{éɽo} `around'}
%
%What category?
%
%\\
%\z



\subsection{Lexical adpositions}\label{sec:ch13:lexical} 

Lexical adpositions are morphologically complex, consisting of a locative prefix and a lexical noun which denotes a body part or general spacial location. The locative prefix can be either inessive or adessive. 

\ea \textit{Lexical adpositions}\\
\begin{tabular}{llcll} %todo what is lədwoŋ?
	\lsptoprule
{éðə́pé} & `on top of' & $\leftarrow$ & {e-ðapeə} & \textsc{loc.in}-top.of.head \\
{ékə́ɾél} & `beside' & $\leftarrow$ &  {ék}-{eɾél} & \textsc{loc.in}-side \\
{ékáɾé} & `inside' & $\leftarrow$ & {ék}-{aɾa} &  \textsc{loc.in}-stomach \\ 
ékarjánó & `inside' & $\leftarrow$ & {ék}-{aɾa}-ánó &  \textsc{loc.in}-stomach-inside \\
{ílika} & `in the middle' & $\leftarrow$ &  {í}-{lika} & \textsc{loc.in}-middle \\
{ílikánó} & `between' & $\leftarrow$ &  {í}-{lika-ánó} & \textsc{loc.in}-middle-inside \\
%ílədwoŋ & `behind' &  & $\leftarrow$ &  {í}-{lədwɜŋ} & \textsc{loc.in}-? \\
{ndrt̪u} & `beneath, behind' & $\leftarrow$ &  n-rt̪u & \textsc{loc.ad}-back \\
%ńdrea & & `around' & $\leftarrow$ &  n-rea & \textsc{loc.ad}-round? \\
{ndɜwuɾ} & `outside' & $\leftarrow$ & {n}-{ɜwuɾ}-ánó & \textsc{loc.ad}-door \\
{ndɜwuɾánó} & `outside' & $\leftarrow$ & {n}-\textit{ɜwuɾ}-ánó & \textsc{loc.ad}-door-inside \\
	\lspbottomrule
\end{tabular}
\z 
This table reveals that a number of lexical adpositions additionally contain the enclitic adposition \textit{ánó}, sometimes without an obvious change in meaning. The exact contribution of the enclitic in these forms is unclear.

Syntactically, lexical adpositions almost always occur with a nominal complement. As with all adpositions, the head noun of this complement occurs to the left of the adposition while the modifiers occur to its right \REF{ex:ch13:under}. The head noun either does or does not take a locative case marker.

\ea 
\gll kúku g-andr-ó (n)-tərəbésá éðə́pé\\
\\ Kuku \textsc{cl}g-sleep-\textsc{pfv} \textsc{loc.ad}-table on.top\\
\glt Kuku slept on top of the table		\hfill		EJ060611
\z 

%\ex \gll ɲagadaŋó Kuku-(??ŋ) ékɾél\\ %todo: can acc appear here?
%\\
%\glt we sat around Elyasir	\hfill 	3/16/2010\\


While the nominal head of lexical adpositions does not need to take a case marker, passivization of that head always triggers the appearance of a locative clitic on the verb.

\ea \gll ugi g-ɜ-n:-ʌ́ʧ-ən-iə́=u ndrt̪u\\
tree \textsc{cl}g-\textsc{rtc}-hear-\textsc{loc.appl-pass-ipfv=loc} under\\ 
\glt	the tree was listened under	\\
\z

This fact also holds for the nominal clitic of enclitic adpositions, and indicates that they together form a syntactic class of locative objects, discussed in \sectref{sec:ch12:locobj}.

%
%\ea \gll kaɾagó tɾambílí ékáré\\
%\\
%\glt he crawled under the car		\hfill 4/18/201
%\ex \gll kɜɾəgí tərəbésá ékáré\\
%\\
%\glt he passed it under the table’	\hfill	4/18/2011
%\ex \gll égadaŋó tərəbésá ékáɾé\\
%\\
%\glt he sat under the table		\hfill	6/6/2011
%\ex \gll kúku gɜninú rəmwɔ loandra ékáɾé\\
%\\
%\glt Kuku looked for a snake under the rock	\hfill	6/6/2011
%\z 
%
%\ea \gll kúku gɜgú ŋeɾá ugi ékɾél\\
%\\
%\glt Kuku put the child next to the tree	\hfill 6/6/2011\\
%\\					Kuku put the tree next to the child
%
%
%\ex \gll ɲagadaŋó yasir ékɾél\\
%\\
%\glt we sat around Elyasir	\hfill 	3/16/2010\\
%
%\ex \gll ígɜdɜt̪rwɜ ege ékɾél ikɜ loándra gógəná\\
%\\
%\glt I am standing beside the house that is made of big rocks\\
%\z 


%\ea \gll kúku kɜsɜʧú uɽi ilbʌ́mbəre ilikano\\
%\\
%\glt Kuku saw a mouse between/ in middle of stools		\hfill 6/6/2011\\
%\z 
%

%\ea \gll katéðét̪a áʧə́váŋ ndɜwuɾ\\
%\\
%\glt	he is scraping the food up out(side)	\hfill	6/23/2011\\
%\ex \gll wáɾá ndɜwuɾ\\
%\\
%\glt	outside the animal pen		\hfill	6/23/2011
%\z

%\ea \gll ugi gɜnnʌ́ʧəniə́u undrt̪u\\
%\\
%\glt	the tree was listened under	\\
%\\
%\gll mbú  ánoánəðe   ege ndúrtu\\
%\\
%\glt	go look at back of /behind the house	4/20/2012\\
%\z

\section{Adverbs} \label{sec:ch13:adverbs}

There are several different classes of adverbs in Moro, whose syntactic properties vary along two parameters. The first is syntactic: some adverb classes, such as manner adverbs, must follow the verb, while others, such as temporal adverbs, can appear anywhere. The second parameter of variation is morphological: some adverbs agree with a specified argument in the clause, typically the subject. The different classes of adverbs and their properties are outlined below. %Npi?

\subsection{The syntax of adverbs}\label{sec:ch13:adverbsyntax}

In texts, adverbs typically occur clause-finally, after all nominal arguments and locative phrases. However, elicitation judgment tasks reveal that adverbs order order freely, although different classes of adverbs occur in different positions.

Different classes of adverbs pattern differently in terms of their clausal position. There are three positions that adverbs can appear in: in the verb phrase, after the verb, before the subject, in topic position, and between the subject and the verb. There are adverbs which seem to be restricted to all possible combinations of patterns. (\tabref{tab:ch13:adverbs}).

Some adverbs also are oriented towards particular noun phrases in the clause, in the sense that they are interpreted as adverbial quantifiers that take that noun phrase as their restriction. These include focus adverbs, floating quantifiers, and argument-oriented adverbs (\sectref{sec:ch13:argorient}), the latter of which show agreement with their associated noun phrase. 

When adverbs can occur in the VP, they are able to freely order with objects, in distinction with, for example, oblique objects and adpositional phrases, which often must  follow nominal objects.

\begin{table}
	\caption{Syntactic positions classes of adverb classes} \label{tab:ch13:adverbs}
\begin{tabular}{p{0.4\textwidth}lll}
\lsptoprule
	 & \textbf{Adv}-Subj & Subj-\textbf{Adv}-V & V-\textbf{Adv} \\
\midrule 
Manner adverbs (\sectref{sec:ch13:manner}) & no & no & yes \\
Focus adverbs (\sectref{sec:ch13:argorient}) & no & yes & yes \\
Measure phrases (\sectref{sec:ch13:temporal}) & yes & no &  yes\\
Temporal adverbs (\sectref{sec:ch13:temporal}) & yes & yes & yes \\
\lspbottomrule
\end{tabular}
\end{table}


%\ea 
%\ea	kúku ga-nwan-at̪-ó	úri	tərəbésá	ékáre	\textbf{ɲómón}	\textbf{ɲ-iɡítʃín}
%\ex 	kúku g-a-nwan-at̪-ó	úri	\textbf{ɲómón}	\textbf{ɲ-iɡítʃín} tərəbésá	ékáre	\\
%\ex 	kúku g-a-nwan-at̪-ó	\textbf{ɲómón}	\textbf{ɲ-iɡítʃín} úri	 tərəbésá	ékáré	\\
%\ex[*]{kúku \textbf{ɲómón}	\textbf{ɲ-iɡítʃín}	g-a-nwan-at̪-ó	 úri	 tərəbésá	ékáré }
%\ex \gll \textbf{ɲómón}	\textbf{ɲ-iɡítʃín}	kúku g-a-nwan-at̪-ó	 úri	 tərəbésá	ékáré	\\
%	{times}		\textsc{cl}ɲ-{three} Kuku \textsc{cl}g-\textsc{rtc}-see-\textsc{loc.appl-pfv}	mouse	table	under \\
%	\glt `I saw the mouse under the table three times.' (all examples) \hfill EJ072509
%	\z
%\z 	
%
Clause-final position is information-structurally neutral, so it is unsurprising that all adverbs occur there. The pre-subject position is due to topicalization (\sectref{sec:chX:topicalization}). This may explain the restriction on manner adverbs occurring preverbally, as these adverbs otherwise require a manner applicative to be extracted (\sectref{sec:ch12:manapplicative}). If correct, this means there are two separate restrictions: adverbs which can attach above VP (temporal adverbs and focus adverbs, when associated with the subject), and adverbs which can or cannot topicalize (temporal adverbs and measure phrases versus manner adverbs).



%\ea 
%	\ea \gll	kúku g-a-bort̪-ó	éló \textbf{ɲómón}	\textbf{ɲ-iɡítʃín}\\
%	Kuku \textsc{cl}g-\textsc{rtc}-jump.up-\textsc{pfv}	up	{times}		\textsc{cl}ɲ-{three}\\
%\ex[*]{ \gll	kúku \textbf{ɲómón}	\textbf{ɲ-iɡítʃín} g-a-bort̪-ó	éló \\
%	Kuku \textsc{cl}g-\textsc{rtc}-jump.up-\textsc{pfv}	up	{times}		\textsc{cl}ɲ-{three}\\}
%\ex \gll	\textbf{ɲómón}	\textbf{ɲ-iɡítʃín} kúku g-a-bort̪-ó	éló \\
%	{times}		\textsc{cl}ɲ-{three} Kuku \textsc{cl}g-\textsc{rtc}-jump.up-\textsc{pfv}	up	\\
%\glt `Kuku jumped up three times.' (all grammatical examples) \hfill EJ072618
%	\z
%\z 	

\subsection{Temporal adverbs}\label{sec:ch13:temporal}

The following represent a partial list of temporal adverbs, conveying the time at which an event occurs or occurred. Many of these concepts translate as nouns in English, although they do note have the syntactic distribution of nouns in Moro. Evidence for this position is that they do not occur in subject position, and that they can serve the answer to a content question asked with \textit{when}:


\ea
\ea \gll	ŋə́ní	ŋ-aɾ-ó 	éréká	kaɲ\\
		\textsc{cl}ŋ.dog	\textsc{sm.cl}ŋ-cry-\textsc{pfv}	yesterday	loudly\\
\trans		‘The dog barked loudly yesterday.’\\
\ex \gll	ŋə́ní	ŋ-aɾ-ó	ndóŋ	kaɲ?\\
		\textsc{cl}ŋ.dog	\textsc{sm.cl}ŋ-cry-\textsc{pfv}	when	loudly\\
\trans		‘When did the dog bark loudly?’\\
\z
\z
A list of temporal adverbs is provided below.
\ea \textit{Temporal adverbs}\\
\begin{supertabular}[t]{ll}
ŋinɜŋ:i		&	`now, today'			\\%	check tones - high or low?\\
ŋinɜŋətəŋ:i	&	`now exactly, right now'\\
ilɜ́ki		&  `today'\\
úlɜlítú̪		&	`tomorrow'\\
etəkwɔ		&	`day after tomorrow'\\
úlɜlítá̪nó	&	`morning'\\
éréká		&	`yesterday'\\
érékánó		&	`afternoon, evening'\\
érékə́kɜ́i	&	`day before yesterday'\\
úlúŋgí		&	`night'\\
úlúŋguluŋ	&	`all night'\\
úləŋánó		&	`midnight'\\
bə́ŕnibərni	&	`dawn'\\
bəte̪		&	`never'\\
bəte̪bəte̪	&	`never ever'\\
bət̪ukəluŋ	&	`long time ago'\\
bətánoŋ		&	`earlier in the day'	\\%todo 	check dentals
bətéréká	&	`yesterday'\\
bət̪érékɜ́ka / érékɜ́ka	&	`the day before yesterday'\\
ɜðəɲɜ́ðəɲí	&	`all day'		\\ %todo check tones and vowels\\
eto			&	`every time'\\
etoto		&	`always'\\
dʒatʃa		&	`every day'\\
ɜtəni		&	`several days later'\\
fərfər		&	`never'\\
lómanaŋ		&	`once upon a time'\\
pə́ndé		&	`long time ago'\\
t̪wánáŋ		&	`long time'\\
ikrəŋ		&	‘all the time’\\
ododo		&	‘always’\\
\end{supertabular}
\z 
Some of these adverbs involve reduplication. The two words for ‘never’ \textit{fərfər} and \textit{bət̪ebət̪e} both do. The adverbs \textit{etoto} ‘always’ and \textit{ɜ́ðə́ɲɜ́ðə́ɲí} ‘all day’ are probably derived from \textit{eto} ‘every time’ and \textit{ɜ́ðə́ɲí} ‘day’ repeated, with deletion of the last vowel of the first word, ex. /eto-eto/  [etoto]. The word \textit{úlúŋguluŋ} ‘all night’ also seems to involve some reduplication --- it is full reduplication minus the final syllable.

%\textit{In B\&B, not in Th: ananoŋ ‘before’, maijən ‘recently’, ɜðəɲinano ‘midday’, bətaŋəɽan ‘before (not today), agəloŋ ‘before (some time back)’, pənde ram ‘in the beginning’ aten t̪ia ‘after awhile’ ɜtindi ‘this morning (past)’, orn ‘later’ jaica ‘always’}

Many of these adverbs appear to be derived forms from a base word. The word \textit{bət̪e} ‘never’ forms the base for three words all conveying time in the past, the most transparent being \textit{bətéréká} from /bət̪e-éréká/, although its meaning is not compositional (never + yesterday). The enclitic adposition \textit{=ánó} `inside' is added to three words as follows:

\ea \begin{tabular}[t]{llll}
úlɜlítú̪ &	‘tomorrow’ 	&	úlɜlít̪ánó 	&	‘morning’ \\
éréká 	&	‘yesterday’ 	&	érékánó 		&	‘afternoon, evening’ \\
úlúŋgí	&	‘night’		&	úləŋánó		&	‘midnight’\\
ŋínɜ́ŋí	&	'today'\\
\end{tabular}\\
\z

The adverb \textit{éréká} `yesterday' is also the base for \textit{érékə́kɜ́i} ‘day before yesterday’, which could be derived from \textit{éréká} plus the demonstrative medial suffix -\textit{ikːɜi}, literally, `that yesterday.' Other temporal adverbs include determiner elements towards the end as well, including \textit{ŋinɜŋ:i} `now, today,' whose final component \textit{=iŋ:i} which looks like the proximal demonstrative (\sectref{demonstratives}) in concord with the initial component \textit{ŋinɜ-}, which does not seem to have an independent meaning. The word for `once upon a time' seems like a reduced version of the indefinite pronoun \textit{lomən-lənəŋ} `time-some' (\sectref{sec:ch8:indpro}).

Temporal adverbs typically appear at the right edge of the verb phrase.
\ea \gll á-g-oása  ndréð   eðá    ŋínɜ́ŋí?\\
2\textsc{sg-cl}g-\textsc{rtc}-wash-\textsc{ipfv}  \textsc{cl}.clothes  why  today\\
\glt `Why are you washing clothes today?'\\
\z 
But the distribution of temporal adverbs is the freest amont adverbs, as they can occur between any major constituent in the clause, illustrated in example \REF{ex:ch13:yesterday}.

A distinct subclass of temporal adverbs is adverbial measure phrases, which resemble quantified noun phrases but with a time-describing head. Unlike other temporal adverbs, temporal measure phrases cannot occur in the position between subjects and verbs.
\ea \textit{Distribution of adverbial measure phrases}\\
\ea	kúku ga-nwan-at̪-ó	úri	tərəbésá	ékáre	\textbf{ɲómón}	\textbf{ɲ-iɡítʃín}
\ex 	kúku g-a-nwan-at̪-ó	úri	\textbf{ɲómón}	\textbf{ɲ-iɡítʃín} tərəbésá	ékáre	\\
\ex 	kúku g-a-nwan-at̪-ó	\textbf{ɲómón}	\textbf{ɲ-iɡítʃín} úri	 tərəbésá	ékáré	\\
\ex[*]{kúku \textbf{ɲómón}	\textbf{ɲ-iɡítʃín}	g-a-nwan-at̪-ó	 úri	 tərəbésá	ékáré }
\ex \gll \textbf{ɲómón}	\textbf{ɲ-iɡítʃín}	kúku g-a-nwan-at̪-ó	 úri	 tərəbésá	ékáré	\\
	{times}		\textsc{cl}ɲ-{three} Kuku \textsc{cl}g-\textsc{rtc}-see-\textsc{loc.appl-pfv}	mouse	table	under \\
	\glt `I saw the mouse under the table three times.' (all examples) \hfill EJ072509
	\z
\z 	
This restriction may be due to the fact that measure phrases narrowly scope over the event described by the VP, but not over the higher tense and aspect information included in the preverb and aspectual affixes on the verb. Yet the pre-subject position is a topic position, which the measure phrase can occupy.

\subsection{Manner adverbs}\label{sec:ch13:manner}

%check: do true manner adverbs have to be postverbal?
Manner adverbs are are VP modifiers. They describe aspects of the event described in the VP. Several syntactic features distinguish manner adverbs: they must appear postverbally, they can be answers to content questions with \textit{t̪au} `how' (\sectref{sec:chx:how}), and they can be pronominalized with \textit{tia} `thus, in this way,' which always triggers a manner applicative (\sectref{sec:ch12:manapplicative}).

There are two classes of manner adverbs, descriptive manner adverbs and degree adverbs. They are primarily distinguished notionally, as they seem to essentially pattern the same for syntactic purposes.
\ea \textit{Manner adverbs}\\
\begin{tabular}[t]{ll}
\multicolumn{2}{l}{\textit{Descriptive manner adverbs}} \\
aten		&	‘quietly’\\
atenaten	&	‘slowly’\\
ŋgíljáŋa	&	‘loudly’\\
pə́léló		&	‘high (pitch), loudly’\\
ram			&	‘early’\\
ramram		&	‘quickly’\\
táltal		&	‘quickly’\\
ŋwəɲoŋ		&	‘fast’\\
ŋópeá		&	‘well’\\	
təmor		&	`plainly' \\
t̪iə			&	‘like this, in this manner’\\
\multicolumn{2}{l}{\textit{Degree adverbs}} \\
káɲ			&	‘very, really’\\
pr			&	‘a lot’\\
dət̪əl/rət̪əl	&	‘more’\\
po			&	‘not filled, missing, little amount’\\
təb(ə́)tə́bu	&	‘all, completely'\\
d̪et̪əm		&	‘truly’\\
durri		&	‘exactly’\\
%iðurt̪u		&	‘after’\\
%t̪ɜge		&	‘enough’\\
\end{tabular}
\z 
Many manner adverb involve reduplication, although the simple adverb and its reduplicated variant often have distinct semantics.  

Manner adverbs can only occur postverbally, illustrated below with the degree adverb \textit{kaɲ}. 
\ea \textit{Distribution of degree adverbs}\\ \label{ex:ch13:degree}
	\ea \gll kúku g-obəð-ó \textbf{kaɲ}\\
			Kuku \textsc{cl}g.\textsc{rtc}-flee-\textsc{pfv} very\\
		\glt `Kuku ran away quickly.' (\textit{lit.}: `Kuku really ran away.') \hfill EJ072618
	\ex[*]{kúku \textbf{kaɲ} g-obəð-ó}
	\ex[*]{\textbf{kaɲ} kúku gobəðó }
	\z 
\z 
This example also illustrates that the use of degree adverbs in Moro is somewhat broader than their translational counterparts in some other languages. In the example above, \textit{kaɲ} `very, really' is able to intensify the typical manner of `running' to be interpreted as `quickly' in this context.

%todo example of descriptive manner adverbs not being able to occur preverbally?

Manner adverbs typically occur at the right edge of the VP. Some examples of manner adverbs in texts are provided below.
\ea \gll é-g-a-v-álə́ŋ-a ŋgíljáŋa\\
\textsc{1sg}-\textsc{cl}g-\textsc{rtc}-\textsc{prog}-sing-\textsc{ipfv} loud\\
 \glt `I'm singing loudly'	\hfill	EJ031710
\ex \gll é-g-a-v-álə́ŋ-a pə́léló	\\
\textsc{1sg}-\textsc{cl}g-\textsc{rtc}-\textsc{prog}-sing-\textsc{ipfv} high\\
\glt	`I'm singing high.' \hfill	EJ031710
\ex \gll k-a-doát-á kaɲkaɲ	\\
\textsc{cl}g-\textsc{rtc}-talk-\textsc{ipfv}\\
\glt `He is talking loudly' \hfill	EJ100611
\ex
\gll k-a-r-at̪-ó 	ŋeniə	təmor\\
\textsc{sm.cl}g-\textsc{rtc-iter}-say-\textsc{pfv}  	\textsc{cl}ŋ.word   plainly\\
\glt ‘he said words plainly’\\
\z 
The adverb \textit{t̪iə} is the pro-form of manner adverbs. When it occurs, it triggers a manner applicative suffix on the verb (\sectref{sec:ch12:manner}). 
\ea
\gll k-ið-iðit̪-ú 	ŋə́mə́gə́niə 	t̪iə 	ɜðəɲɜ́ðəɲiɲ\\
\textsc{cl}g-do-\textsc{man.appl-pfv}  	work 	like.this 	all.day  	\\
\glt ‘She did work like this all day.’
\ex
\gll álə́ŋ-áðát̪-ó 	t̪iə 	ilɜ́ki\\
sing-\textsc{man.appl-ipfv}	like.this 	today	\\
\glt ‘Sing this way today!’ (i.e. it’s your only opportunity) 		\hfill		EJ070213
\z
The manner applicative suffix is required for the extraction of manner adverbs, for example in \textit{wh}-clefts involving \textit{t̪au} `how' as well as in some cases of adverbial focus (\sectref{sec:ch12:manapplicative}). Manner adverbs are the only class of adverbs that have a dedicated applicative form.


%Additional examples of focus adverbs and intensifiers are included below.
%\ea 
%%\gll matʃó	g-a-tə́m-á 	ŋén	áŋəno	ododo	\\
%%man	\textsc{cl}g-\textsc{rtc}-detail-\textsc{ipfv}  word body 	all\\
%%\glt ‘The man is talking about everything’		\hfill	AN062813\\
%\ea
%\gll á-g-a-wət̪-ó 	ɲəwa 	tə́lːɜ́ŋ\\
%2\textsc{sg.sm-cl-rtc}-choose-\textsc{pfv}  	\textsc{cl}young.girls  only.all\\
%\glt `You chose all grown girls only’\\
%\ex
%\gll é-g-a-v-álə́ŋ-a 	tʃáŋtʃáŋ	\\
%1\textsc{sg.sm-cl-rtc-prog}-sing-\textsc{ipfv}	only	\\
%\glt ‘I only sing (I don’t dance)’\\
%\z 


%'One by one', on purpose/by mistake, speaker-oriented particles (hopefully, unfortunately), any others.
%

\subsection{Floating quantifiers and focus adverbs}\label{sec:ch13:focus}

Floating quantifers and focus adverbs are quantificational adverbs which take a nominal argument as their semantic restriction. In addition to contributing quantificational force, focus adverbs associate focus with that argument. Examples of each are provided below.\textit{ododo} `all' 
\begin{tabular}{ll}
\multicolumn{2}{l}{\textit{Floating quantifiers}} \\
ododo & `all' \\
\multicolumn{2}{l}{\textit{Focus adverbs}} \\
ʧáŋəʧáŋ		&	‘only’\\
ikərəŋ		&	‘only’\\
ŋwúlɜ́		&	‘only’\\
təl:ɜŋ		&	‘only all’\\
tʃom		&	‘also, too’\\
ekworəv		&	‘even more, additional’\\
\end{tabular}
 There is only one floating quantifier which is not obviously a nominal modifier of some kind, the universal quantifier \textit{ododo} `all'
 
Floating quantifiers such as `all' seem to be able to associate with any quantifier which they follow. Thus, clause-final floated quantifiers can associate with either the subject or object of a transitive clause, while floating quantifiers which follow the subject must associate with the subject (See \sectref{sec:ch8:universal} for additional details, including information about scope.)

Focus adverbs share these properties with floated quantifiers. They associate with a constituent in the clause, although this can be the verb as well as one of its arguments.  Focus adverbs are restricted to the VP position when the meaning they intensify is located in the VP, as we saw earlier for \textit{kaɲ} `really' \REF{ex:ch13:degree}. Two additional examples of this are provided below.
\ea
\gll á-g-a-wət̪-ó 	ɲəwa 	tə́lːɜ́ŋ\\
2\textsc{sg.sm-cl-rtc}-choose-\textsc{pfv}  young.girls  only.all\\
\glt `You chose all grown girls only.’\\
\ex
\gll é-g-a-v-álə́ŋ-a 	tʃáŋtʃáŋ	\\
1\textsc{sg-cl}g-\textsc{rtc-prog}-sing-\textsc{ipfv}	only	\\
\glt ‘I only sing (I don’t dance).’\\
\z 
In the first example, the adverb scopes over the whole VP, while the focus adverb in the second example associates focus with the lexical content of the verb.

Focus adverbs can occur postverbally or between the subject and the verb. When they occur after the subject, they must associate their focus with the subject. They cannot appear before the subject.
\ea  \textit{Focus adverb}
\ea \gll 	kúku ɡ-ɜɡər-í	nádama	\textbf{tʃáŋtʃáŋ}\\
	1\textsc{sg}-\textsc{cl}g.\textsc{rtc}-read-\textsc{pfv}	books		only\\
	`Kuku only READ the books.' \hfill EJ072509
\ex \gll	kúku ɡ-əɡərí		\textbf{tʃáŋtʃáŋ}	nádama	\\
	1\textsc{sg}-\textsc{cl}g.\textsc{rtc}-read-\textsc{pfv} only	books		\\
\ex	\gll kúku \textbf{tʃáŋtʃáŋ} ɡ-əɡərí		nádama \\ 
	`Kuku only READ the books.' \hfill EJ072509
only 1\textsc{sg}-\textsc{cl}g.\textsc{rtc}-read-\textsc{pfv}	books		\\
	\glt `I only read the BOOKS.', \textit{or} `I only READ the books.'	(all examples)  \hfill EJ072509
	\z 
\z
The generalization is that focus adverbs must associate their focus with a constituent to their left, just like floating quantifiers.

%EVEN? ALSO?

%Focus adverbs can occur with fragment answers:
%
%\ea 
%\gll ɲ́ɲí ʧom\\
%\textsc{1sg.pro} also\\
%\glt `Me, too.'		\hfill	EJ031710
%\z

\subsection{Agreeing adverbs and depictives}\label{sec:ch13:agree}

A small number of adverbs in Moro have the syntax of manner adverbs but contribute descriptive information about an that they associate with, and they show person, number, or gender agreement with those arguments. These elements can associate and agree with any argument which they follow in their clause. While there are several simple adverbs which have this behavior \REF{ex:ch13:advagr}, this category also includes depictive adverbial clauses and the similative adverb \textit{-arno} `like, as', which also agrees with an associated argument.  
\ea \textit{Agreeing adverbs}\\ %can these occur after the subject?
\begin{tabular}{ll}
gwónto & `alone'  \textsc{3sg} (\textit{lit}: \textsc{3sg}-`one')\\
l:əɖó & `together' \textsc{3pl}\\
\end{tabular}
\z 
Just one form of these adjectives is given above. Complete person and number paradigms for both adverbs are provided below (see Chapter \ref{chap:7:pronouns} for Moro person and number paradigms). `Together' can only occur with plural noun phrases. Many forms in the paradigm are suppletive, but third person singular and plural forms simply match the weak concord or verb agreement appropriate for the associates noun class (\sectref{sec:ch6:nclasses}).
\ea \textit{Paradigms for `alone', `together'}\\
\begin{tabular}{lll}
& `alone' & `together' \\
1\textsc{sg} 		& ɲé-nto	&  - \\ 
2\textsc{sg} 		& ŋá-nto &  - \\
3\textsc{sg.hum} 	& gwó-nto	 & - \\
1\textsc{in}.\textsc{du} & lʌŋʌndʌm & l:əká\\
1\textsc{in}.\textsc{pl} & ər:t̪woð & eɖó\\
1\textsc{ex}.\textsc{pl} & ər:t̪woð & ? \\
2\textsc{pl} 		&  ər:t̪woð & ?\\
3\textsc{pl} 		&  ər:t̪woð & lləɖó\\
\end{tabular}
\z 

These adverbs have the distribution of manner adverbs in that they must be postverbal, even when they agree with the subject. While `together' can occur either before or after the object \REF{ex:ch13:togex}, one speaker judged `alone' to only be possible after objects \REF{ex:ch13:aloneex}.

\ea \textit{Syntax of `together'} \label{ex:ch13:togex}
\ea \gll alə-g-a-s-ó acevan l:əká\\
\textsc{1du.in}-\textsc{cl}g-\textsc{rtc}-eat-\textsc{pfv} food \textsc{1du.in}.together\\
\glt `You and me ate food together.' 
\ex aləgasó l:əká acevan\\
\glt `You and me ate food together.'  \hfill AN102015
\z 
\ex \textit{Syntax of ‘alone'}\label{ex:ch13:aloneex}
\ea é-g-a-s-ó acevan ɲé-nto\\
\textsc{1sg}-\textsc{cl}g-\textsc{rtc}-eat-\textsc{pfv} food \textsc{1sg}-alone\\
		`I ate (food) alone'
\ex[*]{égasó ɲénto acevan \hfill AN102015}
		\z
\z 
The similative adverb \textit{-ánó} `like, such as' also occurs clause finally, and associates with an argument that it shows agreement with. One common use of \textit{-ánó} is to modify the general noun \textit{ŋen} `fact, way, matter', which has an adverbial use akin to `in the manner.' See \sectref{sec:ch12:manapplicative} for additional details.

%todo examples of -a(r)no(?) check what EJ prefers

Depictive adverbial clauses have essentially the same distribution --- that of a manner adverb, preferentially at the right edge of VP, and the same function, providing additional description about a verbal argument while the event described by the verb take place. The subject of the depictive clause is always the argument which is controlled by the associated argument, the verb agrees with this argument, like `alone' and `together', and the clause itself occurs with morphology which is typically associated with subject relative clauses \sectref{relativeclause}. A depictive secondary predicate is illustrated in the example below, in bold.
\ea \gll  é-g-a-wað-at̪-ó kúku-ŋ i-ki \textbf{g-é-cánó} \textbf{kaɲ}	\\
\textsc{1sg-cl}g-\textsc{rtc}-find-\textsc{loc.appl-pfv} Kuku-\textsc{acc} \textsc{loc.in}-field \textsc{cl}g-\textsc{dpc1}-angry very\\
	\glt 	`I found Kuku in the field very angry.' \hfill AN110515
\z 
Proper names, such as the object \textit{kukuŋ}, can not ordinarily be modified by relative clauses. As such, this constituent must be distinct from the object. This is a simple example, but depictives can be complete clauses headed by verbs, and they can agree with either subject or object, showing full subject agreement with their associated argument. 


%aləgasór acevan eɖó		‘1pl.in ate food together.’
%lasó acevan lləɖó		‘They ate food together.’



%ágasó ŋánto			‘You ate alone’
%gasó gʷónto			‘He ate alone’
%
%áləgasó lʌŋʌndʌm		‘We ate only us two.’  	(dual form is suppletive)
%áləgasó lʌŋʌndʌcan		‘We ate as two.’
%áləgasór aceva ərrt̪ʷoð		‘1pl.in ate alone’
%ɲagasór aceva ərrt̪ʷoð		‘1pl.ex/2pl ate alone’
%lasó aceva ərrtʷoð		‘They ate food alone’ 	





%REF

%\section{Secondary predicates?}

