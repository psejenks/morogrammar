\part{Complex clauses}

\chapter{Auxiliaries}\label{chapter:auxiliaries}

This section describes auxiliary verbs in Moro. Three criteria identify auxiliary verbs as a class 1) they always taking another verb phrase and its arguments as their complement, which is realized in an infinitive verb form, 2) they lack standard \textsc{amd} affixes which regular verbs have, but instead typically occur in one mostly invariant form 3) they cannot take extension suffixes. Auxiliaries in Moro are used to mark negation, modality, tense, and aspect.

All Moro auxiliaries are obligatory raising-to-subject verbs. This means that whatever the subject of their complement is will be the subject of the auxiliary. In other words, auxiliaries do not assign a semantic role to their subject. This is a definitional property of auxiliaries.

Auxiliaries are also restricted to finite clauses. These verbs only show nonfinite inflection when they are embedded under other auxiliaries, and they do not seem to occur in the infinitive clausal environments described in Section \ref{sec:ch15:infinitives}.

Moro auxiliary verbs fall into X main classes. The negative auxiliary verb \textit{-n:a} and the past auxiliary verb \textit{-w:o} are each in a class of their own, as their selectional properties, ordering properties, and distribution is not shared with any other auxiliaries. In contrast, the future and immediate future auxiliaries seem to occupy essentially the same position in Moro clauses, as they cannot co-occur, and in fact show somewhat complementary distributions in different Moro dialects.

Past > Neg > Fut

Auxiliary semantics and order, p. 49 (different from what I found later??)
lʌwó lʌsʌtʃú					‘They had thought.’
lavəla lʌsʌtʃú					‘They had been thinking’
lawó lʌsʌtʃʌ					‘They finished thinking.’ (they were thinking?)

machandá lawó lassó acevaŋ			‘The men had eaten food.’ (they had finished)
machandá lawó lassa acevaŋ			‘The men were eating food.’ (when I arrived)
*machandá lavəla lassó acevaŋ
machandá lavəla alase acevaŋ			‘The men are going to eat food.’
machandá lawó lavədó (na) ləse acevaŋ	‘The men went and ate.’
machandá lavədó (na) ləse acevaŋ		‘The men went and ate.’

tutu gawó gíðí naŋatoðe naŋase naŋandre	‘Tutu used to wake up and eat and sleep again.’
			  *nəŋese nəŋandre

gíðí áŋáve Khartoum		‘He will be in Khartoum.’

*gíði gawó gabwaɲa		 impossible auxiliary sequence

gavánna giði naŋatoðe			‘He will not wake up.’
gawó genná naŋatoðʌ			‘He did not wake up.’
Kuku gawó gatoðó ŋen niɲetó		‘Kuku had woken when I arrived.’
Kuku gatoðo ŋen niɲetó		‘Kuku woke up when I arrived.’
			 clear evidence for past perfective reading.
 

\section{The past auxiliary}\label{sec:ch14:pstipfv}

egawó egalágá gi (eteto)		‘I used to cultivate the field.’
egawó égalágá gi sa yento erreka	‘I cultivated the field for an hour yesterday’
					Comment: did not finish
égalagó gi sa yento erreka		‘I cultivated the field in an hour yesterday’

egawó ʌnni				‘I’m here./ I live here.’
egawégawo ʌnni			‘I used to live here.’



\section{The negative auxiliary}\label{sec:ch14:negaux}

\ea \gll  	l-anːá               alə-wað-a	\\
\textsc{cl}l-not.\textsc{pfv}   \textsc{3pl.inf}-poke-\textsc{inf2}\\
\glt ‘They did not poke.’
\z 

Given that negation is a propositional operator, we expect it to be a raising verb, like the other TAM auxiliaries. 

As expected, there is strong evidence that negation is an obligatory raising-to-subject predicate. For one, complementizers are unavailable under negation, as we have already mentioned: 

\ea \gll  Kúkːu g-a-nːá (*t̪á/*nə́-) áŋə́-$^{↓}$tóð-á\\
Kuku \textsc{clg}-\textsc{rtc}-not.\textsc{pfv}   {\ \ \textsc{comp1/2}} \textsc{3sg.inf}-move-\textsc{inf2}\\
\glt ‘Kuku’s not moving.’
\z 

Recall that all of the instances of raising we have seen so far similarly prohibit complementizers.

Additionally, athematic subjects are fine with negation \REF{ex:ch14:neg1}, and the negative auxiliary can embed other raising verbs, such as TAM auxiliaries \REF{ex:ch14:neg2}:

\ea \gll  ŋáw ŋ-a-n:á  áŋə́-$^{↓}$d̪ə́n-é\\
\textsc{cl}ŋ.water \textsc{cl}ŋ-not.\textsc{pfv}   \textsc{3sg.inf}-rain-\textsc{inf1}\\ \label{ex:ch14:neg1}
\glt ‘It's not raining.’ \footnote{It is not clear why this example has the \it{e}-infinitive.}
\z 

\ea  \label{notgo} 
\ea \gll  Kúkːu k-án:(a) áŋə́-$^{↓}$və́l-á áŋə́-ndr-$^{↓}$é\\
Kuku \textsc{clg}-not.\textsc{ipfv} \textsc{3sg.inf}-go-\textsc{inf2} \textsc{3sg.inf}-sleep-\textsc{inf1}\\ \label{ex:ch14:neg2}
\glt `Kuku isn't going to fall asleep.' 
\ex \gll   é-g-a-nːá i-gíð-í ɲe-bə́ð-é ŋə́ní		\\
\textsc{1sg}-\textsc{cl}g-\textsc{rtc}-not.\textsc{pfv} \textsc{1sg.inf}-will-\textsc{inf1} \textsc{1sg.inf}-pet-\textsc{inf1} \textsc{cl}ŋ.dog\\
\glt ‘I will not pet the dog.’	
\z 
\z 


\section{The immediate future auxiliary}

\ea   \gll Kúkːu g-a-və́l-á (*t̪á) áŋə́-$^{↓}$ʤóm-\ovalbox{é} \\
Kuku \textsc{cl}g-\textsc{rtc}-go-\textsc{ipfv} {\ \textsc{comp1}} \textsc{3sg.inf}-move-\textsc{inf1}\\
\glt ‘Kuku is going to move.'
\z 

\ea \gll  jamala j-a-və́l-á al-oaɲ-ət̪-é\\
\textsc{cl}j.camel \textsc{cl}j-\textsc{rtc}-go-\textsc{ipfv} \textsc{3pl.inf}-many-\textsc{appl}-\textsc{inf1}\\
\glt ‘The camels are going to be more.'
\z 

\ea \gll  ísːíə j-a-və́lá áŋə́-búg-ə́n-i	\\
{cl}j.gun \textsc{cl}j-\textsc{rtc}-go-\textsc{ipfv}  \textsc{3sg.inf}-$^{↓}$hit-\textsc{pas}-\textsc{inf1}\\
\glt ‘The gun is going to be shot.’
\z 

gavə́lá oro áŋélá kʌdugli		‘Then she’s going to come from Kadugli’
gʌviðiə oro áŋélá kʌdugli (ulʌlítu)	‘Then she will come from K. tomorrow.’
ígʌviðia oro ɲese acevan		‘Then I will eat food.’	
 NB FUTURE tense must be tomorrow or later

gʌviðiʌ ŋʌmʌgʌnia oro aŋiði aŋela ulʌlitu	
‘He’s going to do work then will come back tomorrow.’
gíði aŋase oro aŋandre		‘He will eat then sleep’
gavəlaŋase oro aŋandre	‘He’s going to eat then sleep’ (near future, not tomorrow)



\section{The future auxiliary}

\ea \gll Kúkːu g-íð-á (*t̪a) áŋə́-$^{↓}$ʤóm-\ovalbox{é}\\
Kuku \textsc{cl}g-(\textsc{rtc})-do/will-\textsc{ipfv} {\ \textsc{comp1}} \textsc{3sg.inf}-move-\textsc{inf1}\\
\glt ‘He will move.’
\z 




\section{Progressive deictic auxiliaries}\label{sec:ch14:progaux}

kuku gʌtú kalágá gi (tú) (sa ŋento)	‘Kuku was cultivating the fields there’ (for an hour)
			Comment: worked for an hour, did not finish
gʌ́ní galaga gí				same as above, but HERE
gʌ́nwʌŋ galaga gí			same as above, but in location of addressee
gʌtwe galaga gí			same as above, but far away

gʌt̪u gét̪o kʌdugli	‘She’s on the way from Kadugli.’/’She’s intending to come from K’

gʌ́t̪:u gásó (oro aŋela)		‘He’s there eating and then will come.’
 NB the verb is in the venitive
gʌ́t̪:u gándró			‘He’s there spending the night.’
ígʌnni égándró (orəɲela)	‘I’m here spending the night then I’ll come.'
 “	“	*orəɲédé	‘I’m here spending the night then I’ll go.'	
 impossible because ‘go’ requires a local origo

ígʌnni əni égándra (orəɲela)	‘I’m here spending the night then I’ll come.'
“ 	“	“     (orəɲédé)	‘I’m here spending the night then I’ll go.'
 shows that the “proximal” is default in a way, does not induce clash with either cont.


kuku gʌtú galágá gi			‘Kuku is there cultivating the field.’
gʌtú geto				‘He is coming’
kuku gʌtu gálagó gi			‘Kuku is there cultivating the field and then will come.’
kuku kʌtú kalagó gi			‘Kuku is there having finished cultivating.’
kuku kʌtú na kalagó gi		‘Kuku is there and has finished cultivating.’ 
(different meanings)

kuku kʌnni kalagó gi			‘Kuku is here having cultivated the field.’
kuku kʌnni kálagó gi			‘Kuku is here cultivating then he will go.’
kuku kʌnni kalágá gi			‘Kuku is here cultivating.’

kuku gʌnwʌŋ galagó giʌ		‘Kuku is there having finished cultivating’
kuku gʌnwʌŋ gálagó gi		‘Kuku is there cultivating then will come.’
kuku gʌnwʌŋ galágá gi		‘Kuku is cultivating and will come.’

kúku gʌ́nni gavánná aŋalágá gi	‘Kuku is here not cultivating’ 
kúku gʌ́tú gavánná aŋalágá gi	‘Kuku is there not cultivating’ 

kukú gawó galaga gi erreka/ŋʌnʌŋi	‘Kuku was cultivating yesterday/today.’
ŋini ŋawó ŋʌnduðʌ ŋera erreka	‘The dog was about to bit the child yesterday.’
kuku gawó galagó gi erreka		‘Kuku was there cultivating the field.’ (diff. from aux form?)

These examples illustrate that all of the locative verbs can be auxiliaries; the general locative has the most general meaning, though, indicating just past tense.


\section{`Go' and `come' as auxiliary verbs}\label{sec:ch14:goaux}

egavəla ɲelage gi sa yento		‘I’m going to plow the field.’
eget̪ó elage gi				‘I’m coming to plow the field’

ígiðí ɲede ɲelaga gi			‘I will go plow the field.’
igiði evəde ɲelaga gi			same as above
igiði ɲela ɲelaga gi			‘I’ll come plow the field’
igʌnni égavəla ɲelaga gi		‘I’m here going to go plow the field.’
igʌnni egavanne egavəla ɲelaga gi	‘I’m not here going to plow the field.’
egawó égavə́lá elágá gí		‘I was going to plow the field’ (when something occurred…)



\section{Modal auxiliaries}

\ea \gll  jamala já-j-a-bwáɲ-á al-oaɲ-ət̪-é		\\	
\textsc{cl}j.camel \textsc{pst}-\textsc{cl}j-\textsc{rtc}-want-\textsc{ipfv} \textsc{3pl.inf}-many-\textsc{appl}-\textsc{inf1}\\
\glt ‘The camels were supposed to / needed to be more.’
\ex \gll  Kúkːu g-a-bantá áŋ-anːá áŋə́-$^{↓}$və́l-á áŋə́-$^{↓}$ndr-é\\	
Kuku \textsc{cl}g-\textsc{rtc}-should  \textsc{3sg}-neg.\textsc{pfv}  \textsc{3sg}-go-\textsc{inf2}  \textsc{3sg}-sleep-\textsc{inf1}  \\
\glt `Kuku should not be about to fall asleep.'
\z 

also: can?

\section{Multiple auxiliaries and auxiliary ordering}

\ea 
\ea \gll  é-g-a-nːá ɲ-íð-í ɲe-ndr-é\\
\textsc{1sg-clg}-\textsc{rtc}-neg.\textsc{pfv} \textsc{1sg}-will-\textsc{inf1} \textsc{1sg}-sleep-\textsc{inf1}  \\
\glt `I won’t be sleeping.'
\ex \gll  Kúkːu g-a-bantá áŋ-anːá áŋə́-$^{↓}$və́l-á áŋə́-$^{↓}$ndr-é\\	
Kuku \textsc{cl}g-\textsc{rtc}-should  \textsc{3sg}-neg.\textsc{pfv}  \textsc{3sg}-go-\textsc{inf2}  \textsc{3sg}-sleep-\textsc{inf1}  \\
\glt `Kuku should not be about to fall asleep.'
\z 
\z 

égawó igiði ɲelaga gi sa yento	I was cultivating the field for an hour’
					Comment: distant past

ŋen newo elad̪at̪a, egawó ígiðí nélage gi sa yento
	‘When I was in youth (in a powerful state), I would finish cultivating the field in an hour.’


ŋénéa égedó ŋera ŋetá, égawígiði nəɲeti lʌŋgaɲ et̪á: “égacoɲá” et̪et̪o
			‘When I was a small boy, I would always say to 
