\chapter{Syllable structure}

Moro allows the following types of syllables:

\begin{table}
  \begin{tabular}{lll}
    \lsptoprule
V	&	a.ró.bá	&	‘whey’	\\
N	&	ḿ.bú	&	‘go!’\\
	&	ǹ.dóŋ	&		‘when’\\
R	&	r.ða	&	‘meat’\\
	&	ɽ.diə́	&	‘dalib fruit’\\
CR	&	pr		&	‘very’\\
	&	br.lá.gá&		‘slime’\\
CV	&	la.və.ra&		‘stick’\\
VC	&	ɜr.pú.la&	‘animal skin’\\
CVC	&	e.ma.ðén&	‘same age peer’\\
	&	ŋáŕlá	&	‘spear’\\
\lspbottomrule
  \end{tabular}
  \caption{Syllable types in Moro.}
  \label{tab:ch3:1}
\end{table}


Word-internal consonant sequences are attested, but there are no obstruent-obstruent sequences. All consonant sequences consist of two sonorants or an obstruent and a sonorant. The majority of sonorant-obstruent combinations are coda-onset sequences with rising sonority. 

There is one word, \textit{ɜdniə} which has the sequence obstruent-nasal as a coda-onset. 


There are several stop-r sequences in which the [r] is either the nucleus or forms a complex onset with the preceding stop. 


The sonorant-sonorant sequences are l-m, l-n (one example each) r-m, r-n, r-l, which all have rising sonority. There is also ɾ-m. There are no attested sonorant sequences of falling sonority, such as m-l, m-r, n-l, n-r or l-r. There are no sonorant sequences involving the nasals ŋ and ɲ. 

\begin{table}
  \begin{tabular}{lll}
    \lsptoprule
l-m	&		almoʧána	&	‘tobacco pipe’	\hspace{0.8cm}< Arabic		\\
l-n	&		əlná		&	‘room’		\hspace{1.95cm}< lənːá		\\
r-m	&		ŋərmɜt̪iə	&	‘blindness’\\
	&		kakárma		&	‘s/he finds guilty and fines’\\
r-n	&		kɜmúrnəniə	&	‘s/he pretends, acts like’\\
	&		ðə́ŕná		&	‘leather strips, animal skin, patch’\\
r-l	&		ðərliə́		&	‘root, artery’\\
ɾ-m	& 		ðəɾmbégʷa	&	‘lyre’\\
\lspbottomrule
  \end{tabular}
  \caption{Sonorant-sonorant sequences.}
  \label{tab:ch3:2}
\end{table}

Finally, all the ɽ-C sequences, including ɽ-r and ɽ-ɾ, occur in initial position. This suggests that ɽ functions syllabically in this environment. 

\begin{table}
  \begin{tabular}{p{3pt}p{3pt}p{3pt}p{3pt}p{3pt}p{3pt}p{3pt}p{3pt}p{3pt}p{3pt}p{3pt}p{3pt}p{3pt}p{3pt}p{3pt}p{3pt}p{3pt}p{3pt}p{3pt}p{3pt}p{3pt}p{3pt}p{3pt}}
    \lsptoprule
	&	p&	t̪&	t&	tʃ&	k&	b&	d̪&	d&	dʒ&	g&	f&	v&	ð&	s&	m&	n&	ɲ&	ŋ&	l&	ɾ&	r&	ɽ\\
p	&	&	&	&	&	&	&	&	&	&	&	&	&	&	&	&	&	&	&	&	&	x&	\\
t̪	&	&	&	&	&	&	&	&	&	&	&	&	&	&	&	&	&	&	&	&	&	x&	\\
t	&	&	&	&	&	&	&	&	&	&	&	&	&	&	&	&	&	&	&	&	&	&	\\
tʃ	&	&	&	&	&	&	&	&	&	&	&	&	&	&	&	&	&	&	&	&	&	&	\\
k	&	&	&	&	&	&	&	&	&	&	&	&	&	&	&	&	&	&	&	&	&	x&	\\
b	&	&	&	&	&	&	&	&	&	&	&	&	&	&	&	&	&	&	&	&	&	x&	\\
d	&	&	&	&	&	&	&	&	&	&	&	&	&	&	&	&	x	&	&	&	&	&	\\
dʒ	&	&	&	&	&	&	&	&	&	&	&	&	&	&	&	&	&	&	&	&	&	&	\\
g	&	&	&	&	&	&	&	&	&	&	&	&	&	&	&	&	&	&	&	&	&	x&	\\
f	&	&	&	&	&	&	&	&	&	&	&	&	&	&	&	&	&	&	&	&	&	&	\\
v	&	&	&	&	&	&	&	&	&	&	&	&	&	&	&	&	&	&	&	&	&	&	\\
ð	&	&	&	&	&	&	&	&	&	&	&	&	&	&	&	&	&	&	&	&	&	&	\\
s	&	&	&	&	&	&	&	&	&	&	&	&	&	&	&	&	&	&	&	&	&	&	\\
m	&	&	&	&	x&	&	x&	&	&	&	&	&	&	&	&	&	&	&	&	&	&	&	\\
n	&	&	x&	x&	&	&	&	x&	x&	&	&	&	&	&	&	&	&	&	&	&	&	&	\\
ɲ	&	&	&	&	&	&	&	&	&	x&	&	&	&	&	&	&	&	&	&	&	&	&	\\
ŋ	&	&	&	&	&	&	&	&	&	&	x&	&	&	&	&	&	&	&	&	&	&	&	\\
l	&	&	x&	x&	&	x&	x&	x&	x&	&	&	x&	x&	&	&	x&	x&	&	&	&	&	&	\\
ɾ	&	x&	x&	x&	x&	x&	&	&	&	&	&	&	&	x&	&	x&	&	&	&	&	&	&	\\
r	&	&	x&	x&	x&	x&	x&	&	&	&	x&	x&	&	x&	&	x&	x&	&	&	x&	&	&	\\
ɽ	&	&	&	x&	x&	&	&	&	x&	&	&	&	&	&	&	&	&	&	&	&	x&	x&	\\
\lspbottomrule
  \end{tabular}
  \caption{C1/C2.}
  \label{tab:ch3:3}
\end{table}


There are sequences of three consonants word-internally, all of which include a nasal-stop sequence. While this might suggest that the sequence is a prenasalized stop, it should be noted that stop-r sequences, as in g-r and d-r also occur independently. 

\begin{table}
  \begin{tabular}{lll}
    \lsptoprule
		&	ðəɾmbégʷa	&	‘lyre’ \\
		gr&	aləŋgréma	&	‘bed’ \\ 
		dr&	ándŕeá		&	‘saddle’\\
	\lspbottomrule
  \end{tabular}
  \caption{Stop-r Sequences}
  \label{tab:ch3:4}
\end{table}

