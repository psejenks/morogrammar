\chapter{Noun phrases}\label{chapter:nounphrase}

This section describes the morphological marking and syntactic distribution of nominal modifiers in Moro, including adjectives, numerals, demonstratives, genitive phrases, and relative clauses. Additional discussion of adjectives can be found in Chapter \ref{adjectives}, while relative clauses are discussed in more detail in section \ref{relativeclause}. Previous descriptions of Moro noun phrase syntax include Jenks (2013), from which this chapter draws.

\section{Weak and strong nominal concord}\label{concord}

Most nominal modifiers in Moro agree with the head noun in noun class, an instance of what is commonly called nominal concord. Moro has two distinct sets of nominal concord prefixes, which we call weak and strong concord. Weak concord is identical to noun class subject agreement on verbs, and is simply the concordial prefix \textit{C-}. Strong concord is of the shape \textit{íC:-}, where \textit{C:} is the geminated noun class concord marker.  Both series are illustrated in Table \ref{tab:ch8:1}.

\begin{table}
	\begin{tabular}[t]{cc}
\lsptoprule
Weak concord (\textsc{cl}) & Strong concord (\textsc{scl}) \\
\midrule
g-	&	íkː-	\\
l-	&	ílː-	\\
n-	&	ínː-	\\
ŋ-	&	íŋː-	\\
ɲ-	&	íɲː-	\\
ð-	&	íðː-	\\
r-	&	írː-	\\
j- &	ís- 	\\
\lspbottomrule	
\end{tabular}
  \caption{Weak versus strong concord}
  \label{tab:ch8:1}
\end{table}
The geminated variant of \textit{j-} concord is realized as [s]. Strong concord can only occur in definite noun phrases, and only once, on the leftmost element in that noun phrase, while all other modifiers take weak concord. Strong concord is obligatory on demonstratives, the proximal form of which it closely resembles. Only weak concord is possible on numerals and indefinite modifiers. 

Strong concord closely resembles the proximal demonstrative \textit{íC:i}. In fact, in Written Moro, strong concord is always written as a separate word \textit{iCi} distinct from the noun and modifier. It is unclear if this orthographic decision reflects any dialectal differences in this area. It could be that Moro speakers intuitively feel that strong concord and the proximal demonstrative are identical, or that other dialects of Moro such as Werria might not surface with the same phonological fusion processes described below for Thetegovela. %CHECK Is strong concord impossible on -ero ``NO"??

\section{Nominal modifiers}\label{sec:ch8:nporder}

This section surveys the syntax and morphological marking on nominal modifiers. All nominal modifiers follow the head noun in Moro. The basic syntactic arrangement of complex noun phrases in Moro is Noun-Demonstrative-Adjective-Numeral, illustrated in the following example.

\ea \label{ex:ch8:1}
	\gll	\textit{nád̪ám}	\textit{́ín-ɜtínːə}	\textit{n-əɡətʃan}	\textit{n-óɾé}	   \\
			books	\textsc{scl}n-that	\textsc{cl}n-two	\textsc{cl}n-\textsc{dpc1}.red-\textsc{adj} \\
	\glt	‘those two red books’
\z 
In this example, strong concord only occurs on the leftmost modifier while weak concord occurs on all of the modifiers that follow that one. 

In addition to occurring immediately after nouns, nominal modifiers frequently occur discontinuously in positions to the right of the nouns they modify, as discussed in section \ref{sec:ch12:extraposition}. This pattern is especially common with relative clauses, likely an effect of their syntactic heaviness. In most cases, concord on the extraposed modifier is sufficient to disambiguate its reference.


%EXAMPLES OF EXTRAPOSITION HERE


\subsection{Demonstratives}\label{demonstratives}

There are three degrees of demonstrative determiners in Moro. The C indicates noun class agreement:

\ea Demonstratives in Moro\\
\begin{tabular}[t]{lll}
íCːi		&	`this’	 			&	proximal\\
íCːɜj		&	`that’	 			&	medial\\
íCːɜtiCːɜ	&	`that over there’	&	distal\\
\end{tabular}
\z %TODO Peter Elyasir Werria seems to use ɜkel as well. Angelo also gave me other forms, íC:ɜtú ``that", which the `yonder' form seems to be built on, íkwɜŋ 'that where you are, `íkɜtikɜi `yonder', with the same intensive `ɜj' ending as `that' above. We need to ask Elyasir about these.
The initial \textit{íCː-} of all demonstratives is strong concord; note that the distal form reduplicates the strong concord component of the stem.

Demonstratives can occur independently as demonstrative pronouns, in which case they occur in the forms above.

When they modify nouns, demonstratives fuse with the noun in normal speech, a process which extends to all modifiers exhibiting strong concord. Phonologically, this means that the initial [í] of the strong concord marker is reduced, fused or dropped. If the noun ends in a consonant, the first vowel of the suffix may be dropped or is reduced to [ə́]. If it is dropped, the consonant is not geminated. However, for the g-class, it is still realized as [k]:

\ea		etám	`neck’		\\
		etám-ki	`this neck’	
	 \z 
%		% TODO Peter Elyasir Need more

If the noun ends in a vowel, there is fusion of the two vowels, often resulting in [ə]. In the following example, the noun ends in an [i] with high tone (\textit{wədʒí}), and the demonstrative beings with the same vowel, so they are fused to form an identical vowel [í]. However, this vowel is subject to vowel reduction, and surfaces as [ə́]. The C is filled with [k]. 

\ea \begin{tabular}[t]{ll}
	wədʒí		&	`(the) woman’ \\
	wədʒə́kːi	&	`this woman’\\
	wədʒə́kːɜj	& 	`that woman’\\
	wədʒə́kːɜtikɜ &		`that woman over there’\\
	\end{tabular}
\z 

In the following example, the final [a] of the noun and the initial [i] of the determiner fuse to form [ɜ]. The high tone of the demonstrative appears on this vowel:

\ea  \begin{tabular}[t]{ll}
	ðamala		&	`(the) camel'\\ 
	ðamalɜ́ðːi	&	`this camel\\
	ðamalɜ́ðːɜj	&	`that camel\\
	ðamalɜ́ðːɜtiðːɜ	&	`that camel over there’	\\
  \end{tabular}
\z 

Vowel hiatus resolution patterns for other vowel endings are below. The combinations resulting from final peripheral vowels /i e o u/ reduce to [ə], whereas the combinations resulting from central vowels /a ɜ/ fuse to [ɜ]:

\ea \begin{tabular}[t]{llll}
/e-i/ → [ə] &	ome-íkːi 	& [omə́kːi] 		& ‘this fish’\\
/u-i/ → [ə]	& ɜðu-ísːi 		& [ɜðə́sːi] 		& ‘this breast’\\
/o-i/ → [ə]	& ŋombogó-íŋːi 	& [ŋombogə́ŋːi]	& ‘this calf’\\
/ɜ-i/ → [ɜ]	& ðuwːɜ-íðːi 	& [ðuwɜ́ðːi] 	& ‘this smoke’\\
/eə-i/ → & & & \\
/iə-i/ → & & & \\	
 \end{tabular}
\z 
Noun class agreement paradigms are shown in Table \ref{tab:ch8:2}. The g-class may have either [g] or [k] concord, and the j-class may have either [k] or [s]. In this case, the [k] and [s] versions are selected. %really? I haven't seen this...

\begin{table}
\begin{tabular}[t]{llll}
	\lsptoprule
		& proximal	&	medial	&	distal\\
\midrule		
g-class	&	íkːi	&	íkːɜj	&	íkːɜtikɜ\\
l-class	&	ílːi	&	ílːɜj	&	ílːɜtilɜ\\
n-class	&	ínːi	&	ínːɜj	&	ínːɜtinɜ\\
ŋ-class	&	íŋːi	&	íŋːɜj	&	íŋːɜtiŋɜ\\
ɲ-class	&	íɲːi	&	íɲːɜj	&	íɲːɜtiɲɜ\\
ð-class	&	íðːi	&	íðːɜj	&	íðːɜtiðɜ\\
r-class	&	írːi	&	írːɜj	&	írːɜtirɜ\\
j-class &	ísːi 	&	ísːɜj	&	ísːɜtisɜ\\
\lspbottomrule
	\end{tabular}
  \caption{Demonstrative paradigms}
  \label{tab:ch8:2}
\end{table}

%add more forms to this table?

\subsection{Genitive phrases}\label{sec:ch8:genitive}

Genitive constructions in Moro are marked by strong or weak concord in addition to a possessive prefix on the possessor. There are two prefixes, \textit{Cə́-} and \textit{Ca-}. Note that possessive pronouns are discussed in Section \ref{section:posspro}.

Beginning with the \textit{Cə́ } possessor, the structure is possessee \textsc{scl}x.ə́--possessor, as shown below:

\ea \gll	ŋ-ədərːeə 	ŋə́-ɲeɾá 		ŋ-aŋəɾ-á \\
	\textsc{cl}ŋ-nursing 	\textsc{scl}ŋ.\textsc{gen}-\textsc{scl}ɲ.child 	\textsc{scl}ŋ-good-\textsc{adj} \\
	\glt	`Nursing of babies is good’
\z 

Note that this noun phrase involves generic reference to an activity. In definite noun phrases modified by a single possessor, the possessor occurs with strong concord. As with demonstratives, strong concord fuses with the noun, resulting in the template possessee-í\textsc{scl}xː.ə́--possessor:

\ea \gll áŋə́nó 	     ís-umːiə 		j-a-daŋ-á \\ 
			\textsc{cl}j.body   \textsc{scl}j.\textsc{gen}-\textsc{cl}g.boy	\textsc{sm}.\textsc{cl}j-\textsc{rtc}-dirty-\textsc{adj}\\
			\glt the boy’s body is dirty
%	\ex 	ŋíní ŋ-ummiə\\
%		‘boy’s dog’ \\
 \z 

A full possessive paradigm with a proper noun possessor is provided below:

\ea \begin{tabular}[t]{lll}
g&	udʒə́-kː-ə́-t̪út̪u &‘Tutu’s person’ \\
l&  	lidʒə́-lː-ə́-kúkú &	‘Kuku’s people’ \\
n &	nəmertʌ̪-nː-ə́-kúkú &	‘Kuku’s horses’ \\
ŋ &	ŋeɾɜ-ŋː-ə́-kúkú &	‘Kuku’s child’\\
ɲ &	ɲeɾɜ-ɲː-ə́-kúkú 	&‘Kuku’s children’\\
ð&	ðápːɜ-ðː-ə́-kúkú &	‘Kuku’s friend’\\
r& 	rápːɜ-rː-ə́-kúkú &	‘Kuku’s friends’\\
j&  	ajén-sː-ə́-kúkú 	&‘Kuku’s mountain’ \\
\end{tabular} \z

Proper names lack the accusative suffix \textit{-ŋ} in the  possessive. This is no mistake: genitive is a particular case-marked form of the noun. Thus, nouns which show a surface alternation for nominative versus accusative case can only exhibit the nominative form, the root, with the possessive prefix:

\ea 
\begin{tabular}[t]{lllll}
Nominative & Accusative & & Genitive & \\
ŋaw & ŋaw-a & `water' & gə́-ŋaw & ‘of the water’ \\
ómón & ómón-a & `leopard' &  g-ómón & `of the leopard'
\end{tabular} 
\z

In case the noun is also modified by a demonstrative or multiple genitive phrases, only the leftmost modifier occurs with strong concord, while the other modifiers take weak concord: 

\ea 
	\ea \gll jamal -ɜ́sː-i jə́--kúkːu	\\	
		camel \textsc{scl}j-this \textsc{cl}j.\textsc{gen}-Kuku\\	
		\glt ‘these camels of Kuku’s
	\ex	\gll  ogəŋ -ɜ́kː-ɜj   gə́-kúkːu		\\
			   hote \textsc{scl}j-that \textsc{cl}g.\textsc{gen}-Kuku\\
		\glt 	`that hoe of Kuku’s
	\ex \gll súr -ɜ́-sːə́--kúku jé-ðamala	\\
			picture \textsc{scl}j.gen-Kuku \textsc{cl}j.\textsc{gen}-camel\\
			‘Kuku’s picture of the camel’ 
	\z
\z 



%The other genitive marker is \textit{Ca-}, which appears in compounds and complements, and always has weak concord. This prefix harmonizes with the vowel height of the noun stem. If the possessed noun begins with a vowel, the vowel of the prefix is dropped, although its tone is recuperated on the noun it modifies. %What exactly is a `generic genitive construction'?
%\ea 
%	\ea	\gll ajén 	  ja-ŋálːo\\
%			\textsc{cl}j.hill   \textsc{cl}j.\textsc{comp}-Ngalo\\
%		\glt 	`Ngalo’s hill’  	(the hill Ngalo is from)
%	\ex	\gll	ajén jɜ-kúkːu \\
%			\textsc{cl}j.hill   \textsc{cl}j.\textsc{gen}-\textsc{cl}g.Kuku \\
%		\glt	`Kuku’s hill’  		(the hill Kuku is from)
%	\z
%\z %these are clauses

The genitive is also the main way that compounds are formed in Moro. We have only noted compounds composed of two nouns with genitive marking. In these cases, the relationship is not clearly possessed-possessor, but can be part-whole or some other relational meaning. In the last two examples, there is also a locative on the noun, so the order is \textsc{gen}-\textsc{loc}-\textsc{noun}. %Is there any systematic decision between the two genitives for possessive forms? 

\ea Cə forms
	\ea \gll	loðeə 		l-ánó\\
		\textsc{cl}l.bone  	\textsc{cl}l.\textsc{gen}-\textsc{cl}g.inside	\\
		\glt `spine’  
	\ex \gll	ŋalːə́tʃa		ŋə́-láj\\
		{\textsc{cl}ŋ.sweet.substance}  	\textsc{cl}ŋ.\textsc{gen}-\textsc{cl}l.bee\\
		\glt `honey’
	\ex \gll ɜniŋé 		g-úgi	\\
			\textsc{cl}g.ear  	\textsc{cl}g.\textsc{gen}-\textsc{cl}g.tree\\
		\glt `leaf’
	\ex \gll ðəpúndri 			ð-é-ŋáw\\
			{\textsc{cl}ð.wooden.object}  	 \textsc{cl}ð.\textsc{gen}-\textsc{loc}-\textsc{cl}ŋ.water\\
		\glt `boat’ (lit. `wooden thing of in the water')
	\ex	\gll kwaŋ 		k-é-ŋáɲá	\\			
		\textsc{cl}g.thing  	 \textsc{cl}g.\textsc{gen}-\textsc{loc}-\textsc{cl}ŋ.grass/forest \\
		\glt `animal’ (lit. `things of in the forest') 
		%lidʒi lel:a `women'
%lidʒi leloraŋ `men'
%lidʒi 'people'


		
		%this is an argument for the prefix/case status of é; all the porrige types in `What do Moro people do?' list genitive-ON-N compounds, do these have predictable meaning?
	\z 
\z

%\ea Ca forms:\\
%\gll ŋene 		ŋa-d̪et̪am			2/24/2012 \\
%\textsc{cl}ŋ.word  	 \textsc{cl}ŋ.\textsc{gen}-\textsc{cl}ð.truth {} \\
%\glt truth  (words of truth)  
%\z


%I THINK THE noun-complement examples from my head movement paper should be added here as well.

\subsection{Numerals}

The Moro numeral system uses a quinary (base-five) system until nine, and then switches to a decimal (base-ten) system. The numbers for 1-5 are as follows. Noun class marking is observed with numbers 1-3. The following basic numerals use the default g/l noun class pairing, so `one' is marked with \textit{g(w)-}, and `two' and `three' with \textit{l-}. `Four' and `five' are invariant and never agree with the noun they modify.

\ea \begin{tabular}[t]{ll}
1&	gwənto \\ 
2&	ləgətʃan \\ 
3&	ləgətʃín \\
4&	marlon \\
5&	ðénə́ŋ \\
 \end{tabular}
\z

The numbers for 6-9 are composed of 5, the conjunction \textit{nə-} ‘and’ and the number required to add up to the desired numeral, so 6= 5 and 1, 7 = 5 and 2, 8 = 5 and 3, 9 =  5 and 4. Note that the conjunction nə- does not appear before words beginning with [l] in 7 and 8, due to a phonotactic restriction against n(ə)-l sequences. If a different noun class were used, it would be present, ex. \textit{ðénə́ŋ nə-ɲəgətʃan}

\ea \begin{tabular}[t]{ll}
6 &	ðénəŋ nə-gwənto \\
7 &	ðénəŋ ləgətʃan \\ 
8 &	ðénəŋ ləgətʃín\\ 
9 &	ðénəŋ nə-marlon\\
  \end{tabular}
\z

The word reð is 10, and subsequent numbers use the formula 10 and 1, 10 and 2, etc.. 

\ea \begin{tabular}[t]{ll}
10 & reð\\
11 & reð nə-gwənto\\
12 & reð ləgətʃan\\
13 & reð ləgətʃín\\
14 & reð nə-marlon\\
15 & reð nə-ðénə́ŋ\\
16 & reð nə-ðénə́ŋ nə-gwənto  \\
17 & reð nə-ðénə́ŋ ləgətʃan\\
18 & reð nə-ðénə́ŋ ləgətʃín  \\
19 & reð nə-ðénə́ŋ nə-marlon  \\
   \end{tabular}
\z
	
The words for 5 and 10 may be derived from the word for ‘hand’. The singular for ‘hand’ in Thetogovela is \textit{ðéj} and the plural is \textit{réj}.

The word for 20 is the plural of 10 with a following numeral 2, literally `two tens'. The numerals 2 and 3 that follow 10 to convey multiples of 10 agree with the plural form of `ten' for noun class, which is \textit{n-}, so \textit{nəgətʃan} and \textit{nəgətʃín}. Beyond 20, the decimal system begins again, culminating with 100, which is 10 (x) 10. There is no Moro number for 1000; the Arabic term \textit{ɜlf} is used. The Arabic word \textit{mijɜ} can also be used for 100. 

\ea \begin{tabular}[t]{ll}
20	&ńdréðeə nəɡətʃan                  \\
21	&ńdréðeə nəɡətʃan nə-gwənto \\
22	& ńdréðeə nəɡətʃan ləgətʃan\\
{\ldots} & \\
30	&ńdréðeə nəgətʃín \\
40	&ńdréðeə marlon \\
50	&ńdréðeə ðénə́ŋ \\
60	&ńdréðeə ðénə́ŋ nəgwənto  \\
70	&ńdréðeə ðénə́ŋ ləgətʃan \\
80	&ńdréðeə ðénə́ŋ ləgətʃín  \\
90	&ńdréðeə marlon \\
100	&ńdréðeə reð   \\
200	&ńdréðeə reð ləgətʃan   \\
1000 & 	ɜlf \\
    \end{tabular}
\z

Numerals follow the noun in Moro, and the lower numbers (1-3) agree in noun class with the noun via weak concord. The numbers 4, 5 (and derivatives of base 5) and 10 do not agree with the head noun at all. The numeral ‘one’ agrees with the singular noun class, and ‘two’ and ‘three’ agree with the plural form of the noun. The weak concord prefix \textit{j-} is realized as [e] in the numeral ‘two’ and [i] in the numeral ‘three’, determined by the vowel height of the numeral root: 

\ea \ea \begin{tabular}[t]{ll}
ðamala ðənto	&	‘one camel’\\
jamala egətʃan	&	‘two camels’\\
jamala igətʃín	&	‘three camels’\\
jamala marlon	&	‘four camels’\\
jamala ðénə́ŋ	&	‘five camels’\\
jamala reð 		&	‘ten camels’ \\
\end{tabular}
	\ex \begin{tabular}[t]{ll}
indiə gənto		&	‘one drum’	\\
nəndiə nəgətʃan	&	‘two drums’\\
nəndiə nəgətʃín	&	‘three drums’\\
nəndiə marlon	&	‘four drums’\\
nəndiə ðénə́ŋ	&	‘five drums’\\
nəndiə reð 		& 	‘ten drums’  	\\
 \end{tabular}
	\z
\z

The complete paradigm of the numerals 1-3 are given in Table \ref{tab:ch8:3} for the nine main noun class pairings. The plurals of ‘two’ and ‘three’ surface with an initial /ŋ/ in the plural ɲ-class rather that /ɲ/, e.g. ŋ̘ɡətʃan, due to nasal assimilation to the following velar stop. The agreement forms above are instances of weak concord.


\begin{table}
 \begin{tabular}[t]{llll}
\lsptoprule
noun class & one & two & three\\
\midrule
g/l & gwənto & ləgətʃan & ləgətʃín\\
g/n	& gwənto & nəgətʃan & nəgətʃín\\
j/j & ento & egətʃan & igətʃín	\\
l/ŋ & l(ə)nto & ŋgətʃan & ŋgətʃín	\\
l/ɲ & l(ə)nto & ɲəgətʃan & ɲəgətʃín	\\
ŋ/ɲ & ŋwənto & ɲəgətʃan & ɲəgətʃín	\\
ð/r & ðənto & rəgətʃan & rəgətʃín	\\
ð/g & ðənto & gəgətʃan & gəgətʃín	\\
ð/j & ðənto & egətʃan & igətʃín	\\	
\lspbottomrule
 \end{tabular}
  \caption{Numeral paradigms}
  \label{tab:ch8:3}
\end{table}


Numerals allow H tone to spread from the last syllable of a preceding H-toned noun onto the first two syllables of L-toned numerals:

\ea \begin{tabular}[t]{ll}
	 ebəl egətʃan &	‘two birds (species that hang upside down)’ \\
	 ején éɡə́tʃan &	‘two mountains’ 	 \\
 \end{tabular}
 \z 

%DATA IN JENKS (2013) has some mistakes - check to see if tone spreading is found with 3, which has a LLH pattern in isolation. 

Like other modifiers, numerals can be used elliptically, without a head noun, where they occur with weak concord. The noun ‘horse’ belongs to the class g/n:

\ea 
	\ea \gll	á-ɡ-a-bwáɲ-á 				nəmərtá̪ 	↓məńáw\\
				2\textsc{sg}.\textsc{sm}-\textsc{cl}g-\textsc{rtc}-want-\textsc{\textsc{ipfv}} 	\textsc{cl}n.horse 	how.many\\
		\glt 	‘How many horses do you want?’
	\ex	\gll	(é-ɡ-a-bwáɲ-á) 				ɡwənto\\
				1\textsc{sg}.\textsc{sm}-\textsc{cl}g-\textsc{rtc}-want-\textsc{\textsc{ipfv}}	\textsc{cl}g.one\\
		\glt 	‘(I want) one (horse)’
	\ex	\gll 	(é-ɡ-a-bwáɲ-á) 				nəgətʃan\\
				1\textsc{sg}.\textsc{sm}-\textsc{cl}g-\textsc{rtc}-want-\textsc{\textsc{ipfv}}	\textsc{cl}n.two\\
		\glt 	‘(I want) two (horses)’
	\z 
\z
Recall that elliptical possessors and demonstratives occurred with strong concord in these contexts. As expected, the distribution of strong versus weak concord in these elliptical anaphora correspond to the definiteness of the anaphor.

%ORDINAL NUMERALS?	None exist, right?


\subsection{Adjectives and subject relative clauses}\label{section:adjsubrelative}

Adjectives in Moro are a predicative category that can occur as the main predicate of a clause (Chapter \ref{adjective}) As such, when adjectives are used to modify a noun, they have the structure of subject relative clauses, in that both follow the noun, use the \textit{é-} dependent clause prefix, and agree with the head noun. The adjectives below occur with strong concord, as evident by the raising of the final vowel of the noun: \textit{eməɾt̪á} → \textit{eməɾt̪ɜ́}  or \textit{jamala} → \textit{jamalɜ́}, as well as by the use of [k] and [s] instead of [g] and [j] as noun class concord:

\ea %downstep?
	\ea \gll	é-g-a-bwáɲ-á 	eməɾt̪ɜ́ 	-g-é-bəg-á \\
				1\textsc{sg}.\textsc{sm}-\textsc{cl}g-\textsc{rtc}-want-\textsc{\textsc{ipfv}}  	\textsc{cl}g.horse	\textsc{scl}g\textsc{sm}.\textsc{dpc}1-strong-\textsc{adj}\\
		\glt	‘I want the strong horse’
	\ex	\gll	é-g-a-bwáɲ-á 	eməɾt̪ɜ́ 	-g-í-munw-ɜ́\\
				1\textsc{sg}.\textsc{sm}-\textsc{cl}g-\textsc{rtc}-want-\textsc{\textsc{ipfv}}  	\textsc{cl}g.horse	\textsc{scl}g\textsc{sm}.\textsc{dpc}1-black-\textsc{adj}\\
		\glt	‘I want the black horse’
	\ex	\gll	í-g-ɜ-sɜtʃ-ú 	jamalɜ́ 	-s-é-bəg-á\\
				1\textsc{sg}.\textsc{sm}-\textsc{cl}g-\textsc{rtc}-see-\textsc{pfv}  	\textsc{cl}j.camel	\textsc{scl}j\textsc{sm}.\textsc{dpc}1-strong-\textsc{adj}\\
		\glt 	‘I saw the strong camels’
	\z
\z

The dependent clause prefix \textit{é-} is the same one used in subject cleft and relative clause constructions (Chapter \textit{relative}), so the attributive adjective may be analyzed as a type of relative clause. Compare the following two structures:

\ea %downstep?
	\ea \gll	í-g-ɜ-sɜtʃ-ú 	jamalɜ́ 	-s-é-bəg-á \\
				1\textsc{sg}.\textsc{sm}-\textsc{cl}g-\textsc{rtc}-see-\textsc{pfv}  	\textsc{cl}j.camel	\textsc{scl}j\textsc{sm}.\textsc{dpc}1-be strong-\textsc{adj}\\
		\glt ‘I saw strong camels’

	\ex \gll 	í-g-ɜ-sɜtʃ-ú 	jamalɜ́ 	-s-é-kəɾ-ó 	íɾíə\\
			1\textsc{sg}.\textsc{sm}-\textsc{cl}g-\textsc{rtc}-see-\textsc{pfv}  	\textsc{cl}j.camel	\textsc{sm}.\textsc{scl}j.\textsc{dpc}1-break-\textsc{pfv} \textsc{cl}g.fence\\
		\glt 	‘I saw the camels who broke the fence’
	\z
\z

If the adjective is vowel-initial, the clause prefix \textit{é-} does not appear due to vowel hiatus (the first vowel is deleted), but its H tone is recuperated on the first vowel of the adjective. The root ogən has no H tone in the predicate version of the adjective, but acquires a H tone when it is used attributively in the ‘relative’ form:

\ea %downstep?
	\ea \gll	tərəbésá 	ð-ogən-á	\\
	\textsc{cl}ð.table 	\textsc{sm}.\textsc{cl}ð-be big-\textsc{adj}	\\
		\glt	‘the table is big’
	\ex \gll	k-a-daŋ-ó 	tərəbésá 	ékáɾé 	ð-ógən-á \\
			\textsc{sm}.\textsc{cl}g-\textsc{rtc}-sit-\textsc{pfv}	\textsc{cl}ð.table 	under	\textsc{sm}.\textsc{cl}ð-\textsc{dpc}1.be big-\textsc{adj}\\	
		\glt	‘s/he sat under the big table’	
	\z
\z

Like other modifiers, adjectives and subject relative clauses can occur without a head noun, in which case strong concord shows up in its full form, in this case ísː-:

\ea 
	\ea \gll	ŋw-ajén 	j-áŋga 	n-ɜ́-sɜtʃ-ú?\\
\textsc{foc}-\textsc{cl}j-mountain	\textsc{cl}j-which	\textsc{comp}-2\textsc{sg}.\textsc{sm}-see-\textsc{pfv}\\
		\glt		‘which mountain did you see?
	\ex	\gll 	í-g-ɜ-sɜtʃ-ú 	ísː!-ógən-á\\
	1\textsc{sg}.\textsc{sm}-\textsc{cl}g-\textsc{rtc}-see-\textsc{pfv}	\textsc{cl}j\textsc{dem}-be big-\textsc{adj}	\\
		\glt		‘I saw the big one’
	\z
\z

Table \ref{tab:ch8:4} provides the paradigm for an adjectival subject relative with strong concord. The same observations apply as before: a final H is inserted on all-L nominal roots, and final /a/ undergoes raising.%mention or remove downstep?

\begin{table}
 \begin{tabular}[t]{llllll}
 \lsptoprule
Class &	\textsc{sg}.N& \textsc{scl}-\textsc{src}-A	& 	\textsc{pl}.N 	& 	\textsc{scl}-\textsc{src}-A	& Gloss\\
 \midrule
 g/l	& udʒí 	& ↓k-é-bəgá	& 	lidʒí	& ↓l-é-bəgá	& ‘strong person(s)’\\
g/n	& emert̪ɜ́& 	↓lébəgá	& 	nəmert̪ɜ́	& ↓nébəgá	& ‘strong horse(s)’\\
j/j	& ajén	& ↓sébəgá	& 	ején	& ↓sébəgá	& ‘strong mountain(s)’\\
l/ŋ	& ləvərɜ́& 	↓lébəgá	& 	ŋəvərɜ́	& ↓ŋébəgá& 	‘strong stick(s)’\\
l/l	& láw	& 	↓lébəgá	& 	ɲáwː	& ↓ɲébəgá	& ‘strong mosquito(s)’\\
ŋ/ɲ	& ŋeɾɜ́	& ↓ŋébəgá	& 	ɲeɾɜ́	& ↓ɲébəgá	& ‘strong child(ren)’\\
ð/r	& ðápːɜ́	& ↓ðébəgá	& 	rápːɜ́	& ↓rébəgá	& ‘strong friend(s)’\\
ð/j	& ðamalɜ́& 	↓ðóɡəná	& 	jamalɜ́	& ↓sébəgá	& ‘strong camel(s)’\\
 \lspbottomrule
 \end{tabular}
  \caption{Moro adjective/subject relative clause inflection: \textit{bəgá} ‘strong’}
  \label{tab:ch8:4}
\end{table}

Like possessives (cf. 10-11), relative clauses can occur without strong concord. This occurs in two contexts: if other modifiers intervene between relatives and their head noun, and in object position. The first context is shown below with an intervening demonstrative:

\ea \gll jamalɜ́		-sːɜtísːə	j-é-bəɡ-á	j-a-j-ó\\
 		\textsc{pl}.camel	\textsc{scl}.that	\textsc{cl}j-\textsc{src}-big-\textsc{adj}	\textsc{cl}-\textsc{rtc}-die-\textsc{pfv}\\
 	\glt	‘Those camels that are big died.’
\z 

In object position the presence of strong concord marks definiteness:

\ea 
	\ea \gll	é-ɡ-a-bwáɲ-á		jamalɜ́		-↓sː-é-bəɡá\\
 				1\textsc{sg}-\textsc{cl}g-\textsc{rtc}-like-\textsc{\textsc{ipfv}}	\textsc{pl}.camel	\textsc{scl}-\textsc{src}-strong-\textsc{adj}\\
 		\glt 	‘I like the camels that are strong.’
	\ex \gll	 é-ɡ-a-bwáɲ-á	jamala		j-é-bəɡá\\
 			1\textsc{sg}-\textsc{cl}-\textsc{rtc}-like-\textsc{\textsc{ipfv}}	\textsc{pl}.camel	\textsc{cl}-\textsc{src}-strong-\textsc{adj}\\
 		\glt ‘I like camels that are strong.’
	\z
\z

The absence of geminate concord in (23b) correlates with normal tone and vowel quality on the final syllable of \textit{jamala} ‘camels’, as expected. However, the modifier \textit{jébəɡá} ‘which are strong’ is still identifiable as a subject relative clause based on the \textit{é-} prefix. 

%That the alternation in (23) is only permitted in object position is due to a requirement that Moro subjects must be definite or specific. is this true// Further evidence for this position is the fact that Moro does not allow wh-elements in situ in subject position.

\subsection{Non-subject relative clauses}\label{nonsubjectrelative}

Relative clauses formed on objects, oblique arguments, and adverbs such as `when' and `how' form a grammatical class. An example of an object relative clause is provided below:

\ea 	\gll  jamalɜ́-sː-ə		(nә́=↓)kúk:u		ɡ-ə́-sɜtʃ-ú 	\\
 			\textsc{pl}.camel-\textsc{scl}-this 	\textsc{comp}2=Kuku	\textsc{cl}g-\textsc{dpc}2-see-\textsc{pfv}	\\
		\glt ‘The camel that Kuku saw.ʼ
\z 

The clause vowel \textit{ə́-} in object relatives, distinguishing them from the \textit{a-} of main clauses or \textit{é-} of subject relatives (See Section \ref{clausevowel}). Additionally, the head noun of an object relative takes a suffix segmentally identical to the proximal demonstrative (Section 4). The final /i/ of this demonstrative reduces to schwa before an object relative. Last, relative clauses include the proclitic \textit{nə́=}, analyzed as a complementizer due to the fact that it also introduces certain subordinate clauses (Chapter \ref{subordination}). With full nominal subjects, \textit{nə́=} can appear before both the subject and the verb phrase, although such multiple occurrences are unattested in texts. For more details on the subject agreement paradigm in non-subject relative clauses see Section \ref{relativeclause}. 

%TODO Peter Elyasir Indefinite non-subject relative clauses?? Can NSRCs occur without a demonstrative element?

%Evidence that the demonstrative element in non-subject relative clauses is the correlate of strong concord comes again from fragment answers. As the answer to the question “Which camel is hungry?” (cf. 20a) one could respond  \textit{ðamalɜ́ðːənísɜtʃú} ‘the camel that I saw’ (cf. 24), but never *\textit{ínísɜtʃú}. While íðːənísɜtʃú  ‘this one that I sawʼ is a grammatical noun phrase, it would not be an appropriate answer to the question in (20a). This might be because the demonstrative element íðːi ‘this’ is interpreted in such a response, resulting in too many foci.


\section{Definiteness and quantification}%separate chapter?

This section surveys the expression of definiteness and quantification in Moro. This includes discussion of the semantic properties of bare nouns, the marking of indefiniteness, and the scope of various quantifiers, including the interactions between quantifiers and negation.

\subsection{Bare nouns}\label{section:baren}

The two main means of marking definiteness are through syntactic position, as there is a tendency for definite, topical noun phrases to occur as subjects, and via the strong concord markers reviewed above. The following example demonstrates that bare nouns in Moro can be used both as indefinite expressions in subject position but also as anaphoric definite expressions in subsequent clauses:

\ea 
	\ea \gll	éréká	í-ɡ-ɜ-sɜtʃ-ú		ówːá		n-óráŋ \\
		yesterday 	1\textsc{sg}-\textsc{cl}g-\textsc{rtc}-see-\textsc{pfv} 	woman 	and-man\\
		\glt	‘Yesterday I saw a woman and a man.’ \label{indefbaren}
	\ex \gll	óráŋ 	ɡá-ɡ-oval-á	n-ówːá		ɡá-ɡ-obəl-á\\
		man 	\textsc{pst}-\textsc{cl}g-tall-\textsc{adj}	and-\textsc{sg}.woman 	\textsc{pst}-\textsc{cl}-short-\textsc{adj}\\
		\glt ‘The man was tall, but the woman was short.’ \hfill %EJ
	\z
\z 
Similarly, uniquely identifiable objects are translated with bare nouns. When shown a picture of a single bird in a single tree, a speaker chose to refer to them both via bare singular nouns: 

\ea Context: A single bird is in a single tree.\\
	\gll  ugɜfiə g-a-w-ó ík-ugi		\\
		   bird  \textsc{cl}g-\textsc{rtc}-be.loc-\textsc{pfv} \textsc{loc}-tree\\
	\glt ‘The bird is in the tree.’  \hfill %AN
\z 
While bare nouns can be definite or indefinite, we saw in the previous section that strong concord plays a role in marking definiteness. Basically, demonstratives always occur with strong concord while genitive phrases and relative clauses only surface strong concord when the noun phrase is definite. Thus, because strong concord can only occur once in a noun phrase, strong concord can be seen as a definite marker which is restricted to modified noun phrases.

The interplay between definite bare nouns and strong concord can be clearly seen in texts. One story in the \textit{Moro Story Corpus} focuses on two men, one disabled and another blind. When the identity of the blind versus disabled man is in question, strong concord is always used (the examples below have been translated into Thetogovela Moro):

\ea \gll matʃe ík-i g-írmɜt̪u n-ə́ŋ-ɜit̪-i matʃe ík-i g-é-kər-ó undr t̪á {\ldots}\\
	man \textsc{cl}g-this \textsc{cl}g-be.blind \textsc{comp}2-3\textsc{sg}.\textsc{cons}-say.to-\textsc{cons}.\textsc{pfv} man \textsc{scl}g-this \textsc{cl}g-\textsc{dpc}1-break-\textsc{pfv} waist \textsc{comp}1 {} \\
	\glt	`The blind man said to the crippled man\ldots' \hfill (from `The cripple and the blind man')
\z 
But when the identity of the referent is clear in the story, whether the blind man or the disabled one, the bare noun \textit{matʃó} (\textit{maje} in Written Moro) `man' is used.

Plural nouns freely allow generic interpretations in Moro. Singular nouns in generic contexts seem to result in definite readings:
\ea 
	\ea \gll  	eða		j-a-ŋəɾ-á\\
		\textsc{pl}.meat 	\textsc{cl}j-\textsc{rtc}-good-\textsc{adj}\\
		\glt 	‘Meat is good.’
	\ex \gll rða		r-a-ŋəɾ-á \\
		\textsc{sg}.meat 	\textsc{cl}-\textsc{rtc}-good-\textsc{adj}\\
		\glt ‘The piece of meat is good.’  \hfill %EJ
	\z
\z 

\ea 	
	\ea[]{ \gll 	ŋénéə 	nə́-ɲe-d-ó		úmːiə, 	é-ɡ-a-bwáɲ-á		eða \\
		when 	\textsc{cmp}-1\textsc{sg}-be-\textsc{pfv}	\textsc{sg}.boy	1\textsc{sg}-\textsc{cl}-\textsc{rtc}-like-\textsc{\textsc{ipfv}}  	\textsc{pl}.meat \\
		\glt ‘When I was a boy, I liked meat.'}
	\ex[\#]{
		\gll ŋénéə 	nə́-ɲe-d-ó		úmːiə, 	é-ɡ-a-boáɲ-á		rða \\
		when 	\textsc{cmp}-1\textsc{sg}-be-\textsc{pfv}	\textsc{sg}.boy	1\textsc{sg}-\textsc{cl}-\textsc{rtc}-like-\textsc{\textsc{ipfv}} 	\textsc{sg}.meat\\
		\glt ‘When I was a boy, I liked the piece of meat.’  } %EJ} 
\z 
\z %TODO Peter Elyasir more generic nouns?

In object position, bare singular nouns can receive indefinite interpretations, as the following example and example \REF{indefbaren} show:

\ea \gll i-g-ɜ-dwɜdʒ-it̪-u kúku ádámá\\
		1\textsc{sg}-\textsc{cl}g-\textsc{rtc}-send-\textsc{appl}-\textsc{pfv} Kuku book\\	
	\glt 	‘I sent Kuku a book.’	 \hfill %AN	
\z
With the exception of with types of nonverbal predicate, however, bare nouns in subject position are preferentially interpreted as definite. Instead, preverbal indefinite expressions tend to have overt indefinite modifiers, described in Section \ref{sec:ch8:indefinite}. 

\subsection{Universal quanfication}\label{sec:ch8:universal}


%1. preð %Is preð part of ðotegovela? NO-- IT's WERRIA
%
%-- `all' with plural nouns, typically right after the noun before modifiers:
%
%ldǝbǝleđe dia ŋǝmaŋa pređ eŋen na ldiɽi ndǝm.
%`they pulled the cow with all their strength and both of them fell down,'
%
%keđa na laŋge pređ ildi leđa ləlosa,
%keđa
%`the leftover wine and everything left over from people's meals'
%
%Alǝṯwer đwala pređ iđi đǝṯaṯënu!
%Let us sell all livestock that are left!
%
%
%-- Example below shows distributive meaning
%Chalendor and Hyena:
%Lorba ildi lǝɽijin lafo lamamǝña ṯalerldeṯo nwaldaŋ ñoman pređ ulëldiṯano na ëđǝñinano.
%those her three sisters, they went out far away every morning and at midday.
%
%-- `whole' with singular nouns
%
%“Đappa, obǝđaŋ ram lǝbate larna dia pređ!”
%"My friend, hurry up the ground is swallowing up the (whole) cow!"
%
%2. təmtəm %can this occur as adverb? between N and dem? Check Thetogovela: NO! It's WERRIA
%
%-- has adnominal looking uses:
%
%na nǝŋǝṯwe laŋge tǝmtǝm elden,
%and sold all their belongings,
%
%-- with extraposition
%
%na algǝrus yamǝndađo tǝmtǝm isi igapǝlo nayen.
%and the money which I brought from Nuba Mountains is finished,
%
%-- with pronoun:
%
%``Igënǝñi igaɽo oralo goɽra na igabërnia Ñaŋpređ,''
%I-g-ënǝ-ñi i-g-a-ɽ-o or-alo g-oɽra na i-g-a-b-ërni-a Ñaŋ-pređ,
%1sg-\textsc{cl}g-be.1d-1sg.om 1sg-\textsc{cl}g-rtc-be.\textsc{rt}-pfv brother-2.poss \textsc{cl}g-big and 1sg-\textsc{cl}g-rtc-called.\textsc{rt}-\textsc{ipfv} 2pl-all.
%"I am your elder brother and my name is, "All of you",
%	
%Can this occur between N and Dem?

The general purpose universal quantifier is \textit{ododo} `all.' This quantifier is an adverb, and has the distribution of an adverb (see Chapter \ref{chapter:adverbs}), but the quantifier must follow the plural noun phrase that it takes as its restriction. The examples in \REF{allpost} illustrate the adverbial distribution \textit{ododo}: it can occur either between the subject and the verb or after the object, but not between the object and the verb. In the examples below, curly brackets represent the possible positions that a single word can occupy, though the word can only occupy one of these positions. 

\ea Context: A tree with four birds in it (Bruening \#1)\footnote{The specific images which were used for elicitation are those from the \textit{Scope 
Fieldwork Project}, available at \texttt{http://udel.edu/$\sim$bruening/scopeproject/scopeproject.html}. Mr. Naser was the primary consultant who participated in scope judgments tasks with the Bruening materials.} \label{allpost}
	\ea \gll ugi g-ert-ó \{*ododo\} ndəfí-ánó \{ododo\}\\
			tree \textsc{cl}g-have-\textsc{pfv} \{all\} birds-inside \{all\}\\
		\glt	‘The tree has all the birds in it.’
	\ex \gll ndəfi \{ododo\} n-a-w-ó ík-ugi \{ododo\}\\
			 birds \{all\} \textsc{cl}n-\textsc{rtc}-be.loc-\textsc{pfv} \textsc{loc}-tree \{all\}\\
		\glt	‘All the birds are in the tree.’  \hfill %AN(date)
	\z
\z 
The universal quantifier \textit{ododo} does not form a constituent with the noun phrase. First, when this quantifier occurs between the noun and its modifiers in object position, the modifiers are obligatorily extraposed, which is evident in that they show strong concord which is not fused with the preceding word (See Section \ref{sec:ch12:extraposition} for a discussion of extraposition).

\ea \gll dwɜdʒ-ɜt̪-í-ɲí ɲibrmir ododo íɲ:-i-ogəná			\\
send-\textsc{loc}.\textsc{appl}-\textsc{ven.imp}-1\textsc{sg}.\textsc{om} barrels all \textsc{scl}ɲ-this-big\\ 
	\glt 	‘Send me every barrel that is big.’ 
\z  
Second, in subject position, \textit{ododo} can occur between the noun phrase and the verb, like all adverbs, but it cannot occur between the noun and modifiers.

\ea 
\ea 	\gll jamal-ɜ́s-í-ogəná ododo j-a-coɲ-á\\
camels-\textsc{scl}-\textsc{dpc1}-big all \textsc{cl}j-\textsc{rtc}-hungry-\textsc{adj}\\ 
			\glt 	‘All the camels that are big are hungry.’ 
	\ex[*]{ \gll jamala ododo ís-í-ogəná  j-a-coɲ-á\\
camels all \textsc{scl}-\textsc{dpc1}-big \textsc{cl}j-\textsc{rtc}-hungry-\textsc{adj}\\ 
	}	\z  
\z 

%6. égawarató lemmia enega ndʌngen ododo		‘I insulted every boyj in their/j houses.’
%SCOPE: shows that ododo must be higher than ‘house’.
%7. ŋweneg ndaoŋken ini égwaratəlú lʌmia ododo	‘It was in theirj house that I insulted every boy.’

%TODO Peter Elyasir demonstrative/scl is a bit weird in this exmaple. Why can it occur with ododo?

%I bought every camel that is strong.
%I didn't buy every camel that is strong.
%All camels that are strong eat sorghum.(with \& without scl?)
%Most camels that are strong eat sorghum?






The quantifier \textit{ododo} can typically scope above or below negation regardless of its syntactic position.  The following examples show that inverse scope, with negation scoping over the adverb, is possible even when the adverb precedes the negative marker.  %todo Peter Elyasir what about in object position 
 %Wide scope is freely available, as illustrated in example \REF{allpost}.

\ea
	\ea Context: Four men are catching four fish, but two fish remain in the water ($\neg > \forall $, Bruening \#16).\\
	\gll ləme \{ododo\} l-enná ɜl-ɜ́nd-ən-iə \{ododo\}		\\
		 fish \{all\}	\textsc{cl}l-\textsc{neg.aux} 3\textsc{pl.inf}-catch-\textsc{pass}-\textsc{\textsc{ipfv}} \{all\}\\
	\glt ‘Every fish was not caught’
	\ex Context: Three birds are in three of four trees, one bird is on the ground and one tree is empty ($\neg > \forall $, Bruening \#3).\\
	\gll ndəfiə ododo n-enná al-áv-éa í-lúgwí		\\
		 birds all	\textsc{cl}n-\textsc{neg.aux} 3\textsc{pl.inf}-be.loc-\textsc{ipfv} \textsc{loc}-trees \\
	\glt ‘Not all the birds are in the trees’
	\z
\z

%Todo - Add discussion of universal quantifiers scoping above negation.

%Does the same flexibility occur for universal quantifier follows negation? we saw that this was not so for an indefinite.

%With multiple quantifiers, on the other hand, there is a preference for surface scope. That is, with \textit{ododo} taking the subject as its restriction and an indefinite quantifier in object position, there is a preference for the quantifier to appear preverbally if it is being interpreted with wide scope.
%\ea  Context: A tree has every lantern in it, three trees have no lantern. ($  \exists > \forall $, Bruening \#5)\\
%	\gll tinar \{ododo\} ð-a-w-ó ík-úgí g-ə́n:əŋ \{??ododo\} \\
%		lantern {all} \textsc{cl}ð-\textsc{rtc}-be.loc-\textsc{pfv} \textsc{loc}-tree {\ \ all} \\
%	\glt  `Every lantern is in a tree.' 
%\z 
%This example suggests that \textit{ododo} must precede another operator in order to scope above it, but additional work is needed to determine if this is true, particularly in the realm of negation.


%4. \textit{ndem} 'both' %TODO Peter Elyasir check thetogovela
%
%also: all three, all four
%
%Lënŋulu ndǝm lafo laɽo lëđǝm,\\
%Both of them were young men,\\
%
%oro đǝge eloman lomǝn tǝŋ aliđi alǝɽe lǝŋǝndǝm,\\
%then the next time we will go both of us,

\subsection{Indefinite modifiers}\label{sec:ch8:indefinite}

This section describes five adjectives in Moro which are always indefinite:
\ea 
 \begin{tabular}[t]{ll}
{-ə́n:əŋ}& `some (\textsc{sg}.)'\\
{-ə́m:ə́n} &  `some (\textsc{pl}.)' \\
{-oaɲá} & `a lot of \\ 
{-ɜ́mɜtɜŋ} &  `a bit of' \\
{-érto} & `a different' \end{tabular}
\z  
%These modifiers typically occur in indefinite noun phrases. 
These modifiers show weak concord. They do not occur with demonstratives or modifiers marked with strong concord, either in texts or in elicitation. This point provides evidence that they are restricted to indefinite noun phrases, as strong concord is associated with definiteness. %can these occur with a demonstrative?

The indefinite modifier \textit{-ə́n:əŋ} is impossible in a context where there is a unique identifiable referent, a definite context, indicating it is indefinite.
\ea[\#]{Context: Four birds in one tree (Bruening \#1)\\
	 \gll ndəfi ododo nawó ík-ugi g-ənəŋ	\\	
				 birds all \textsc{cl}n-\textsc{rtc}-be.loc-\textsc{pfv} \textsc{loc}-tree \textsc{cl}g-\textsc{indef}\\	
	\glt		`All the birds are in a tree'\\
		Comment: ``It’s strange because there’s only one tree.''}
\z 
Compare example \REF{allpost} above, where a bare noun for tree is used in this same context, as it is a definite environment. 

The plural indefinite \textit{-ə́m:ə́n} `some (pl.)' is restricted to plural count nouns, it is often rejected with mass nouns (though such uses in texts do sometimes occur --- see, for example, \REF{ex:ch8:mud} below).
\ea 
	\ea \gll jamala j-ə́mə́n j-a-w-ó n-ayen\\
			camels \textsc{cl}j-some \textsc{cl}j-\textsc{rtc}-be.loc-\textsc{pfv} on-mountains\\
		\glt	`Some camels are in the mountains.'
	\ex[*]{ \gll  ŋaw ŋ-ə́mə́n j-a-w-ó n-ayen\\
			water \textsc{cl}ŋ-some \textsc{cl}j-\textsc{rtc}-be.loc-\textsc{pfv} on-mountains\\
			}
	\z 
\z 
The quantifiers \textit{-ə́n:əŋ} and \textit{-érto} `a different' are restricted to singular count nouns. In contrast, he quantity modifiers \textit{-oaɲá} and `a lot of and  
\textit{-ɜ́mɜtɜŋ} `a bit of' can occur with either plural count nouns or mass nouns.

Generally, these modifiers qualify as adjectives in that in addition to modifying nouns they can serve as the main predicates of sentences. 
\ea
	\ea \gll  ðamala ð-ə́n:əŋ\\
			water \textsc{cl}ŋ.some \textsc{cl}j-\textsc{rtc}-be.loc-\textsc{pfv} on-mountains\\
		\glt	`There's a camel.'
	\ex  \gll jamala j-ə́m:ə́n\\
			camels \textsc{cl}j-some\\
		\glt 	`There are some camels.'
	\ex \gll jamala j-oaɲá\\
			camels \textsc{cl}j-many\\
		\glt 	`There are a lot of camels.'
\z
\z
This distinguishes these modifiers from, for example, numerals, which can only occur as main predicates after the predicate nominal copular (\sectref{sec:ch9:nompred}). 

The quantity indefinite \textit{-oaɲá} can take the adjectival comparative marker \textit{-ət̪} (\sectref{sec:ch10:comp}), resulting in a comparative use of the adjective.
\ea \gll jamala j-oaɲ-it̪-ú nemert̪a\\
			camels \textsc{cl}j-many-\textsc{comp}-\textsc{pfv} horses\\
	\glt 	`There are more camels than horses.'
\z
The comparative marker cannot attach to the other indefinite modifiers.

%WIDE SCOPE: From notes

\subsubsection{Scope of indefinites}

This section describes the scopal possibilities of indefinite noun phrases, with particular attention to \textit{-ə́n:əŋ} and \textit{-ə́m:ə́n}. In the first part of this section, I show that both indefinites have specific and non-specific indefinite uses.

The second part of the section focuses on how the position of indefinite modifiers affect their scop. While indefinite modifiers in object position can have high in low scope, indefinites associated with subjects can appear postverbally, in which case they obligatorily scope under verbal auxiliaries such as negation. The examples in this section below combine textual data from the Moro Story Corpus, presented in Written Moro, with examples from targeted elicitation, which are presented in Thetogovela Moro. 

First, the examples in \REF{ex:ch8:speca} illustrate specific indefinite uses of \textit{-ə́n:əŋ}. In the first example, the indefinite is making reference to a specific river flowing from a region of the Nuba mountains. In the second example, the indefinite is used in a partitive context, which are always specific due to their association with a contextually supplied set.
%\ea \gll na ŋǝɽañ ŋ-ǝnǝŋ n-ǝŋ-ënṯ-i e-sǝɽo na oɽo ṯ-ay-ay-o y-aŋa-l-ay-a.\\
%and disease \textsc{cl}ŋ-indef comp2-3sg.cons-enter.\textsc{rt}-\textsc{cons}.pfv \textsc{loc}-goats and goats comp1b-cly.inf-die.\textsc{rt}-pfv with-3sg.inf-3pl.om-die.\textsc{rt}-\textsc{ipfv}\\
%\glt and a certain disease caught the goats and sheep and cause them to die.
%\z 

\ea Textual examples of specific indefinites \label{ex:ch8:speca}
\ea \gll  đǝbarlda đ-ǝnǝŋ đ-eṯ-o Irǝfi na đ-omǝn n-ǝđ-el-a Ecǝmwarre,\\
stream \textsc{cl}ð-indef \textsc{cl}ð-come.\textsc{rt}-pfv Irvi and clđ-other comp2-clđ.inf-come.\textsc{rt}-\textsc{ipfv} Echmwarre\\
\glt one branch of stream flows from Irvi area, and another branch from Echmwarre hill,\hfill  (from `Moving') 
\ex \gll Loman-nǝŋ maj-anda l-a-fo l-ǝmǝn ndǝjan l-ǝ-lǝŋ-ǝn-u alo N-ayen Ende,\\
day-\textsc{indef} men-assoc.\textsc{pl} \textsc{cl}l-\textsc{rtc}-\textsc{past.aux} \textsc{cl}l-\textsc{indef} two \textsc{cl}l-\textsc{dpc}2-give.birth.\textsc{rt}-\textsc{pass}-\textsc{pfv} place on-mountains Ende\\
	\glt `Once upon a time there were two brothers who lived in the Ende Mountains,\ldots \label{ex:ch8:some1}
	\ex \gll maje g-ǝnǝŋ g-a-b-ërn-ia Kwëlira\\
man \textsc{cl}g-\textsc{indef} \textsc{cl}g-\textsc{rtc}-\textsc{prog}-be.called.\textsc{rt}-\textsc{ipfv} Kolira\\
\glt `\ldots one of them was called Kolira.' \hfill (from `Kuku and the dragon')
\z
\z 
Example \REF{ex:ch8:some1} illustrates a specific indefinite use of \textit{-ə́m:ə́n} to introduce individuals in a story. In that example, the indefinite does not occur next to the subject, which it modifies, but instead in an position after the verb discontinuous from the noun it agrees with. This phenomenon is common in Moro (\sectref{sec:ch12:extraposition}), it plays an important role in the scope of indefinites. What is important to see here is that a specific indefinite use of  \textit{-ə́m:ə́n}  obtains despite occurring postverbally. 

 The examples below offer additional examples of \textit{-ə́m:ə́n} referring to a specific group or substance. 
\ea Textual examples of specific indefinites
	\ea 	\gll  Na leđa l-ǝmǝn n-lde-ɽǝṯ-e alo Ekau, na l-ǝmaṯan n-ldǝ-f-eṯ-e alo Noge.\\
					and people \textsc{cl}l-some \textsc{comp}2-\textsc{cl}l.inf-stop.\textsc{rt}-\textsc{cons}.\textsc{pfv} place Ecow and \textsc{cl}l-\textsc{indef} \textsc{comp}2-\textsc{cl}l.\textsc{inf}-be.\textsc{loc}-\textsc{loc}.\textsc{appl}-\textsc{cons}.\textsc{pfv} place Noge\\
			\glt Some families moved to Ecow village and a few settled in Noge village,\ldots \hfill (from `Moving')
	\ex 	\gll n-an-ëbəđ-ən-i ŋaca-ŋa ŋ-əmən ŋ-ore ŋ-ə-đam-o ŋawa ŋə-đənia,\\
			\textsc{comp}2-cln.\textsc{inf}-build.\textsc{rt}-\textsc{pass}-\textsc{cons}.\textsc{pfv} mud-\textsc{cl}ŋ.with \textsc{cl}ŋ-some \textsc{cl}ŋ-red \textsc{cl}ŋ-\textsc{dpc}-resist.\textsc{rt}-\textsc{pfv} water \textsc{cl}ŋ.\textsc{poss}-rains\\
			\glt \ldots it is built with a red colored mud which resisted rain,\ldots  \label{ex:ch8:mud} \hfill (from `What do Moro people do?')
	\z 
\z 
%NARROW SCOPE
In summary, then, \textit{-ə́m:ə́n} can function as a specific indefinite for plurals and mass nouns.


%\ea \gll Pǝnde ram maje g-a-fo g-a-f-o g-ǝnǝŋ g-ǝ-b-ërn-ia Ukuwa,\\
%Past early man \textsc{cl}g-\textsc{rtc}-\textsc{past.aux} \textsc{cl}g-\textsc{rtc}-be.\textsc{loc}-\textsc{pfv} \textsc{cl}g-indef \textsc{cl}g-\textsc{dpc}-\textsc{prog}-be.called.\textsc{rt}-\textsc{ipfv} Aukowa\\
%\glt Once upon a time there was a man called Aukowa,
%\z 


%\ea Textual examples
%\ea \gll Abalimi n-ǝŋ-eɽǝđ-e Apaḏuḏ ṯa aŋ-iđi aŋ-el-a loman-nǝŋ\\
%Abalimi \textsc{comp}2-3\textsc{sg}.\textsc{inf}-ask.\textsc{rt}-\textsc{cons}.\textsc{pfv} Apadud \textsc{comp}1 3\textsc{sg}.\textsc{inf}-\textsc{fut.aux} 3\textsc{sg}.\textsc{inf}-come.\textsc{rt}-\textsc{ipfv} day-\textsc{indef}\\
%	\glt Abalimi invited Apadud to come and visit him one day. \hfill (from `Abalimi rode over Apadud') \label{ex:ch8:invite}
%\ex \gll ŋenŋanṯa ndǝ eđa g-ǝ-đëm-ǝn-u g-ǝnǝŋ,\\
%because if man \textsc{cl}g-\textsc{dpc}-defeat.\textsc{rt}-\textsc{pass}-\textsc{pfv} \textsc{cl}g-\textsc{indef}\\
%	\glt For when somebody is defeated,\ldots \hfill (from `Moving') \label{ex:ch8:condition}
%\z 
%\z 

%Interestingly, in texts \textit{-ə́n:əŋ} occurs with plural nouns in limited contexts, all of which seem to be episodic existential environments.
%\ea Textual examples of \textit{-ə́n:əŋ} with plural nouns
%	\ea \gll “I-g-afo i-g-oɽ-ǝṯ-o i-lǝbu n-ǝñ-arr-ǝbǝc-e ñere ñ-ɜ́n:ǝŋ ñ-ǝ-ɽ-o ñowa,\\
%1sg-\textsc{cl}g-\textsc{past.aux} 1\textsc{sg}-\textsc{cl}g-go.\textsc{rt}-\textsc{loc}.\textsc{appl}-\textsc{pfv} \textsc{loc}-well \textsc{comp}-clɲ.\textsc{inf}-\textsc{iter}-raise.\textsc{rt}-\textsc{cons}.\textsc{pfv} girls clɲ-\textsc{indef} clɲ-\textsc{dpc}-be.\textsc{rt}-\textsc{pfv} young women\\
%	\glt "I went down the well and helped some young women to climb up,\hfill \hfill (from `The boys and the bears')
%	\ex \gll I-liga l-akǝl leđa l-ǝnǝŋ l-a-fo l-a-ɽǝñ-ǝđ-ia Ṯur-đa iđ-i đore\\
%			\textsc{loc}-time \textsc{scl}l-that people \textsc{cl}l-\textsc{indef} \textsc{cl}l-\textsc{rtc}-\textsc{past.aux} \textsc{cl}l-\textsc{rtc}-kill.\textsc{rt}-\textsc{ap}-\textsc{ipfv} Turkish.government-with \textsc{scl}đ-this \textsc{cl}đ.red\\
%		\glt At that time people were fighting with the Turkish army. \hfill (from `The cripple and the blind man')
%	\z 
%\z 
%These examples indicate that \textit{-ə́n:əŋ} may be used for plural nouns when existential meanings are needed while \textit{-ə́m:ə́n} is restricted to referential or specific uses of plural or mass nouns. With singular nouns, \textit{-ə́n:əŋ} has both specific and nonspecific uses.

%TODO check plurals with ənəŋ in these contexts: ``At that time some people were living in Khartoum"

%likɜ́lla

%NEGATION

Similarly, in the presence of the negative auxiliary (\sectref{sec:ch14:negaux}), bare plural subjects typically scope above negation, as does and the indefinite \textit{-ə́m:ə́n} in subject position. 

\ea Elicited examples of bare and indefinite plural subjects scoping above negation
\ea \gll lɜdʒí l-é-ðat̪-ó ʌni  l-a-n:á al-íɲé-noan-a\\
people \textsc{cl}l-\textsc{dpc1-}pass-\textsc{pfv} here \textsc{cl}l-\textsc{rtc}-\textsc{not.aux} \textsc{3pl.inf}-\textsc{1sg.om}-watch-\textsc{inf2}\\
\glt `People who passed by here didn't see me.' (Nobody saw me)
\ex \gll lɜdʒí l-ə́:ə́n l-é-ðat̪-ó ʌni  l-a-n:á al-íɲé-noan-a\\
people \textsc{cl}l-some \textsc{cl}l-\textsc{dpc1-}pass-\textsc{pfv} here \textsc{cl}l-\textsc{rtc}-\textsc{not.aux} \textsc{3pl.inf}-\textsc{1sg.om}-watch-\textsc{inf2}\\
\glt `Some of the people who passed by here didn't see me.' ($\exists>\neg$, Comment: `Some saw you, some didn't.') \label{ex:ch8:someb}
\z 
\z 
Example \ref{ex:ch8:someb} illustrates that \textit{-ə́m:ə́n} `some' can have quantificational uses, no specific individuals are being referred to.

In object position, \textit{-ə́m:ə́n} `some' can have both narrow and wide scope uses relative to negation, while a bare noun interpreted indefinitely has only narrow scope.
\ea 
\ea \gll  é-g-a-n:á ɲi-sɜ́tʃ-ɜ́  lɜdʒí l-é-ðat̪-ó ʌni \\
\textsc{1sg-cl}g-\textsc{rtc}-\textsc{not.aux} \textsc{1sg.inf}-see-\textsc{inf2} people \textsc{cl}l-\textsc{dpc1-}pass-\textsc{pfv} here \\
\glt `I didn't see people who passed by here.' ($\neg>\exists$) \label{ex:ch8:someaa} 
\ex \gll  é-g-a-n:á ɲi-sɜ́tʃ-ɜ́ lɜdʒí  l-ə́:ə́n  l-é-ðat̪-ó ʌni \\
\textsc{1sg-cl}lg\textsc{rtc}-\textsc{not.aux} \textsc{1sg.inf}-see-\textsc{inf2} people \textsc{cl}l-some \textsc{cl}l-\textsc{dpc1-}pass-\textsc{pfv} here \\
\glt i) `I didn't see any people who passed by here.' ($\neg>\exists$)\\ \label{ex:ch8:somebb} 
	 ii) `I didn't see some of the people who passed by here.' ($\exists>\neg$) 
\z 
\z 
While the modified bare noun with weak concord on the modifier in \REF{ex:ch8:someaa} can only scope under negation, the indefinite in \REF{ex:ch8:somebb} can either scope above or below negation. More concretely, this sentence can be used either in a context where nobody was seen ($\neg>\exists$) or in a context where some people were seen but others weren't ($\exists>\neg$).

Similarly, singular indefinites formed with \textit{-ə́n:əŋ} in object position can scope above or below negation. Because this modifier is much more common in texts, we can illustrate with the following naturally occurring examples.

\ea High and low scope of \textit{-ə́n:əŋ} relative to negation in object position \label{ex:ch8:objindefhilo}
	\ea  \gll  aŋ-erṯe g-ə-pəđ-ia  			ig-ëpwa 	y-enəŋ \\
			\textsc{3sg.inf}-not.aux  \textsc{cl}g-\textsc{dpc1}-fight-\textsc{ipfv} \textsc{loc}-fight, \textsc{cl}j-\textsc{indef}, \\
		\glt `\ldots and does not fight in certain fighting events' \hfill ($\exists>\neg$, from `Stickfighting')  \label{ex:ch8:stick1}
	\ex \gll  Abalimi  n-ǝŋ-aṯ-a, ``Ndo, i-g-a-b-er i-g-ǝ-sëc-ia	 waŋge 	g-ǝnǝŋ kwai kwai." \\
			 Abalimi \textsc{comp2}-\textsc{3sg.cons}-say-\textsc{ipfv} no 1\textsc{sg}-\textsc{cl}g-\textsc{rtc-prog}-not.aux 1\textsc{sg}-\textsc{cl}g-\textsc{dpc2}-see-\textsc{ipfv} thing \textsc{cl}g-\textsc{indef} never never \\
		\glt `Abalimi answered, "No, I can’t see anything at all."' \hfill ($\neg > \exists$ from `Abalimi rode over Apadud') \label{ex:ch8:no}
	\z 
\z 
In \REF{ex:ch8:stick1}, the translation `certain fighting events' indicates the existence of events is not being negated, hence, that the locative object is scoping above negation. In \REF{ex:ch8:no}, however, the indefinite in object position is clearly in the scope of negation.

We now focus on additional syntactic factors influencing the scope of the indefinites, focusing on the singular indefinite \textit{-ə́n:əŋ}. We have already seen that \textit{-ə́n:əŋ} can be negated, forming negative existential statements. This can done in two ways. First, the negative copula \textit{-eró} can occur as a nominal modifier immediately before indefinite \textit{-ə́n:əŋ}, resulting in a negative indefinite quantifier.

\ea Textual examples of negation scoping above \textit{-ə́n:əŋ} ($\neg>\exists$) \label{neg1} %(translate?)
\ea \gll na eđa g-ero g-ənəŋ g-ero ŋəmađa.\\
		and person \textsc{cl}g-not.have \textsc{cl}g-\textsc{indef} \textsc{cl}g-not.have.\textsc{rt} peer\\
	\glt \ldots and nobody who did not have a peer group.\hfill (from `Bringing up children and naming them')
	\ex \gll na eđa g-ero g-ǝnǝŋ g-iđi aŋ-ŋa-nac-e wag-ǝnǝŋ bipi kwai kwai,\\
and person \textsc{cl}g-\textsc{neg.aux} \textsc{cl}g-\textsc{indef} \textsc{cl}g-\textsc{fut.aux} 3\textsc{sg}.\textsc{inf}-3\textsc{sg.om}-give.\textsc{rt}-\textsc{cons}.\textsc{pfv} something only never never\\
	\glt  no one can give you something generously. \hfill (from `Moving')
	\ex \gll Ŋen ŋ-ero ŋ-ǝnǝŋ ŋ-ǝ-b-ǝđ-ǝn-ia đǝmǝtianǝđa,\\
matter \textsc{cl}ŋ-\textsc{neg.aux} \textsc{cl}ŋ-\textsc{indef} \textsc{cl}ŋ-\textsc{dpc}-\textsc{prog}-do.\textsc{rt}-\textsc{pass}-\textsc{ipfv} without\\
	\glt Nothing is done without money. \hfill (from `Moving')
		\z 
\z
	\ea Elicited examples of negation scoping above \textit{-ə́n:əŋ} ($\neg>\exists$) \label{neg2}\\
	\gll matʃó g-eró g-ə́n:əŋ g-et̪-o ega dɜŋgaɲ\\
man \textsc{cl}g-not.have \textsc{cl}g-\textsc{indef} \textsc{cl}g-come.to-\textsc{pfv} house household-\textsc{1p} \\ %what's this possessive pronoun?
	\glt ‘Nobody came to our house.’
\z 
This strategy is restricted to subjects. Attempts to construct these examples in object position are rejected Instead, speakers prefer to use the strategy of negating an object indefinite with the negative clausal auxiliary \textit{-en:á}.

\ea Strategies for scoping object-associated indefinites below negation ($\neg>\exists$) \label{neg2}\\
	\ea  \gll é-g-a-nná ɲi-sɜ́tʃ-ɜ́  udʒí g-ə́n:əŋ\\
\textsc{1sg-cl}lg\textsc{rtc}-\textsc{not.aux} \textsc{1sg.inf}-see-\textsc{inf2}  man \textsc{cl}g-\textsc{indef}\\ %what's this possessive pronoun?
  		\glt `I didn't see anybody'
	\ex[*] { \gll ígɜsɜtʃú udʒi g-eró g-ə́n:əŋ \\
\textsc{1sg-cl}lg\textsc{rtc}-see-\textsc{ipfv}  person \textsc{cl}g-not.have \textsc{cl}g-\textsc{indef} \\ %what's this possessive pronoun?
			\glt ‘I didn't see anybody.’ (intended)}
\z 
\z

When indefinites like \textit{-ə́n:əŋ} `a(n)' and \textit{-oaɲá} `many' are associated with the subject in a sentence with a negative auxiliary, their interpretation depends on their position relative to the negative auxiliary. Only indefinites which precede the negative auxiliary can scope above it. If a subject-associated indefinite occurs after the verb it must have a low scope interpretation (see Section \ref{sec:ch12:extraposition} on split noun phrases like these).

\ea  Scope of \textit{-ə́n:əŋ} `a(n)' and clausal negation
	\ea \gll ðamala ð-ə́n:əŋ ð-en:á áŋ-avel-a	    \\
			camel \textsc{cl}ð-\textsc{indef} \textsc{clð}-not \textsc{3sg.inf}-go-\textsc{inf2}\\
			\glt ‘One camel of the camels did not come.’ ($\exists > \neg$) \hfill 	AN111015 \label{ex:ch8:awide}
	\ex 	\gll ðamala ð-en:á áŋ-avel-a ð-ə́n:əŋ	    \\   
			 camel  \textsc{clð}-not \textsc{3sg.inf}-go-\textsc{inf2} \textsc{cl}ð-\textsc{indef} \\
			\glt ‘No camels went.’ ($\neg>\exists$)
	\z 
\ex Scope of \textit{-oaɲá} `many' and clausal negation
	\ea \gll jamala j-oáɲa j-en:á al-avəl-a	\\
			camels \textsc{cl}j-many \textsc{clð}-not \textsc{3sg.inf}-go-\textsc{inf2} \\
		\glt ‘Many camels didn’t go’  (many $> \neg$)
	\ex \gll  jamala j-en:á ala-vəl-a j-oáɲa\\
	camels \textsc{cl}j-many \textsc{clð}-not \textsc{3sg.inf}-go-\textsc{inf2} \\	
		\glt ‘Not many camels went. ($\neg >$ many) \hfill AN111015
	\z
\z
The examples above are the translations offered by one of our consultants for these examples. Below we will see that while indefinites in subject position are able to take scope below a following auxiliary, subject-associated indefinites which are extraposed to a position following the verb must take low scope.

Evidence that postverbal indefinites must take low scope is illustrated in the following judgment.
\ea Context: Three birds in three of four trees, one bird is on the ground and one tree is empty ($\exists>\neg$, Bruening \#3)
	\ea  \gll ugufia g-əneŋ g-enná aŋə-v-ea ik-ugi	\\
			 bird \textsc{cl}g-\textsc{indef} \textsc{cl}g-neg.\textsc{aux} 3\textsc{sg}-be.\textsc{loc}-\textsc{inf}2 \textsc{loc}-tree\\
		\glt 	‘A bird isn’t in the tree.’ \label{ex:ch8:indefneg}
	\ex[\#]{ \gll ugufia genná aŋave ík-ugi gə́nəŋ\\
				 bird  \textsc{cl}g-neg.\textsc{aux} 3\textsc{sg}-be.\textsc{loc}-\textsc{inf}2 \textsc{loc}-tree \textsc{cl}g-\textsc{indef} \\
			\glt	‘No bird is in the tree’\\
		Comment: ‘Means all the birds are on the ground.’} \label{ex:ch8:negindef}
	\z
\z
The context is one where there exists a bird who is not in the tree. In order the existence of such a bird to be asserted, the indefinite modifier of the subject \textit{ugufia} `bird' must precede negation. 

When an indefinite modifier precedes a negative auxiliary, on the other hand, high scope for negation seems to be only a preference. Speakers often judge inverse scope for preverbal indefinites to be acceptable. For example, the two sentences below were accepted in a context where there were no arrivals at the house.
\ea Elicited examples of negation scoping above pre and post-verbal \textit{-ə́n:əŋ}, $\neg>\exists$ %what's this possessive pronoun?
	\ea \gll matʃó g-enná g-ét̪-ó ega dɜŋg-aɲ g-ə́n:əŋ\\
			man \textsc{cl}g-\textsc{neg.aux} \textsc{cl}g-come.to-\textsc{pfv} house household-\textsc{1p} \textsc{cl}g-\textsc{indef}\\ 
			\glt ‘Nobody came to our house.’
	\ex \gll matʃó g-ə́n:əŋ g-enná g-ét̪-ó ega dɜŋg-aɲ\\
			man \textsc{cl}g-\textsc{indef} \textsc{cl}g-\textsc{neg.aux} \textsc{cl}g-come.to-\textsc{pfv} house household-\textsc{1p} \\
	\glt ‘Nobody came to our house.’ \hfill AN10192015
	\z
\z

Confirmation from texts that indefinite modifiers associated with a subject can scope below a verb are found in \REF{textanya}. Furthermore, in texts we find cases where an extraposed indefinite can occur between an auxiliary and a main verb \REF{textanyb}, and in a position after the verb \REF{textanyc}, where such an interpretation is required.  The negative auxiliary is \textit{-(b)er(te)} in Written Moro, with the different forms determined by aspect. 

\ea Textual examples of negation scoping above \textit{-ə́n:əŋ}, $\neg>\exists$   \label{textany} %(note that thetogovela has a different negative marker) - translate?.
\ea \gll  na eđa g-ənəŋ aŋ-erṯe g-iđi aŋə-m-ënṯ-i nano kwai kwai.\\
and man \textsc{cl}g-\textsc{indef} 3\textsc{sg}.\textsc{inf}-not \textsc{cl}g-\textsc{fut.aux} 3\textsc{sg}.\textsc{inf}-3\textsc{sg}.\textsc{om}-enter.\textsc{rt}-\textsc{cons}.\textsc{pfv} near never never\\
\glt  `\ldots and nobody is allowed to enter near her at all.' \hfill (from `Bringing up children and naming them') \label{textanya}
\ex \gll Na eđa n-ǝŋ-erṯe g-ǝnǝŋ g-ǝ-lǝŋeṯ-o ŋǝɽwata eŋen!\\
and man \textsc{comp}2-\textsc{3sg.cons}-\textsc{neg.aux} \textsc{cl}g-\textsc{indef} \textsc{cl}g-\textsc{dpc}-know.\textsc{rt}-\textsc{pfv} speech 3\textsc{pl}.\textsc{poss}\\
\glt `\ldots and no one was able to understand their language!' \hfill (from `Moving') \label{textanyb}
\ex \gll Nanda ña-g-a-b-er ña-g-ǝ-fa-fiđ-ia ŋǝsa alo y-enǝŋ\\
1\textsc{ex.pro} 1\textsc{ex}-\textsc{cl}g-\textsc{rtc}-\textsc{prog}-\textsc{neg.aux} 1\textsc{ex}-\textsc{cl}g-\textsc{dpc}2-\textsc{iter}-find.\textsc{rt}-\textsc{ipfv} food down \textsc{cl}y-\textsc{indef}\\
\glt `We didn’t find any food anywhere.'\hfill (from `Chalendor and hyena') \label{textanyc}
\z
\z 
To summarize, then, indefinites associated with subjects can take scope above or below negation. Both of these interpretations are possible when the indefinite modifier occurs preverbally, in subject position. But the low scope interpretation is obligatory when an indefinite modifier follows negation. 

Indefinites following negation can take high scope under two conditions. The first context where an indefinite following negation can scope above it is when it is associated with a postverbal argument, as in \REF{ex:ch8:objindefhilo}. Second, the low scope restriction for postverbal indefinites does not hold for negative existential sentences (See Section \ref{existcop} for further discussion of negative existential sentences). For example, in \REF{ex:ch8:negexindef}, indefinite modifiers take scope above the the negative existential predicate even when they are extraposed to a position following it.
\ea  \label{ex:ch8:negexindef}
	\gll	írtə́	  g-eró g-ə́n:əŋ\\
				knife 	\textsc{cl}g-not.be \textsc{cl}g-\textsc{indef}\\
		\glt 	‘Some knife isn't here/there.’  ($\exists>\neg$)
\ex	\gll	ŋáwá	ŋ-eró ŋ-ə́mat:aŋ\\
				water 	\textsc{cl}ŋ-not.be \textsc{cl}ŋ-\textsc{indef}\\
		\glt 	‘Some (of the) water isn't here/there.’ ($\exists>\neg$)\\
				Comment: You bought a bunch of bottles of water and some of them were left behind.  %\hfill EJ72116
	\z 
Intuitively, the reason why these examples have specific interpretations is that their counterparts without any indefinite modifier have the negative existential reading. That is, \textit{írtə́ geró} means `There is no knife.' 


To conclude, one indefinite modifier, \textit{-ə́m:ə́n}, must scope above negation, and is often associated with specific indefinite uses. The interpretation of the other indefinites depend on the syntactic status of the noun they modify and their position: postverbal indefinites which modifying the subject must take scope below verbal operators like negation. Indefinites in other positions allow scopal ambiguity. Additional work is needed to determine whether other scopal operators in the clause behave like negation with respect to quantifier scope, and the interpretation of sentences with multiple quantifiers.






%TODO NEED to Do MorE ELICITATion WIth -əmən
%I didn't see any camels. ənəŋ? 
%I didn't see any camel.



%\textit{-emən} occurs below negation in the following example:
%
%\ea \gll  g-ə-b-er g-ə-b-ađ-ia laŋge l-əmən,\\
%\textsc{cl}g-\textsc{dpc}-\textsc{prog}-\textsc{neg.aux} \textsc{cl}g-\textsc{dpc}-\textsc{prog}-do.\textsc{rt}-\textsc{ipfv} things \textsc{cl}l-\textsc{indef}\\
%\glt not doing any cooking,
%\z
%


%
%
%Indefinite expressions occur before all other modifiers except for negative \textit{-ero} (example \ref{neg1-neg2}), and never occur with demonstratives:
%
%\ea \gll n-ǝŋǝ-rmaṯ-e wara nano g-ǝnǝŋ g-oɽra,\\
%\textsc{comp}2-3\textsc{sg}.coms-arrive.\textsc{rt}-\textsc{cons}.\textsc{pfv} tabaldi.tree at \textsc{cl}g-\textsc{indef} \textsc{cl}g-big,\\
%\glt he arrived at a big tabaldi tree,
%\z 
%
%\ea \gll Loman-nǝŋ maj-anda l-a-fo l-ǝmǝn ndǝjan l-ǝ-lǝŋ-ǝn-u alo Nayen Ende,\\
%day-\textsc{indef} men-assoc.\textsc{pl} \textsc{cl}l-\textsc{rtc}-\textsc{past.aux} \textsc{cl}l-\textsc{indef} two \textsc{cl}l-\textsc{dpc}2-give.birth.\textsc{rt}-\textsc{pass}-\textsc{pfv} place mountains Ende\\
%\glt Once upon a time there were two brothers who lived in the Ende Mountains,
%\z 
%
%\ea \gll n-an-ëbəđ-ən-i ŋaca-ŋa ŋ-əmən ŋ-ore ŋ-ə-đam-o ŋawa ŋə-đənia,\\
%\textsc{comp}2-cln.\textsc{inf}-build.\textsc{rt}-\textsc{pass}-\textsc{cons}.\textsc{pfv} mud-\textsc{cl}ŋ.with \textsc{cl}ŋ-some \textsc{cl}ŋ-red \textsc{cl}ŋ-\textsc{dpc}-resist.\textsc{rt}-\textsc{pfv} water \textsc{cl}ŋ.\textsc{poss}-rains\\
%\glt it is build with a red colored mud which resisted rain,
%\z 

\subsection{Indefinite pronouns}%decide what to do with this section and where to put it...
\label{sec:ch8:indpro}

Moro has a single series of indefinite pronouns, setting aside \textit{wh}-words or content question words, which are only used in questions. These `indefinite pronouns' are not really pronouns, but rather are generic nouns with the indefinite modifier \textit{-ə́n:əŋ} (Section \ref{sec:ch8:indefinite}) In the truncated version of the human indefinite pronoun listed below, class concord is absent from the indefinite marker. In the others there is class agreement on the indefinite.

\ea Indefinite pronouns\\ \label{indefpro}
\begin{tabular}[t]{lll}
iðə́-n:əŋ		& man-\textsc{indef} & ‘somebody’ \\
kwaŋa-g-ə́n:əŋ	& thing-\textsc{cl}g-\textsc{indef} & ‘something’\\
anə-j-ə́n:əŋ		& place-\textsc{cl}j-\textsc{indef} & ‘somewhere’\\
%nɜ́ŋwə́t̪iə	& situation-how	& ‘somehow’\\
ɲómə́n-ɲ-ə́n:əŋ		& days-\textsc{cl}j-\textsc{indef} & ‘sometimes, some days’\\
lómə́n-l-ə́n:əŋ	& day-\textsc{cl}l-\textsc{indef} & ‘sometime, one day’\\\end{tabular} 
\z

These expressions have the same scopal possibilities as the indefinites described in Section \ref{sec:ch8:indefinite}. The examples below illustrate the use of these indefinite pronouns in narrow scope positions, below negation.

\ea Textual examples of indefinite pronouns with low scope
\ea \gll orn n-ld-erṯe l-ǝ-fiđ-ia eđa g-ǝnǝŋ l-ǝ-lǝŋeṯ-ǝ-ma,\\
but \textsc{comp}2-\textsc{cl}l.\textsc{inf}-not.aux \textsc{cl}l-\textsc{dpc}-find.\textsc{rt}-\textsc{ipfv} man \textsc{cl}g-indef \textsc{cl}l-\textsc{dpc}-know.\textsc{rt}-\textsc{pfv}-3\textsc{sg}.om\\
\glt \ldots but they did not find anyone they knew,\ldots \hfill (from `Moving')
\ex \gll  na n-ǝŋ-erṯ ŋ-iđi aŋ-oɽǝbaṯ-e i-gi loma-nǝŋ tǝŋ.\\
and \textsc{comp}2-3\textsc{sg}.\textsc{cons}-\textsc{neg.aux} \textsc{cl}ŋ-\textsc{fut.aux} 3sg.\textsc{inf}-come.\textsc{rt}-\textsc{inf}1 \textsc{loc}-field day-indef again.\\
\glt \ldots and he did not come back to the field anymore.  \hfill (from `Chalendor and hyena')\label{anymore}
\z 
\z 
The forms in these textual examples are slightly different from those in the table above, as they represent the Written Moro variants. In some cases, these include truncated forms which were not provided in elicitation, e.g. \textit{loma-nǝŋ} for `sometime, anymore' in \REF{anymore} versus the longer form with class agreement in \REF{indefpro}. The exact differences between these full versus truncated forms requires further investigation. 


%`Somehow' differs from the other forms in that the expression `how' occurs after the indefinite marker. This enclitic distribution is typical for \textit{=tia} `how' (See section \ref{how})

%\subsection{Distributive universal quantifiers}\label{section:distributive}


%\ea Universal pronouns\\
%\begin{tabular}[t]{lll}
%éjə-ðɜ́n:əŋ-gə́n:əŋ	& ?-\textsc{indef} & ‘everybody’ \\%TODO what is the root that this is attaching to?
%éjə-kwaŋá-gə́n:əŋ	& thing \textsc{cl}g-\textsc{indef} & ‘everything’\\
%ék-án-ódodo		& \textsc{loc}-place-all & ‘everywhere’\\
%ðáðá-ódodo	& way-every	& ‘every way’\\
%ét̪et̪o	& allways & ‘always’\\
%\end{tabular}  \z %these are arabic according to Angelo & Hannah. Not sure what to do with these...


%TODO Check the plural indefinite pronouns behave in Thetogovela:
%
%\ea \gll  aŋ-iđi la-l-əmən na aŋə-b-orkwađ-aṯ-e orra nano g-əlëɽəŋu ṯwañ!\\
%3\textsc{sg}.\textsc{inf}-do.\textsc{rt} thing-\textsc{cl}l-indef and 3\textsc{sg}.\textsc{inf}-\textsc{prog}-approach.\textsc{rt}-\textsc{loc}.\textsc{appl}-\textsc{cons}.\textsc{pfv} husband near \textsc{cl}g-3\textsc{sg}.\textsc{poss} near\\
%\glt and cook food and approach near her husband!
%\z
%
%\ea \gll ŋenŋanṯa g-afo g-ero g-ǝ-s-a la-l-ǝmǝn liga l-walano.\\
%because \textsc{cl}g-\textsc{past.aux} \textsc{cl}g-not.aux \textsc{cl}g-\textsc{dpc}-eat.\textsc{rt}-\textsc{ipfv} \textsc{cl}l.\textsc{poss}-\textsc{cl}l-indef time \textsc{cl}l-long\\
%\glt because he had not eat anything for a long time.
%\z 
%

%
%

%

%
%\ea \gll Abalimi n-ǝŋ-aṯa
%, “Ndo, i-g-a-b-er i-g-ǝ-sëc-ia waŋge-g-ǝnǝŋ kwai kwai."\\
%Abalimi \textsc{comp}2-3sg.cons-say.\textsc{rt} no 1sg-\textsc{cl}g-\textsc{rtc}-\textsc{prog}-not.aux 1sg-\textsc{cl}g-\textsc{dpc}2-see.\textsc{rt}-\textsc{ipfv} thing-\textsc{cl}g-indef never never\\
%\glt Abalimi answered, "No, I can’t see anything at all."
%\z 
%

%\subsection{Specific indefinite \textit{-əmən}} vs. =əmatan?? (seems to occur with mass nouns?)

%INDEFINITE PRONOUNS
%\subsection{-omən `other'}
%
%\ea 
%	\ea \gll na ig-i g-omǝn ṯ-aŋ-ërn-u Kaloka na ŋere ṯ-aŋ-ërn-u Kwajerna.\\
%			and s\textsc{cl}g-this \textsc{cl}g-indef \textsc{comp}1b-3sg.\textsc{inf}-be.called.\textsc{rt}-d.\textsc{inf}2 Kaloka and girl \textsc{comp}1b-3sg.\textsc{inf}-be.called.\textsc{rt}-d.\textsc{inf}2 Kwaje\\
%		\glt and the other is called Kaloka and the daughter is called Kwajema.
%	\ex \gll ñ-erṯe ñ-a-g-ǝm-a đađ iđ-i đ-omǝn đ-ǝ-b-ǝṯ-a ne-ṯa.\\
%		2\textsc{pl}-not.aux 2\textsc{pl}-\textsc{rtc}-\textsc{cl}g-take.\textsc{rt}-imp road sclđ-this sclđ-other sclđ-\textsc{dpc}-go.\textsc{rt}-\textsc{ipfv} on-pool\\
%		\glt do not take the other road which goes by the pool.
%	\z 
%\z 

%\subsection{Adjectives of quantity?}


%-- Least common in our texts, occurs (only?) in an adverbial position:
%
%"Ŋen ŋanṯau nǝñarrǝpaŋa egal isi ywalano ododo oro ŋela ŋañǝsi rǝmwa?
%"How can I carry you all this way, then you let me eat the snake?


%4. Each
%
%ŋini ŋavárla ŋónt̪o ŋónt̪o		‘The dogs are coming one by one.’
%ɲini ɲavarla ɲónto ɲónto		‘(same?)’
%ɲini ɲavarla ɲeneɲəneŋ		‘(same?)’
%
%lədʒi lavarla gənəgənəŋ		‘Each woman came.’
%*lədʒi gənəgənəŋ lavarla	
%jamala javarla ðənəðənəŋ		‘Each camel came.’
%jamala ðənəðənəŋ javarla		
%


%
%(spoken to a single person)



