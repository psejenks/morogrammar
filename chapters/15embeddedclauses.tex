\chapter{Embedded clauses}\label{chapter:embeddedclauses}


\section{Embedded clause types}\label{sec:ch15:clausetypes}


\subsection{Inflectional patterns}

\ea  {Finite complementation}
\ea  \gll  Kúkːu	g-a-rə́mə́t̪-iə      $[$    (*t̪á)     g-{é}-$^{↓}$ʧə́ð-á                   ugi  $]$	\\
K. \textsc{cl}g-\textsc{rtc}-continue-\textsc{ipfv}  {}  \textsc{comp1} \textsc{cl}g-\textsc{dpc1}-chop-\textsc{ipfv}  \textsc{cl}g.tree {} {}\\
\glt ‘Kuku kept chopping the tree.’
\ex \gll 	é-g-a-mwandəð-ó Kúkːu-(ŋ)   $[$ t̪á g-{ə́}-$^{↓}$noán-á ðamala $]$\\
 \textsc{1sg}-\textsc{cl}g-\textsc{rtc}-ask-\textsc{pfv} Kuku-\textsc{acc} {} \textsc{comp1} \textsc{cl}g-\textsc{dpc2}-watch-\textsc{ipfv} \textsc{cl}j.camel {}\\
\glt ‘I asked Kuku to watch the camel.’
\z 
\ex  {Infinitival complementation}
\ea  \gll  Kúkːu g-íð-á $[$ (*n-)áŋə́-$^{↓}$ʤóm-é $]$ \\
Kuku \textsc{cl}g-(\textsc{rtc}-)do/will-\textsc{ipfv} {} {\ \textsc{comp1}}-\textsc{3sg.inf}-move-\textsc{inf1} {}\\
\glt ‘He will move.’ \label{2a}
\ex  \gll  	Kúkːu  g-əndəʧin-ú   $[$     (n)-áŋə́-$^{↓}$lə́və́ʧ-a              ŋálːo(-ŋ) $]$ \\
 	Kuku  \textsc{cl}g-(\textsc{rtc})-try-\textsc{pfv}   {}  \textsc{comp2-3sg.inf}-hide-\textsc{inf2}  Ngalo-\textsc{acc} {}\\
\glt   ‘Kuku tried to hide Ngalo.' \label{2b}
\z 
\z 

The table below summarizes the different kinds of finite clauses in Moro. The analyses suggested above have been outlined, and the different classes of predicates which occur in each class is listed as well.

\begin{table} 
	\caption{Summary: Finite complements}
\begin{tabular}[t]{llll}
\lsptoprule
\textsc{Clause vowel} & \textsc{Predicate Class} & \textsc{Complementizer} & \textsc{Analysis} \\
\lspmidrule
	a-/ʌ- 	& Assertive 		& t̪á 		& Normal finite complement \\
 		  & Perceptive	 	& t̪á 		& Normal finite complement \\
   		& Factive 			& t̪á 		& Normal finite complement \\ 
\midrule 
	 ə́- 	& Communicative	 & t̪á 	& Subjunctive \\ 
	 	& Desiderative &  t̪á 	& Subjunctive \\
\midrule 
	é-/í- & Perceptive 		& - 		& Raising-to-object \\
   		& Desiderative		 & - 		& Raising-to-object \\
  		 & \textit{seem} 		& - 		& Raising-to-subject \\ 
  		 & Aspectual 		& - 		& Raising-to-subject \\
\lspbottomrule
\end{tabular}
\end{table}

\subsection{Complementizers}

\section{Embedded finite clauses} \label{sec:ch15:finiteclauses}

%\ea  \ea  \textbf{a/ʌ-}: Finite root clauses with no extraction (root clause \textsc{rtc}, \REF{ex:ch15:fin1})
%\ex \textbf{é/í-): Subject relative clauses (dependent clause 1 \textsc{dpc1}, \REF{ex:ch15:fin2})
%\ex \textbf{ ə́-}: Non-subject relative clauses (dependent clause 2 \textsc{dpc2}, \REF{ex:ch15:fin3})
%	\z 
%\z 

\ea	\ea \gll  {Kúkːu} {g-{ʌ}-sʌtʃ-ú} {jamala}\\	
		Kuku  \textsc{cl}g-\textsc{rtc}-see-\textsc{pfv}	\textsc{cl}j.camel	\\
	\glt‘Kuku saw the camels.’ \label{ex:ch15:fin1}
	\ex	\gll \textit{jamalʌ́}			$[$	\textit{-sː-$^{↓}${í}-sʌtʃ-ú} 		\textit{Kúkːu}	$]$ \\	
		\textsc{cl}j.camel		{}	\textsc{scl}j-\textsc{dpc1}-see-\textsc{pfv}		Kuku\\
	\glt 		\glt ‘the camels that saw Kuku’ \label{ex:ch15:fin2}
	\ex	\gll \textit{jamalʌ́}	 $[$ \textit{-sːə}		\textit{(nә́-$^{↓}$)Kúkːu}	\textit{(nə́-$^{↓}$)ɡ-{ə́}-sʌtʃ-ú} $]$\\	
 		\textsc{cl}j.camel {}	-\textsc{scl} 	\textsc{comp2}-Kuku		\textsc{comp2}-\textsc{cl}g-\textsc{dpc2}-see-\textsc{pfv} {} \\
		\glt ‘the camels that Kuku sawʼ \label{nsrc} \label{ex:ch15:fin3}
	\z 
	\z 


\subsection{Embedded root clauses}\label{sec:ch15:rtc}

Standard finite complement clauses in Moro feature the same verbal morphology that occurs in matrix clauses, including finite subject agreement and the presence of the root-clause vowel \textit{-a/ʌ-}. These clauses are also introduced by the complementizer \textit{t̪á}, allowing distinct subjects, and allow distinct tense and aspect in the two clauses, as the examples below demonstrate:

\ea  \gll 		í-ɡ-ʌ-ʧ-ʌ́		nano	 			$[$	t̪á		Kúkːu 	g-{a}-koreð-ó		 ŋálːo-(ŋ) $]$\\
		\textsc{1sg}-\textsc{cl}g-\textsc{rtc}-bad-\textsc{adj}	\textsc{prt} {}	 \textsc{comp1}	 Kuku		\textsc{cl}g-\textsc{rtc}-scratch-\textsc{pfv} Ngalo	{}\\
	\glt‘I’m sad that Kuku scratched Ngalo.’
\ex \gll 		é-ɡ-a-lə́ŋét̪-a					$[$	t̪á	Kúkːu kʌ́-ɡ-{ʌ}-t̪und̪-ú  $]$\\
		\textsc{1sg}-\textsc{cl}g-\textsc{rtc}-know-\textsc{ipfv}	{} \textsc{comp1} Kuku	\textsc{pst}-\textsc{cl}-\textsc{rtc}-cough-\textsc{pfv} {}\\
		\glt ‘I know that Kuku had coughed.ʼ \label{src1}
\ex \gll  í-g-ʌləf-ət̪-ú                           or-áɲ-ó          $[$             	 t̪á 	é-g-{a}-ŋó-naʧ-a                          ut̪əɾə   $]$\\
\textsc{1sg}-\textsc{cl}-promise-\textsc{appl}-\textsc{pfv}  brother-\textsc{1pos}-\textsc{acc} {} \textsc{comp1} \textsc{1sg-cl}g-\textsc{rtc-3sgo}-give-\textsc{ipfv}  \textsc{cl}g.pig {}\\
\glt `I promised my brother that I’d give him a pig.'
\z 

The following classes of predicates occur with standard finite complement clauses in Moro:

\ea 
\ea  \textbf{Assertive:} \textit{-at̪-} `say/think', \textit{-nd- ŋéné} `believe,' \textit{-ʌlə́f-} `promise', \textit{-doát- ŋgámá} `whisper,'  \textit{-lúgə́t̪-} `tell (say + \textsc{appl})'
\ex  \textbf{Perceptive:} \textit{-nː-} `hear', \textit{-sʌ́tʃ-} `see,' \textit{-wə́ndat̪-} `watch'
\ex  \textbf{Factive:} \textit{-lə́ŋét̪-} `know', \textit{-lʌ́lːəŋəʤəʧən-} `remember,' \textit{-ʌdʒívʌ́tʃən-} `forget,' 
\ex  \textbf{Evaluative adjectives}: \textit{-tʃ- nano} `sad', \textit{-tʃ-} `bad,' \textit{-ŋər-} `good'
\z 
\z 

While the verbs and this class are both factive and non-factive, it is worth pointing out that the adjectives above are all factive. Thus, all finite factive complements in Moro show up in standard finite complement clauses. %show?

\subsection{Subjunctive complements}\label{sec:ch15:dpc2}

Other finite complement clauses in Moro occur with the \textit{non-subject relative clause} vowel \textit{-ə́-}, as in the following examples:

\ea \ea  \gll 	é-g-a-mwandəð-ó Kúkːu-ŋ   $[$ t̪á k-{ə́}-$^{↓}$noán-á ðamala $]$\\
\textsc{1sg-cl}g-\textsc{rtc}-ask-\textsc{pfv} Kuku-\textsc{acc} {} \textsc{comp1} \textsc{cl}g-\textsc{dpc2}-watch-\textsc{ipfv} \textsc{cl}ð.camel {}\\
\glt ‘I asked Kuku to watch the camel.’ \label{ex:ch15:camel1}
\ex  \gll   é-g-a-neð-ó   $[$ t̪á Kúkːu g-{ə́}-$^{↓}$noán-á ðamala $]$\\
\textsc{1sg-cl}g-\textsc{rtc}-refuse-\textsc{pfv} {}  \textsc{comp1} Kuku \textsc{cl}g-\textsc{dpc2}-watch-\textsc{ipfv} \textsc{cl}ð.camel {}\\
\glt ‘I refused/don’t like that Kuku watch the camel.’ \label{ex:ch15:camel2}
\z 
\z 

While the example in \REF{ex:ch15:camel1} looks like object control, example \REF{ex:ch15:camel2} cannot be an instance of object control because the lower verb has an overt subject.  This pattern has primarily been found with verbs of communication, including the examples above as well as -\textit{lúgə́t̪}- `tell.' Like non-subject relative clauses, these examples allow the normal `finite' complementizer \textit{t̪á}.

Given these issues, there are two possibilities. The first is that both examples above are instances of object control, and that  \REF{ex:ch15:camel2} is a case of Backward Control \citep{polinsky02}, where it is the higher clause which surfaces with a null argument. The second hypothesis is that these are not instances of control at all.

There is good reason to favor the second hypothesis: \citet{landau13} observes that there are no attested cases where finite complement clauses with agreement serve as the complements of obligatory control predicates:

\ea \textit{The finiteness rule for Obligatory Control}  \citep{landau13}  \label{ocrule}
In a fully specified complement clause (i.e., the I$^0$ head carries slots for both [T] and [Agr]):
\ea  If T$_0$ carries both semantic tense and agreement ([+T,+Agr]), No Control obtains.
\ex Elsewhere, Obligatory Control obtains.
\z
\z  

For Landau, No Control predicates refers to those that may allow a normal pronominal subject in the embedded clause, which is realized in languages like Moro with rich agreement on the verb. However, that subject need not be coindexed with a matrix argument; and when it is, it constitutes a normal cases of semantic binding of a pronominal anaphor (i.e. analogous to English \textit{John$_i$ told Mary that he$_i$ was sick.}). If these are instances of No Control, non-coreferential subjects should be freely available in the lower clause, and the null subject of the embedded clause in above (a) should be able to refer to someone besides the subject. The following example shows that the first of these predictions is correct:

\ea \gll í-g-ʌ-lug-ət̪-ú Kúkːu-ŋ  $[$ t̪á ŋálːo g-ə́-$^{↓}$nóan-á ðamala $]$\\
\textsc{1sg-cl}g-\textsc{rtc}-say-\textsc{appl}-\textsc{pfv} Kuku-\textsc{acc} {} \textsc{comp1} Ngalo \textsc{cl}g-\textsc{dpc2}-watch-\textsc{ipfv} \textsc{cl}ð.camel {}\\
\glt ‘I told Kuku for Ngalo to watch the camel.’
\z 

In this example, the embedded subject co-occurs with the non-subject relative clause vowel, and this subject is not coindexed with a matrix argument.\footnote{The second prediction has not been systematically tested. A context could be established where there is a topic DP which is an individual distinct from the subject. The null subject in the second clause would then be expected to be preferentially co-indexed with this topic.} Thus, this is not an instance of control. In a way this conclusion is a relief: obligatory control and non-subject relative clauses do not seem to form a natural morphosyntactic class, and it would be difficult to reconcile their shared morphology from a theoretical perspective.

If these are not instances of obligatory control, what feature is the ``non-subject relative clause'' vowel \textit{ə́}- an exponent of? We would like to suggest that this morpheme may be an exponent of subjunctive or irrealis mood. If correct, this hypothesis would provide a natural explanation for why this vowel occurs in the examples above: these are unrealized actions, requested or imagined by the speaker, and hence a natural environment for the subjunctive.  The more difficult question is why the subjunctive would occur in non-subject relative clauses. We would like to suggest that this vowel is also simply a reflex of clauses marking A-bar dependencies, and that irrealis clauses and questions form a natural class syntactically.


Tests from Farkas 1992:

same verb, in declarative vs. subjunctive:

1. I. said that P. left.
2. I said that P leave immediately.

3a. I believes that P left.
3b. I doesn't believe that P leave

Subjunctive governors: (Farkas p. 73)
1. `want', `order', modals (be possible, necessary), epistemic predicates expressing neutral or negative commitment: doubt, not believe, be possible/impossible



\subsection{Finite raising complements}\label{sec:ch15:dpc1}

A number of complement clauses occur with the \textit{subject relative clause} vowel.

\ea  \gll  Kúkːu	g-a-rə́mə́t̪-iə      $[$    (*t̪á)     g-{é}-$^{↓}$ʧə́ð-á                   ugi  $]$	\\
K. \textsc{cl}g-\textsc{rtc}-continue-\textsc{ipfv}  {}  \textsc{comp1} \textsc{cl}g-\textsc{dpc1}-chop-\textsc{ipfv}  \textsc{cl}g.tree {} {}\\
\glt ‘Kuku kept chopping the tree’\label{ex:ch15:tree3}
\ex \gll 	oráŋ   g-a-nː-ó 	            Kúkːu-ŋ     $[$ (*t̪á)   g-{é}-land̪-ó                     ʌwúr  $]$\\
	man    \textsc{cl}g-\textsc{rtc}-hear-\textsc{pfv}  Kuku-\textsc{acc} {} \textsc{comp1}   \textsc{cl}g-\textsc{dpc1}-close-\textsc{prfv}     \textsc{cl}j.door {}\\
\glt‘The man heard Kuku close the door’ \label{ex:ch15:door3}
\z 


A complementizer is prohibited in the embedded clause in these examples, as in subject relative clauses. Neither of the embedded clauses above has an overt subject, and in fact overt subjects are prohibited in these examples.\footnote{The unavailability of an overt lower subject militates against another possible analysis of these facts, as instances of \textit{prolepsis}, or ellipsis of a lower bound argument \citep{davies05}. Thanks to an anonymous reviewer for this point.} Additionally, the argument which is interpreted as the agent of the embedded verb occurs as the subject of the matrix clause in \REF{ex:ch15:tree3} and as the object of the matrix clause \REF{ex:ch15:door3}, as indicated by its accusative case marking.

Despite not allowing an overt subject, these clauses are finite. The most basic evidence comes from the verbal morphology of the embedded verbs, which is identical to that in main clauses, with the exception of the clause type vowel. In addition, the two clauses can receive independent aspect marking:

\ea \gll é-g-a-nː-ó Kúkːu-ŋ g-í-$^{↓}$kíð-íə ʌwur\\
1\textsc{sg}-\textsc{cl}g-\textsc{rtc}-hear-\textsc{pfv} Kuku-\textsc{acc} \textsc{cl}-\textsc{dpc1}-open-\textsc{ipfv} \textsc{cl}j.door\\
\glt `I heard Kuku opening the door.'
\z 

As we will see in the following sections, the perfective/imperfective distinction only occurs in finite clauses in Moro.

While the embedded clause is finite, these clauses still constitute cases of of raising-to-subject \REF{ex:ch15:tree3}, and raising-to-object \REF{ex:ch15:door3}.  This can be shown in that the promoted argument does not receive a semantic (theta) role from the higher verb. For example, both positions allow inanimate nouns such as `water' to occur there:

\ea 
\ea  \gll  ŋáw ŋ-a-rə́mə́t̪-iə ŋ-é-$^{↓}$d̪ə́n-éə\\
\textsc{cl}ŋ.water \textsc{cl}ŋ-\textsc{rtc}-continue-\textsc{ipfv} \textsc{cl}ŋ-\textsc{dpc1}-rain-\textsc{ipfv}\\
\glt `It keeps on raining.'
\ex  \gll  é-g-a-nː-á ŋáw ŋ-é-$^{↓}$d̪ə́n-éə\\
1\textsc{sg}-\textsc{cl}g-\textsc{rtc}-hear-\textsc{ipfv} \textsc{cl}ŋ.water \textsc{cl}ŋ-\textsc{dpc1}-rain-\textsc{ipfv}\\
\glt `I hear it raining.'
 \z 
 \z 
 
Additionally, a passivized argument of the lower clause can undergo raising. This is significant as the raised argument in these cases has already received its theta role from the lower verb. Moreover, the lower verb in this case is an idiom meaning `shoot the gun' (lit: `hit the gun'); the idiomatic meaning is preserved under raising:

\ea 
\ea  \gll  ísːíə j-ʌ-rə́mə́t̪-iə j-í-p-ə́n-íə \\
\textsc{cl}j.gun \textsc{cl}g-\textsc{rtc}-continue-\textsc{ipfv} \textsc{cl}j-\textsc{dpc1}-beat-\textsc{pas}-\textsc{ipfv}\\
\glt `The gun kept being fired.'
\ex  \gll  é-g-a-nː-ó ísːíə j-í-bug-ən-ú\\
1\textsc{sg}-\textsc{cl}g-\textsc{rtc}-hear-\textsc{pfv} \textsc{cl}j.gun \textsc{cl}j-\textsc{dpc1}-hit-\textsc{pas}-\textsc{pfv}\\
\glt  `I heard the gun be shot.'
\z
\z 

Other arguments could be adduced but we take these examples to be conclusive.

The following classes of predicates select finite raising/\textsc{dpc1} complements:

\ea 
\ea  \textbf{Perception (R-t-O):} \textit{-nː-} `hear', \textit{-sʌ́tʃ-} `see,' \textit{-wə́ndat̪-} `watch'
\ex \textbf{Desiderative (R-to-O):} \textit{-bwáɲ-} `want'
\ex \textbf{Modal (R-to-S):} \textit{-ánː-} `seem' 
\ex \textbf{Aspectual (R-to-S):} \textit{-rə́mə́t̪-} `continue'
\z 
\z 

Regular perception verbs allow both finite complements and finite raising complements. This fact itself supports the raising analysis of these clauses, because it demonstrates that the raising predicates themselves do not semantically require a nominal object. However, these verbs can also occur with a nominal object instead of a clausal one, in which case that individual itself is the theme argument.

\section{Infinitive clauses}\label{sec:ch15:infinitives}

\ea \textbf{Infinitival verb template:} \\
\textsc{sagr} - \textsc{om} - \textsc{iter} - $\sqrt{\textsc{root}}$ - \textsc{extension} - \textsc{inf}
\z 


\ea \textbf{Infinitival subject agreement}, Proximal Infinitive 2,  \textit{wəndat̪} `watch'  \citep{rose13}\\
\begin{tabular}[l]{lrr}
\hline
	& \textsc{Singular} 				& \textsc{Plural} \\
\hline
1 	& \textit{ɲe-wə́ndat̪-a} 			&  \textit{ɲa-wəndat̪-a} \\
1+2 	& \textit{ál(ə́)-$^{↓}$wə́ndat̪-a} 		& \textit{ál(ə́)-$^{↓}$wə́ndat̪-a-r} \\
2 	& \textit{a-wə́ndat̪-a} 			&  \textit{ɲa-wə́ndat̪-a}\\
3 	& \textit{áŋ(ə́)-$^{↓}$wə́ndat̪-a} 	&  \textit{alə-wəndat̪-a} \\
\hline
\end{tabular}
\z 


\ea \textbf{Basic morphological distinctions on Moro verb forms}, \textit{wəndat̪} `watch' \citep{rose13}.\footnote{The names of the classes are slightly different in \citep{rose13}, where the \textit{e}-infinitive is ``Subordinate 1'' and the \textit{a}-infinitive is ``Subordinate 2''. Since the finite verb forms above are just as subordinate as those described in this section, we have opted for the term infinitive for reasons described in the previous footnote.}\\
 \begin{tabular}[l]{lll}
\hline
Finite & Perfective & \textit{-wəndat-ó} \\
& Imperfective proximal & \textit{-wə́ndat-a} \\
& Imperfective distal &  \textit{-á-wəndat-ó}\\
\hline
\textit{e}-infinitive & Proximal & \textit{-wə́ndat-e} \\
&  Distal &  \textit{-wə́ndat-a} \\
\hline
\textit{a}-infinitive & Proximal & \textit{-wə́ndat-a} \\
&  Distal &  \textit{-wə́ndat-ó} \\
\hline
 \end{tabular}
\z
 

\ea \textbf{Distribution of infinitival clauses in Moro}
\ea  \textbf{\textit{e}-infinitive (Infinitive 1):} i) Structurally reduced complements of raising-to-subject tense, aspect and modal auxiliaries; ii) \textit{logophoric control}
\ex \textbf{\textit{a}-infinitive (Infinitive 2):} i) Complements of some obligatory \textit{predicative control} verbs; ii) complements of the negative auxiliary
\z 
\z 

\ea \textbf{Summary: Infinitival complements}\\ \label{infsum}
\begin{tabular}[t]{llll}
\hline
\textsc{Infinitive class} & \textsc{Predicate Class} & \textsc{Comp.} & \textsc{Analysis} \\
\hline
\textit{e}-infinitive & TAM & - & Raising-to-subject \\
   & `want', `let' & - & Raising-to-object \\
   & `want', `let,'  & \textit{t̪á} & Object control  \\ 
      & Communication & \textit{t̪á} & Object control  \\ 
\textit{a}-infinitive & Desiderative & \textit{nə́} & Subject control  \\
   & Implicative & \textit{nə́} & Subject control (predicative) \\
   & Negation & - & Raising-to-subject \\ 
\hline
\end{tabular}
\z 

\subsection{\textit{e}-infinitives (Infinitive 1)\label{einf}}

Moro \textit{e}-infinitives occur after three semantically distinct classes of verbs. The first class of verbs which occur with \textit{e}-infinitives is TAM auxiliaries. The \textit{e}-infinitives following TAM auxiliaries do not allow a complementizer:

\ea \ea  \gll  Kúkːu g-íð-á (*t̪a) áŋə́-$^{↓}$ʤóm-{é}\\
Kuku \textsc{cl}g-(\textsc{rtc})-do/will-\textsc{ipfv} {\ \textsc{comp1}} \textsc{3sg.inf}-move-\textsc{inf1}\\
\glt ‘He will move.’
\ex  \gll  Kúkːu g-a-və́l-á (*t̪á) áŋə́-$^{↓}$ʤóm-{é} \\
Kuku \textsc{cl}g-\textsc{rtc}-go-\textsc{ipfv} {\ \textsc{comp1}} \textsc{3sg.inf}-move-\textsc{inf1}\\
\glt ‘Kuku is going to move.'
\z 
\z 

Other members of the TAM auxiliary class include \textit{-tóð-} `start to' (lit.: `move') as well as the modals \textit{-dwadat̪ó} `can,' and \textit{-mantá} `should.'

TAM auxiliaries selecting \textit{e}-infinitives are best analyzed as obligatory raising-to-subject predicates, similar to the \textit{restructuring} predicates studied in \citep{wurm03}.\footnote{Wurmbrand's main evidence for restructuring come from phenomena such as clitic climbing, scrambling, and long passives in Germanic and Romance. We have not been able to replicate any of these tests in the relevant Moro sentences. The absence of clitic climbing might be related to the observation that object marker incorporation or cliticization in Moro seems more phonological than syntactic \citep{jenks15}.} Supporting evidence for this claim comes from the fact that these auxiliaries do not place any semantic restrictions on their subjects, or on the semantics of their complement, as in the stative adjectival predicate in \REF{ex:ch15:camels1}. 

\ea \gll jamala j-a-və́l-á al-oaɲ-ət̪-é\\
\textsc{cl}j.camel \textsc{cl}j-\textsc{rtc}-go-\textsc{ipfv} \textsc{3pl.inf}-many-\textsc{appl}-\textsc{inf1}\\
\glt ‘The camels are going to be more.' \label{ex:ch15:camels1}
\z 

Further evidence that TAM auxiliaries are obligatory raising-to-subject predicates come from passives: example the `hit the gun' idiom test, whose meaning is preserved under passivization and raising:

%todo add example!!?

Interestingly, the intransitive variant of the verb \textit{-bwáɲ-} `want' also occurs in this pattern. Example \REF{ex:ch15:camels1} shows that the matrix subject in this construction does not need to be semantically volitional, as the complement is a comparative adjective (cf. \REF{ex:ch15:camels1}). As the translation indicates, being plentiful is not a state that the matrix subject would be expected to have `control' over in these examples. Thus, these raising-to-subject instances of \textit{-bwáɲ-} are likely quasi-modals comparable to English `need': 

\ea \gll jamala já-j-a-bwáɲ-á al-oaɲ-ət̪-é		\\		
\textsc{cl}j.camel \textsc{pst}-\textsc{cl}j-\textsc{rtc}-want-\textsc{ipfv} \textsc{3pl.inf}-many-\textsc{appl}-\textsc{inf1}\\
\glt ‘The camels were supposed to / needed to be more.’ \label{ex:ch15:camels2}
\z 

The negative auxiliary, which will be discussed in the following section, has an interesting distribution relative to TAM auxiliaries, apparently preferring to come after modal auxiliaries while preceding putative tense and aspectual auxiliaries:

\ea \ea  \gll  é-g-a-nːá ɲ-íð-í ɲe-ndr-é\\
\textsc{1sg-clg}-\textsc{rtc}-neg.\textsc{pfv} \textsc{1sg}-will-\textsc{inf1} \textsc{1sg}-sleep-\textsc{inf1}  \\
\glt `I won’t be sleeping.'\footnote{This example is surprising in that the auxiliary following negation is marked with infinitive 1 rather than infinitive 2. More work is still needed in understanding the distribution of the two infinitives in sequences of auxiliaries.}
\ex  \gll  Kúkːu g-a-bantá áŋ-anːá áŋə́-$^{↓}$və́l-á áŋə́-$^{↓}$ndr-é\\
Kuku \textsc{cl}g-\textsc{rtc}-should  \textsc{3sg}-neg.\textsc{pfv}  \textsc{3sg}-go-\textsc{inf2}  \textsc{3sg}-sleep-\textsc{inf1}  \\
\glt `Kuku should not be about to fall asleep.'
\z 
\z 

This finding indicates that clauses with an auxiliary and an \textit{e}-infinitive can be analyzed as monoclausal \citep[again, cf.][]{wurm03}. These ordering preferences between auxiliaries are reminiscent of the templatic syntactic orderings proposed by \citet{cinque99}.

A second environment for \textit{e}-infinitives is after the transitive variants of \textit{-bwáɲ-} `want' and the periphrastic causative verb \textit{-ŋ́git̪-} `let.' These two predicates are syntactically different from the TAM auxiliaries in that they take an object noun phrase in addition to the infinitival complement:

\ea  \gll 	k-ʌ́-ndʌ-ŋgit̪-iə                    ɲa-tʃə́ð-é         ugi\\
	\textsc{clg}-\textsc{rtc}-\textsc{2plo}-let-\textsc{ipfv}   \textsc{2pl.inf}-chop-\textsc{inf1}  \textsc{cl}g.tree \\
		\glt ‘He is letting you all chop the tree.’ \label{ex:ch15:tree1}
	\ex  \gll  é-g-a-bwáɲ-á ŋáw (*t̪á) áŋə́-$^{↓}$d̪ə́n-é \\
\textsc{1sg-cl-rtc}-want-\textsc{ipfv} \textsc{cl}ŋ.water (\textsc{comp1}) \textsc{3sg.inf}-rain-\textsc{inf1} \\
		\glt `I want it to rain.' \label{ex:ch15:rain1}
	\z 

In \REF{ex:ch15:tree1}, the second person plural object marker is incorporated into the higher verb, demonstrating that it is a syntactic object of that verb. However, the higher object is not necessarily a semantic object of the higher verb, and in fact the most plausible analyses of these predicates is as raising-to-object predicates. For one, \REF{ex:ch15:rain1} shows that the object does not need to be a potential agent. This same example also shows that the complementizer is prohibited in these examples, with an athematic object.

However, with the same predicates, a complementizer is sometimes judged acceptable. The third environment where we find \textit{e}-infinitives, then, is with exactly the same class of predicates but with an animate, potentially agentive object. In these environments, the complementizer is allowed by Moro speakers, but is optional:

\ea 
	\ea	\gll  	é-g-a-bwáɲ-á                Kúkːu-ŋ       (t̪á)       áŋə́-$^{↓}$váð-é     ŋálːo-ŋ\\
\textsc{1sg-clg-rtc}-want-\textsc{ipfv}  Kuku-\textsc{acc} \textsc{comp1}  \textsc{3sg.inf}-shave-\textsc{inf1}   Ngalo-\textsc{acc}\\
		\glt `I want Kuku to shave Ngalo.'
	\ex  \gll  Kúkːu g-ʌ-ŋgít̪-iə ŋálːo-ŋ (t̪á) áŋə́-$^{↓}$noán-é ðamala úlʌlítu  \\
Kuku \textsc{cl}g-\textsc{rtc}-let-\textsc{ipfv}  Ngalo-\textsc{acc} \textsc{comp1}  \textsc{3sg.inf}-watch-\textsc{inf1}   \textsc{cl}.ð.camel  tomorrow\\
		\glt `Kuku is forcing Ngalo to watch the camels tomorrow.'
\z 
\z 

The complementizer plus \textit{e}-infinitives pattern also occurs with transitive \textit{-mwándəð-} `ask', \textit{-ámadat̪-} `help', and \textit{-lúgə́t̪-} `tell':\footnote{The complementizer \textit{nə́-} is sometimes used with these examples instead:

\ea \gll  é-g-amadat̪-ó ŋálːo-ŋ n-áŋə́-$^{↓}$noán-é ðamala	 \\
\textsc{1sg-cl}g-(\textsc{rtc})-help-\textsc{ipfv}  Ngalo-\textsc{acc} \textsc{comp2}- \textsc{3sg.inf}-watch-\textsc{inf1}   \textsc{cl}ð.camel\\
\glt ‘I helped Ngalo to watch the camel.’ (hasn’t happened yet) 
\z

More investigation is needed if any syntactic or semantic differences obtain in these cases.}

\ea
	\ea	\gll é-g-a-mwandəð-ó kúkːu-ŋ t̪á áŋə́-$^{↓}$búg-í ísːiə́   \\
\textsc{1sg-cl}g-\textsc{rtc}-ask-\textsc{pfv}  Kuku-\textsc{acc} \textsc{comp1}  \textsc{3sg.inf}-give-\textsc{inf1}  \textsc{cl}j.gun \\
		\glt ‘I asked Kuku to shoot the gun.’
	\ex  \gll   é-g-amadat̪-ó ŋálːo-ŋ t̪á áŋə́-$^{↓}$pə́g-é ŋóréðá	 \\
\textsc{1sg-cl}g-(\textsc{rtc})-help-\textsc{ipfv}  Ngalo-\textsc{acc} \textsc{comp1} \textsc{3sg.inf}-pick-\textsc{inf1}  \textsc{cl}ŋ.sesame\\
	\glt ‘I helped Ngalo to pick the sesame.’ (hasn’t happened yet) \label{help}
\z 
\z 

We have seen that this same class of predicates occurs with subjunctive No Control complements as well as normal finite complements.

We would like to suggest that the complementizer plus \textit{e}-infinitive pattern comprise cases of obligatory object control, a fact which is supported by the availability  of complementizers. Even stronger support for this conclusion comes from the idiom-chunk test, in which an inanimate, non-agentive object is semantically anomalous (indicated by #) in the object position of the main clause, as it attributes animacy to the gun:

\ea \gll \# é-g-a-mwandəð-ó ísːiə́ t̪á áŋə́-$^{↓}$búg-ə́n-i\\
\textsc{1sg-cl}g-\textsc{rtc}-ask-\textsc{pfv}  \textsc{cl}j.gun \textsc{comp1}  \textsc{3sg.inf}-hit-\textsc{pas}-\textsc{inf1}   \\
\glt ‘I asked the gun to be fired.’
\z 

This finding is unsurprising under an object-control analysis, as the object of the higher verb is its semantic argument as well, and thus, in this example, is required to have the agentive properties typical of control arguments.

\subsection{\textit{a}-infinitives (Infinitive 2)}

The second inflectional category for infinitives, the \textit{a}-infinitive, occurs in two environments: after certain subject control verbs (implicatives and desideratives) and after negation.\footnote{We will see below that \REF{ex:ch15:kukushoots} is problematic for the generalizations about control types. This example is actually ambiguous between a proximal \textit{a}-infinitive and a distal \textit{e}-infinitive. However, as we lack any clear evidence that \REF{ex:ch15:kukushoots} must receive a distal interpretation, and its complementizer matches the other \textit{a}-infinitives, we have left this example in this section.} We will begin with the subject control predicates, which are shown below:

	\ea	\gll 	Kúkːu  g-əndəʧin-ú        (n)-áŋə́-$^{↓}$lə́və́ʧ-{a}              ŋálːo(-ŋ) \\
          	Kuku  \textsc{cl}g-(\textsc{rtc})-try-\textsc{pfv}     \textsc{comp2-3sg.inf}-hide-\textsc{inf2}  Ngalo-\textsc{acc}\\
          	\glt ‘Kuku tried to hide Ngalo.'
	\ex	\gll  Kúkːu	g-a-neð-ó                  (n)-áŋə́-sː-{a}	\\	
		Kuku \textsc{cl}g-\textsc{rtc}-refuse-\textsc{pfv}   \textsc{comp2-3sg.inf}-eat-\textsc{inf2}\\
		\glt ‘Kuku refused to eat.’
	\ex	\gll	 Kúkːu g-a-bwáɲ-á (n)-ʌ́ŋə́-$^{↓}$pʷ-{ʌ́} ísːíə	 \\
		Kuku \textsc{cl}g-\textsc{rtc}-want-\textsc{ipfv}   \textsc{comp2-3sg.inf}-beat-\textsc{inf2} \textsc{cl}j.gun\\
		\glt ‘Kuku wants to fire the gun.’\label{ex:ch15:kukushoots}
\z 

Semantically and syntactically, the behavior of these predicates is as expected for control predicates.  These verbs impose thematic restrictions on their complement, requiring agentive subjects and complements \REF{ex:ch15:touched}. Similarly, idiomatic meanings are not preserved with these predicates \REF{ex:ch15:shot}.\footnote{Other evidence suggests that \textit{-néð-} `refuse' may have a raising variant as well, perhaps one which is semantically distinct in ways which need more study.}

\ea[\#]{  \gll tərbésá ð-a-neð-ó n-ʌ́ŋə́-$^{↓}$bə́r-n-iə	\\
\textsc{cl}ð.table \textsc{cl}ð-\textsc{rtc}-refuse-\textsc{inf2}   \textsc{comp2-3sg.inf}-touch-\textsc{pas-inf2}\\
\glt ‘The table refused to be touched.’} \label{ex:ch15:touched}
\ex[\#]{  \gll ísːíə j-a-$^{↓}$bwáɲ-á n-ʌ́ŋə́-$^{↓}$pw-ə́n-íə \\
\textsc{cl}j.gun \textsc{cl}j-\textsc{rtc}-want-\textsc{ipfv}  \textsc{comp2-3sg.inf}-beat-\textsc{pas-inf2}\\
\glt `The gun wants to be beaten.’ (*`The gun wants to be shot.')} \label{ex:ch15:shot}
\z 
\z 

Another piece of evidence supporting the control analysis of these cases is the availability of a complementizer, in this case the complementizer \textit{nə́-}, which also occurs in non-subject relative clauses (see ex. \ref{nsrc}). The occurrence of a relative clause complementizer after control predicates is compatible with analyses of control which posit predicational semantics for control complements \citep{wil80,chierch84}, particularly via PRO-movement  \citep{clark90}. Under such theories, these complements could be analyzed as CP-sized predicates, like relative clauses, which serve as the internal argument of the control verb.

The second environment where \textit{a}-infinitives occur is after the negative auxiliary verb \textit{ánː}. In these cases, no complementizer occurs:

\ea \gll 	l-anːá               alə-wað-a	\\
\textsc{cl}l-not.\textsc{pfv}   \textsc{3pl.inf}-poke-\textsc{inf2}\\
\glt ‘They did not poke.’ \label{ex:ch15:poke}
\z 

Given that negation is a propositional operator, we expect it to be a raising verb, like the other TAM auxiliaries. 

As expected, there is strong evidence that negation is an obligatory raising-to-subject predicate. For one, complementizers are unavailable under negation, as we have already mentioned: 

\ea \gll Kúkːu g-a-nːá (*t̪á/*nə́-) áŋə́-$^{↓}$tóð-á\\
Kuku \textsc{clg}-\textsc{rtc}-not.\textsc{pfv}   {\ \ \textsc{comp1/2}} \textsc{3sg.inf}-move-\textsc{inf2}\\
\glt ‘Kuku’s not moving.’
\z 

Recall that all of the instances of raising we have seen so far similarly prohibit complementizers.

Additionally, athematic subjects are fine with negation \REF{ex:ch15:raining}, and the negative auxiliary can embed other raising verbs, such as TAM auxiliaries \REF{ex:ch15:dog}:

\ea \gll ŋáw ŋ-a-n:á  áŋə́-$^{↓}$d̪ə́n-é\\
\textsc{cl}ŋ.water \textsc{cl}ŋ-not.\textsc{pfv}   \textsc{3sg.inf}-rain-\textsc{inf1}\\
\glt ‘It's not raining.’ \label{ex:ch15:raining} \footnote{It is not clear why this example has the \textit{e}-infinitive.}
\ex  \ea  \gll  Kúkːu k-án:(a) áŋə́-$^{↓}$və́l-á áŋə́-ndr-$^{↓}$é\\
Kuku \textsc{clg}-not.\textsc{ipfv} \textsc{3sg.inf}-go-\textsc{inf2} \textsc{3sg.inf}-sleep-\textsc{inf1}\\
\glt `Kuku isn't going to fall asleep.'  \label{notgo}
\ex  \gll    é-g-a-nːá i-gíð-í ɲe-bə́ð-é ŋə́ní		\\
\textsc{1sg}-\textsc{cl}g-\textsc{rtc}-not.\textsc{pfv} \textsc{1sg.inf}-will-\textsc{inf1} \textsc{1sg.inf}-pet-\textsc{inf1} \textsc{cl}ŋ.dog\\
\glt ‘I will not pet the dog.’	\label{ex:ch15:dog}
\z 

In summary, then, \textit{a}-infinitives occur after two major classes of predicates. While many of these cases are instances of obligatory subject control, negation also takes an \textit{a}-infinitive complement.



\subsection{Ordering among auxiliaries and verbs?}

Another source of insight about infinitival complements comes from ordering restrictions between them, another type of data which is difficult to gather in the absence of elicitation. These ordering restrictions suggest that the differences between \textit{e}-infinitives versus \textit{a}-infinitives with respect to raising predicates can be reduced to differences in their structural size \citep[cf.][]{wurm03}. The clearest evidence for this claim comes from negation. Notably, all of the different predicates discussed above can embed the negative auxiliary except some TAM predicates, which must follow negation (\REF{ex:ch15:asleep}, cf. \ref{notgo}):

\ea[*]{ \gll Kúkːu g-a-və́l-á áŋ-$^{↓}$ánː-e áŋə́-ndr-$^{↓}$á	 			\\
Kuku \textsc{cl}g-go-\textsc{ipfv} \textsc{3sg.inf}-not-\textsc{inf1} \textsc{3sg.inf}-sleep-\textsc{inf2} \\
\glt (Intended: \glt ‘Kuku is going to not fall asleep.’)} \label{ex:ch15:asleep}
\z 

This indicates that the TAM predicates that must follow negation are lower on the clausal spine than negation.

If we take the negative auxiliary to be relatively high on the clausal spine, at TP or above, we might conjecture that its complement is always realized as a \textit{a}-infinitive, and hence, that an  \textit{a}-infinitive is smaller than TP. In contrast, the complement of the aspectual auxiliaries, which can be analyzed as verbal  heads (V/$v$)\footnote{In general, the final vowels of the relevant forms can be analyzed as $v$ heads; see \citet{jenks15} for phonological evidence that the $v$P is a phonological domain.} are always in the form of \textit{e}-infinitives. These verbal heads would also be able to select each other recursively, subject to further ordering constraints, resulting in multiple \textit{e}-infinitives. Under this view, the unavailability of negation as complement of some TAM auxiliaries follows from the general requirement that TP always occur above VP within a single clause \citep[cf.][]{cinque99}.\footnote{We might further conjecture that in the absence of a higher auxiliary like negation, the lexical verb would occur in T (by head movement or some equivalent) and take finite morphology.}

Supporting evidence for this proposal comes from the distribution of finite agreement, which occurs on negation after \textit{a}-infinitive-selecting control predicates.

\ea  
	\ea  \gll  Kúkːu g-ʌ-v-ə́ndətʃən-iə g-án:a áŋə́-ndr-$^{↓}$á		\\
Kuku \textsc{cl}g-\textsc{rtc}-\textsc{ipfv}-try-\textsc{ipfv} \textsc{cl}g-not.\textsc{impf} \textsc{3sg.inf}-sleep-\textsc{inf2}\\
\glt ‘Kuku is trying to not fall asleep.’
\ex  \gll  *Kúkːu g-ʌ-v-ə́ndətʃin-iə n-áŋ-$^{↓}$ánːa áŋə́-ndr-$^{↓}$á	 		\\
Kuku \textsc{cl}g-\textsc{rtc}-\textsc{ipfv}-try-\textsc{ipfv} \textsc{comp2}-\textsc{3sg.inf}-not.\textsc{impf} \textsc{3sg.inf}-sleep-\textsc{inf2}\\
\z 
\z 

If finite agreement always occurs on T, and negation is always in T, accounting for its ability to take finite agreement, an explanation is available for why finite agreement and negation are always correlated. When \textit{a}-infinitival complements occur as the complement of these control verbs, they would be occupying lower positions on the clausal spine, staying in their VP position, but still the direct complement of T. Thus, agreement on infinitives might be the realization of agreement on a V head, rather than T, explaining why multiple instances of agreement sometimes occur in putatively monoclausal structures, such as with negative or aspectual auxiliaries.

In this light consider the example below where modals occur above negation, as we saw above:

\ea \gll Kúkːu g-a-mantá áŋə́-nːá áŋə́-$^{↓}$və́l-á áŋə́-ndr-é\\
Kuku \textsc{cl}g-\textsc{rtc}-should  \textsc{3sg.inf}-neg.\textsc{prfv}  \textsc{3sg.inf}-go-\textsc{inf2}  \textsc{3sg.inf}-sleep-\textsc{inf1}  \\
\glt `Kuku should not be about to fall asleep.' \label{ex:ch15:asleep2}
\z 

We can now analyze this example as follows: both modals and negation are types of T heads. When in the same clause, only one element can realize the finite T head, the modal in \REF{ex:ch15:asleep2}, resulting in negation occupying a lower head below T, hence realizing an \textit{a}-infinitive form. Likewise, the fact that the inchoative auxiliary \textit{və́l} also occurs with the \textit{a}-infinitive corresponds to the fact that it is the complement of negation. However, the lexical verb which is the complement to the inchoative auxiliary emerges with the \textit{e}-infinitive, because it is the complement to a higher V head, and thus is the structurally smallest of the three.

The diagram below summarizes the distribution of the two infinitives according to this theory:

\ea  CP - $\underbrace{{$T(Mod)$} - \underbrace{$T(Neg) $-$V(Asp)$ - \overbrace{$VP$}^\text{\textit{e-}infinitive}}_\text{\textit{a-}infinitive}}_{finite\  clause}$
\z 

Again, in the absence of a higher head, the lexical verb or any higher auxiliary will simply move to the highest position and take the regular finite morphology. Crucially, though, this does not affect the size of its complement, which must stay in the low position. Additionally, agreement is realized on each verbal within a single clause. To summarize, then, we can see that the ordering diagnostics make sense of the distribution of the raising predicates in \ref{infsum} by virtue of attributing them different `sized' complements, a result that has been clearly established for Germanic and Romance languages \citep{wurm03}.

