\chapter{Relative clauses and clefts}\label{chapter:relative}




%TODO data to check
a) pronoun clefts --- is there any more to be said?
b) id complementizer in clefts: does it agree with the pivot?


%TODO add examples from section 4.1 example 19-23
\ea
\ea	  Subject Question \\
\gll 	ŋwɜ́ʤɜ́kːi	ɡ-é-mː-ó	ówːá	ɡ-óal-á?\\
		\textsc{cl}g.who	\textsc{sm.cl}g-\textsc{dpc1}-take-\textsc{pfv}	\textsc{cl}g.woman	\textsc{sm.cl}g-tall-\textsc{adj}\\
\trans	‘Who married the tall woman?’\\
\ex	Subject Focus\\
\gll	ŋwə́-matʃə́-kːi	ɡ-é-mː-ó	ówːá	ɡ-óal-á  \\
	\textsc{cl}f-\textsc{cl}g.man-\textsc{cl}g.\textsc{dem}	\textsc{sm.cl}g-\textsc{dpc1}-take-\textsc{pfv}	\textsc{cl}g.woman	\textsc{sm.cl}g-tall-\textsc{adj}\\    
\trans		‘This is the man who married the tall woman’\\
\z
\z

\ea	 	Subject Relative\\
\gll	matʃə́-kːi	ɡ-é-mː-ó	ówːá	ɡ-óal-á  	ɡ-ɜnd-ú	oɡómá\\       
	\textsc{cl}g.man-\textsc{cl}g.\textsc{dem}	\textsc{sm.cl}g-\textsc{dpc1}-take-\textsc{pfv}	\textsc{cl}g-woman	\textsc{\textsc{sm.cl}}g-tall-\textsc{adj}    
	\textsc{sm.cl}g-catch-\textsc{pfv}	\textsc{cl}g.thief\\
\trans	‘The man who married the tall woman caught the thief’\\
\z

\ea	Declarative\\
\gll	matʃə́-kːi	ɡ-a-mː-ó	ówːá	ɡ-óal-á  \\
	\textsc{cl}g.man-\textsc{cl}g.\textsc{dem}	\textsc{sm.cl}g-\textsc{rtc}-take-\textsc{pfv}	\textsc{cl}g.woman	\textsc{cl}g-tall-\textsc{adj}\\
\trans	‘This man married the tall woman’\\
\z

\ea
\ea	Object Question\\
\gll	ŋwɜ́ndə́kːi	(n-)úʤí	(nə́-)ɡ-ə́-wəndat̪-ó?\\                        
	\textsc{cl}g.what	(\textsc{comp-})\textsc{cl}g.person	(\textsc{comp-})\textsc{sm.cl}g-\textsc{dpc2}-see-\textsc{pfv} \\  
\trans		‘What did the person see?’\\
\ex	Object Focus\\
\gll ŋw-óɡovél-kːi	(n-)úʤí (nə́-)ɡ-ə́-wəndat̪-ó \\
\textsc{cl}f-\textsc{cl}g.monkey-\textsc{cl}g.\textsc{dem}	(\textsc{comp-})\textsc{cl}g.person (\textsc{comp-})\textsc{sm.cl}g-\textsc{dpc2}-see-\textsc{pfv}   \\
\trans		‘This is the monkey that the person saw’\\
\ex	Object Relative\\
\gll oɡovél-kːi	(n-)úʤí		(nə́-)ɡ-ə́-wəndat̪-ó	ɡ-obəð-ó	\\
\textsc{cl}g.monkey-\textsc{cl}g.\textsc{dem}	(\textsc{comp-})\textsc{cl}g.person (\textsc{comp-})\textsc{sm.cl}g-\textsc{dpc2}-see-\textsc{pfv}	\textsc{sm.cl}g-run-\textsc{pfv}\\
\trans		‘The monkey that the tall person saw ran away’\\
\z
\z

\ea	 	Declarative\\
\gll	uʤí	ɡ-a-wəndat̪-ó	oɡovél-kːi    \\
 \textsc{cl}g.person	\textsc{sm.cl}g-\textsc{rtc}-see-\textsc{pfv}	\textsc{cl}g.monkey-\textsc{cl}g.\textsc{dem}\\
\trans	‘The person saw this monkey’\\
\z

\section{Relative clauses}\label{relativeclause}

While the complementizer \textit{nə́=} is usually optional, it becomes obligatory in some object relatives forms involving a pronominal subject:

%TODO MAKE EXAMPLE BELOW INTO TABLE

\ea
\begin{tabular}[t]{clll}
	ðamalɜ́-ðː-  &    *(n)=í-sɜtʃ-ú	&-1\textsc{sg}-		&‘the camel that I saw’\\
 		    …	&     *(n)=ɜ́-sɜtʃ-ú	&-2\textsc{sg}-		&‘…that you saw’\\
		    …	&  (nә́)=ɡ-ə́-sɜtʃ-ú	&-3\textsc{sg}-		&‘…that she saw’\\
 		    …	&  *(n)=ɜ́lə́-sɜtʃ-ú	&-1\textsc{du.in}-	&‘…that you and I saw’\\	
		    …	&  *(n)=ɜ́lə́-sɜtʃ-ú-r	&-1\textsc{pl.in}-	&‘…that we (\textsc{incl}.) saw’\\
		    …	&   (nә́)=ɲɜ́-sɜtʃ-ú	&-1\textsc{pl.ex}- 	&‘…that we (\textsc{excl}.) saw’\\
		    …	&   (nә́)=ɲɜ́-sɜtʃ-ú	&-2\textsc{pl} - 		&‘…that you (\textsc{pl}.) saw’\\
		    …	& (*nә́)=l-ə́-sɜtʃ-ú 	&-3\textsc{pl}-		&‘…that they saw’\\
\end{tabular}
\z
Following standard practice, the asterisk outside of parentheses indicates obligatoriness while the asterisk inside of parentheses indicates that the clitic is disallowed. These examples illustrate that the varying optionality of nə́= is phonologically conditioned: \textit{nə́=} is obligatory with vowel-initial prefixes and impossible before /l/ in 3rd person plural forms due to a /*nl/ co-occurrence constraint in Moro, apparently also active across schwa (cf. the numerals 12 and 13 in Table 8; see also Gibbard et al 2009, p. 113).

The forms in TABLE XXX also illustrate a distinction between object relative clauses on the one hand (and other embedded verbs taking the \textit{ə́}-prefix) and root clauses and subject relatives on the other in that object relative clauses lack the “extra” ɡ-class marker which occurs between first and second person prefixes and the clause-typing vowel (e.g. (23-24)).

\section{Topicalization}