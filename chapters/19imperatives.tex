\chapter{Imperatives}\label{chapter:imperative}

%TODO add discussion of jussive here

Thetogovela Moro has two kinds of imperatives, the proximal/itive imperative and the distal/ventive imperative. The proximal imperative is used for actions that are near to the speaker, or indicate motion away from the speaker. The distal/ventive is used for actions that are far away from the speaker or indicate motion towards the speaker. It can also be used to indicate emotional distance or uninvolvement in the action. We will use the terms ‘proximal’ and ‘distal’ to refer to these forms from now on. See section X for more discussion of the distinction. The proximal form is more common that the distal, and is the form used when location or motion is unexpressed. This type of ‘deictic’ distinction is also found in the imperfective, and in some of the subordinate constructions (infinitives, consecutive perfective), but not in the main clause perfective. See XXX for details. The distinction is found in other Kordofanian languages such as Koalib (Quint 2006) where it is labeled centripetal/centrifugal, as well as Nilo-Saharan languages, particularly Nilotic languages (Dimmendaal 2003). 
%TODO XXX 

The two forms are differentiated by the final vowel and by the tone pattern:

\ea
\begin{multicols}{2}
\gll və́léð-ó\\
pull-\textsc{prox.imp}\\
\trans “pull!”\\
 
\columnbreak 
\gll vəleð-a\\
pull-\textsc{dist.imp}\\
\trans “pull (from there to here)!”\\
\end{multicols}	
\z

\subsection{Proximal imperative}
The proximal imperative is formed from the verb root and a final suffix \textit{–ó}. All tone-bearing units in the verb root bear high tone:

\ea Consonant-initial verb roots\\
\begin{tabular}[t]{ll}
və́léð-ó				&	“pull!”\\
tə́ŋát̪-ó				&	“lick!”\\
pə́ɡə́ðó				&	“pay!”\\
váð-ó				&	“shave!”\\
ɡə́ɲ-ó				&	“kill!”\\
lánd-ó				&	“close!”\\
\end{tabular}
\z

\ea Vowel-initial verb roots\\
\begin{tabular}[t]{ll}
ábə́ɾ-ó		&	“fly!”\\
ódə́ɲ-ó		&	“squat!”\\
ámádát̪-ó	&	“help!”\\
áɾ-ó		&	“cry!”\\
áp-ó		&	“carry, pick up!”\\
	
\end{tabular}
\z

Verb roots with high vowels /i ɜ u/ and some /ə/, cause the final suffix to be realized as raised [ú]:

\ea High vowel roots\\
\begin{tabular}[t]{ll}
sɜ́ð-ú	&		“defecate!”\\
kíð-ú	&		“open!”\\
íð-ú	&		“make!”\\
də́ɾ-ú	&		“stop, stand!”\\
túnd-ú	&		“cough!”\\
ɜ́ŋɜ́ʧ-ú	&		show!”\\
ílíð-ú	&		“buy!”\\
ɜ́wút-ú	&		“throw!”\\
mə́ɲɜ́ʧ-ú	&		“peel, remove layer!”\\
\end{tabular}
\z

Verb roots consisting only of a consonant show two stategies for forming the imperative. Sonorant-initial roots have a geminate consonant, with the first half of the geminate functioning as the first tone-bearing syllable. Obstruent-initial roots have an epenthetic [ə] preceding the root in the imperative:

\ea C roots\\
\begin{tabular}[t]{ll}
m-ó		&	“take, marry!”\\
ŕ̩r-ó	&	“kick (once), pound, stab!”\\
s-ó		&	“eat!”\\
ə́p-ú		&	“beat!”\\
t̪-ú		&	“drink!”\\
\end{tabular}
\z

Verb roots that begin with a diphthong [oa] (or [wa]) have reduction of the diphthong to [a] in the imperative:

\ea Diphthong-initial roots\\
\begin{tabular}[t]{llll}
ás-ó	&	“wash!” \hspace{35pt} cf.&   	k-oása 	&		“he is washing”\\
áð-ó	&	“grind!”&	k-oaða		&		“he is grinding flour”\\
ár-ó	&	“curse, badmouth!”	&	k-oara	&		“he is cursing”\\
ándət-ó	&	“dry!”	&	k-oándəta	&	“it is drying”\\
\end{tabular}
\z

Some verb roots begin with [wə] sequences when prefixed, but in the imperatives this sequence is realized as [u] instead:

\begin{tabular}[t]{lllll}
úɾ-ú	&	“dig!”  &	 cf.&	kɜwə́ɾɜ́	&	“he is digging”\\
úd-ú 	&	“fry!”	&		&	kɜwə́dɜ́	&	“he is frying”\\
\end{tabular}

% TODO George thinks the imperfectives are kɜ-ud-ɜ.  

The verb kavə́lá ‘to go’ has a suppletive form of the imperative: ḿbú. 

\subsection{Distal imperative}
The distal imperative is formed from the verb root and the suffix \textit{–a}. This vowel is raised to [ɜ] when attached to roots with high vowels. The verb form is low-toned:

\ea
\begin{tabular}[t]{ll}
vəleð-a	&	“pull!”\\
ɡəɲ-a	&	“kill!”\\
land-a	&	“close!”\\
abəɾw-a	&	“fly!”\\
ap-a	&	“carry, pick up!”\\
kið-ɜ	&	“open!”\\
ilið-ɜ	&	“buy!”\\
ɜwut-ɜ	&	“throw!”\\
əs-a	&	“eat!”\\
əp-ɜ	&	“beat!”\\
að-a	&	“grind!”\\
uɾ-ɜ	&	“dig!” \\
\end{tabular}
\z
% TODO abəɾw-a was highlighted

However, when followed by a noun phrase, an object marker, or a locative or instrumental marker, the final suffix vowel has high tone:

% TODO "a locative" was highlighted
% Is this also true when followed by adverbs?

A similar pattern is found with (proximal) imperfective verbs that are low-toned in utterance final position. 
% TODO Check and see if this is true of –lo.

\subsection{Plural}
The imperative plural is expressed by the addition of the suffix \textit{-r}. This is added to both the proximal and distal imperatives:

% TODO EXAMPLES

\subsection{Use of the imperatives}
SAY MORE on the distal/proximal distinction
