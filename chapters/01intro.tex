\chapter{Introduction}\label{chapter:introduction}


Moro is a language spoken in the Moro Hills in the Nuba Mountains region of South Kordofan State (in Arabic,  Janub Kurdofan) in the Republic of Sudan. The Nuba Mountains are shown on the following map of Sudan, in the south-central region, north of the South Sudan border. 

\begin{figure}
  \includegraphics[width=\linewidth]{figures/Map-of-Sudan-UN.jpg}
    \caption{Map of Republic of Sudan}
  \label{fig:1-1}
\end{figure}

The Moro hills (Jebel Moro in Arabic) are located in the southern, central area of the Nuba Mountains,  approximately 50km east of Kadugli. There are numerous Moro-speaking villages in the hills. \citep{nadel47} reports that the ancestral home of the Moro was on Lebu Hill in the western massif of the Moro area. Subsequent migrations were to the north and east of the massif.

\begin{figure}
  \includegraphics[width=\linewidth]{figures/Morolocation.png}
    \caption{Moro Hills (map created from glottolog.org)}
  \label{fig:1-1}
\end{figure}

The Nuba Mountains is the most linguistically dense areas in Sudan. Estimates are that between forty and fifty languages are spoken in this area, depending on language and dialect divisions (Schadeberg and Blench 2013). The languages are generally classified as belonging to one of three main phyla: Nilo-Saharan, Kadu and Niger-Congo. Moro is classified as a Kordofanian language, part of the Niger-Congo phylum. A map of Nuba Mountain languages is provided below:

\begin{figure}
  \includegraphics[width=\linewidth]{figures/NubaMountainLanguages.png}
    \caption{Map of Nuba Mountain Languages (map available at https://commons.wikimedia.org/}
  \label{fig:1-1}
\end{figure}

The Glottolog code for Moro is moro1285, its Ethnologue code is mor and its ISO code is ISO 639-3.  Schadeberg (1981) splits Kordofanian into four main groups: Heiban, Talodi, Rashad and Katla. Further research on the Katla languages (Katla, Tima, Julud) suggest that they are quite distinct from Talodi and Heiban languages (Dimmendaal 2009, Hammarström 2013). Talodi and Heiban may form a group to the exclusion of Rashad, as proposed by Dimmendaal (2015). New research on Rashad languages being conducted at the U. of Khartoum may shed more light on this question. Moro is part of the Heiban group, and is further classified as West Heiban along with the Tira language (Schadeberg 1981). Heiban languages are divided into three branches: West, Central and East, which refer to geographical locations relative to one another. The other Heiban languages include the Central languages, Koalib (Rere), Logol, Laru (Laro), Otoro, Ebang, and Shwai (Shirumba), and the East languages, Ko and Warnang. A chart of the Heiban sub-groupings is given below based on Schadeberg (1981):

\begin{figure}
  \includegraphics[width=\linewidth]{figures/Heiban.pdf}
    \caption{Heiban language sub-groupings}
  \label{fig:1-2}
\end{figure}


Geographically, Moro is the most south western of the Heiban languages. It is geographically close to Tocho, Lumun and Acheron (Asheron or Aceron), all Talodi languages, and Caning (Shatt), which is classified as a Daju (Eastern Sudanic, Nilo-Saharan) language. 

\begin{figure}
  \includegraphics[width=\linewidth]{figures/fig-Heibanmap.png}
    \caption{Map of Heiban languages (map created from glottolog.org)}
  \label{fig:1-3}
\end{figure}

Moro is reported to have seven dialects corresponding to ethnic clan divisions (Naser 2013). Although there are seven dialects, according to Blench (2005), based on information provided by Angelo Naser, there are only six clans. A document Angelo wrote under the name Angelo Ali for the SIL linguist, Elizabeth Guest, also confirms six clans. Naser (2013) provides the following list of dialect names: Lətogəfəlda, Uləba, Nubwa, Ndərria, Layene, Wërria, and Ləŋorban, with the last two corresponding to the Ləŋorban clan, and these may be the same dialect. On both Ethnologue and in Blench (2015), the dialects are listed as Umm Gabralla (Toberelda), Ulba, Nubwa, Nderre, Laiyen, Werria, and Umm Dorein (Longorban). 


The standard dialect is that known as Longorban or Werria. This is the dialect that is the subject of the only previous grammar of Moro, The Moro Language Grammar and Dictionary (1971) by Black \& Black. Note that both names are listed as separate dialects in the table above. Our consultants usually use the term Werria to refer to it. Ləŋorban is the dialect that was used in the original New Testament translation. According to information reported in 1997 by the SIL linguist Elizabeth Guest (\verb http://www.rogerblench.info/Language/Niger-Congo/Kordofanian/Moro/guest_moro-nt-history1997.pdf), speakers of other dialects had difficulty understanding the original New Testament, and it was subsequently revised, completed in 1993. The current methodology employed by the Moro Language Committee is for representatives of all the dialects to meet and develop a consensus on appropriate words. This newly developed standard is used in primers and other literacy materials, and especially in Christian religious worship. See Naser (2013) and Lamoureaux (2017) for more information. 

The dialect described in this book is \textit{ðət̪ogovə́lá} or in the Moro writing system, Đət̠ogovəla. This is the dialect identified as Toberelda, tobəɽelda, Lət̪opəɽelda, or Lətogəfəlda, which may represent the pronunciation of this dialect’s name in the standard dialect. The sound [ɽ] in Werria frequently corresponds to [g] in Thetogovela, and the b/p often corresponds to [v]. However, note that these two sounds appear to have switched order. The Moro language is referred to as \textit{ðəmwaɾə́ŋá}. \textit{ð(ə)-} is a noun class marker. Moro is the Arabic word for the language, presumably derived from this designation. However, the term Moro is not viewed as pejorative. \textit{ð(ə)-} is replaced with \textit{o-} or \textit{u-}  for a single Moro person, and \textit{l(ə)-} for plural. Hence, the \textit{lə-} in the names above is probably the noun class marker:

\begin{exe}
\ex Moro Thetogovela endonyms
	\begin{xlist}
		\ex	ðəmwaɾə́ŋá	'Moro language'
		\ex omwaɾə́ŋá	'Moro person'
		\ex ləmwaɾə́ŋá	'Moro people'
	\end{xlist}
\end{exe}

The names of the dialects and people who speak the dialect (as pronounced in Thetogovela) are provided below: 

\ea  
\begin{supertabular}[t]{llll}
 
&	Dialect	&	Person	&	People 		\\
\midrule
a.	&	ðəlbwa		&	ulbɜ	&	ulbɜ\\
b.	& 	ðənəwːɜ		& 	unuwːɜ	& 	(lə)nuwːɜ	 \\
c.	&	ðəndəriə	& 	oldəriə	& 	ndəriə	 \\
d.	&	ðaiɲa		&	oaiɲ	&	laiɲ \\
e.	&	ðəwɜriə		&	wɜriə 	&	wɜriə\\
f.	&	ðəŋorban(a)	&	oŋorban(a)	&	ləŋorban(a)	 \\
g.	&	ðət̪ogovə́lá	&	ot̪ogovə́lá	 &	lət̪ogovə́lá\\
\end{supertabular}\
\z 

We recognize the importance of the standard written variety to the Moro people, particularly those who are displaced to Omdurman/Khartoum. The standard written form unifies the Moro language and people, and provides evidence of the continued vitality of the language faced with ongoing pressure from Arabic. This book is not an attempt to undermine or replace this work. Our goal is to document the grammatical properties and spoken characteristics of one spoken variety of the language. Our hope is that this may inspire others to work on other varieties to gain a full picture of the complexities of the Moro language as a whole.  
	
\section{Methods and Data collection}
The data collected for this book have been amassed over the course of a decade, from 2005-2018, beginning with a graduate field methods class held at the University of California, San Diego. The main consultant for this class was Mr. Elyasir Afsos Julima, and he has been our primary consultant ever since. Data were also provided by Elyasir’s wife, Ms. Ikhlas Elahmer, over the course of these years, and by Mr. Angelo Naser Kuku, the current head of the Moro Language Committee, who visited San Diego in June-July of 2013 and Berkeley and San Diego in X of 2015. Extensive fieldwork in the Nuba Mountains was not possible due to instability and war returning to the Nuba Mountains, but also due to our own personal circumstances which prevented travel to Khartoum for any reasonable length of time. Rose was able to travel to Khartoum in 2017, and some of the texts collected on that trip with Moro speakers in Omdurman  are included in the book. This methodology has obvious drawbacks in that the speakers were displaced from the language area, and we were limited in the number of speakers which whom we could conduct research. On the other hand, it also had advantages due to availability of the speakers and the ability to cross-check data regularly. 

Elyasir Julima was born in approximately 1968, and comes from the village of Karakaray-Al Byeara. He is a member of the X clan. He was raised in Omdurman rather than the Nuba Mountains due to war-induced displacement and the death of his mother when he was a small child. Nevertheless, there is a sizeable Moro population in Omdurman, and his primary caregiver during childhood was his paternal grandmother, who was monolingual in Moro. Elyasir also made frequent trips to the Nuba Mountains as a child and spent summers with his uncles in Karakaray. Elaysir is also fluent in Sudanese Arabic and English. He left Sudan in 19XX, spent two years in Cairo, Egypt and then arrived in San Diego, California in 20XX. 


Ikhlas Elahmer is approx. 42 years old. She was born and raised in the village of XXXX and is a member of the X clan. After the town was attacked during the civil war when she was approx. nine years old, she and her family moved to Khartoum. She did not reside in a Moro speaking neighbourhood in Khartoum, but still maintained the language. She left Sudan at age 20, spent two years in Cairo, Egypt and arrived in San Diego, California in 20XX. She is fluent in Sudanese Arabic and English. Elyasir and Ikhlas converse with each other primarily in Moro and Arabic, with English added when necessary. They speak to their children in Arabic and English. Although they used to speak to the children in Moro, the children never gained fluency, and are now dominant in English. 

Angelo Kuku Naser is approx. 57 years old. He was born in the village of X and lived in Karakarai for about 20 years. Although his parents were Kain, he grew up speaking Thetogovela Moro. He now resides in Omdurman, Sudan, and uses Moro on a daily basis with his family and other speakers. He worked for the Bible Society for a number of years translating the Old Testament into Moro. He has also been active in preparing literacy materials and teaching Moro as a member of the Moro Language Committee. He is familiar with the other dialects of Moro, as his family is originally Moro Kain, and he has worked extensively with the written standard based on Werria. 

In addition to these speakers, we also recorded data from Mr. Israel Aldelong, which was used to analyze the vowel system. Several speakers residing in Sudan recited stories or short pieces which are included in the text collection. We  acknowledge the assistance of Safa Ashmek Madwa Angalo, who assisted in the translation of the stories. 

\begin{figure}
  \includegraphics[width=\linewidth]{figures/Villagesmap.png}
    \caption{Map of Place Names (UN map)}
  \label{fig:1-1}
\end{figure}

There is no reliable estimate of the number of Moro speakers. In 1955-56, the Moro population was 28,311 based on the Sudan Census. Ethnologue lists the number in 1982 as 30,000 based on an SIL survey. The Second Sudanese Civil War lasted from 1983-2005, and thousands of people were killed or displaced from the area. Given these issues, it is hard to ascertain overall numbers of speakers. The 2008 Sudan Census lists around 102,000 people in the Kadugli area, including Um Dorein. However, this is a large area that includes other people besides Moro, and the census may not be reliable. Moreover, war broke out again in 2011 and is still ongoing. In the current conflict, many Nuba people have been killed, and many displaced, this time across the border into South Sudan. As a result, Mous (1998) classifies the entire Kordofanian family as endangered, ‘partly by genocide’ and Williamson \& Blench (2000) state that Kordofanian speakers ‘have been displaced through political insecurity and their status is now uncertain’. The Moro are one of the larger populations in the Nuba Mountains; this was reported as early as Nadel (1947), and repeated by other researchers such as Guest (19XX), and Moro people themselves. The large numbers have surely contributed to the maintenance of the language even among displaced populations in the Omdurman/Khartoum area, despite the pressure of Arabic. However, the almost continuous civil strife in the Nuba Mountains has rendered the language threatened at the very least. 


\section{Previous literature}
Stevenson (1956-57) includes information on Moro. 
Until recently, Moro was the only Kordofanian language for which there was a grammar. Keith and Betty Black wrote a 1971 grammar and glossary, \textit{The Moro Language and Dictionary}. The grammar is based on 12 months of fieldwork on Moro, and is primarily based on the speech of one main consultant, identified as Kapiroma (or Kapirma). It describes the Umdorein dialect, which seems to be the Werria or Loŋorban dialect. The grammar contains some important information, but it contains no data with tone, since the Blacks did not identify Moro as a tone language. 
Thilo Schadeberg's 1981 overview work on Kordofanian contains some wordlists from Moro.
Elizabeth Guest worked for SIL in the 1990s in Khartoum, and created wordlists, dialect information, and brief descriptions of Moro. Although they were not published, these works are available online and provide a good deal of information on verbs and their basic forms. 
Our own research began in the mid-2000s, and we and our colleagues have published a number of papers that describe and analyze aspects of the Thetogovela dialect of the Moro language, ranging from phonetic and phonological analyses to syntax. Some of the papers are descriptive in nature, and some theoretical. The main members of our team included Farrell Ackerman, George Gibbard, Laura Kertz, Andrew Strabone and Hannah Rohde. Papers published by members of this team are included in the bibliography. Our research was funded by the National Science Foundation. 

The Moro Language Committee has produced a number of basic primers and other educational materials written entirely in Moro. An overview of their work is described in Naser (2013). Recently, Angelo Naser and other members of the Moro Language Committee have been teaching Moro people in Omdurman to read and write, and have been encouraging them to write stories. A collection of these stories was published locally in 2018. In addition, Angelo collaborated with Peter Jenks to produce the Moro story corpus, a searchable online repository of written stories with English glossing and translation (http://linguistics.berkeley.edu/moro/\#/)

\section{Acknowledgments}
First and foremost, we thank all the Moro who helped us with understanding their language over many years. In particular, Elyasir Julima was indispensable and incredibly patient, steady, and consistent. Angelo Naser's dedication to the Moro language and its survival has provided the Moro in exile with a strong sense of community. He has rich knowledge and is committed to encouraging Moro to speak, read and write their language. We especially appreciate his willingless to travel to the US, and to work with us to produce this grammar. Ikhlas Elahmer welcomed us into her home countless times. We thank her for her humor, her delicious food, and her friendship. We also thank all their children whom we have watched grow up. 

%\section{About literature}
%\citet{Comrie1981} is a useful introduction to typology \is{typology}. %\is = index of subjects
%It deals with languages of the whole world, not restricting itself to \ili{Indo-European languages}. %\il = index of languages
%\iai{Dionysios Thrax} was also an important figure. %\ia = index of authors. Not necessary for authors whose work you cite.
%
%\ea\label{ex:1:descartes}
%\langinfo{Latin}{}{personal knowledge}\\
%\gll cogit-o ergo sum \\
%     think-1{\sg}.{\prs}.{\ind} hence exist.1{\sg}.{\prs}.{\ind}\\
%\glt `I think therefore I am'
%\z
%
%\begin{table}
%\caption{Frequencies of word classes}
%\label{tab:1:frequencies}
% \begin{tabular}{lllll} % add l for every additional column or remove as necessary
%  \lsptoprule
%            & nouns & verbs & adjectives & adverbs\\ %table header
%  \midrule
%  absolute  &   12 &    34  &    23     & 13\\
%  relative  &   3.1 &   8.9 &    5.7    & 3.2\\
%  \lspbottomrule
% \end{tabular}
%\end{table}
%
%
%
%\begin{figure}
%\caption{Some XML}
%\begin{lstlisting}
%<LogEvent Action = "1" Value = "107" Cursor = "477">
%<LogEvent Action = "1" Value = "107" Cursor = "477">
%<LogEvent Action = "1" Value = "107" Cursor = "477">
%<LogEvent Action = "1" Value = "107" Cursor = "477">
%\end{lstlisting}
%\end{figure}
