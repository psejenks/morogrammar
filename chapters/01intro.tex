\chapter{Introduction}\label{chapter:introduction}

Moro is a language spoken in the Nuba Mountains region of South Kordofan State in the Republic of Sudan. There are numerous Moro-speaking villages and the main town is Umm Dorein. \citep{nadel47} reports that the ancestral home of the Moro was on Lebu Hill in the western massif of the Moro area. Subsequent migrations were to the north and east of the massif.

INSERT MAP

Moro is classified as a Kordofanian language, part of the Niger-Congo phylum. Schadeberg (1981) splits Kordofanian into four main groups. Moro is part of the Heiban group, and is further classified as West Heiban along with the Tira language. The West, Central and East terms refer to geographical locations relative to one another. 

INSERT FIGURE

Geographically, Moro borders Asheron and XXX

Moro is reported to have seven dialects, according to Ethnologue (XXX) and Blench (2005), corresponding to ethnic clan divisions. However, a document written by Angelo Ali for the SIL linguist, Elizabeth Guest, lists six clans.

The standard dialect is that spoken in the town of Umm Dorein, known as Longorban or Werria. This is the dialect that is the subject of the only previous grammar of Moro, The Moro Language Grammar and Dictionary (1971) by Black \& Black. Note that both names are listed as separate dialects in the table above. Our consultants usually use the term Werria to refer to it. Ləŋorban is the dialect that was used in the original New Testament translation. According to information reported in 1997 by Elizabeth Guest (\verb http://www.rogerblench.info/Language/Niger-Congo/Kordofanian/Moro/guest_moro-nt-history1997.pdf), speakers of other dialects had difficulty understanding the original New Testament, and it was subsequently revised, completed in 1993. The current methodology employed by the Moro Language Committee is for representatives of all the dialects to meet and develop a consensus on appropriate words. This newly developed ‘standard’ is used in primers and other literacy materials. SEE ANGELO’S ARTICLE in NML for more info. 

The dialect described in this book is \textit{ðət̪ogovə́lá} or in the Moro writing system, Đət̠ogovəla. This is the dialect identified as Toberelda, tobəɽelda or Lət̪opəɽelda which may represent the pronunciation of this dialect’s name in the standard dialect. The sound ɽ in Werria frequently corresponds to [g] in this dialect, and the b/p often corresponds to [v]. However, note that these two sounds appear to have switched order. The Moro language is referred to as \textit{ðəmwaɾə́ŋá}. \textit{ð-} is a noun class marker. Moro is the Arabic word for the language. However, it is not viewed as pejorative. \textit{ð(ə)-} is replaced with \textit{o-} for a single Moro person, and \textit{l(ə)-} for plural. Hence, the \textit{lə-} in the names above is probably the noun class marker:

\begin{exe}
\ex Moro endonyms
	\begin{xlist}
		\ex	ðəmwaɾə́ŋá	Moro language
		\ex omwaɾə́ŋá	Moro person
		\ex ləmwaɾə́ŋá	Moro people
	\end{xlist}
\end{exe}

	
\section{Data collection}
The data collected for this book have been amassed over a period of nine years, from 2005-2014, beginning with a graduate field methods class held at the University of California, San Diego. The main consultant for this class was Mr. Elyasir Afsos Julima, and he has been our primary consultant ever since. Data were also provided by Elyasir’s wife, Ms. Ikhlas Elahmer, over the course of the nine years, and by Mr. Angelo Naser Kuku, the current head of the Moro Language Committee, who visited San Diego in June-July of 2013. Fieldwork in Sudan was not possible due to instability and war returning to the Nuba Mountains, but also due to personal circumstances which prevented travel to Khartoum for any reasonable length of time. This methodology has obvious drawbacks in that the speakers were displaced from the language area, and we were limited in the number of speakers which whom we could conduct research. On the other hand, it also had advantages due to availability of the speakers and the ability to cross-check data regularly. 

Elyasir Julima is approx. 45 years old, and comes from the village of Karakaray-Al Byeara. He is a member of the X clan. He was raised in Omdurman rather than the Nuba Mountains due to war-induced displacement and the death of his mother when he was a small child. Nevertheless, there is a sizeable Moro population in Omdurman, and his primary caregiver during childhood was his paternal grandmother, who was monolingual in Moro. Elyasir also made frequent trips to the Nuba Mountains as a child and spent summers with his uncles in Karakaray. Elaysir is also fluent in Sudanese Arabic and English. He left Sudan in 19XX, spent two years in Cairo, Egypt and then arrived in San Diego, California in 20XX. 

Karakaray: clinic, elementary school. Angelo's parents are Kain, but grew up in Karakarai. ELyasir insists his Moro is Thetogovela. Market in Karakarai. Like city heights... (more than 1,000) (Angelo's family is Kain)

Ikhlas Elahmer is approx. 37 years old. She was born and raised in the village of XXXX and is a member of the X clan. After the town was attacked during the civil war when she was approx. nine years old, she and her family moved to Khartoum. She did not reside in a Moro speaking neighbourhood in Khartoum, but still maintained the language. She left Sudan at age 20, spent two years in Cairo, Egypt and arrived in San Diego, California in 20XX. She is fluent in Sudanese Arabic and English. Elyasir and Ikhlas converse with each other primarily in Moro and Arabic, with English added when necessary. They speak to their children in Arabic and English. Although they used to speak to the children in Moro, the children never gained fluency, and are now dominant in English. 

Angelo Kuku Naser is approx. 52 years old. He was born in the village of X and lived there for about 20 years. He resides in Omdurman, Sudan, and uses Moro on a daily basis with his family and other speakers. He works for the Bible Society translating the Old Testament into Moro, and preparing literacy materials as a member of the Moro Language Committee. He is familiar with the other dialects of Moro, as his family is originally Moro Kain, and he has worked extensively with the written standard based on Werria.  

There is no reliable estimate of the number of Moro speakers. In 1955-56, the Moro population was 28,311 based on the Sudan Census. Ethnologue lists the number in 1982 as 30,000 based on an SIL survey. The Second Sudanese Civil War lasted from 1983-2005, and thousands of people were killed or displaced from the area. Given these issues, it is hard to ascertain overall numbers of speakers. The 2008 Sudan Census lists around 102,000 people in the Kadugli area, including Um Dorein. However, this is a large area that includes other people besides Moro, and the census may not be reliable. Moreover, war broke out again in 2011 and is still ongoing. In the current conflict, many Nuba people have been killed, and many displaced, this time across the border into South Sudan. As a result, Mous (1998) classifies the entire Kordofanian family as endangered, ‘partly by genocide’ and Williamson \& Blench (2000) state that Kordofanian speakers ‘have been displaced through political insecurity and their status is now uncertain’. The Moro are one of the larger populations in the Nuba Mountains; this was reported as early as Nadel (1947), and repeated by other researchers such as Guest (19XX), and Moro people themselves. The large numbers have surely contributed to the maintenance of the language even among displaced populations in the Omdurman/Khartoum area, despite the pressure of Arabic. However, the almost continuous civil strife in the Nuba Mountains has rendered the language threatened at the very least. 


%\section{About literature}
%\citet{Comrie1981} is a useful introduction to typology \is{typology}. %\is = index of subjects
%It deals with languages of the whole world, not restricting itself to \ili{Indo-European languages}. %\il = index of languages
%\iai{Dionysios Thrax} was also an important figure. %\ia = index of authors. Not necessary for authors whose work you cite.
%
%\ea\label{ex:1:descartes}
%\langinfo{Latin}{}{personal knowledge}\\
%\gll cogit-o ergo sum \\
%     think-1{\sg}.{\prs}.{\ind} hence exist.1{\sg}.{\prs}.{\ind}\\
%\glt `I think therefore I am'
%\z
%
%\begin{table}
%\caption{Frequencies of word classes}
%\label{tab:1:frequencies}
% \begin{tabular}{lllll} % add l for every additional column or remove as necessary
%  \lsptoprule
%            & nouns & verbs & adjectives & adverbs\\ %table header
%  \midrule
%  absolute  &   12 &    34  &    23     & 13\\
%  relative  &   3.1 &   8.9 &    5.7    & 3.2\\
%  \lspbottomrule
% \end{tabular}
%\end{table}
%
%
%
%\begin{figure}
%\caption{Some XML}
%\begin{lstlisting}
%<LogEvent Action = "1" Value = "107" Cursor = "477">
%<LogEvent Action = "1" Value = "107" Cursor = "477">
%<LogEvent Action = "1" Value = "107" Cursor = "477">
%<LogEvent Action = "1" Value = "107" Cursor = "477">
%\end{lstlisting}
%\end{figure}
