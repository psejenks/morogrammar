\part{Simple clauses}

\chapter{Copular clauses}\label{Chapter:copular}

This chapter describes predicational and non-predicational copular clauses in Moro. Copular clauses in Moro are clauses that are headed by a copula, an element connecting a subject, typically a noun phrase to a nonverbal predicate or another noun phrase. There are two kinds of copular clauses in Moro, predicational copular clauses and non-predicational copular clauses. Predicational copular clauses are headed by one of two copular verbs, one of which selects for nominal predicates and the other of which selects for locative predicates:


\ea Predicate nominal copular clause \label{ex:ch9:1}\\
	 \gll  kúku		g-a-d-ó	     	udʒí  		g-é-ŋerá 	 	g-é-lɜŋgitʃ-in-ú\\
	 Kuku 	\textsc{cl}g-\textsc{rtc}-be-\textsc{pfv}   	person	\textsc{cl}g-\textsc{dpc}1-good	 \textsc{cl}g-\textsc{dpc}1-know-\textsc{pass}-\textsc{pfv}\\
	\glt	 ‘Kuku is a good person to know.’		\hfill 		 EJ61213 
	\z 

\ea Locative copular clauses \label{ex:ch9:2}\\
	\gll  logopájá 	 l-a-w-ó   				n-tərəbésá\\
	 		cup		 \textsc{cl}l-\textsc{rtc}-be.loc-\textsc{pfv}			loc-table\\
	\glt ‘The cup is on the table.’	 \hfill 			AN62413 \z


 Non-predicational copular clauses lack verbs and serve the function of either identification or equation between two individuals, marked with a non-verbal copular clitic.
 
\ea  	Identificational copular clause \label{ex:ch9:3}\\
	\gll údʒə́-k:ɜtíkɜ 	kətʃə́-k:i \\ 
		person-\textsc{cl}g.\textsc{2d} 	Kachi-\textsc{id} \\
	\glt ‘That person is Kachi.’		\hfill 	EJ61013  \z 

\ea  	Equative copular clause \label{ex:ch9:4}\\
	\gll 	Kúku ŋúŋ elaŋ\\
				Kuku \textsc{cop:eq} chief\\
		\glt	`Kuku is the chief.' \hfill AN111015 \z 
	
	
The apparent copula is different in these two examples. In the identificational clause in \ref{ex:ch9:3}, it is an enclitic on the second noun phrase which resembles the proximal demonstrative \textit{íki}, glossed \textsc{id}. In the equative sentence in \ref{ex:ch9:4}, it is the word \textit{ŋúŋ}, which is optional and resembles the third person human pronoun.

Predicational versus non-predicational copular clauses differ in the extent to which they exhibit syntactic properties of verbal clauses. While predicative copula show normal verbal morphology, the complement of predicate nominal copula show special syntactic restrictions which indicate they are predicates rather than normal nominal objects. In contrast, non-predicational copula are not verbs, and these clauses lack many of the morphosyntactic distinctions available in normal verbal clauses.


\section{Predicational copular clauses}

There are two predicational copula in Moro. Both are verbs, but they differ in whether they select for nominal or locative predicates: 

\ea  \label{ex:ch9:5}
\begin{tabular}[t]{cl}
\lsptoprule
Predicational copula & Complement type \\
\midrule 
-d- & Nominal predicate\\
-w- & Locative predicate \\ 
\lspbottomrule
\end{tabular}
\z 

This section describes copular clauses headed by each of these predicates.

\subsection{Predicate nominal copular clauses}\label{sec:ch9:nompred}

The copula \textit{d} `be' patterns with other verbal predicates in Moro, agreeing with a referential subject and inflecting for the basic perfective-imperfective aspect distinction, both diagnostic properties of verbs:

\ea Predicate nominal copular clause in perfective and imperfective \label{ex:ch9:6}
	\ea  \gll  é-g-a-d-ó oraŋ\\
		 1SG-\textsc{cl}g-\textsc{rtc}-be-\textsc{pfv} man\\
		 \glt  ‘I am a man.’	
	\ex  \gll é-g-a-v-ə́d-eá oraŋ\\
		 1SG-\textsc{cl}g-\textsc{rtc}-PROG-be-I\textsc{pfv} man\\		
	  	\glt ‘I am about to be a man.’  \z \z 
	  	
As these examples show, the meaning of \textit{d} ranges between `be' and `become' in English, relative to the aspectual form of the verb. The alternation between a present tense stative and inchoative reading is typical of stative verbs in general (Section \ref{imperfectivesem}). 

Further evidence for the verbhood of the copula \textit{d} also comes from its ability to occur after negative and inceptive auxiliaries (\ref{ex:ch9:7a}, \ref{ex:ch9:7b}), and its ability to occur in the imperative (\ref{ex:ch9:7c}):

\ea Evidence for verbal nature of predicate nominal copular clausees
	\ea  	\gll  	ŋerá		ŋ-a-n:á					ŋá-d-e		oraŋ	\\
					child		\textsc{cl}ŋ-\textsc{rtc}-not.aux.pfv		\textsc{3sg.inf}-be-\textsc{pfv}	man \\
			\glt  	‘A child is not a man.’		\label{ex:ch9:7a}				
	\ex 	\gll 	ŋerá		ŋ-a-və́lá			áŋá-və́d-é		oraŋ	\\
	 				child		\textsc{cl}ŋ-\textsc{rtc}-incep.aux	\textsc{3sg.inf}-be-\textsc{inf}1	man\\
		 	\glt	‘The child is going to be a man.’		\label{ex:ch9:7b}	
	\ex 	\gll 	ə́dó 		oraŋ		    			\\
					be.\textsc{imp}		man\\
			\glt	‘Be a man!’				\hfill 					EJ81214 \label{ex:ch9:7c}
	\z \z 
	
Two instances of agreement hold in predicate nominal copular clauses. The copula itself shows normal subject agreement. Additionally, the noun following the complement must have the same number specification as the subject:

\ea  \gll kúku nə ŋálo-ŋ l-a-d-ó lədʒ-!ə́l-!í-tʃ-!ə́\\
		Kuku and ŋalo-\textsc{acc} \textsc{cl}l-\textsc{rtc}-be-\textsc{pfv} people-\textsc{scl}l-\textsc{dpc}1-bad-\textsc{adj}\\
	\glt 		`Kuku and Ngalo are bad people' \hfill EJ72116 \z 

However, predicate nominal copular clauses are subject to several restrictions which show they are distinct from clauses with normal transitive verbs. First, they cannot occur in the periphrastic causative (Section \ref{section:causative}), which is otherwise fully productive:

\ea[\#]{	\gll	kúku		g-ɜ-ŋgit-ú      		ðamalə́-ðɜtíðə		ð-a-d-ó	 	oraŋ\\
	 		Kuku 	\textsc{cl}g-\textsc{rtc}-cause-\textsc{pfv}	camel-clð.that		\textsc{cl}ð-\textsc{rtc}-be-\textsc{pfv}	man\\
		 	\glt	‘Kuku made that camel be male.’	(intended) 		\hfill 		EJ8814}
\z 
This restriction indicates that predicate nominal copular clauses lack an external argument or semantic causer, a necessary precondition for causative formation.

Additionally, the nominal complement of \textit{d} must be interpretable as a predicate. This predicative requirement can be seen in several restrictions which do not obtain for normal argumental objects.

First, this nominal cannot be referential:

\ea 	\ea[*]{ \gll Kúku 		g-a-d-ó			udʒə́-kɜtíkə	\\
					Kuku		\textsc{cl}g-\textsc{rtc}-be-\textsc{pfv}		person-\textsc{cl}g.\textsc{2d}\\
	 			\glt ‘Kuku is that person.’ (intended)}
	\ex[*] {		\gll Tútu		g-a-d-ó	 		Shone \\
					Tutu		\textsc{cl}g-\textsc{rtc}-be-\textsc{pfv}		Sean	\\
	 			\glt ‘Tutu is Sean.’ (intended) \hfill 	EJ8714}
\z  \z 

Additionally, nominal predicates cannot be modified by indefinite quantifiers, such as the indefinite modifiers discussed in Section \ref{indef}:

 \ea 	\ea[*]{ \gll Kúku 		g-a-d-ó			umiə g-ənəŋ g-é-ŋer-á	\\
					Kuku		\textsc{cl}g-\textsc{rtc}-be-\textsc{pfv}		boy \textsc{cl}g-indef \textsc{cl}g-\textsc{dpc}1-good-\textsc{adj}\\
	 			\glt ‘Kuku is a good boy.’ (intended)}
	\ex[*] {		\gll Kúku nə ŋálo-ŋ  l-a-d-ó	 		lə́miə l-ɜmən l-é-ŋer-á \\
					Kuku and Ngalo-ACC		\textsc{cl}g-\textsc{rtc}-be-\textsc{pfv}		boy \textsc{cl}l-indef \textsc{cl}l-\textsc{dpc}1-good-\textsc{adj}\\
	 			\glt ‘Kuku and Ngalo are good boys.’ (intended) \hfill 	EJ8116}
\z  \z 

Both of the sentences above are fine when the indefinite modifier is omitted. These semantic restrictions on the nominal complement reveal that it is not a normal object, but is in fact a semantic predicate which is applied to the subject of the copular clause.

Along with the semantic restriction on the nominal complement are several morphosyntactic ones. First, the object is not case marked, To see this, we must look at the restricted set of human common nouns which mark accusative case such as \textit{matʃo} `man':

	\ea  \gll  Kúku g-a-d-ó matʃo-(*ŋ)\\
		 Kuku \textsc{cl}g-\textsc{rtc}-be-\textsc{pfv} man-(ACC)\\
		 \glt  ‘I am a man.’	
 	\z
Additionally, the ``object'' or predicate nominal cannot be promoted in the passive \ref{nopassncc} or extracted in a cleft or content question \ref{nowhncc}: 

\ea[*]{ \gll 	oraŋ		g-a-d-ən-ó	 	\\
				man		\textsc{cl}g-\textsc{rtc}-be-caus-\textsc{pfv}\\
	 		\glt ‘The male was made to be.’ (intended)	 		\hfill 					EJ8814 \label{nopassncc}}
\z 

\ea \label{nowhncc} \ea  \gll ðamala	g-a-d-ó	 	wánde	íðɜtíðɜ\\
		camel	\textsc{cl}g-\textsc{rtc}-be-\textsc{pfv}	what		clð.that\\
		\glt 	‘What (gender) is that camel?’
	\ex[*]{	\gll ŋwənd(ə́ki)	ðamalə́-ðɜtíðə		g-ə́-d-ó	 	\\
		foc-what-?		camel-clð.that		\textsc{cl}g-nsrc-be-\textsc{pfv}\\
	 	\glt ‘What (gender) is that camel?’ (intended)	\hfill 					EJ8814 \label{nopassncc}}
\z \z 
Together, these restrictions on the nominal complement of \textit{d} are due to its status as a predicate, rather than a normal internal argument of a verb. 

Finally, predicate nominal copula can occur with a number of specific classes of predicts. First, and somewhat unsurprisingly, \textit{d} can also take nominals which describe a person with a particular property as their complement (\sectref{sec:ch6:pcnominal}):

\ea 
\ea \gll g-a-d-ó aməda\\
\textsc{cl}g-\textsc{rtc}-be-\textsc{pfv} joker\\
\glt `(S)he is a joker.'
\ex \gll g-a-d-ó ðaŋ kaɲ\\
\textsc{cl}g-\textsc{rtc}-be-\textsc{pfv}  useless very\\
\glt  `(S)he is very useless.'
\ex \gll g-a-d-ó ywarra\\ %check with Elyasir
\textsc{cl}g-\textsc{rtc}-be-\textsc{pfv} clueless\\
\glt  `(S)he is incapable.'
\z
\z
The copula \textit{d}  also occurs with predicative uses of numerals, which agree with the subject where possible:

\ea 
\ea \gll ɲerá ɲ-a-d-ó ɲ-egetʃaŋ\\
girls \textsc{cl}g-\textsc{rtc}-be-\textsc{pfv} \textsc{cl}ɲ-two\\
\glt `The girls are two.' (=`There are two girls.')
\ex \gll ɲerá ɲ-a-d-ó marlon\\
girls \textsc{cl}g-\textsc{rtc}-be-\textsc{pfv} four\\
\glt `The girls are four.' (=`There are four girls.')
\z
\z
Finally, the predicate nominal copula \textit{d} can be followed by ideophones:

\ea
	\ea \gll ŋgwɘɲ ŋ-a-d-ó tʃírbibibi\\
			 letters \textsc{cl}ŋ-\textsc{rtc}-be-\textsc{pfv} IDPH\\
		\glt `The letters are hard to read as the writing is small and compressed
	\ex \gll átʃə́vá g-a-d-ó trəbebebe\\
			food \textsc{cl}g-\textsc{rtc}-be-\textsc{pfv} IDPH\\
		\glt `The food is very soft, like liquid.'
	\ex \gll ŋəðәmana ŋ-a-d-ó gәja gәja\\
			\textsc{cl}ŋ.bean SM.\textsc{cl}ŋ-\textsc{rtc}-be-\textsc{pfv} IDPH\\
		\glt ‘the beans are crunchy’ \hfill Naser \& Rose (To appear)
\z 
\z

These examples clearly demonstrate that ideophones are predicates. However, it is not clear that ideophones should be considered nouns, so these examples potentially constitute an exception to the generalization that \textit{d} must be followed by a noun. For more details on ideophones, including in these contexts, see Chapter \ref{chapter:ideophones}.

%TODO See Jenks et al. for discussion of a similar phenomenon in Basaea??
%TODO Add discussion of comparatives if I can find data; recall that a periphrastic construction was used for comparison in these casees


\subsection{Locative copular clauses}\label{section:loccop}

Locative copular clauses are clauses containing the locative copula \textit{v} `live, be at.'  When this copula has an intransitive subjects, it expresses a simple spacial relationship. When \textit{v} occurs with an animate subject, it is best translated as `live.' The locative copula is a normal verb, meaning it takes all verbal prefixes and marks an aspectual distinction between perfective and imperfective. Note that like other instances of the phoneme /v/, this copula is realized as [w] before round vowels (Section \ref{section:v}):

	\ea 	\gll 	nala		n-a-w-ó 			ég-ə́tám	\\
					necklace	\textsc{cl}n-\textsc{rtc}-be.loc-\textsc{pfv} 	loc-neck\\
			\glt 	‘The neckace is on (my) neck.’	 \z 

\ea 
	\ea	\gll 	kúku		g-a-w-ó	      	lɜmú	/ n-ajén	\\
	 			Kuku 	\textsc{cl}g-\textsc{rtc}-be.loc-\textsc{pfv}	Khartoum / on-mountain 		\\
	 	\glt 	‘Kuku lives in Khartoum/in the mountains.’
	\ex	\gll 	kúku		g-a-v-eá	  	lɜmú/ n-ajén		\\
	 			Kuku 	\textsc{cl}g-\textsc{rtc}-be.loc-\textsc{pfv}	Khartoum / on-mountain 	\\
	 	\glt 	‘Kuku is about to live in Khartoum / in the mountains.’ \z \z 

Like with stative verbs more generally (Section \ref{imperfective}), the imperfective form of \textit{-v-} has an inchoative meaning.

Locative copular clauses are normal transitive clauses. They form imperatives, and can be embedded under negation and other auxiliaries:

\ea \gll 		ə́wó 		lɜmú\\      
				live.\textsc{imp}	Khartoum\\
	\glt 	‘Live in Khartoum!’ \z 

\ea 
\ea \gll  	é-g-a-nná			é-v-á				ɜnni 	\\
			\textsc{1\textsc{sg}}-\textsc{cl}g-\textsc{rtc}-not.\textsc{pfv}	1\textsc{sg}-live-\textsc{inf}2		here\\
	\glt	 	‘I don’t live here.’										

\ex \gll    í-g-iðí 		ɲe-v-é		lɜmú\\
	 		1\textsc{sg}-\textsc{cl}g-will	1\textsc{sg}-be.loc-\textsc{inf}1	Khartoum\\
	\glt  	‘I’ll be in Khartoum.’					\z 					
\z 

Further evidence that locative copular clauses are normal transitive sentences comes from their ability to be passivized:

\ea \gll  ajén j-e-v-ən-ú-u\\
			mountains \textsc{cl}j-rtc-be.loc-pass-pfv-loc\\
	\glt 		`The mountains are lived in.'
\z 
Similarly, the object of locative copula can be extracted in a cleft:
\ea
\ea \gll  	Kuku		n-a-w-ó	 		ŋgá \\ 
			k.		\textsc{cl}g-\textsc{rtc}-be.loc-\textsc{pfv}	where\\ 
	\glt  	‘Kuku lives where?’
\ex \gll  	ŋŋgwa		nə́=	    kúku 	g-ə́-w-ó-u			 	\\
			foc.where	comp2= K.		\textsc{cl}g-rtc-be.loc-\textsc{pfv}-LOC \\
	 \glt 	‘Where does Kuku live?’ 						\z 
\z 
Like all locative objects, passivization or extraction of the object of the locative copula leaves behind a locative clitic on the verb. %TODO SectionXXX; what about contexts where you are identifying locations on your body? these seem different for deictic copular clauses

\section{Non-predicational copular clauses}\label{section:nonpredcop}

This section describes two kinds of non-predicational copular clauses: identificational copular clauses and equative copular clauses. These clause types are distinct from other clause types in Moro in that they lack predicates, and also because they lack any elements which agrees with the subject. Instead, these clauses equate or identify two nominal expressions as referring to the same individual. 

Equative clauses are distinguished from identificational clauses largely based on their information structural profile. In order to see the profile more clearly most of these examples will be embedded in question-answer pairs.

		
\subsection{Identificational copular clauses}


Identificational copular clauses are those where a known referent is supplied with a contextually new identity. In these clauses, a clitic which looks just like a proximal demonstrative occurs on the referential noun phrase which supplies the new identity, for example, on a proper name. I will call this element the identificational clitic. Identificational copular clauses can either have an overt subject, or they can simply include the focused material and identificational clitic. An example is given in \REF{ex:copular:id1} and \REF{ex:copular:id2}. In both of these examples, the element hosting the identificational clitic corresponds to the word which in the question hosts the focus clitic \textit{ŋ́\super{w}=} (on which see the following section and Chapter \ref{chapter:questions}):

\ea Identificational answer to question \label{ex:copular:id1}
 	\exi{Q:} \gll  	ŋ́\super{w}-ɜ́dʒí 	=k:ɜtíkɜ	\\			
						 \textsc{foc} -who	   			S\textsc{cl}g.2d		\\
			\glt 	‘Who is that?’ 
	\exi{A:} \gll 	 	údʒí=kɜtíkɜ 		kətʃə́-k:i 	\\
		     		person-\textsc{cl}g.\textsc{2d} 		Kachi-ID		\\
		    \glt   ‘That person is Kachi.’
\z 

\ea  Identificational confirmation request \label{ex:copular:id2}
	\exi{Q:} 	\gll  ŋ́\super{w}-kətʃi kətíkɜ\\
					 \textsc{foc} -Kachi	   S\textsc{cl}g.2d	\\
				\glt   ‘Is that Kachi?’ 
	\exi{A:}	\gll aa, 	kətʃə́-k:i	\\
					 yes	Kachi-ID	\\
				\glt ‘Yes, it’s Kachi’	\hfill	 (AN62413)
\z  
Identificational clitics are distinct from normal adnominal demonstratives (see Section \ref{demonstratives}). Unlike demonstratrives, the identificational clitic only occurs in the proximal form \textit{=íC:i}, but it lacks deictic semantic content. Additionally, the identificational clitic can occur on referential expressions such as names, which demonstratives cannot. Identificational clitics also occur in clefts (Section \ref{question}) and resemble demonstrative relative operators (Section \ref{nonsubjectrelative}), suggesting that these constructions are syntactically related, possibly by ellipsis of an embedded clause in the case of identificational copular clauses. 

In the following example, we see that the information structure of the question is inverted from \REF{ex:copular:id2}, as the pronoun whose identity is being sought is in its focused form. Nevertheless, the answer looks like other identificational copular clauses as the new, focused information hosts the identificational clitic:

\ea  Identity crisis\label{ex:copular:id3}
	\exi{Q:} 	\gll 	ɲíɲí ŋ́\super{w}-ɜdʒɜ-ki? \\
						1SG.PRO. \textsc{foc} 	   \textsc{foc} -who-ID \\
				\glt  	‘Who am \underline{I}?’ 
	\exi{A:}	\gll 	(ŋŋá) kúkə-ki	\\
						2SG.PRO	Kuku-ID\\
				\glt 	‘You're kuku’ \hfill AN111015 \label{ex:copular:id3b}
\z 
Identificational clefts often make use of independent personal pronouns, as in \REF{ex:copular:id3b}, both because pronouns are often focused in these contexts and because there is no verbal agreement to track subject reference.

Identificational copular clauses cannot be negated with a negative auxiliary verb in the same clause. Instead, these clauses must be negated by a special subordination construction:

\ea  Identificational copular clause under negation
	\exi{Q:} 	\gll 	 	ŋ́\super{w}-kətʃi kɜtíkɜ		\\
					 \textsc{foc} -Kachi	   S\textsc{cl}g.2d	\\
				\glt  		‘Is that Kachi?’ 
	\exi{A:}		\gll 		ndo, 		k-an:á  	=t̪á		kətʃə́-k:i\\
			no,		\textsc{cl}g-neg.aux	 =\textsc{comp1}	Kachi-\textsc{id}\\
				\glt 	‘No, it’s not Kachi.’		\hfill	 (AN62413)
\z  
In the special subordination construction, the negative auxiliary occurs alone in a clause before the complementizer \textit{tá}.

%TODO Add section references to negation and complementizer?

%TODO CAN INDEFINITES BESIDE S WH-WORDS OCCUR IN THESE CONSTRUCTIONS?

\subsection{Equative copular clauses}

Equative copular clauses identify two individuals. We have found two variants of equative copular clauses which are preferred by one or the other one of the Moro consultants on this project; the exact distribution of these forms requires further study. In the first variant, in \ref{ex:ch9:eq1}, the equative clause is headed by the copula \textit{ŋúŋ}, which resembles the third person human independent pronoun and does not agree with its subject or mark any inflection of any kind:

\ea Equative copular clause\label{ex:ch9:eq1}
	\ea \gll	umiɜ́-ki g-é-ŋer-á ŋúŋ kúku \\	
				boy-S\textsc{cl}g \textsc{cl}g-\textsc{dpc}1-good-\textsc{adj} \textsc{cop:eq} Kuku\\
		\glt	‘The best boy is Kuku.’
	\ex	\gll 	matʃ-ɜ́k-ɜtikɜ ŋúŋ kúk:u \\
				man-S\textsc{cl}g-that \textsc{cop:eq} Kuku\\
		\glt	‘That man is Kuku’
	\ex	\gll 	Kúku ŋúŋ élàŋ\\
				Kuku \textsc{cop:eq} chief\\
		\glt	‘Kuku is the chief.’ \hfill AN111015
	\z 
\z

The second variant of equative copular clause is with no copula at all, but two juxtaposed noun phrases:

\ea Bare equative copular clause
	\ea	\gll matʃ-ɜ́k-ɜtik:ɜ	kúku \\
			man-S\textsc{cl}g-that 	Kuku \\
		\glt ‘That man is Kuku’ 
	\ex	\gll kúk:u	élàŋ  	\\
			Kuku	chief 	\\
		\glt ‘Kuku is the chief.’	\hfill 	EJ081516
	\z
\z

These sentences have neutral information structure, meaning that the existence of, say, `that man' and `Kuku' have already been established, and this sentence serves only to identify the two. 

%It is intereesting that the 

%TODO NEGATION OF THIS CONSTRUCTION?
%TODO CAN INDEFINITES OCCUR IN THESE CONSTRUCTIONS? I'm guessing no...did I get this?

The proclitic \textit{ŋ́\super{w}=}, which also occurs in clefts and questions, is arguably a variant of the equative copula which emerges when one of the identified individuals is focused. Notice that the question below has both the proclitic equative copula and the identificational clitic:

\ea Identity confusion\label{id4}
	\exi{Q:} \gll (ŋ́w=)kúku ɲ́w-ɜ́dʒ!ɜ́-ki?	\\
				 \textsc{foc} -Kuku  \textsc{foc} -who-ID\\
			\glt ‘Who is Kuku?
	\exi{A:} \gll ŋ́w=!íɲi kúku(*=íki).	\\
					 \textsc{foc} -1SG.PRO Kuku(-ID)\\
			\glt ‘\underline{I} am Kuku.' \hfill AN111015
\z

When the speaker identifies himself in the response above, the name \textit{Kúku} is already part of the common ground, which is why the identificational clitic is impossible. Additionally, the pronoun itself also cannot be new information. However, the pronoun `I' is contrastively focused as the correct identity of Kuku, and as such, we see the fronted, arguably focused variant of the equative copula and pronoun in this context.

To see this more clearly, compare (\ref{id4}) with the `identity crisis' context from the previous section (\ref{id3}), which form a minimal pair in terms of their information structure. In the response clause in (\ref{id3}), the identity Kúku is new information, while in the response clause in (\ref{id4}) there is no new information, making an equative clause appropriate, here with contrastive focus on one of its arguments. 
