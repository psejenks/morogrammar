\chapter{Questions, conditionals, and adverbial clauses}\label{chapter:questions}

questions here

topic construction:

orn eđe yobaŋǝno agiđi ṯǝŋalǝnanaco ñere
orn eđe  y-obaŋǝno a-g-iđi ṯǝ-ŋa-lǝ-na-nac-o ñere
but meat cly-soft  2sg-clg-fut.aux comp1b-2sg.om-3pl.om-iter-give.rt-ipfv children 
but the soft meat you will give to my children (Fox, hyena, and rabbit)


\section{Polar questions}\label{ynquestion}


\section{Clefts}\label{clefts}

\section{Content questions}\label{content}


%Agreeing ‘how’!
%ðamala ðáw-íði            ‘What is the camel for?’
%ʌgʌlit̪ú ðamala ŋən ŋantau        ‘For what purpose did you buy the camel?’
%ŋen ŋaw?                ‘How’s the situation?’
%ŋen ŋero                ‘Nothing (doing) = ‘there’s no situation’’
%udʒí gaw íki?                ‘What is the woman for?’
%lavəra láw ílli?                ‘What is the stick for?’
%lavəra láwənde ílli?            ‘Which kind of stick is this’’ (Elyasir)
%yefia yaw ísi?                ‘What is the lion for?’
%día raw íri?                ‘What is the cow for?’
%día rawand(e)-íri?            ‘What kind of cow?’  (EJ62017)
%ðoala ðaw íði                ‘What is the livestock for?’
%
%Ŋenəŋaw kúku gamma káka ‘it’s true that K. married Kaka’
%
%Ŋen ŋantau ‘for what situation’
%
%Q: (*ŋw)ðóálá ðaw íði nə́kuku gə́natʃó ŋál:o (*ðaw)?    
%‘For what purpose did Kuku give ŋalo the livestock?’
%    “What is the money for, the one that Kuku gave to Ngalo” (Elyasir trans 6/20/17)
%Q: (*ŋw)ðóálá ðawanda íði nə́kuku gə́natʃó ŋál:o (*ðaw)?    
%‘For what purpose did Kuku give ŋalo the livestock?’
%A: ðóálá ðaðama            ‘Dowry livestock.’ “for marriage”
%
%*Kuku ganatʃó ŋalo ðoala ðaw.     (confirmed: EJ62017)
%*Kuku ganatʃó ŋalo ðoala ðawíði


In many languages, the formation of constituent questions, or wh-questions, involves the question word appearing in the standard or canonical position in the sentence, a strategy known as \textit{in-situ}. In others, the question word appears displaced external to the clause, leaving a ``gap" in the canonical position, a strategy known as \textit{ex-situ}.  Some languages uniformly utilize one strategy for constituent question constructions while some languages exclusively utilize the other. There are, however, some languages that possess both in-situ and ex-situ constructions (Cheng 1997; Potsdam 2006). Moro, a Kordofanian (Niger-Congo) language spoken in the Nuba Mountains of Sudan, belongs to this latter class.  Schadeberg (1981) classifies Moro as belonging to the Western group of West-Central Heiban Kordofanian languages. 

The two types of wh-question constructions in Moro display strikingly different properties. In the typical in-situ strategy, a question word appears in the canonical position. In the example in (1), the declarative sentence (1a) is juxtaposed against an in-situ object question (1b). The question word appears in the post-verbal object position. In the ex-situ strategy in (1c), in contrast, the form of the question word itself is different (\textit{wánde} vs. \textit{ŋwɜ́ndə́kːi}) and the verb has a different prefix (\textit{a-} glossed as Root Clause (\textsc{rtc}) as it occurs in declaratives, in-situ questions, and complements of bridge verbs, and \textit{ə́-} in the ex-situ question, which we gloss as \textsc{dependent clause}2 (\textsc{dpc}2); Jenks 2013, Rose 2013). In addition, a particle \textit{nə́-}, which we will analyze as a complementizer, is optionally attached to the subject and/or the verb (1c). All data are from the Thetogovela dialect (in Moro orthography, Dətogovəla).  Moro has two tones. High tone is marked with an accent (\hspace{2pt}  ́ ) and low tone is unmarked.

\ea
\ea  \gll kúku         ɡ-a-sː-ó               	eða\\
	\textsc{cl}g.Kuku   \textsc{sm.cl}g-\textsc{rtc}-eat-\textsc{pfv}	\textsc{cl}j.meat     	\\		     	     \trans ‘Kuku ate the meat.’\\
\ex  \gll kúku	ɡ-a-sː-ó	wánde?\\
	\textsc{cl}g.Kuku	\textsc{sm.cl}g-\textsc{rtc}-eat-\textsc{pfv}   	\textsc{cl}g.what\\
	\trans ‘What did Kuku eat?’\\
\ex	\gll ŋwɜ́ndə́kːi	(nə́-)kúku     	(nə́-)ɡ-ə́-sː-ó?                \\
	what.\textsc{cl}g	(\textsc{comp-})Kuku  	(\textsc{comp-})\textsc{sm.cl}g-\textsc{dpc}2-eat-\textsc{pfv}  \\
	\trans ‘What did Kuku eat?’\\
\z
\z

Subject wh-questions only use the ex-situ strategy as in (2). This is surmised from the form of the question word, and the prefix on the verb. Unlike object questions, there is a different prefix on the verb, \textit{é-}, glossed as \textsc{Dependent clause} 1 (\textsc{dpc}1). In addition, the particle \textit{nə́-} prefixed to the verb in (1c) is never attested in these constructions. 

\ea
\ea		\gll ŋwɜ́ndə́kːi	ɡ-é-sː-ó        	eða?		\\
what.\textsc{cl}g	\textsc{sm.cl}g-\textsc{dpc}1-ate-\textsc{pfv}	\textsc{cl}j.meat\\
\trans		‘What ate the meat?’	\\
\z
\z

The goals of this article are threefold. First, we provide a basic description of constituent or wh-question constructions in Thetogovela Moro. In the grammar of a related Moro dialect (Black and Black 1971), in-situ questions are reported for all wh-phrases (p. 73), but only a few examples of ex-situ constructions are given for ‘why’ and ‘how’. Nevertheless, the structure of the ex-situ constructions differs from Thetogovela. There is a dearth of descriptive material on the syntactic properties of Kordofanian languages in general, and this article aims to contribute to a better understanding of one of these languages. Second, we outline the ways in which ex-situ constituent question constructions share structural parallels with cleft and relative clause constructions. We propose that ex-situ questions are, in fact, a type of wh-cleft construction. Third, we provide an analysis of the morphological markers found in ex-situ questions.  The verb prefixes \textit{ə́-} and \textit{é-}, observed in (1c) and (2) respectively, and the particle \textit{nə́-}, pose analytical challenges. We argue that evidence from other constructions in the language point to the verb prefixes as dependent clause markers, as they appear in other dependent clause constructions. The distribution of \textit{nə́-} suggests that it is a type of complementizer that can appear cliticized to the verb or the subject. It, too, appears in other dependent clause constructions where its status as a complementizer is clearer. 

The paper is organized as follows. In Section 2 we present wh-in-situ constructions, comparing them to corresponding declarative clauses. Section 3 explores wh-ex-situ constructions identifying the basic differences between subject and non-subject wh-constructions. Section 4 demonstrates similarities between wh-ex-situ questions and relative clauses and clefts, leading to the conclusion that wh-ex-situ questions constitute a wh-cleft construction. We provide arguments from negation for the biclausality of clefts, evidence from tone that all three types employ dependent clauses, and examples demonstrating that the verb prefixes \textit{é-} and \textit{ə́-} are employed in other dependent clause constructions. In section 5, we address properties of non-subject wh-ex-situ questions, clefts and relative clauses, including alternate morphological marking in different persons, the distribution of resumptive pronouns, and evidence that the marker \textit{nə́-} in (1c) is a complementizer. Finally, we conclude in section 6 with some typological considerations. 

\section{In-situ content questions}\label{sec:ch18:insitu}

%TODO ADD SECTION 2 OF MOROWHPAPER HERE
In this section we describe the behavior of wh-in-situ questions. We begin with those bearing the lexical category noun (N): this is the lexical category in Moro that determines class agreement both internal to the noun phrase (NP) as well as with subject agreement on the verb in a clause.  

Before presenting the relevant examples it is important to introduce some aspects of the noun class system of Moro. As in other Niger-Congo languages, nouns in Moro are divided into a number of noun classes (Stevenson 1956-7; Black and Black 1971; Schadeberg 1981; Gibbard et al. 2009). Noun class is marked by the first segment, usually a consonant, on the noun, and indicates singular, plural or invariable, e.g. \textit{ŋeɾá} ‘girl, child’ (class marker \textit{ŋ}) vs. \textit{ɲeɾá} ‘girls, children’ (class marker \textit{ɲ}). Subject agreement on verbs and nominal modifiers shows class agreement with the noun through use of a corresponding consonant. Some nouns are vowel-initial; these nouns have either \textit{ɡ} or \textit{j} noun class agreement. We indicate noun class with cl followed by the agreement consonant, following Gibbard et al. (2009). 

Declaratives and corresponding in-situ object wh-questions are illustrated in (3).

\ea
\ea \gll	kúku	ɡ-a-t̪að-ó	eða\\
	\textsc{cl}g.Kuku	\textsc{sm.cl}g-\textsc{rtc}-leave-\textsc{pfv}	\textsc{cl}j-meat   \\  			     
\trans	‘Kuku left the meat behind.’\\
\ex \gll	kúku	ɡ-a-t̪að-ó	wánde?\\
	\textsc{cl}g.Kuku	\textsc{sm.cl}g-\textsc{rtc}-leave-\textsc{pfv}	\textsc{cl}g.what\\
\trans	‘What did Kuku leave behind?’	\\
\z
\z

\ea
\ea \gll	kúku	ɡ-a-t̪að-ó	ówːá     \\
	\textsc{cl}g.Kuku	\textsc{sm.cl}g-\textsc{rtc}-leave-\textsc{pfv}	\textsc{cl}g.woman/wife\\
\trans	‘Kuku left the woman/wife behind.’\\
\ex \gll	kúku		ɡ-a-t̪að-ó             	ɜʤɜ́ŋɡaŋo?	\\
		\textsc{cl}g.Kuku	\textsc{sm.cl}g-\textsc{rtc}-leave-\textsc{pfv}	\textsc{cl}g.who\\
\trans		‘Whom did Kuku leave behind?’\\
\z
\z

As can be seen, the wh-phrase functioning as an object occupies the same clausal position as the NP object in a declarative clause.
The nominal form \textit{ɜʤɜ́ŋɡaŋo} has a shorter form \textit{ɜʤɜ́}, which is used in particular constructions, such as with comitatives, glossed here as instrumental (\textsc{inst}) as the same marker is used for both senses. 

\ea
\ea \gll k-a-t̪að-ó-ŋó		sára-ɡa\\
\textsc{sm.cl}g-\textsc{rtc}-leave-\textsc{pfv}-3\textsc{sgom}	\textsc{cl}g.Sara- \textsc{cl}g.\textsc{inst}\\
\trans ‘S/he left him/her with Sara.’\\
\ex	\gll k-a-t̪að-ó-ŋó		ɜʤa-ɡá?\\
\textsc{sm.cl}g-\textsc{rtc}-leave-\textsc{pfv}-3\textsc{sgom}	\textsc{cl}g.who-\textsc{cl}g.\textsc{inst}\\
\trans ‘With whom did s/he leave him/her?’\\
\z
\z

Nominal expressions associated with non-subject functions containing the modifiers ‘which’ and ‘whose’ may also appear in-situ. The expression “whose NP” is a genitive construction, which is formed by prefixing the possessor with \textit{Cə́-} (C-́ before vowel-initial stems) where C represents a noun class marker that agrees with the class of the possessed (Jenks 2013).  This can be seen in (6) where the wh-modifier functioning as possessor bears the class prefix \textit{ŋ-}, determined by the class of the possessed nominal.  

\ea

\ea \gll	ŋálːo	ɡ-a-mː-ó	ŋeɾá	ŋ-ɜ́ʤɜ́?\\
		\textsc{cl}g.Ngalo	\textsc{sm.cl}g-\textsc{rtc}-take-\textsc{pfv}	\textsc{cl}ŋ.girl	\textsc{cl}ŋ.poss-who\\
\trans		‘Whose daughter did Ngalo marry?’\\
\ex	\gll ŋálːo	ɡ-a-mː-ó	ŋerá	ŋ-áŋɡa?\\
		\textsc{cl}g.Ngalo 	\textsc{sm.cl}g-\textsc{rtc}-take-\textsc{pfv}	\textsc{cl}ŋ.girl	\textsc{cl}ŋ.poss-which\\
\trans		‘Which daughter did Ngalo marry?’\\
\z
\z

In contrast to these in-situ non-subject nominal constructions, all wh-elements occupying the subject role relative to a verb occur only in the ex-situ constructions; their discussion will be deferred to section 3 where we address this strategy.
	
Turning to time and spatial adverbials, their wh-forms can also appear in-situ.  Moreover, they, like nominals, typically appear in the clausal position associated with that specific adverbial. Sentential temporal adverbs such as \textit{éréká} ‘yesterday’ may appear in multiple positions in declarative sentences, but usually appear post-verbally and following the object, if one is present. The order of manner adverbials with respect to time adverbials is not fixed: some manner adverbials are more flexible than others with respect to linear order; however, unlike temporal adverbs, manner adverbials do not appear between subject and verb or between verb and object. In (7) and (8), the position of the time adverbial ‘yesterday’ in (7a) and (8a) is occupied by the question word ‘when’ in (7b) and (8b), but the reverse order of adverbs in both sentences is also possible. 

\ea
\ea \gll	ŋə́ní	ŋ-aɾ-ó 	éréká	kaɲ\\
		\textsc{cl}ŋ.dog	\textsc{sm.cl}ŋ-cry-\textsc{pfv}	yesterday	loudly\\
\trans		‘The dog barked loudly yesterday.’\\
\ex \gll	ŋə́ní	ŋ-aɾ-ó	ndóŋ	kaɲ?\\
		\textsc{cl}ŋ.dog	\textsc{sm.cl}ŋ-cry-\textsc{pfv}	when	loudly\\
\trans		‘When did the dog bark loudly?’\\
\z
\z

The spatial wh-adverb ‘where’ displays a similar distribution:

\ea
\ea \gll		á-ɡ-erl-et̪-ó	n-ején	joáɲa\\
		2\textsc{sgsm-cl}g-walk-\textsc{loc}.\textsc{appl}-\textsc{pfv}	\textsc{loc-}\textsc{cl}j.mountain	\textsc{cl}j.many\\
\trans           		‘You went to different countries/regions.’	\\
\ex \gll	á-ɡ-a-v-ət̪-ó	ŋɡá?		     \\
		2\textsc{sgsm-cl}g-\textsc{rtc}-go-\textsc{loc}.\textsc{appl}-\textsc{pfv}	where\\
\trans		‘Where did you go?’	  \\
\z
\z

Finally, the wh-adverbials denoting ‘how’ and ‘why’ also appear in-situ: 

\ea
\ea \gll		á-ɡ-áfː-a	d̪át̪áo	eɡea?\\
	2\textsc{sgsm-cl}g-build-\textsc{ipfv}	how	\textsc{cl}g.house \\
\trans	‘How are you building the house?’\\
\ex \gll	á-ɡ-oás-a	ndréð	eðá	ŋínɜ́ŋí?\\
	2\textsc{sgsm-cl}g-wash-\textsc{ipfv}	\textsc{cl}n.clothes	why	today\\
\trans	‘Why are you washing clothes today?’\\
\z
\z

A summary of the wh-in-situ words is provided in the following chart. There are also plural forms of ‘what’ and ‘who’. Wh-words have the singular/plural class pairing \textit{g/l} used primarily for humans. The words ‘which’ and ‘whose’ also have noun class agreement, shown here as \textit{g/l}, but for these words, noun class can vary depending on the lexical noun, as expected given the structure of genitive constructions. 

\begin{table}
\caption{In-situ wh-words}
\label{Ch19:1}
	\begin{tabular}[t]{lll}
\lsptoprule
& Singular	&	Plural\\
what	&	wánde	&	lánde\\
who	&	ɜʤɜ́ŋgaŋo / ɜʤɜ́	&	ɜʤɜ́lánda\\
which	&	N  ɡáŋɡa	&	N láŋɡa\\
whose	&	N  ɡɜ(n)ʤɜ́	&	N lɜ(n)ʤɜ́\\
where	&	ŋgá	&	n/a\\
when	&	ndóŋ	&	n/a\\
why	&	eðá	&	n/a\\
how	&	(d̪á)t̪áo	&	n/a\\
\lspbottomrule
	\end{tabular}
\end{table}

In conclusion, the ability of wh-elements to appear in-situ depends on their syntactic position: while all non-subject wh-elements may optionally appear in-situ, subject forms cannot. These latter must appear in ex-situ constructions. Consequently, we turn to a discussion of this question formation strategy.


\section{Ex-situ content questions}\label{sec:ch18:ex}

%TODO ADD SECTION 3 OF MOROWHPAPER HERE
Ex-situ question constructions contain a wh-phrase in sentence initial position, followed by a modifying dependent clause. In section 4, we provide arguments that these constructions are best analyzed as clefts. In this section, we simply describe the basic properties of ex-situ wh-question constructions, beginning with subject questions and then turning to non-subject questions. 

\subsection{Subject questions}

Consider the following pairs of sentences, where (10a) and (11a) illustrate declarative clauses, and (10b) and (11b) represent their interrogative analogues with the non-human variant of the wh-element.

\ea
\ea \gll	uɡviə	ɡ-a-sː-ó	uðɜ\\
		\textsc{cl}g.bird	\textsc{sm.cl}-\textsc{rtc}-eat-\textsc{pfv}	\textsc{cl}g.worm\\
\trans		‘A bird ate a worm.’\\
\ex \gll	ŋwɜ́ndə́kːi	ɡ-é-sː-ó	uðɜ?		\\
		what.\textsc{cl}g	\textsc{sm.cl}g-\textsc{dpc}1-hit-\textsc{pfv}	\textsc{cl}g.worm\\
\trans		‘What ate a worm?’\\
\z
\z
\ea
\ea \gll	jáŋála	j-a-t̪ːw-ó	\\	
		\textsc{cl}j.sheep	\textsc{sm.cl}j-\textsc{rtc}-get.lost-\textsc{pfv}	\\
\trans		‘The sheep got lost’\\
\ex \gll	ŋwɜ́ndə́lːi	l-é-t̪ːw-ó?\\		
		what.\textsc{cl}l	\textsc{sm.cl}l-\textsc{dpc}1-get lost-\textsc{pfv}	\\
\trans		‘What (plural) got lost?’	\\
\z
\z

These ex-situ questions are the only allowable means for forming a subject question: no in-situ subject question strategy is available. Note that for the interrogatives in (10b) and (11b), the verbal prefix \textit{é-}, glossed as \textsc{Dependent Clause} 1 (\textsc{dpc}1), is observed, as opposed to the \textit{a-} verbal prefix seen in the declaratives in (10a) and (11a). The wh-expression \textit{ŋwɜ́ndə́kːi} ‘what’, which appears in clause-initial position in (10b) can be decomposed into the prefix \textit{ŋwə́-}, the word \textit{wánde} ‘what’, and the demonstrative \textit{-íkːi}. Note, however, that the vowel /a/ of \textit{wánde} has been raised to [ɜ]. Typically, \textit{-íkːi} does not trigger vowel raising on a root. The occurrence of vowel harmony in this case, however, serves as an indication that the word has become lexicalized. (Height harmony in Moro raises /e a o/ to [i ɜ u] respectively.) The [i] of the demonstrative regularly fuses with the final vowel of the stem (Strabone and Rose 2012), and in this case is reduced to [ə]. The word \textit{ŋwɜ́ndə́lːi} in (11b) is the plural form of ‘what’; plurality is expressed by the noun class of the demonstrative \textit{-ílːi} and the noun class subject agreement on the verb.
The sentences below illustrate a declarative sentence and a corresponding subject wh-question containing the human wh-question form ‘who’ \textit{ŋwɜ́ʤɜ́kːi}. 

\ea
\ea \gll	ŋeɾá       	ŋ-a-sː-at̪-ə́-ɲé	áʧə́váŋ	\\
\textsc{cl}ŋ.child	\textsc{sm.cl}ŋ-\textsc{rtc}-eat-\textsc{loc}.\textsc{appl}-\textsc{pfv}-1\textsc{sgom}	\textsc{cl}g.food  \\
\trans ‘A girl ate my food.’\\
\ex \gll	ŋwɜ́ʤɜ́kːi   ɡ-é-sː-at̪-ə́-ɲé	áʧə́váŋ?	\\
\textsc{cl}g.who	\textsc{sm.cl}g-\textsc{dpc}1-eat-\textsc{loc}.\textsc{appl}-\textsc{pfv}-1\textsc{sgom}	\textsc{cl}g.food  \\
\trans ‘Who ate my food?’\\
\z
\z

The word \textit{ŋwɜ́ʤɜ́kːi} in (12b) is composed of \textit{ɜʤɜ́} ‘who’, the prefix \textit{ŋwə́-} (which is responsible for the first high tone on \textit{-ɜ́ʤɜ́-}), and the demonstrative \textit{-íkːi}. 
The same basic ex-situ question strategy obtains for phrasal wh-questions involving ‘which’ and ‘whose’, where the \textit{ŋwə́-} element can be seen marking a lexical noun, without a co-occuring demonstrative (13a-b). In each question, the verb form contains the dependent clause é- prefix on the verb, in this case raised to [í] due to vowel harmony. 

\ea
\ea \gll	 	ŋwə́-ŋeɾá  [ŋ́ŋwerá]	ŋ-áŋɡa	ŋ-í-t̪únd̪-ɜ?\\
		\textsc{cl}f-\textsc{cl}ŋ.girl	\textsc{cl}ŋ-which	\textsc{sm.cl}ŋ-\textsc{dpc}1-cough-\textsc{ipfv}\\
\trans		‘Which girl is coughing?’\\
\ex \gll	ŋwə́-ŋeɾá	ŋ-ɜ́(n)ʤɜ	ŋ-í-t̪únd̪-ɜ?\\	            
		\textsc{cl}f-\textsc{cl}ŋ.girl	\textsc{cl}ŋ-who	\textsc{sm.cl}ŋ-\textsc{dpc}1-cough-\textsc{ipfv}\\
\trans		‘Whose girl is coughing?’\\
\z
\z

In sum, irrespective of the structural status of the wh-element as head of an NP or modifier, subject wh-phrases obligatorily appear ex-situ. For modified wh-phrases, the question word may appear with a prefix \textit{ŋwə́-} in one variant or with a demonstrative suffix in another, but the verb is always marked by a dependent clause prefix \textit{é-}. 

\subsection{Non-subject questions}

We have already seen how objects and adverbials behave in in-situ question formation. In this section we examine the varieties of non-subject wh-questions that also permit ex-situ wh-constructions. 

\subsubsection{Object questions}
Object ex-situ question words appear in clause-initial position. Wh-phrases in this position are prefixed with \textit{ŋwə́-} and suffixed with the demonstrative \textit{-íkːi}. While they share these characteristics with subject questions, two additional properties are unique to non-subject questions: 1) a prefix \textit{ə́-} between the subject class marker and the verb root, and 2) an optional complementizer \textit{nə́-} on the subject, verb, or both (see section 5.3 for further analysis). We take the prefix \textit{ə́-} to be a second type of dependent clause marker (\textsc{dpc}2), used for non-subject wh-question constructions, alternating with \textit{é-} which marks subject questions (see section 4.4 for further discussion of these prefixes). The prefix \textit{ə́-} marks non-subject wh-questions, rather than objects, since verbs occurring with adverbial question words also show the same prefix. In each of the examples below, an in-situ question is contrasted with the ex-situ version (those in (14) are repeated from (1b,c)): 

\ea
\ea \gll	kúku	ɡ-a-sː-ó	wánde?\\
	\textsc{cl}g.Kuku	\textsc{sm.cl}g-\textsc{rtc}-eat-\textsc{pfv}   	\textsc{cl}g.what\\
\trans	‘What did Kuku eat?’\\
\ex \gll	ŋwɜ́ndə́kːi	(nə́-)kúku     	(nə́-)ɡ-ə́-sː-ó?                \\
	what.\textsc{cl}g	(\textsc{comp-})Kuku  	(\textsc{comp-})\textsc{sm.cl}g-\textsc{dpc}2-eat-\textsc{pfv}  \\
\trans	‘What did Kuku eat?’\\
\z
\z

The in-situ question has the root clause prefix \textit{a-} on the verb, whereas the ex-situ question has the prefix \textit{ə́-}. In addition, the subject and the verb in the ex-situ question are optionally marked with the particle \textit{nə́-} in (14b). The wh-word \textit{wánde} ‘what’ occurs in the in-situ question, but is additionally marked with \textit{ŋwə́-} and with the demonstrative pronoun in the ex-situ question. Although we have argued that it is morphologically complex, we gloss \textit{ŋwɜ́ndə́kːi} here as ‘what’, only indicating its noun class, for ease of exposition.

%Note extraction of comitative objects is impossible; extraction of the subject is required:

\ea \ea \gll  ŋw-ɜ́dʒ-ɜ́ki l-é-rl-ó Kúk:u-ga\\
\textsc{foc}-who-\textsc{scl}g \textsc{cl}l-\textsc{dpc1}-walk-\textsc{pfv} Kuku-\textsc{inst.cl}g\\
\glt `Who walked with Kuku?'
\ex[*]{ \gll  ŋw-ɜ́dʒ-ɜ́ki nə-Kúk:u l-erl-é-ja\\
\textsc{foc}-who-\textsc{scl}g \textsc{comp2}-Kuku \textsc{cl}l-\textsc{dpc2}-walk-\textsc{pfv}=\textsc{inst}\\
\glt `Who did Kuku walk with?'}
\z 
\z 

\subsubsection{Adverbial wh-questions}
The adverbial question words ‘when’, ‘where’, ‘how’, and ‘why’ can also occur in ex-situ constructions. The word ‘when’ may or may not be preceded by \textit{ŋwə́-}, the cleft element. However, irrespective of the presence of \textit{ŋwə́-} the non-subject dependent clause prefix \textit{ə́-} appears on the verb (except if the verb stem is vowel-initial), and \textit{nə́-} optionally occurs on the subject and verb. 

\ea
\ea
\gll	ópːó	ɡ-a-vədað-ó	eɡea        ŋópéa   ndóŋ?\\
	\textsc{cl}g.grandmother	\textsc{sm.cl}g-\textsc{rtc}-clean-\textsc{pfv}	\textsc{cl}g.house	well	when?\\
\trans		‘When did Grandmother clean the house thoroughly?’\\
\ex
\gll	(ŋwə́-)ndóŋ	(n-)ópːó	(nə́-)ɡ-ə́-vədað-ó eɡea	ŋópéa?  \\               
	\textsc{cl}f-when	(\textsc{comp-})\textsc{cl}g.grandmother	(\textsc{comp-})\textsc{sm.cl}g-\textsc{dpc}2-clean-\textsc{pfv} 
\textsc{cl}g.house	well?\\
\trans		‘When did Grandmother clean the house thoroughly?’\\
\z
\z

As for the locative adverbial question element ‘where’, it can also appear in ex-situ position. If it does, the cleft element \textit{ŋwə́-} is obligatory, and it is accompanied by all of the concomitant characteristics of ex-situ questions (\textit{ŋwə́-ŋɡa}=[ŋŋ́ɡwa])       

\ea
\ea \gll	k-afː-ó	eɡea	ŋɡá?\\
		\textsc{sm.cl}g-build-\textsc{pfv}	\textsc{cl}g.house	where\\
\trans		‘Where did s/he build the house?’\\
\ex \gll	ŋŋ́ɡwa	(nə́-)ɡ-áfː-ó-u	eɡea?\\
		\textsc{cl}f.where	(\textsc{comp-})\textsc{sm.cl}g-build-\textsc{pfv}-loc	\textsc{cl}g.house\\
\trans		‘Where did s/he build the house?’\\
\z
\z

Ex-situ questions with the manner adverbial constituents ‘how’ also appear with an obligatory \textit{ŋwə́-} marker prefixed to a shorter version of \textit{d̪at̪áo}, the form that appears in in-situ questions. Note that the complementizer \textit{nə́-} does not appear in this example due to a phonological constraint against /n(ə)-l/ sequences (Gibbard et al. 2009; Jenks 2013). 

\ea
\ea \gll		lədʒí	l-a-dat-togat̪-ó	eɡea	d̪át̪áo  \\
	\textsc{cl}l.person	\textsc{sm.cl}l-\textsc{iter}-repair-\textsc{pfv}	\textsc{cl}g.house	how  \\
\trans	‘How did the people repair the house?’\\
\ex \gll	 ŋwə́-t̪áo	lədʒí	l-ə́-dat-togat̪-ó	eɡea?\\
	\textsc{cl}f-how	\textsc{cl}l.person	\textsc{sm.cl}l-\textsc{dpc}2-\textsc{iter}-repair-\textsc{pfv}	\textsc{cl}g.house\\
\trans	‘How did the people repair the house?’\\
\z
\z

As for interrogatives requesting causal explanations with ‘why’, these may be formed as ex-situ structures, but there is no occurrence of \textit{ŋwə́-}. In ‘why’ questions, the verb displays the typical ex-situ form with the dependent clause prefix \textit{ə́-}, while the subject and verb can optionally host a \textit{nə́-} element.

\ea
\ea \gll	ówːá	ɡ-oás-a	ndréð	eðá	ŋínɜ́ŋí?\\
	\textsc{cl}g.woman	\textsc{sm.cl}g-wash-\textsc{ipfv}	\textsc{cl}n.clothes	why	today\\
\trans		Why is the woman/wife washing clothes today?\\
\ex \gll	eðá	(n-)ówːá	(nə́-)ɡ-oás-a	ndréð	ŋínɜ́ŋí?\\
	why	(\textsc{comp-})\textsc{cl}g.woman	(\textsc{comp-})\textsc{sm.cl}g-wash-\textsc{ipfv}	\textsc{cl}n.clothes	today\\
\trans		‘Why is the woman/wife washing clothes today?’\\
\z
\z

To review, there is variability among adverbial wh-elements concerning the occurrence of the \textit{ŋwə́-} marker. It is obligatory with ‘where’ and ‘how’, optional with ‘when’ and disallowed with ‘why’. Furthermore, none of the adverbials bear the demonstrative \textit{-íkːi} found with nominals. Nevertheless, these constituent interrogatives display the same dependent clause marker \textit{ə́-} and optional \textit{nə́-} marking. The following chart summarizes the forms of ex-situ wh-words: 

\begin{table}
	\caption{Ex-situ wh-words}
	\label{Ch19:2}
	\begin{tabular}[t]{lll}
	\lsptoprule
&	Singular	&	Plural\\
what	&	ŋwɜ́ndə́kːi	&	ŋwɜ́ndə́lːi\\
who	&	ŋwɜ́ʤɜ́kːi	&	ŋwɜ́ʤɜlándə́lːi\\
which	&	ŋwə́-N  ɡáŋɡa	&	ŋwə́-N láŋɡa\\
whose	&	ŋwə́-N  ɡɜ(n)ʤɜ́	&	ŋwə́-N lɜ(n)ʤɜ́\\
where	&	ŋŋ́ɡwa	&	n/a\\
when	&	(ŋwə́-)ndóŋ	&	n/a\\
why	&	eðá	&	n/a\\
how	&	ŋwə́t̪áo	&	n/a\\
	\lspbottomrule
	\end{tabular}
\end{table}

%TODO ADD SECTION 5 OF MOROWHPAPER HERE, up to `quotative complementizer ma'

\subsection{Properties specific to non-subject filler-gap constructions}

This section presents more detailed descriptions of three properties which are characteristic of ex-situ wh-questions from non-subject positions. These properties also occur in non-subject relative clauses and clefts, solidifying the relationship between the three constructions. Section 5.1 addresses morphological properties of subject-verb agreement in these clauses which distinguishes them from main clauses. In Section 5.2 the distribution of resumptive pronouns is reviewed, and Section 5.3 presents evidence that the proclitic \textit{nə́-}, which occurs optionally before subjects and verbs in these clauses, is a complementizer.

\subsubsection{Subject agreement and verb prefixes}

When non-subject relatives and ex-situ wh-questions have 3$^{rd}$ person subjects, the verb exhibits noun class agreement followed by the prefix \textit{ə́-}. When the subject of a main clause declarative is 1st or 2nd person a fixed person/number marker is followed by a default class marker \textit{ɡ-} (33a, 34a). However, in ex-situ non-subject questions, 1st and 2nd person subject agreement does not occur with the \textit{ɡ-} class prefix, and there is no evidence for the presence of the dependent clause prefix ə́- either (33b, 34b):

\ea
\ea \gll	á-ɡ-a-wəndat̪-ó	náláɲá\\
		2\textsc{sgsm-cl}g-\textsc{rtc}-see-\textsc{pfv}	\textsc{cl}n.red ant\\
\trans		‘You saw the red ants.’\\
\ex \gll	ŋwɜ́ndə́kːi	(n-)á-wəndat̪-ó?\\
		\textsc{cl}g.what	(\textsc{comp-})2\textsc{sgsm}-see-\textsc{pfv}\\
\trans		‘What did you see?’   	\\
\z
\z

\ea
\ea	\gll	ɲá-ɡ-a-və́dáð-a	ɜdnə́-ɡá    \\                    
		2\textsc{plsm-cl}g-\textsc{rtc}-clean-\textsc{ipfv}	\textsc{cl}g.young mother-\textsc{cl}g.\textsc{inst}\\
\trans		‘You (all) are cleaning with the young woman.’\\
\ex	\gll	ŋwɜ́ʤɜ́kːi	(nə́-)ɲá-və́dáð-a	lə́kːa?\\
		\textsc{cl}g.who	(\textsc{comp-})2\textsc{plsm}-clean-\textsc{ipfv}	together(dual)\\
\trans		‘Who are you (all) cleaning with?’\\
\z
\z

It is not immediately clear if the dependent clause \textit{ə́-} prefix is morphologically absent in these forms or deleted due to vowel hiatus resolution. Since all non-3rd person subject marker prefixes end in a vowel, the absence of the default class marker \textit{ɡ-} leads to vowel hiatus. Although usually the first of two vowels is deleted in vowel hiatus in Moro, if a schwa is one of the vowels, schwa is preferentially deleted.  Thus, /á-ə́-wəndat̪-ó/ would reduce to [áwəndat̪ó] (cf. 33b). The only clue as to the presence of \textit{ə́-} might be the preservation of its tone. The high tone cannot migrate leftwards as the subject prefix is high-toned already, but it also fails to appear on the first vowel of the root:  *[áwə́ndat̪ó]. This indicates that the \textit{ə́-} prefix is not morphologically present in these forms.
The same pattern of prefixation occurs with other non-subject ex-situ questions:

\ea
\gll	ŋwə́-t̪áo	(n)-áfː-ó	eɡea?\\
	\textsc{cl}f-how	(\textsc{comp-})2\textsc{sgsm}.build-\textsc{pfv}	\textsc{cl}g.house\\
\trans	‘How did you build the house?’
\z

This subject agreement pattern also occurs in clefts (36a) and relative clauses (36b):

\ea
\ea \gll ŋw-úmːiə́-kːi	(n-)é-wəndat̪-ó				\\
	\textsc{cl}f-boy-\textsc{cl}g.\textsc{dem}	(\textsc{comp-})1\textsc{sgsm}-see-\textsc{pfv}\\
\trans		‘It is the boy that I saw’\\
\gll	umːíə-kːi	(n-)é-wəndat̪-ó	k-ɜ́-sː-iə             	jáŋála  \\
		boy-\textsc{cl}g.\textsc{dem}	(\textsc{comp-})1\textsc{sgsm}-see-\textsc{pfv}	\textsc{sm.cl}g-\textsc{rtc}-eat-\textsc{caus}.\textsc{ipfv}	\textsc{cl}j.sheep\\
\trans	‘The boy I saw is grazing sheep’\\
\z
\z

Consequently, the absence of the dependent clause prefix and default class agreement prefix with 1st and 2nd subjects is one more way that non-subject clefts, relative clauses and ex-situ questions pattern alike.

\subsection{Resumptive markers in ex-situ object constructions}
Another characteristic of non-subject ex-situ questions is resumptive pronouns. Cross-linguistically, resumptive marking is expressed by several different, functionally equivalent, encoding strategies, e.g., independent pronouns, clitics, affixes or other verbal marking (Ariel 1999; Sharvit 1999; Falk, 2002; de Vries 2005; Marten et al. 2007). In Moro, pronominal object markers appear on the verb. In declarative root clauses, object markers cannot co-occur with the lexical NPs with which they co-refer; this also holds for in-situ wh-questions. The fact that object markers can occur in ex-situ wh-questions and clefts thus provides further support (see Section 4.2) that these constructions are biclausal, consisting of a cleft element and a dependent clause.

The person and number features on object markers in Moro reflect the same person and number features which are marked in Moro pronouns and subject agreement, including inclusive/exclusive 1st plural and dual forms. Their distribution is complex and correlates with tone (Rose 2013). Here we illustrate only the third person singular forms.

The pattern of object marking with ex-situ object questions parallels pronominal object marking more generally: a resumptive third person singular pronoun occurs with human objects (37b), but not with non-human singulars (37a).


\ea
\ea \gll	ŋwɜ́ndə́kːi	(n-)úmːiə	(nə́-)ɡ-ə́-ləvəʧ-ó?	\\
		\textsc{cl}g.what	(\textsc{comp-})\textsc{cl}g.boy	(\textsc{comp-})\textsc{sm.cl}g-\textsc{dpc}2-hide-\textsc{pfv}\\
\trans		‘What did the boy hide?’\\
\ex \gll	ŋwɜ́ʤɜ́kːi	(n-)úmːiə	(nə́-)ɡ-ə́-ləvəʧ-ó-ŋó?\\
		\textsc{cl}g.who	(\textsc{comp-})\textsc{cl}g.boy	(\textsc{comp-})\textsc{sm.cl}g-\textsc{dpc}2-hide-\textsc{pfv}-3\textsc{sgom}\\
\trans		Who did the boy hide?\\
\z
\z

The 3pl object marker \textit{-lo} is used with plural objects regardless of animacy or human status. In (38a), the plural form of the cleft wh-word appears, and \textit{-lo} occurs on the verb. 

\ea
\ea \gll	ŋwɜ́ndə́lːi	(nə́-)kúku	(nə́-)ɡ-ə́-t̪að-ó-lo?\\
	\textsc{cl}l.what	(\textsc{comp-})Kuku	(\textsc{comp-})\textsc{sm.cl}g-\textsc{dpc}2-leave-\textsc{pfv}-3\textsc{plom}\\
\trans		‘What (pl.) did Kuku leave?			 \\
\ex \gll	ŋwɜ́ʤɜlándə́lːi	(nə́-)kúku	(nə́-)ɡ-ə́-t̪að-ó-lo?\\
	\textsc{cl}l.who	(\textsc{comp-})Kuku  (\textsc{comp-})\textsc{sm.cl}g-\textsc{dpc}2-leave-\textsc{pfv}-3\textsc{plom}\\
\trans		‘Who (pl.) did Kuku leave?’	\\
\z
\z

Object questions with ‘which’ and ‘whose’ show a similar pattern.  Resumptive pronouns occur with extracted plurals regardless of animacy or humanness, and resumptive pronouns can occur with singular wh-phrases, but are optional (39c): 

\ea
\ea \gll	ŋw-ðoála	ð-aŋɡa	(nə́-)kúku	(nə́-)ɡ-ə́-t̪að-ó?\\
		\textsc{cl}f-\textsc{cl}ð.livestock	\textsc{cl}ð-which	(\textsc{comp-})Kuku	(\textsc{comp-})\textsc{sm.cl}g-\textsc{dpc}2-leave-\textsc{pfv}\\
\trans		 ‘Which livestock did Kuku leave behind?’\\
\ex \gll	ŋw-íɾiə	j-aŋɡa	nə́-kúku	(nə́-)ɡ-ə́-t̪að-ə́-lo?	\\
		\textsc{cl}f-\textsc{cl}j.cows	\textsc{cl}j-which	(\textsc{comp-})Kuku	(\textsc{comp-})\textsc{sm.cl}g-\textsc{dpc}2-leave-\textsc{pfv}-3\textsc{plom}\\
\trans		‘Which cows did Kuku leave behind?’\\
\ex \gll	ŋw-úmːiə	ɡ-aŋɡa	(nə́-)kúku	(nə́-)ɡ-ə́-t̪að-ó(-ŋó)?\\
		\textsc{cl}f-\textsc{cl}g.boy	\textsc{cl}g-which	(\textsc{comp-})Kuku	(\textsc{comp-})\textsc{sm.cl}g-\textsc{dpc}2-leave-\textsc{pfv}(-3\textsc{sgom})\\
\trans		‘Which boy did Kuku leave behind?’\\
\z
\z

The distribution of plural resumptive pronouns in clefts and relative clauses is the same as for ex-situ questions: they are required in all three constructions. However, there are some differences with respect to singular resumptive pronouns. In all three constructions, singular resumptive pronouns refer only to humans. In ex-situ questions, resumptive pronouns are optional with human objects in general. In relative clauses, singular resumptive pronouns are restricted to proper names. In clefts singular resumptive pronouns occur with proper names and independent pronouns. Despite these specific restrictions, the occurrence of resumptive pronouns in all three filler-gap constructions provides further evidence for biclausality as object pronouns are elsewhere prohibited with clausemate lexical NPs.

\subsection{The complementizer nə́-}
The last aspect of non-subject wh-constructions that requires further analysis is the use of the particle \textit{nə́-}, which can appear optionally at various positions within the filler-gap domain. To establish the role of \textit{nə́-} in dependent clauses, we compare its distribution with that of the complementizer \textit{t̪á}, and conclude that \textit{nə́-}, too, is a complementizer.

The particle \textit{nə́-} appears optionally on the subject and/or the verb. It can also appear on the clause-level adverb \textit{bə́té} ‘never’ for two out of the three speakers consulted, but Angelo Naser, who rejects this, prefers \textit{bə́té} to appear sentence finally. Example (14b), repeated here as (40), shows the particle appearing on the subject and the verb.  Example (41b) shows the particle on the adverb ‘never’ as well.

\ea
\gll	ŋwɜ́ndə́kːi	(nə́-)kúku	(nə́-)ɡ-ə́-sː-ó?\\
	\textsc{cl}g.what	(\textsc{comp-})\textsc{cl}g.Kuku	(\textsc{comp-})-\textsc{sm.cl}g-\textsc{dpc}2-eat-\textsc{pfv}	\\
\trans	‘What did Kuku eat?’\\
\z

\ea
\ea \gll	bə́té	ɲá-ɡ-!ánː-a	ɲá-bəlw-a          	kúku-ɡa\\
	never   1\textsc{plexc.sm}-\textsc{cl}g.\textsc{rtc}-\textsc{neg}-\textsc{ipfv} 1\textsc{plexc.sm}-wrestle-\textsc{inf}	Kuku-\textsc{inst}\\
\trans		‘We never wrestle with Kuku.’\\
\ex \gll	ŋwɜ́dʒɜ́ki	(nə́-)bə́té	(nə́-)ɲ-ánː-a	 (nə́-)ɲá-bə́lw-á	lə́kːa?  	   \\
		    	who	        (\textsc{comp-})never     (comp)-2\textsc{plsm}-\textsc{neg}-\textsc{ipfv}-sub	
			(comp)-2\textsc{plsm}-wrestle-\textsc{inf}	together(dual)\\
\trans		‘Who do you never wrestle with?’
\z
\z

First, consider the distribution of \textit{nə́-} in a variety of constructions. It appears not only in non-subject filler-gap constructions as in (41), but also in complement clauses, i.e. clauses with \textit{a-} and \textit{ə́-} clause markers, as discussed in section 4.4.  Depending on the verb, such clauses permit the \textit{nə́-} complementizer or else require the \textit{t̪á} complementizer.  The particles \textit{nə́-} and \textit{t̪á} never co-occur.  In addition, dependent clauses in which the \textit{t̪á} complementizer never appears are likewise places in which \textit{nə́-} is unattested:  subject filler-gap constructions (wh-questions, clefts, and relative clauses), as well as for the complement clauses and adjunct clauses illustrated in section 4.4. 

Second, \textit{nə́-} has a similar distribution in clefts and in dependent clauses (non-subject filler-gap constructions, adjunct clauses, and in the complement clause of ‘refuse’). In both cases, it occurs as a proclitic on the subject or the verb. Furthermore, it is optional. 

Third, if a non-subject element of a dependent clause is questioned with a wh-cleft, the \textit{nə́-} can appear in the dependent clause, but only in limited circumstances: i) in complements that are normally marked with \textit{a-} in declaratives and ii) if there is no other complementizer present in the dependent clause. Otherwise, the verb morphology associated with an ex-situ question appears only on the verb of the main clause. In (42), the main clause verb \textit{nː} ‘hear’ (in the sense of informed) selects a complement clause with \textit{t̪á} and a verb that is prefixed with root clause \textit{a-} ([ɜ] due to vowel harmony). In the wh-cleft question in (43), the \textit{nə́-} appears only on the main verb, not on the dependent clause. The main verb bears the verb morphology of an ex-situ non-subject question: it lacks the default class marker \textit{g-} and the \textit{ə́-} (see section 5.1). The lower verb is unaltered morphologically, except for the fact that it bears a resumptive pronoun \textit{-ŋó}.

\ea
\ea \gll	é-ɡ-a-nː-ó	t̪á	kúku	ɡ-ɜ-bəɡ-ú	bitər(-o)? \\
   1\textsc{sgsm-cl}g-hear-\textsc{pfv}	\textsc{comp}	\textsc{cl}g.Kuku	\textsc{sm.cl}g-\textsc{rtc}-hit-\textsc{pfv}   	Peter(-oc)\\
\trans	‘I heard that Kuku hit Peter’\\
\ex \gll	ŋwɜ́ʤɜ́kːi	(n-)á-nː-ó	t̪á	kúku		ɡ-ɜ-bəɡ-ó-ŋó? \\
   \textsc{cl}g.who	(\textsc{comp-})2\textsc{sgsm}-hear-\textsc{pfv}	\textsc{comp} \textsc{cl}g.Kuku	\textsc{sm.cl}g-\textsc{rtc}-hit-\textsc{pfv}-3\textsc{sgom}\\
\trans	‘Who did you hear that Kuku hit?’\\
\z
\z

In contrast, the verb \textit{at̪} ‘think’, does not select a complement clause with \textit{t̪á} (43). In this case, when the object is questioned, the embedded verb is marked with \textsc{dpc}2 and \textit{nə́-} marking can appear in both the matrix and subordinate clauses, as shown in (44).

\ea
\gll	nána	ɡ-at̪-a	bitər	ɡ-a-sː-ó	ləbəmbɜ́j\\
mama	\textsc{sm.cl}g-think-\textsc{ipfv}	Peter	\textsc{sm.cl}g-\textsc{rtc}-eat-\textsc{pfv}	\textsc{cl}l.yam\\
\trans ‘Mama thinks that Peter ate a yam’\\
\z

\ea
\gll	ŋwɜ́ndə́kːi	(nə́-)nána	(nə́-)ɡ-at̪-a	bitər (nə́-)ɡ-ə́-sː-ó \\
what	(\textsc{comp-})mama	(\textsc{comp-})\textsc{sm.cl}g-think-\textsc{ipfv}	Peter (\textsc{comp-})\textsc{sm.cl}g-\textsc{dpc}2-eat-\textsc{pfv}\\
\trans ‘What did Mama think that Peter ate?’\\
\z

All these factors point to an analysis of \textit{nə́-} as a complementizer. It typically co-occurs with \textit{ə́-} in a variety of constructions, not just those that exhibit filler-gap relationships. The \textit{nə́-} is obligatory when the verb is in the infinitive form (with alternate subject marking), but is otherwise optional, and when optional can appear cliticized on either the subject (as the first element in the clause) or the verb or both. Furthermore, it cannot co-occur with another complementizer. Its phonological form is that of a clitic. Moro does not allow words that end in [ə], and so all consonant-only or Cə morphemes cannot be free. In contrast the complementizer \textit{t̪á} can occur as a separate functional word, as can the quotative complementizer \textit{ma}.

\subsection{Agreeing how}\label{how}

%TODO add from notes

\section{Conditional constructions}

%TODO add from notes