\chapter{Clausal syntax}\label{chapter:syntax}

This chapter describes the syntax of indicative verbal clauses, which are clauses where a verb serves as the main predicate rather than a non-verbal predicate or a copular clause. This chapter is restricted to simple clauses, those with a single verb and with no auxiliary. The basic syntax of such clauses is Subject-Verb-Object-Adverb, as in the following example:

While there can be multiple objects and the distribution of adverbs is quite free. Nevertheless, a look at simple transitive clauses in texts reveal that the vast majority of the time, objects are immediately preverbal and adverbs are sentence-final. %todo texts how often?

Moro clauses show nominative-accusative alignment both in terms of their syntax and their morphology. Thus, verbs agree with the sole argument of intransitive verbs as well as the agent of transitive verbs. These arguments pattern alike syntactically as well, occurring before the verb and serving a distinguished role in a number of syntactic and morphological processes such as passivization and coordination.

This chapter describes the morphosyntactic properties of the four classes of syntactic elements which occupy such clauses and their interactions in their typical order of appearance: subjects, verbs, objects, and adverbs. The focus of this chapter is primarily syntactic, with emphasis on word order, valence alternations, and the distribution of arguments versus adjuncts.

\section{Subjects}\label{sec:ch12:subjects}

Subjects in Moro are identified with the following properties. First, the subject is always preverbal, typically immediately before the verb (\ref{ex:ch12:1}):  %check can adverbs appear in this position

\ea Subjects must be preverbal \label{ex:ch12:1}
\ea ŋerá ŋ-ʌ-túnd-ʌ\\
girl CLŋ-RTC-cough-IPFV\\
`The girl is coughing'
\ex *ŋ-ʌ-túnd-ʌ ŋerá\\
CLŋ-RTC-cough-IPFV girl\\
`The girl is coughing'
\z 
\z 

Second, subjects always trigger verb agreement, regardless of whether there is an overt noun phrase before the verb (Section \ref{sec:ch12:agreement}). Third, the subject appears in nominative case, visible only on proper human nouns, which is unmarked (Section \label{ex:ch12:2}):

\ea Subjects must be nominative \label{ex:ch12:2}\\
\ea \gll Kúku g-ʌ-túnd-ʌ\\
Kuku CLŋ-RTC-cough-IPFV\\
\glt `Kuku is coughing.'
\ex \gll *Kúku-ŋ g-ʌ-túnd-ʌ\\
Kuku-\textsc{acc} CLŋ-RTC-cough-IPFV\\
\glt `Kuku is coughing.'
\z 
\z 

Fourth, in-situ content question words cannot appear in subject position, while they can occur object or adjunct positions (\citealt{roseetal14}, Section \ref{sec:ch12:contentquestion}): 

\ea No in-situ subject wh-question \label{ex:ch12:3}
\ea \gll ʌ́nʤʌ	ŋ-í-t̪únd̪-ʌ?	            \\
		CLF-CLŋ.girl	CLŋ-who	SM.CLŋ-DPC1-cough-IPFV\\
    \glt `Who is coughing' (Intended)
\z 
\z 

Fifth, subjects have a privileged role for purposes of binding. Subjects must be the antecedent in reflexive and reciprocal binding (Section \ref{sec:ch12:passive}, Section \ref{sec:ch12:antipassive}), 

\ea Reflexives are subject oriented
\ea Subject as antecedent in reflexive
\ex object cannot be antecedent in reflexive
\z 
\z 

\ea Reciprocals are subject oriented
\ea subject as antecedent in reciprocal
\ex object cannot be antecedent in reciprocal
\z 
\z 

Subjects also can be the antecedent in cases of semantic binding or variable binding, although in these cases of variable binding objects can bind one another (Section \ref{sec:ch12:objects}):

\ea Variable binding from subject position
\z 

Sixth, subject position is privileged for the purposes of a number of valence affecting processes. The passive suffix \textit{-en} decreases the valence of a verb by promoting an object to subject position while suppressing the subject of its active counterpart. Passive \textit{-en} also occurs in the context of body-affecting actions only when these are committed by the subject (Section \ref{sec:ch12:passive}). Similarly, causatives demote their non-causative counterpart to object position and add a new subject, a causer (\ref{sec:ch12:causative}).

Seventh, subjects are the target of syntactic raising and control constructions (Section \ref{section:nonfinite}). 

\section{Verb classes and valence alternations}\label{sec:ch12:verbs}

- discussions of different classes of verbs: unaccusative: 
- appearance, disappearance
- classes of labile verbs
 change of location, change of state

- unergative
- transitive
SO MANY CLASSES...how do I start to go about this? Which ones can undergo causative and which can't? which can take applicatives and which can't?
- ditransitive 


\section{Objects}\label{sec:ch12:objects}

\section{Basic properties of objects}

Properties of objects:
- Objects follow verbs
- Objects are pronominalized as incorporated pronouns or object markers
- Objects can be passivized (or reflexivized? or antipassivized?)

Obligatoriness:
- Non-human objects have a null pronominal variant so they appear optional in many texts
- Human objects are obligatorily expressed either as noun phrases or object markers

Objects are symmetrical in Moro according to a number of tests (ackerman et al.):
ordering, pronouns, passives, 

Nevertheless, there are a number of asymmetries: 
person/number, binding

\section{Valence increasing alternations}\label{sec:ch12:increasing}

\subsection{Causatives}\label{sec:ch12:applicative}

- restrictions on distribution..., periphrastic vs. applied causative

\subsection{Applicative objects}\label{sec:ch12:applicative}

- benefactive applicative objects

- locative applicative objects

\section{Valence decreasing alternations}\label{sec:ch12:decreasing}

Two verbal extension suffixed decrease the valence of the verb.

\subsection{Passives and reflexives}\label{sec:ch12:passive}

- Passives
- Body-affected action/possessor raising
- Reflexives


%IDIOMS RETAINED IN PASSIVE
11. égandró nano	‘I’m sleeping on it.’/covering/hiding it.’

12. égandró ŋén nano	‘I’m sleeping on the word.’ = ‘I didn’t tell the truth.’

13. ŋén ŋʌndranú nano	‘A lie was told.’

14. ŋén ŋʌndranúu nano	‘A lie was told (there).’ (?)

ígʌŋʌtʃú umiə adama					‘I told the boy about the book (?)
ígʌŋgitú lʌmia ododo llʌsetʃsi ntam enen ek-almiraja	‘I let each boy see himself in the miror.’
ígʌŋʌtʃú lʌmiə ntam en-en				‘I told each boy about themselves.’
*ígʌŋʌtʃú ntam en-en lʌmiə				
 binding which is independent of passive extension suffix
lʌmiə lʌŋʌtʃənú ntam enen				‘The boys were told about themselves.’
*ntam enen lemmiə lʌŋʌtʃənú		AN1115		
 
lemmiə ododo lʌsʌdʒʌtʃinú enega dəŋgen		‘Each boy was seen in his house.’

Passivization enabling binding; passive not occurring in binding between objects.
 

\subsection{Antipassives and reciprocals}\label{sec:ch12:antipassive}

- Antipassivization
- Reciprocals
- Pluractionality?

\section{Instrumental objects}


Instrumental adjuncts must occur after themes:

\ea \ea \gll Kúku g-ɜ-f:-ət-ú ŋálo-ŋ gəla loandra-la\\
K. \textsc{cl}g-\textsc{rtc}-shoot-\textsc{appl-pfv} Ngalo-\textsc{acc} plate stone-\textsc{cl}l.with\\ 
\glt `Kuku shot the plate with the stone'
\ex[*]{Kúku g-ɜ-f:-ət-ú ŋálo-ŋ loandra-la gəla  }
\z 
\z 

- instrument? comitative? 
- take the instrumental suffix
- trigger the instrumental clitic on the verb in extraction contexts

\section{Locative objects}\label{sec:ch12:locobj}

- covers any locative object of the verb that is not a object of a directional verb (e.g. go to X) or an applied locative object.
- triggers locative cliticization on the verb


\section{Discontinuous constituents in the clause}

`discontinuous constituents' in the clause including 1) extraposition 2) secondary predicates and 3) floating quantifiers)

\subsection{Extraposition}

(is this necessary? can't any of the classes be locative?)

\subsection{Quantifier float}

(does this need to be different from floating quantifiers, below?)

\subsection{Secondary predicates}


\section{Ellipsis in the clause?}

Can auxiliaries delete their VP complements?