\part{The sound system}

\chapter{Segmental phonetics and phonology}

This chapter adapter outlines the phonetics and phonology of consonants and vowels in Thetogovela Moro. 

\section{Vowels}

Thetogovela Moro contrasts the vowels in \ref{tab:ch2:1}

\begin{table}
\begin{tabular}{cc}
	\begin{minipage}{.5\linewidth}
	\centering
  \begin{tabular}{llll}
    \lsptoprule
    front  & central  & back & \\
    \midrule
	i 	& 		& 	u 	& 	high \\
		& 	ɘ	& 		& 	high-mid\\
	e 	&	ɜ	&  	o	& 	mid \\
		&  	ə	& 		&	mid-low \\
		& 	a	&		&	low \\
\lspbottomrule
  \end{tabular}
  \caption{Vowels of Thetogovela Moro}
  \label{tab:ch2:1}
  \end{minipage} &
  \begin{minipage}{.5\linewidth}
  \centering
	\begin{vowel}
		\putcvowel{i}{1}
		\putcvowel{e}{2}
		\putcvowel{a}{15}
		\putcvowel{o}{7}
		\putcvowel{u}{8}
		\putvowel{\textreve}{49pt}{15pt}
		\putcvowel{\textschwa}{12}
		\putvowel{\textrevepsilon}{50pt}{30pt}
	\end{vowel}
  \end{minipage}
\end{tabular}
\end{table}



The contrast between /ɘ/ and /ə/ is perceptually difficult, and only ə is used in the writing system. In previous research on Moro, only one short central ə was recognized. The distinction between the two vowels is reflected in the vowel harmony system, but few words contain only ə or ɘ. Therefore, the tradition of transcribing only ə will be continued in this book, unless it is necessary to point out the distinction.    

In addition to the contrastive vowels, allophonic variants of /e/ and /i/ are [ɛ] and [ɪ] respectively, while [ɔ] occurs as a variant of both /o/ and /ɘ/. 

\subsection{/i/}
The vowel /i/ is a high front close vowel, pronounced [i]. It may be pronounced [ɪ] in closed syllables or in open syllables between consonants. The following are examples of [i] in different positions in the word. There are no restrictions on the type of consonants that [i] can precede or follow. Examples of /i/ are given in \tabref{tab:ch2:3}.

\begin{table} 
\caption{Examples of /i/}	
 \label{tab:ch2:3}
\begin{tabular}[t]{lp{2cm}lp{2cm}lp{2cm}}
\lsptoprule
\multicolumn{2}{l}{Initial} &  \multicolumn{2}{l}{Medial}	 &	\multicolumn{2}{l}{Final}  \\
\midrule
it̪əlí 	& 	‘year’ 	&  iɾiniə	& ‘snot’ &	umədí   &	‘small ant’\\
idəvíni	& 	‘shoe’	&  lɪ́ŋgwɔ	& ‘frog’ & ə́sːí	& ‘eye’\\
iðú		& ‘breasts’	&  kɘdʒivɘʧənú &	‘s/he forgot’ & ɘðúni	&‘hearthstone’ \\
ikúrkuriə &	‘butterflies’	&  ŋísíə &	‘fever, bile’ & lɜ́ɾí &	‘calf (lower leg)’\\
igəlje	& ‘devil, satan’	& 	kɘɾíðíə	&	‘s/he's about &  lɘ́mí &	‘beard’\\
irpúlɘ	& ‘animal pelts’ & & to change’ & & 		\\

\lspbottomrule	
\end{tabular}
\end{table}


		
\subsection{/u/}
The vowel /u/ is a high back round vowel, pronounced [u]. It can often induce lip rounding or labialization on preceding consonants. There are no restrictions with respect to consonants that precede or follow it.

\begin{table}
\caption{Examples of /u/}
 \label{tab:ch2:4}
\begin{tabular}[t]{lp{4.28cm}lp{4.28cm}}
 \lsptoprule
\multicolumn{2}{l}{Initial} &	\multicolumn{2}{l}{Final}  \\
\midrule
uríθ		&	‘chain’			&	ɘðú	&	‘breast’	\\
umədí	& ‘small biting ant’		& 	umú	 	&‘Arab (perjorative)’\\
uɾi 	&	‘rat’ 			&	gɘvɘgú	&	‘s/he miscarried’\\
ut̪ɘdiə	&	‘grandfather, elder’	&	 t̪íðú	&	‘thread, roll!’\\
undə́r	&	‘backside’		\\
\midrule
 \multicolumn{2}{l}{Medial} & & \\
 \midrule
 ðugi	& ‘wood plank’		&	lə́rúðí	&	‘grape’\\
uməɾtín&	‘co-wife’			&	abugʷala	& ‘papers’\\
ɘrpúla	&‘animal skin’			&	ɘ́lúŋ	&	‘promiscuous person’\\
\lspbottomrule
\end{tabular}
\end{table}


\subsection{/ɜ/}

The vowel /ɜ/ is a mid central unrounded vowel, pronounced [ɜ] or [ɘ]. This vowel is written as ë in Moro orthography. 

\begin{table}
\caption{Examples of /ɜ/}
 \label{tab:ch2:5}
\begin{tabular}[t]{lp{5.5cm}lp{3.85cm}}
 \lsptoprule
\multicolumn{2}{l}{Initial} &	\multicolumn{2}{l}{Final}  \\
\midrule
ɜdniə	&	‘young woman with a few children’	&	ɜtúlɜ	&	‘big spear’ \\
ɜniŋíə	&	‘ear’						&	ðə́wɜ́	&	‘smoke’\\
ɜðú	&	‘breast’		\\
ɜwə́ɾí	&	‘door’		\\
\midrule
 \multicolumn{2}{l}{Medial} & & \\
 \midrule			
lɜ́mí	&	‘beard, chin’	&	ðɜ́ŕtí	&	‘anus, urethra’\\
ŋɜ́və́ní	&	‘blood’		\\
\lspbottomrule
\end{tabular}
\end{table}


Under influence of labialization, the vowel /ɜ/ may be pronounced [ɔ] in the final syllable, as in the examples in (5). We surmise that the vowel is /ɜ/ rather than another vowel for several reasons. Co-occurrence restrictions due to vowel harmony mean that the final vowel must be from the ‘higher’ set of vowels, /i u ɜ ɘ/ which appear elsewhere in the stem. The short central /ə/ and /ɘ/ cannot occur word-finally except as part of the diphthongs iə or eə. The vowel /i/ is compatible with labialization (\textit{áŋə́rə́ðwí} ‘..that he sharpen’ and /u/ usually suppresses underlying labialization (/g-ɜ-tʃəndəŋw-u → [gɜtʃəndəŋú] ‘s/he went down’). This leaves only /ɜ/ as a candidate. Alternations in verbal paradigms confirm the ɜ/ɔ alternation adjacent to labialization: /g-ɜ-ɾəðw-ɜ/ --  [gɜɾə́ðwɜ́ ~  gɜɾə́ðwɔ́] ‘he is sharpening’

\ea \begin{tabular}[t]{llll}
pə́lúŋʷɔ́	& ‘devil, evil eye’ & 	ɜɲɔʷŋ	& ‘mouth’\\
ibɔgʷɔ́	& ‘fog’	& rəmʷɔ & 	‘snake, God’ 	\\
 \end{tabular} \label{ex:ch2:1}
\z


\subsection{/e/}

The vowel /e/ is a mid front unrounded vowel pronounced [e]. It is found in both open and closed syllables. However, it may be pronounced [ɛ] in some closed syllables and before [r].

\begin{table}
\caption{Examples of /e/}
 \label{tab:ch2:6}
\begin{tabular}[t]{lp{3.2cm}lp{5cm}}
 \lsptoprule
\multicolumn{2}{l}{Initial} &	\multicolumn{2}{l}{Final}  \\
\midrule
ebamba	&	‘drum’			&	ole	&	‘sound, voice, words, language’\\
emaðén	&	‘same age peer’ 	&	ome	&	‘fish’\\
evaja	&	‘poor person’		&	eləŋe&	‘king, leader’\\
ɛlːe		&	‘feather, wing’		&	ɽdré&	‘earth, ground’\\
eréθ	&	‘clothing’			&	ɘniŋé&	‘ear’\\
\midrule
 \multicolumn{2}{l}{Medial} & & \\
 \midrule		
ðəbérá	&	‘air, wind’		&	údáɾén&	‘uncle’\\
ogowélá	&	‘monkey’			&	alét̪a	&	‘wall’\\
aveja	&	‘liver’			&	léɲá	&	‘egg, penis’\\
ŋerá	&	‘girl, child’		&	ləpɛ́r	&	‘tail’\\
\lspbottomrule
\end{tabular}
\end{table}


Examples of [ɛ] in closed syllables and before [r]. Note that these pronunciations are variable, and [e] is also acceptable. 

\ea \begin{tabular}[t]{llll}
	
wɛn		&	‘liar’			&	ɛ́réká	&	‘yesterday’\\
ɛlːe	&	‘feather, wing’	&	ləpɛ́r	&	‘tail’
\end{tabular} \label{ex:ch2:2}
\z

The vowel /e/ also triggers secondary palatalization of preceding consonants, as in \textit{ondəðé} [ondəðjé] ‘louse’ In the Werria dialect of Moro, final /e/ is pronounced as a diphthong and is written ea in the orthography. 

\subsection{/o/}

The vowel /o/ is a mid back rounded vowel pronounced [o]. However, it can be pronounced [ɔ] in closed syllables and before [r].

\begin{table} 
\caption{Examples of /o/}	
 \label{tab:ch2:7}
\begin{tabular}[t]{lp{3.5cm}lp{3.5cm}}
\lsptoprule
\multicolumn{2}{l}{Initial} &	\multicolumn{2}{l}{Final}  \\
\midrule
ópá	& ‘grandmother’ & 	ŋombogó	& ‘calf (baby cow)’ \\
ómóná & 	‘tiger, big cat’ & 	rːlo	& ‘female goat’ \\
ot̪ə́mba & 	‘ostrich’ & 	áɾó	& ‘cry, howl!’\\
ondəðé	& ‘louse’ & 	kalaŋó & 	‘s/he sang’\\
ogowélá	& ‘monkey’	& kobəðó & 	‘s/he ran away’\\
\midrule
 \multicolumn{2}{l}{Medial} & & \\
 \midrule 
ðórə́ná	& ‘dust’ & 	ndogá & 	`stick for lower lip'\\
paɖóla	& 	‘jute’ & 	lókógóŋ	& 	‘scorpion’ \\
ðopa	& ‘star’& 	ɽdóŋ	& ‘occipital bun’\\
aróbá	& ‘whey’ & 	ləbopwá	& ‘mushroom’\\
agɔ́ l & ‘two lower teeth removed for beauty’ & ɔrːáɲ	 & ‘my sibling/cousin’ \\
\lspbottomrule	
\end{tabular}
\end{table}

The last two examples illustrate pronunciation of /o/ as [ɔ].

\subsection{/a/}

The vowel /a/ is a low central unrounded vowel, pronounced [a] in all contexts. 

\begin{table} 
\caption{Examples of /a/}	
 \label{tab:ch2:8}
\begin{tabular}[t]{lp{3.5cm}lp{3.5cm}}
\lsptoprule
\multicolumn{2}{l}{Initial} &	\multicolumn{2}{l}{Final}  \\
\midrule
áɾó	&	‘cry, howl!’						&	ðórə́ná		&	‘dust’\\
agɔ́l		&	‘two lower teeth removed for beauty’	&	paɖóla		&	‘jute’\\
ád̪ámá	&	‘book’							&	ləbopwá		&	‘mushroom’\\
ándə́mé	&	‘flea’							&	aróbá		&	‘whey’\\
áwáná	&	‘sugar cane’						&	ndogá		&	‘stick inserted under lower lip’\\
\midrule
 \multicolumn{2}{l}{Medial} & & \\
 \midrule 			
váðó	&	‘shave!’			&	ŋáŕlá	&	‘spear’\\
gálá	&	‘bead’			&	ləŋáθ	&	‘tooth’\\
logopájá	&	‘cup’			&	máŋga	&	‘mango’\\
lamámá	&	‘dove, bathroom’	&	ɔrːáɲ	&	‘gentleman/man’\\
\lspbottomrule	
\end{tabular}
\end{table}


\subsection{/ə/}

The vowel /ə/ is a mid-low central unrounded vowel. It is of short duration compared to the other vowels, averaging around 50ms. This vowel can be epenthetic or a reduced version of front and back vowels. However, it also appears in roots with no obvious synchronic source of reduction, and contrasts with the other vowels of the same harmonic set: /e a o/. The vowel /ə/ is therefore a phonemic vowel, and acoustic evidence suggests that it contrasts with another mid central unrounded short vowel, /ɘ/, which is a higher vowel - see section 2.1.8. /ə/ cannot appear word-finally, except as part of a diphthong [eə], and only appears initially preceding geminates or liquid-initial consonant clusters (ld, rm, rl). Imperative forms of verb roots with a single geminate consonant are preceded by [ə], so we conclude that initial [ə] is epenthetic in these cases. In medial position, /ə/ can appear in any syllable, and surrounded by different consonants:

\begin{table} 
\caption{Examples of /ə/}	
 \label{tab:ch2:9}
\begin{tabular}[t]{lp{3.5cm}lp{3.5cm}}
\lsptoprule
\multicolumn{2}{l}{Initial} &	\multicolumn{2}{l}{Medial}  \\
\midrule
(ə)ldəmáná	&	‘bean’	&	ðəbérá		&	‘air, wind’\\
ə́rːá		&	‘lizard’	&	lavəra		&	‘stick’\\
ə́sːó		&	‘eat!’	&	gəla			&	‘plate’\\
ərmeə		&	‘rib’		&	lə́bóŋʷá	&	‘bottle’\\
\lspbottomrule	
\end{tabular}
\end{table}

\subsection{/ɘ/}

The vowel /ɘ/ is a mid-high central unrounded vowel, also with a duration of around 50ms. Like /ə/, this vowel can be epenthetic or a reduced version of front and back vowels. However, it also appears in roots with no obvious synchronic source of reduction, and contrasts with the other vowels of the same harmonic set: /i u ɘ/. /ɘ/ is also restricted from appearing word-finally except as part of a diphthong /iɘ/. Imperative forms of verb roots with a single geminate consonant are preceded by [ɘ], so we conclude that initial [ɘ] is epenthetic in these cases. In medial position, /ɘ/ can appear in any syllable, and surrounded by different consonants:

\begin{table} 
\caption{Example of /ɘ/}	
 \label{tab:ch2:10}
\begin{tabular}[t]{lp{3.5cm}lp{3.5cm}}
\lsptoprule
\multicolumn{2}{l}{Initial} &	\multicolumn{2}{l}{Medial}  \\
\midrule
ɘ́pːú	&‘beat!	&ŋɘ́bɘ́ní	&	‘jewelry’\\
ɘ́wːí	&‘boil!’	&ðɘ́gí		&	‘scab’\\
ɘ́sːiə́	&‘eye’	&umɘ́ɾtín	&	‘co-wife - 3poss’\\
		&&ud̪ɘmiɘ			&	‘witch doctor’\\
		&&undɘ́r				&	‘backside’\\
\lspbottomrule	
\end{tabular}
\end{table}

\subsection{Contrasts}

Moro has vowel harmony in which the vowels /e a o ə/ alternate with /i ɜ u ɘ/. Therefore, it is difficult to find minimal pairs contrasting all the vowels, as vowel harmony restricts their co-occurrence. Nevertheless, we present available contrasts here through minimal pairs. Note that tones may sometimes differ between words. 

\ea Higher vowels:
\begin{tabular}[t]{lllll}
i vs u		&	ŋudí	&	dew				&	ŋudú	&	dough\\
i vs. ɜ		&	iðú		&	‘breasts’		&	ɜðú		&	‘breast’\\
i vs. ɘ		&	ŋiðəniə́	&	‘rabbit’ 		&	ŋəðəniə  &	‘honor’	 \\
ɜ vs. u		&	gɜwúndɜ	&	‘s/he filters’	&	gɜwundú	&	‘s/he filtered’\\
ɜ vs. ɘ		&	ðɜ́diə́	&	‘side’			&	ðɘ́diə́	&	‘crevice, crack’\\
			&	lɜmí	&	‘beard, chin’	&	lɘ́mí	&	‘hedgehog’	\\
u vs. ɘ		&	ðugi		&	‘wood plank’		&	ðɘ́gí	&	‘cut, wound	
\end{tabular}\label{ex:ch2:3}
\ex 
Cross-height:\\
\begin{tabular}[t]{lllll}
i vs. e		&	víðú		&	‘vomit!’		&	véðó		&	‘knock!’\\
ɜ vs. a		&	ɽrwɜ́	&	‘testicles’	&	ɽrwa	&	‘small monkey’\\
u vs. o		&	uwːɜ	&	‘moon’		&	ówːá	&	‘woman’\\
e vs. ɜ\\
ɘ vs. ə   	
\end{tabular}\label{ex:ch2:4}
\ex 
Lower vowels:\\
\begin{tabular}[t]{lllll}
e vs. o	&	leɾó	&	‘they don’t have’	&	loɾó	&	‘they mated’\\
		&	eða		&	‘meats’				&	oða		&	‘deer sp.\\
e vs. a	&	ole		&	‘voice, sound’		&	ola		&	‘large covered milk gourd’\\
e vs. ə	&	ŋeɾá	&	‘girl’				&	ŋəɾá	&	‘trash’	\\
o vs. ə	&	ðóla	&	‘rat’				&	ðə́lá		&	‘grave, horn’\\
o vs. a	&	ola		&	‘large covered		&	ala		&	‘grinding stone’
 milk gourd’\\
ə vs. a	&	ŋəma	&	‘power’				&	ŋama	&	‘gum’	
\end{tabular}\label{ex:ch2:5}
\z

\subsection{Diphthongs}

Light diphthongs [iɘ] and [eə] are attested word-finally. In Moro orthography, these are written ia and ea respectively, but in Thetogovela, the second portion of the diphthong is pronounced more like a mid central vowel than a low a.

\ea	
\begin{tabular}[t]{llll}
lwát̪ə́déə́	&	‘heel’	&	ɜgəðíɘ́	&	‘mill floor’\\
opéɾéə	&	‘sword’	&	ugəviɘ	&	‘bird’
\end{tabular}\label{ex:ch2:6}
\z

These diphthongs contrast with their front vowel counterparts [i] and [e]:

\ea	
\begin{tabular}[t]{llll}
ɜ́gíɘ́	&	‘mental person’	&	ləmeə	&	‘fleas’\\
ugi		&	‘rat’			&	ləme	&	‘fish’
\end{tabular}\label{ex:ch2:7}
\z

There may also be diphthongs with rounding, but it is difficult to assess whether these sounds are diphthongs, labialized consonants, or a sequence of a consonant and a glide [w]. Let us consider the distribution. We will assume for now a transcription with [w]. This chart shows the distribution of CwV sequences in major lexical items:

%embed tabular in table with caption
\begin{table}
\caption{Chart of Cw-vowel sequences}
\label{tab:ch2:10}
\begin{tabular}[t]{p{1cm}p{1cm}p{1cm}p{1cm}p{1.25cm}p{1cm}p{1cm}p{1cm}}
\lsptoprule

%Figure 2: Chart of Cw-vowel sequences, make it prettier
	&e		&a	&o	&i		&ɜ	&u	&ə/ɘ\\
\midrule
pw	&		&$\bullet$	&	&		&$\bullet$ (ɔ)&		&$\bullet$\\
bw	&		&$\bullet$	&	&		&$\bullet$	\\	
t̪w	&		&$\bullet$	&	&		&$\bullet$\\		
d̪w	&		&	&	&		\\
tw	&		&$\bullet$	&	&		&$\bullet$		\\
dw	&		&$\bullet$	&	&		&$\bullet$ (ɔ)\\		
tʃw	&		&$\bullet$	&	&		\\	
dʒw	&						\\
kw	&		&$\bullet$	&$\bullet$	&		&$\bullet$	&	&$\bullet$\\
gw	&$\bullet$ (eə)	&$\bullet$	&	&		&$\bullet$	&	&$\bullet$\\
fw	&						\\
sw	&						\\
ðw	&		&$\bullet$	&	&		&$\bullet$		\\
mw	&		&$\bullet$	&	&		&$\bullet$	&$\bullet$	&$\bullet$\\
nw	&		&$\bullet$	&	&			\\
ɲw	&		&$\bullet$	&	&			\\
ŋw	&		&$\bullet$	&$\bullet$	&		&$\bullet$	&$\bullet$	&$\bullet$\\
lw	&		&$\bullet$	&	&		&$\bullet$	&	&$\bullet$\\
rw	&		&$\bullet$	&	&$\bullet$ (i, iə)	&$\bullet$		\\
ɾw	&		&	&	&		&$\bullet$		\\
{ɽw}	&$\bullet$ (eə)	&$\bullet$				\\
jw \\
\lspbottomrule
\end{tabular}
\end{table}					

There is an overwhelming tendency for Cw sequences to be followed by the lower vowels [a] or [ɜ]. There are some [ə] and a few instances of [u] and [o], but these latter alternate with [ə], and may be considered rounded schwa. Finally, there are a handful of words that have a front vowel, but the Cwi or Cwe sequences are otherwise very restricted:

\ea	\begin{tabular}[t]{ll}
	trwi				&	‘police’\\
	ɽrwíə, ŋwuríə		&	‘cucumber, cucumbers’\\
	ɽwéə, ɲə́dwéə 		&	‘ankle, ankles’\\
	ləbəgweə, ɲəbəgweə	&	‘tall flower’
	\end{tabular}\label{ex:ch2:8}
\z 
	
One analysis would hold that wa and wɜ sequences are light diphthongs [oa] and [uɜ]. Another possibility is that there are restrictions on the other sequences, such that they tend not to co-occur. This would make sense for [wu] and [wo], as they share rounding, and there is some evidence from verb paradigms that [w] is not realized before underlying /u/ and /o/. Consider the following verb paradigms, which show sequence of Cw both root-initially and root-finally. In the proximal imperfective forms (9a,d), a [w] appears before the final aspectual/mood/deixis suffix \textit{-a} or \textit{-ɜ}. However, in the proximal imperative and perfective, where the aspectual suffix is \textit{-ó} or \textit{-ú}, no [w] appears. The same pattern is not observed if the [w] is not adjacent to the suffix.

\ea \begin{tabular}[t]{lllll}
	a.	&	g-ɜ-mə́ɾw-ɜ́	&‘he is passing by’	&g-ɘ-mwə́t̪-ɜ	&	‘he is sipping’\\
	b.	&	g-ɜ-məɾ-ú	&‘he passed by’	&g-ɜ-mwət̪-ú		&	‘he sipped’\\
	c.	&	múɾ-ú		&‘pass by!’		&mút̪-ú			&	‘sip’\\
	d.	&	g-a-ɲə́dw-á	&‘he is soaking’	&g-a-ŋwáð-á		&	‘he smells bad’\\
	e.	&	g-a-ɲəd-ó	&‘he soaked’		&g-a-ŋwað-ó		&	‘he smelt bad’\\
	f.	&	ɲə́d-ó		&‘soak!’			&ŋwáð-ó			&	‘smell bad!’
	\end{tabular}\label{ex:ch2:9}
\z 

			
These forms also suggest that not all sequences of [wa] and [wɜ] are underlying diphthongs, but may arise from Cw-vowel sequences.

To complicate the picture further, there is also a distinction between two kinds of stems that begin with [w]. The locative prefix \textit{é-} attaches to nouns. If the noun is vowel initial, then another consonant, either [s] or [k] also appears (below. Some [w] initial nouns also show the extra consonant, suggesting that they are vowel-initial, not consonant-initial. Compare (16c-f) to (16g-j). 

\ea \begin{supertabular}[t]{lllll}
 	&	Singular	&	Plural	&	Locative 	& 	Noun\\
a.	&	ándə́mé		&	nándə́mé	&	ék-ándə́mé		&	‘flea’\\
b.	&	ajén		&	ején		&	ék-ajén		&	‘mountain’\\

c.	&	wárá		&	nwárá	&	ék-wárá		&	‘animal pen’\\
d.	&	wílí		&	nwílí	&	ík-wílí		&	‘picture, dream, spirit’\\
e.	&	wájá		&	lájá	 	&	ék-(ə)wájá	&	‘fly, bee’\\
f.	&	wíjɜ́		&	*		&	ík-wíjɜ́		&	‘dry dirt, ground’\\

g.	&	wáɾá		&	láɾá	&	é-wáɾá		&	‘chicken’\\
h.	&	waɾá		&	laɾá	&	é-waɾá		&	‘baobab tree’\\
i.	&	wálá		&	*		&	é-wálá		&	‘wool, braids’\\
j.	&	wasén		&	ləwasénanda	&	é-wasén	&	‘wife’\\

k.	&	wəliə́		&			&				&	‘flour’\\
l.	&	waŋgaló		&	laŋgaló	&				&	‘animal’\\
m.	&	wut̪ɜ		&	nt̪wɜ	&				&	‘low wall of compound’\\
\end{supertabular}\label{ex:ch2:10}
\z 

A similar pattern occurs in verbs. Verb roots that begin with [w] are distinguished from those that begin with [u] or [o]. The imperative has no prefixes, so the root is also word-initial. The proximal imperfective form typically has a prefix \textit{a-} or \textit{ɜ-}. All vowel-initial roots lack the clause marking vowel, due to vowel hiatus resolution, where the first vowel is deleted. This is observed with roots beginning with [u] or [o]:. 

\ea	
\begin{center}
\begin{tabular}[t]{lll}
	Proximal	&	Proximal 	&	Meaning\\ 
	imperative	&	imperfective \\
 	wát̪-ó		&	g-a-wát̪-á	&	‘sew’\\
	wund-ú		&	g-ɜ-wúnd-ɜ	&	‘filter, strain’\\
	wə́ɲó			&	g-a-wə́ɲ-á	&	‘have sex’\\
	ót̪-ó		&	g-a-wə́t̪-á	&	‘choose’		/wə́t̪-o/ → [ót̪o]\\
	óg-ó		&	g-og-a		&	‘thresh’		/g-a-og-a/\\
	ódə́ɲ-ó		&	g-odə́ɲ-a		&	‘squat, kneel’\\
	ud̪əð-ú		&	g-udə́ð-ɜ		&	‘milk’
	\end{tabular}\label{ex:ch2:11}
\end{center}
\z

However, some verb forms have a [w] that appears to be part of the root, but acts like a vowel rather than a consonant. The clause marking vowel that typically appears between the class marker and the root is missing in these forms, just as in vowel-initial roots. Furthermore, the tone pattern is also suggestive of a vowel-initial root. VC roots usually lack high tone in the imperfective, whereas CVC roots have high tone. The root wan below behaves like a vowel-initial root. 

\ea 
\begin{center}
\begin{tabular}[t]{lll}
	Proximal	&	Proximal 	&	Meaning\\ 
	imperative	&	imperfective \\
	wán-ó		&	g-wan-a		&	‘be anxious’ \\
	wáj-t̪-ó		&	g-waj-á		&	‘be rough, coarse’ (adj.) \\
	wásːə́ð-ó		&	g-wás:əð-eə	&	‘scatter (seeds)’ 
		\end{tabular}\label{ex:ch2:12}
\end{center}
\z

This suggests that the wat̪ begins with a consonant [w], whereas wan begins with a vowel, or a diphthong, and so is /oan/. 

There is a third set, however, that shows even more chameleon-like behaviour. Like the forms in (18), these verbs lack [w] in the imperative, and are missing the clause-marking prefix. The tone pattern, with a high tone on the root, are more consistent with consonant-initial roots, but are not unattested with vowel-initial roots. It does not appear to be possible to predict whether [w] appears in the imperative or not. 

\ea	
\begin{center}
\begin{tabular}[t]{lll}	
	Proximal	&	Proximal \\ 
	imperative	&	imperfective \\ 
	ás-ó		&	g-wás-a		&	‘wash’ \\ 
	áɾ-ó		&	g-wáɾ-a		&	‘badmouth’\\
	ál-t̪-ó		&	g-wal-á		&	‘be long’ (adj.) \\
	ánd̪-ó		&	g-wánd̪-a	&	‘harvest’ \\
	ónd̪áʧ-ó		&	g-wónd̪aʧ-a	&	‘be pregnant’ \\
	óndót̪ó		&	g-wóndət̪-a	&	‘dry up, wither, be strong’ 
		\end{tabular}\label{ex:ch2:12}
\end{center}
\z 

%{CHECK with RELATIVE FORMS OF 1st and 2nd person (GUEST has aŋ-wand-e (versus aŋəwat̪e)  WHAT ARE THE DURATIVES?}

This means that words like \textit{g-a-bwáɲ-á} ‘s/he wants’ may be transcribed as \textit{gabwáɲá} or \textit{gaboáɲá}, since both /oa/ and /wa/ are attested in the language. If the CwV sequence is contained with a morpheme, however, it is difficult to tell which transcription is more accurate, and indeed they have been transcribed both ways in previous publications.

%What about subordinate verbs that take –e or –i?

There are co-occurrence restrictions on [w] appearing with round vowels.  See section X. 

\subsection{Vowel Length and Stress}
Moro does not have contrastive vowel length. There are, nevertheless, some vowels that are long in particular positions. In phrase final position and citation form, penultimate vowels in open syllables may have longer duration than in antepenultimate or ultimate position if the following consonant is short. In addition, the vowel [a] has longer inherent duration than the other vowels, which contributes to the perception of long vowels. Finally, some following consonants such as [ɾ] or [g] may cause increased vowel duration. 

In order to illustrate this general pattern, recordings were taken from a separate intonation study. Four repetitions of subject-verb-object declarative sentences were recorded in which the subject-verb portion did not differ, but the objects varied. The durations of vowels in nouns in the phrase-final position were measured. For trisyllablic nouns, there is no pattern whereby penultimate vowels are consistently longer than ultimate vowels or vice versa. Instead, the pattern is dependent on the nature of the consonant intervening between them. The penultimate vowels are longer if the consonant is /ɾ/ (except for \textit{ŋgaɾá}), but the ultimate vowel is longer if the consonant is a nasal. 

\begin{table}
  \begin{tabular}{llll}
    \lsptoprule
	penultimate	&	ultimate	&	T-test & 
\textit{p value}	\\
\midrule 
ðamala	&	111.82	&	80.51	&	n.s			\\
ðəbáɾá	&	176.39	&	139.56	&	\textit{p} < 0.05	\\
ŋgaɾá	&	161.47	&	147.36	&	n.s.		\\
ŋgáɾá	&	161.96	&	140.48	&	\textit{p} < 0.05	\\	
lamámá	&	103.00	&	147.89	&	\textit{p} < 0.05	\\
lómóna	&	129.90	&	158.99	&	\textit{p} < 0.05	\\
\lspbottomrule
  \end{tabular}
  \caption{Vowel length by position in trisyllabic nouns.}
  \label{tab:ch2:X}
\end{table}


As for bisyllabic nouns, there is no general pattern (TABLE X). Only one case shows a significant difference, with a longer final vowel. The short central vowels [ə] and [ɘ] are not lengthened in penultimate position. 

\begin{table}
  \begin{tabular}[t]{llll}
    \lsptoprule
	penultimate	&	ultimate	&	T-test & 
\textit{p} value	\\
\midrule 
láɾá	&	163.24	&	156.22	&	n.s.	\\
ðáɾá	&	160.23	&	195.23	&	\textit{p}<0.05.	\\
ŋáná	&	158.83	&	172.14	&	n.s	\\
ŋaɲa	&	162.84	&	130.88	&	n.s	\\
\lspbottomrule
  \end{tabular}
  \caption{Vowel length by position in disyllabic nouns.}
  \label{tab:ch2:Y}
\end{table}

The longer duration of vowels in some words leads to the percept of stress, and may have prompted the description of Moro stress in Black \& Black (1971:14):
%not able to do enumerate[(i)]
\begin{quotation}
Stress presents a problem because it fluctuates freely in many words. It also seems to vary with the intonation pattern and is affected by elision. However for the most part a rough prediction can be given.
\begin{enumerate}
	\item If the last syllable is closed, it is stressed.

	\item If it is open the stress moves to the pentultimate unless this syllable contains /ə/. 
	\item If so the stress moves further to the front to the nearest syllable not containing /ə/, or,
	\item If the word is only 2 syllable, the stress returns to the ultimate.	
\end{enumerate}
\end{quotation}

They further comment (p. 15) that “There is a tendency to lengthen vowels in stressed syllables when words are said in isolation. In normal speech however length is not present.”

If Moro has stress, tone is not a correlate of stress. Any type of syllable (open/closed) and any type of vowel may bear either high or low tone. While there is a tendency for high tone to be attracted to closed syllables, this is not an absolute requirement. If tone is not a correlate of stress, two other parameters may express prominence: duration and intensity. 

\section{Consonants}
The consonant inventory of Thetogovela Moro is characterized by contrast between dental and alveolar stops, and several types of rhotics – a trill, flap and retroflex flap. Voiceless fricatives are not common within the language, although /s/ and /f/ are phonemic. 


\begin{table}
	\begin{tabular}[t]{lcccccccccccc}
			\lsptoprule & 
				\multicolumn{2}{c}{{Labial}} &					% Bilabial
				\multicolumn{2}{c}{{Dental}} & 					% Dental
				\multicolumn{2}{c}{{Alveolar}} & 				% Alveolar
				\multicolumn{2}{c}{{Retroflex}} & 				% Retroflex
				\multicolumn{2}{c}{{Palatal}} & 					% Palatal
				\multicolumn{2}{c}{{Velar}}  \\
	\midrule		 Stop / Affricate &  									% Plosive
				p & b &										% Bilabial
				{t̪} & {d̪} &									% Dental
				t   &  d  &							% Alveolar
				{} 		& {} 			&					% Retroflex
				ʧ 		& ʤ 	&					% Palatal
				k & g \\				\midrule				
			 Nasal & 							% Nasal
				\multicolumn{2}{c}{m} &													% Bilabial
				& &								% Dental
				\multicolumn{2}{c}{n} &							% Alveolar
				&  &														% Retroflex
				 \multicolumn{2}{c}{ɲ}  &														% Palatal
				 \multicolumn{2}{c}{ŋ} 														% Velar
         \\		
			\midrule Trill &  								% Trill
				& {} &											% Bilabial
				\multicolumn{2}{r}{}&								% Dental
				\multicolumn{2}{l}{r}&								% Alveolar
				& &														% Retroflex
				& 														% Palatal
				        &         &		% Velar
        \\		% Glottal
			\midrule Tap/Flap &  						% Tap /Flap
				& &													% Bilabial
				%\multicolumn{2}{}{} &					% Dental
				\multicolumn{2}{l}{{ɾ}} &					% Alveolar
				& {ɽ} &														% Retroflex
				& &														% Palatal
				        &         	% Velar
    \\		% Glottal
			\midrule Fricative & 						% Fricative
				{f} & {v} &									% Bilabial
				{} & {ð } &									% Dental
				s & &													% Alveolar
				{} & {} &								% Retroflex
				{{}} & {} &								% Palatal
				 & {} 											% Velar
 \\										% Glottal
			\midrule Lateral & 							% Approx.
				&   &														% Bilabial
				&								% Dental
				\multicolumn{3}{r}{{l}} & 					% Alveolar
				&  &											% Retroflex
				&  & &														% Palatal
        \\		% Glottal
			\midrule Approx & 							% Approx.
				&  w &														% Bilabial
				&								% Dental
				\multicolumn{3}{l}{{}} &					% Alveolar
				&  &											% Retroflex
				& j & &														% Palatal
        \\		% Glottal
			\bottomrule
		\end{tabular}
  \caption{Moro consonant inventory.}
  \label{tab:ch2:5}
\end{table}


%(20)	Consonant inventory
%
%	labial	dental	alveolar	retroflex 	palatal	velar		
%stop/	p	t̪		t			ʧ	k
%affricate	b	d̪		d			ʤ	g
%fricative	f	ð 		s
%	v	
%nasal	m			n			ɲ	ŋ
%trill	            			r    	
%flap				ɾ	ɽ
%lateral				l
%glide	w  						j


In the Moro writing system used in Bible translation and teaching materials, the Roman alphabet is combined with some International Phonetic Alphabet symbols (ŋ ɽ ə). Most vowels and consonants are the same in both alphabets. However, the following letters are used which are different:

\begin{table}
  \begin{tabular}[t]{llll}
    \lsptoprule
\textit{Written Moro} &		\textit{IPA}\\
\midrule 
	ë	&	ɜ		\\
	d	&	ð		\\	
	t	&	t̪	\\
	d	&	d̪	\\	
	c	&	tʃ	\\
	j	&	dʒ	\\
	ñ	&	ɲ	\\
	y	&	j	\\
	r	&	ɾ or r	\\
	rr	&	r	\\
\lspbottomrule
  \end{tabular}
  \caption{Correspondence between written Moro and IPA.}
  \label{tab:ch2:4}
\end{table}

Geminate or long consonants are indicated with double letters in the writing system, ex. \textit{dappa} ‘friend’ or \textit{igannaŋa} ‘I am listening to you’ (examples drawn from \textit{Fətau agəwërdia Ajwab?} or “How to write a letter?”). This is true except for rr, which represents the trill [r] as opposed to r which represents [ɾ]. However, these uses are not consistent, and [r] is often used to indicate [ɾ] as well. 

CHECK – loss of final vowel due to case – what obstruents can occur word-finally?

\subsection{/p/}
/p/ is a voiceless bilabial stop. It can occur word-initially, between vowels and following [ɾ]. We have not observed any occurrences of [lp] [rp] or [mp] sequences. [p] can only occur pre-consonantally before [r]. It does not occur word-finally.

[p] has relatively long duration when it occurs between vowels, so it is difficult to ascertain whether an intervocalic [p] is geminate or not, and there are no clear minimal pairs for gemination. Certain words have consistently long consonants, however, and we conclude that these are geminate. 

\ea
\begin{supertabular}[t]{lp{3cm}lp{3cm}}
\#\underline{\hspace{1cm}}V	&									&	V\underline{\hspace{1cm}}V	&	\\
paɖólwa						&	‘jute’							&	ðopa						&	‘star’\\
pɜ́diə						&	‘place to pray’					&	ed̪apəgá						&	‘nail’\\
pə́ɖúŋwá						&	‘devil, evil eye’				&	lə́pə́ɲíə						&	‘firefly light’\\
pə́gó							&	‘weed, uproot!’					&	ləpér						&	‘tail’\\

púllí						&	‘open gourd by making hole!’ 	&	gadʒópá						&	‘s/he is following to catch up’\\

pə́ndé						&	‘long time’						&	larbapa						&	‘old leather shoe’\\
\midrule
 \multicolumn{2}{l}{Geminate} & & \\
\midrule

C\_\_V						&									&	\#\_\_C						&	\\
ɜɾpúlɜ						&		‘animal skin’				&	gópːəta						&	‘s/he is defending’\\
							&									&	ápːó						&	‘carry!’\\
							&									&	ðapːa						&	‘friend’\\
							&									&	pr							&	‘very’\\
\end{supertabular}\label{ex:ch2:13}	
\z


[p] has short Voice Onset Time (VOT), as can be seen in the following spectrogram of a portion of the word \textit{égapəgó} ‘I weeded’. The silent closure duration of [p] is 126ms, whereas the VOT measures 20ms. 

 


\subsection{/b/}
/b/ is a voiced bilabial stop. It can occur word-initially, between vowels and following a consonant, either [r], [ɾ] [m] or [l], although most [lb] words are loans from Arabic, and the [rb] and [ɾb] sequences are only observed in nouns. [b] can also precede [ɾ] and [r] in nouns. There are no geminate [b] and it does not occur word-finally. [b] is a very frequent sound in Moro nouns in non-initial position. However, most Moro nouns begin with a consonantal noun class marker prefix, a small set that does not include labial consonants. Therefore, nouns beginning with labials are rare and may be borrowings.  

\ea	
\begin{supertabular}[t]{p{2cm}lll}
\#\_\_V	&	&				V\_\_V	\\
bit̪iə	&	‘butter’		&aɾóbá		&‘whey’\\
boʧa	&	‘ashes’		&ðaba		&‘cloud’\\
bə́ɾó		&	‘touch!’		&ðəbiə		&‘pre-wedding feast for bride’\\
bét̪ó	&	‘be satisfied!’&	ebambəɲá	&‘skull,eggshell’\\

bóló	&	‘wrestle!’		&íbín		&‘sibling-in-law’ (3sg.poss).\\
bət̪e	&	‘never’		&ləbú		&‘well’\\
		&					&gabóŕt̪a	&‘s/he is riding’		\\
\midrule
C\_\_V	&	&	V\_\_C	\\
\midrule
tʃambə́ɾa		&‘scab’			&ləbɾea		&‘walking stick’\\
ɜlbɜ́mbɜɾiə	&‘stool’		&embɾeá		&‘ring for balancing pots on head’\\
ðəɾmbégwa	&	‘lyre’		&ɜlibŕiɜ	&‘thread’\\
ʧarbapóða	&‘lung’		\\
gatʃómbəða	&	‘s/he is tickling’	&ombra	&‘branch of doleib palm’\\
\end{supertabular}\label{ex:ch2:14}
\z 

The triconsonantal sequences [ɾmb], [mbɾ] and [mbr] suggest that the sequence mb may be a prenasalized stop. The word \textit{ḿbú} ‘come!’ is the only occurrence of [mb] word-initially, however, and the fact that [m] bears tone may indicate that it is syllabic. 
			
Between vowels /b/ is often pronounced [β]. In this example of the word \textit{égabəɾó} ‘I touched’, the pronunciation is that of a fricative. The [β] is very short compared to the duration of intervocalic [p]:


Word-initially, [b] is a prevoiced stops with negative VOT: %todo Sharon INSERT SPECTROGRAM

Many words transcribed with [b] in the Werria dialect are pronounced with [v] (or [ʋ]) in Thetogovela (Werria data provided by Angelo Naser)

\ea	
\begin{tabular}[t]{lll}
	Werria	&	Thetogovela \\
	baðo	&		váðó		&	‘shave!’\\
	bəleðo	&		və́léðó	&	‘pull!’\\
	biðu	&		víðú		&	‘vomit!’\\
	galəbó	&		galəvó	&	‘s/he filled hand, took scoop’
\end{tabular}\label{ex:ch2:15}
\z 

\subsection{/t̪/}
/t̪/ is a voiceless (APICAL/LAMINAL?) dental stop. It can occur word-initially, between vowels and following a consonant, [r], [ɾ] [l] or [n], although there are only one or two attestations of the latter. [t̪] can precede [r], even word-initially. It does not occur word-finally.
Like the other voiceless stops, [t̪] has relatively long duration when it occurs between vowels. It can be geminated. 

\ea 	
\begin{supertabular}[t]{lllp{4cm}}
\#\_\_V		& & V\_\_V	\\
t̪əbwɜ	&‘bamboo’	&at̪ə́ndŕeá	&‘cloven hoof’\\
t̪ə́mát̪ó	&‘step on!’	&ðat̪á		&‘corn leaf’\\
t̪áðó	&‘leave!’	&ðót̪oŋ		&‘agama lizard’\\
t̪íðú	&‘thread!’	&et̪a		&‘lake, pool’\\
					&&gatəŋat̪ó	&‘s/he licked’\\
					&&gɜwúndət̪ɜ	&‘s/he is about to wring’	\\
&\\
\midrule
C\_\_V		&	&	V\_\_C	\\
\midrule
ðɜbərt̪ulɜ	&‘type of locust’	&ðəbət̪rwá	&‘(shield made of) doleib frond’\\
eməɾt̪á		&‘horse’		\\
ɜ́ŕt̪í		&‘buttock’	&	\#\_\_C	\\
t̪úrt̪ú		&‘wait for!’	&	t̪r		&‘police’\\
ə́lt̪ə́miə́		&‘termite mound’		\\
gɜnt̪ú		&‘s/he entered’	\\
\end{supertabular}\label{ex:ch2:17}
\z 	

%TODO Sharon GEMINATE?

t̪ has short VOT, only slightly aspirated. It sometimes has an ejective quality. 
%TODO Sharon SPECTROGRAM

\subsection{/d̪/}
/d̪/ is a voiced (APICAL/LAMINAL?) dental stop. It can occur word-initially, between vowels and following [n] or [l], although there is only one attestation of the latter. The cluster nd̪ can occur word-initially. d̪ does not occur preceding consonants or word-finally. It cannot be geminated. This sound is not common, particularly word-initially.

\ea
\begin{tabular}[t]{llll}
\#\_\_V	&	&	V\_\_V	\\
d̪et̪əm	&‘truly’		&ád̪ámá		&‘book’\\
d̪oát̪ó	&‘send, forge!’	&id̪əvíní	&‘shoe’\\
d̪oáðó	&‘push!’		&ŋəd̪ərriə	&‘nursing of babies’\\
						&ud̪əmiə		&‘witch doctor’\\
						&kad̪ó		&‘plant!’\\
						&lud̪ɜ́ðɜ		&‘they are milking’\\
\midrule
C\_\_V\\
\midrule		
t̪únd̪ú	&‘cough!’		\\
galánd̪a	&‘s/he is about to close sthg’			\\
ə́ld̪ə́máná	&‘bean’		\\
nd̪əmana	&‘kidney’	\\	
nd̪əmiə	&‘witch doctors’		\\
nd̪urt̪u	&‘behind, last’
\end{tabular}\label{ex:ch2:18}
\z	


\subsection{/t/}
/t/ is a voiceless alveolar (APICAL LAMINAL) stop. It can occur word initially, between vowels and following a consonant, [r], [ɾ], [ɽ], [l] or [n]. The latter is uncommon. It does not occur word-finally.
Like the other voiceless stops, [t] has relatively long duration when it occurs between vowels. It can be geminated only in morphological contexts.  

\ea
\begin{tabular}[t]{lllp{3cm}}
\#\_\_V	&&	V\_\_V	\\
tɜ́sí	&‘shake!’				&lamatáɾá	&‘support pole’\\
tóaðó	&‘stroke, rub!’			&lətaŋgora	&‘mane’\\
tə́mó		&‘describe in detail!’	&lútí		&‘owl’\\
tə́ŋ		&‘again’				&otéleə		&‘mat woven from palm leaves’\\

tu		&‘there’				&gatogó		&‘it pecked’\\
\midrule			
C\_\_V		\\
\midrule
ŋáɾtə́máðá	&‘small lizard’		\\
ə́ɽtú			&‘gazebo, shade structure’		\\
uməɾtín		&‘co-wife’		\\
úrtə́ðú		&‘pull out!’		\\
ɜ́rtə́ŋə́tiə		&‘armpit’		\\
ə́ltóléa		&‘cheek, shouting’		\\
ə́ltə́miə́		&‘barren woman’		\\
gunto		&‘one, on one’s own’		\\
bantalón	&‘trousers’		
\end{tabular}\label{ex:ch2:19}
\z 

GEMINATE?

\subsection{/d/}
/d/ is a voiced (APICAL/LAMINAL?) alveolar stop. It can occur word-initially, between vowels and following [n] or [l], although there is only one attestation of the latter. The cluster nd can occur word-initially. d does not occur preceding consonants except in the cluster ndr, which can occur word-initially and word-internally. Since [d] cannot otherwise occur before [r], it may constitute and insertion in an /nr/ sequence. [d] cannot appear word-finally and it cannot be geminated. 

\ea
\begin{supertabular}[t]{lllp{3cm}}
\#\_\_V	&&	V\_\_V	\\
diə́		&‘cow’					&ðɜ́diə́			&‘side’\\
dógə́t̪ó	&‘get clean!’			&lodóɾə́wa		&‘flower, leaf’\\
doátó	&‘speak, tell!’			&gɜdɜ́dəðɜ		&‘s/he is hiccupping’\\
dúwə́t̪ú	&‘chew with back teeth!’	&gadərnó		&‘s/he pressed’\\
dáŋó	&‘stay!’				&gavədaðó		&‘s/he cleaned’\\
								&ododo			&‘all’\\
\midrule	
C\_\_V	&&		C\_\_C	\\
\midrule
ándə́mé	&‘flea’				&ándŕeá		‘saddle’\\
ɜndiə	&‘leather’			&kańdrá	s/he is sleeping’\\
ɜ́ndú	&‘catch (it)!’		&ńdŕát̪ó	‘be near to’\\
kavəndəɲó	&‘he snapped (it)’		\\
ðɜpəldwɔ	&‘male agama lizard’	&	\#\_\_C	\\
ndəŋ	&‘firm’\\
ndəm	&‘together’\\
\end{supertabular}\label{ex:ch2:20}
\z 


\subsection{/tʃ/}
/tʃ/ is a voiceless alveopalatal affricate. It is written c  in the orthography. It can occur word-initially, between vowels and following the consonants [m] and [r]. It does not occur before a consonant or word-finally. It can be geminated.

\ea 	
\begin{tabular}[t]{lllp{4cm}}
\#\_\_V	&&	V\_\_V	\\
tʃambə́ɾa	&‘scab’				&ðətʃa		&‘wine filter’\\
ʧugúlɜ	&‘pumpkin, gourd’	&lɜtʃuwɜ́	&‘whip made from leather or stiff hair’\\
ʧə́ŋge	&‘cobra’ 			&matʃó		&‘man’\\
tʃə́ndúŋú	&‘go down!’			&gatʃə́ð́a		&‘s/he is chopping legs (of bed, table)’\\

ʧómbə́ðó	&‘tickle!’			&gaʧoɲá		&‘s/he is hungry’\\
tʃom	&‘also, too’		&gɜmədɜʧú	&‘s/he twisted sthng’\\
							&&gɜrɜ́tʃiðiə &‘s/he is gathering (things) together’\\
\midrule
C\_\_V	&&		Geminate	\\
\midrule
ɜmtʃu	&‘loan, feud; clan’	&lətʃːó	&‘animal fat; top of sprout of doleib palm tree seedling’\\
ortʃəl	&‘poisonous tree’	&oʧːa	&‘milk pot made from calabash’
\end{tabular}\label{ex:ch2:21}
\z 

\subsection{/dʒ/}
/dʒ/ is a voiced alveopalatal affricate. It is written j in the orthography. It can occur word-initially, between vowels and following the consonants [ɲ] and [n]. Between vowels it can be pronounced [ʝ]. It does not occur before a consonant or word-finally. It cannot be geminated. [dʒ] is a rare consonant. It occurs word-initially only in verbs, and most of the occurrences in verbs are probably due to dissimilation from a following voiceless consonant (see section X), typically a [tʃ], in what appear to be lexicalized reduplicative durative/iterative prefix. A good example is \textit{gadʒátʃ́aŋgərəða ánó} (see section X.)

\ea
\begin{tabular}[t]{llp{2.5cm}p{3cm}}
\#\_\_V	&&	V\_\_V	\\
dʒópó		&‘follow to catch up!’	&ɲəkawádʒá	&‘white people’\\
ʤómó		&‘move!’				&lə́dʒógádʒógá	&‘coucal bird’\\
dʒátʃə́dwé	&‘implore’				&gadʒə́vá	&‘s/he doesn’t know’\\
									&gaʤə́pə́ðiə	&‘it is rotten’\\

							&&gadʒátʃ́aŋgərəða ánó	&‘s/he is twisting, writhing’\\
								&&gɜdʒívɜ́ʧəniə	&‘s/he forgets’\\
\midrule				
C\_\_V	\\				 	
\midrule
ðə́ɾáɲdʒálá	&‘stone wall’			\\
lɜ́ndʒú		&‘swish water in a bowl!’	
\end{tabular}\label{ex:ch2:22}		
\z 


\subsection{/k/}
/k/ is a voiceless velar stop. It can appear word-initially before a vowel, between vowels and following [r] and [l]. It can also appear geminated. 

\ea
\begin{tabular}[t]{llll}
\#\_\_V	&&	V\_\_V	\\
kájó	&‘tie!’		&ókóra		&‘sap’\\
kə́wó		&‘pinch!’	&lókógóŋ	&‘scorpion’\\
kárðó	&‘worry!’	&gɜkiðú		&‘s/he opened’\\
kúra	&‘ball’		&álə́karðó		&‘we worried’\\
\midrule				
C\_\_V		&&	Geminate	\\
\midrule
ɜkúrkuriə	&‘butterfly’		&gaɡakːoreð-ó	&‘s/he scratched’\\
ɜlkɜnísɜ	&‘church’		
\end{tabular}\label{ex:ch2:23}
\z 	

[k] is pronounced with slight aspiration, a longer VOT than the other voiceless stops. 


\subsection{/g/}\label{sec:ch2:g}
/g/ is a voiced velar stop. It can appear word-initially before a vowel, between vowels and following [ŋ], either preceding a vowel or [r]. It cannot be geminated. 

\ea 
\begin{tabular}[t]{llll}
\#\_\_V	&&	V\_\_V	\\
gí		&‘farm’		&ðugi		&‘wood plank’\\
gálá	&‘bead’		&iməganiə	&‘excrement’\\
ɡə́ɲó		&‘kill!’	&omágá		&‘snail’\\
gwáŋá	&‘thing’	&ðugi		&‘wood plank’\\
					&&goɡət̪ó	&‘s/he jumped’\\
					&&pə́ɡə́ðó	&‘pay!’\\
\midrule
C\_\_V			\\
\midrule
máŋga		&‘mango’	\\		
ləŋɡə́lːəme	&‘pen/crab’		&C\_\_C	\\
ŋɡárá		&‘salt’		&aləŋgréma	&‘bed’
\end{tabular}\label{ex:ch2:24}
\z 


Although there are contrasts between /k/ and /g/ at the beginning of the word, the contrast is generally neutralized in phrase-initial position where both are pronounced as [k]. Consider the following sentences in which the noun class subject agreement marker /g-/ is realized as [g] phrase-internally in (33)a, but as [k] phrase-initially in (33)b.

\ea \ea 	[umːiə gadaŋó ntərəbésa]
	umːiə  	g-a-daŋ-ó          	n-tərəbésa
	CLg.boy       SM.CLg-RTC-sit-PFV	LOC-table
	‘the boy sat on the table‘
\ex	[kadaŋó ntərəbésa]
		g-a-daŋ-ó	n-tərəbésa
		SM.CL-RTC-sit-PFV	LOC-table
     	‘he sat on the table‘
\z \z 

\subsection{/f/}
/f/ is a voiceless labiodental fricative. It is attested word-initially in only one word, and only occurs following a consonant [l] in loanwords from Arabic. Otherwise, [f] occurs between vowels and may be geminated:

\ea	
\begin{tabular}[t]{llll}
\#\_\_V	&&	V\_\_V	\\
fərfər	&‘never’	&ódófá	&‘foam, bubble’\\
					&&ðəgəfiə		&‘tree sp. (green with thorns)’\\
					&&ŋófə́ŋé		&’unskillfulness’\\
					&&galə́fa		&‘s/he swears’\\
\midrule			
C\_\_V	&&		Geminate	\\
\midrule
alfəɾáɾa	&‘small hatchet’		&lə́fːə́ɾəŋeə	&‘bird sp.’\\
								&&áfːó		&‘shoot!’
\end{tabular}\label{ex:ch2:26}
\z

Many words transcribed with f in Werria are pronounced with v in Thetogovela:

\ea 	
\begin{tabular}[t]{lll}
	Werria	& Thetogovela\\
	ðəfia		&ðəviə	&‘type of tree’\\
	ðɜffia		&ðɜ́viə́	&‘lion’\\
	gafərəða	&gavə́rða	&‘s/he is clearing the fields’\\
	gafó		&gavó	&‘s/he lived (somewhere)’		
\end{tabular}\label{ex:ch2:27}
\z 

\subsection{/v/}\label{section:v}	
/v/ is a voiced labiodental fricative, often pronounced as an approximant [ʋ]. It occurs word-initially preceding a vowel and between vowels, but it does not occur before or after a consonant, word-finally or as a geminate. It is also restricted from appearing before round vowels. It is pronounced [w] before round vowels:  \textit{gakə́vá} ‘s/he is about to pinch’ vs. \textit{gakəwó} ‘he pinched’ (compare /v/ with the behaviour of /w/: \textit{gaðə́wá} ‘s/he is about to poke’ \textit{gaðəwó} ‘s/he poked’)

\ea		
\begin{tabular}[t]{llll}
\#\_\_V	&&	V\_\_V	\\
váðó	&‘shave!’			&áveja	&‘spring, rainy season’\\
várt̪ó	&‘lead, be in front’	&ðává	&‘chaff’\\
və́léðó	&‘pull!’			&ðəvəra &‘line’\\
və́rðó	&‘clear the field!’	&evəlːa	&‘wild cat’\\
véðó	&‘knock, slap’		&id̪əvíní	&‘shoe’\\
						&&lóváɾá	&‘guinea fowl’\\
						&&luvəɾíŋí	&‘eyebrow’
\end{tabular}\label{ex:ch2:28}
\z 

\subsection{/ð/}
/ð/ is a voiced interdental fricative. It is one of the most common obstruents in Moro, as it is a noun class prefix. It can occur word-initially, between vowels, following a consonant [r] or [ɾ], and word-finally. When word-final, it is devoiced and pronounced [θ]. For some speakers, it may be pronounced [θ] in other positions, too. It can also be geminate, either lexically, or as the result of morphological concatenation. 

\ea 
\begin{tabular}[t]{lp{3cm}lp{3cm}}
\#\_\_V	&&	V\_\_V		\\\hline
ðolóŋ	&‘eel’		&óvə́ðájá	&‘winter (Dec. – mid Feb.)’\\
ðílí	&‘manure’	&uðɜ	&‘worm’\\
ðərliə́	&‘root, artery’	&úmə́ðí	&‘sharp spear with cerrations’\\
ðáðá	&‘path’	&ləðe	&‘bone’\\
ðúgí	&‘breastfeed!’	&gɜðíɲɜ́	&‘s/he is scared’\\
ðɜ́diə́	&‘path’	&kɜsɜ́ðɜ́	&‘s/he is defecating’	\\
\midrule
C\_\_V	&&	V\_\_\# 	\\
\midrule
ðəɾðiə́	&‘water lizard with long tail’	&uɾíθ	&‘chain’\\
rða		&‘meat’	&eɾéθ	&‘clothing’\\
				&&ləŋáθ	&‘tooth’\\
&\\			
\midrule
Geminate\\			
\midrule
ŋwáðːá	&‘male goat’		\\
gagaðːó	&‘s/he mixed a lot of things’		
\end{tabular}\label{ex:ch2:29}
\z 

\subsection{/s/}
/s/ is a voiceless alveolar fricative. It is not a common sound in Moro, and many occurrences are due to borrowed words. s can occur word-initially, between vowels and as a geminate. It does not occur after consonants or word-finally. There is no voiced counterpart. 

\ea
\begin{supertabular}[t]{llll}
\#\_\_V	&&	V\_\_V		\\
síánó	&‘face’		&ðəsiə	&‘loincloth for women’\\
sɜndúgi	&‘box’		&wasɛ́n	&‘wife’\\
súrá	&‘icture’	&ɲusí	&‘chicks’\\
suwɜ́jə	&‘type of dance’ 	&tərəbésá	&‘table’\\

sɜ́ðu	&‘defecate!’	&úsílá	&‘spirit, shade’\\
sát̪ó	&‘chew!’	&koása	&‘s/he is washing’\\
					&&kɜtɜsú	&‘s/he shook’\\
&\\
\midrule
Geminate\\
\midrule
ŋə́sːa	&‘lot of food’		\\
ə́sːí		&‘eye’		\\
kas:a	&‘s/he is eating’	\\
\end{supertabular}\label{ex:ch2:30}	
\z 

\subsection{/m/}
/m/ is a voiced labial nasal stop. It can occur word-initially, between vowels, before [b], following a consonant ([g], [r] [ɾ] and [l]), and word-finally. It can also be geminated. 

\ea
\begin{tabular}[t]{llll}
\#\_\_V	&&	V\_\_V		\\
matʃó	&‘man’			&ðóŋgómá	&‘nostril’\\
mogʷátá	&‘peanut’		&rumɜ		&‘wild yams\\
músí	&‘banana’		&eməɾt̪á		&‘horse’\\
mə́gú		&‘curse!’		&ɜ́mə́ðíə		&‘celebration’\\
miɲɜ́ʧú	&‘peel, remove layer!’		&imɜgəniə	&‘excrement’\\
mánó	&‘cook!’		&ləmúrgúrgí	&‘type of bird’\\
ḿbú		&‘Come!’		&ləmí		&‘hedgehog’\\
\midrule	
V\_\_C		&& V\_\_\# 	\\
\midrule
ebamba		&‘drum’		&etám	&‘neck’\\
ləbəmbɜ́j	&‘yam’		&tʃom	&‘also, too’\\
lumbɜlwɔ́	&‘calabsh bowl’		&ndəm	&‘together, two’\\
						&&rám	&‘early’\\
\midrule
Geminate	&&		C\_\_V	\\
\midrule
lɜt̪úmːɜ	&‘hollow behind the ear’		&lugma		&‘porridge’\\
gam:ó	&‘he took/married’			&ərmeə		&‘rib’\\
umːiə	&‘boy’						&almoʧána	&‘tobacco pipe’\\
							&&ðəɾmbégʷa	&‘lyre’
\end{tabular}\label{ex:ch2:31}
\z 

\subsection{/n/}

/n/ is a voiced alveolar nasal stop. It can occur word-initially, between vowels, after consonants ([r], [l], [d]), before [d], and word-finally. It can also be geminated. [n] is a very frequent sound, due in part to its status as a noun class prefix. [nd] sequences are observed word-initially, and these may be prenasalized stops. However, the [n] in this position can also bear tone, which may indicate that it is syllabic.
\ea
\begin{tabular}[t]{lp{3.5cm}lp{3.5cm}}
\#\_\_V	&&	V\_\_V		\\
nəməníə	&‘olive trees’		&ŋə́ní	&‘dog’\\
nálá	&‘beads’		&lə́wːáná	&‘porcupine’\\
nɜðəɲín	&‘day, hour’	&	anadáɾá	&‘eyeglasses’\\
néðó	&‘refuse!’		&anó		&‘place’\\
nát̪ó	&‘taste’!		&énéə	&‘community hunting’\\
nínú	&‘search for!’	&	t̪ə́ðə́nú	&‘slip!’\\
						&&úlɜlítá̪nó	&morning’\\
						&&ɜni	&‘here’\\
\midrule				
V\_\_C	&&		V\_\_\# 	\\
\midrule
ɜ́t̪ɪ́ndíə	&‘river bank’	&	emaðén	&‘age-mate’\\
lávándə́ŋé	&‘hard seed of emakəŋe fruit used to play hockey game’	&	lómón	&‘day, finger’\\
ləpə́ndóŋwá	&‘bushbaby’		&aten	&‘quietly, slowly’\\
undrt̪u	&‘under’			\\
&\\		
Geminate	&&		\#\_\_d	\\
ganːó		&‘he heard’ 		&ńdríə	&‘fences, gardens’\\
ŋɜ́dwɜ́nːéə́ŋ	&‘hot drink’		&ndogá	&‘stick inserted under lower lip’\\
							&&nda	&‘head’\\
							&&ńdrá	&‘sleep!’\\
\midrule
C\_\_V\\
\midrule
lɜbərnét̪a	&‘hat’			\\
lóŕná		&‘big basket for grain’			\\
ə́lná			&‘room’			\\
ɜdniə		&‘young woman with children’	
\end{tabular}\label{ex:ch2:32}		
\z 

The sequence [ndr] may result from the insertion of a stop [d] between /n/ and /ɾ/. This occurs, for example, with noun class prefixes. The singular begins with a vowel [e] or [i] and is of noun class [g]. The plural has the prefix /n-/. The sequence /n-ɾ/ → [ndr]. 

\ea	 
\begin{tabular}[t]{lll}
		singular	&plural\\
		eɾéθ		&ndréθ		&‘clothes’\\
		íɾíə		&ńdríə		&‘fence, garden’\\
		iɾíŋ		&ndríŋ		&‘name’
\end{tabular}\label{ex:ch2:33}
\z

\subsection{/ɲ/}
/ɲ/ is a voiced palatal nasal stop. It can occur word-initially, between vowels, and word-finally. The only cluster it can appear in is preceding [dʒ], and this is only observed in one word. Can it be geminated? [ɲ] is a noun class prefix, and so appears frequently in word-initial position.
\ea
\begin{tabular}[t]{lp{3cm}lp{3cm}}
\#\_\_V	&&	V\_\_V		\\
ɲəgəmbə́lliə	&‘top molars’		&abə́ɲá	&‘piece of ostrich egg for facial decoration’\\
ɲogə́l		&‘eagles’			&éŋáɲá	&‘forest’\\
ɲaŋwat̪a		&‘water cup bowls made from gourd’		&léɲá	‘egg, penis’\\
ɲubə́lbə́líə	&‘earlobes’		&ðə́ɲóŋ		&‘large forest’\\
ɲə́dó			&‘be wet!’		&ɜðiɲíní	&‘sun’\\
							&&gɜmiɲɜ́ʧɜ	&‘he is peeling sthg’\\
\midrule
V\_\_C		&&	V\_\_\# 	\\
\midrule
ðə́ɾáɲdʒálá	&‘stone wall’		&ŋgáɲ	&‘sickness, death’’\\
						&&ŋgwə́ɲ	&‘sign, letters, writing’\\
						&&óráɲ	&‘my brother’ \\
						&&káɲ	&‘very, much’\\
						&&t̪wáɲ	&‘near’
\end{tabular}\label{ex:ch2:34}
\z

\subsection{/ŋ/}
/ŋ/ is a voiced velar nasal stop. It can occur word-initially, between vowels, before [g] and word-finally. It can also be geminated. [ŋ] is a noun class prefix, and so occurs frequently word-initially. 
\ea
\begin{tabular}[t]{lp{3cm}lp{3cm}}
\#\_\_V	&&	V\_\_V		\\
ŋombogó	&‘calf (baby cow)’		&ámákə́ŋé	dom &‘palm fruit’\\
ŋəɾá	&‘trash’				&já́ŋála		&‘ewes’\\
ŋelá	&‘oil’					&ðəŋəlá	&‘tongue’\\
ŋíʧa	&‘sin, mistake’			&éŋáɲá	&‘forst’\\
ŋáŋó	&‘scratch’		\\	
\midrule
V\_\_C	&&		V\_\_\# 	\\
\midrule
tʃaŋgwə́ɾá	&‘rhino, large boulder tha?’		&ámʧáŋ	&‘large rock’\\
ðɜ́ŋguri	&‘chameleonv		&ðəbə́ŋ	&‘type of gum tree’\\
ðóŋgómá	&‘nostril’			&ðomóŋ	&‘big, lazy rat’\\
ŋgáɲ	&‘sickness, eath’	&ógəŋŋa	&‘plant that causes itching’\\
\midrule
\#\_\_C\\
\midrule
ŋgə́m		&‘squirrel’			\\
ŋ́ɡít̪ú	&‘let be, allow!’	\\		
ŋgát̪ó	&‘go away, leave’!	
\end{tabular}\label{ex:ch2:35}		
\z

\subsection{/l/}
/l/ is a voiced alveolar lateral. It can appear word-initially, between vowels, before consonants, and word-finally. It can also be geminated. The only consonant that can precede it in a cluster is [r]. However, [l] can precede a large number of consonants due to Arabic noun borrowings, in which [l] is part of the Arabic definite article, ex. \textit{almotʃána} ‘tobacco pipe’. There are no instances of [l] before the voiced fricatives or the rhotics. Sequences of l-coronal are in native words.  

\ea 	
\begin{supertabular}[t]{lp{3cm}lp{3cm}}
\#\_\_V	&&	V\_\_V		\\
lájá	&‘flies, bees’		&odəlóŋá 	&‘fox’\\
ləbú	&‘well’				&ot̪elea		&‘spider’\\
lodóɾə́wa	&‘flower, leaf’		&ulɜngi		&‘night’\\
lútí	&‘owl’				&ðamala		&‘camel’\\
lɜ́ndʒú	&‘swish water in bowl’			&mə́lɜ́ðú	&‘exchange, replace!’\\
límú	&‘put together!’		&úlɜlítú̪	&‘tomorrow’\\
\midrule
C\_\_V	&&		V\_\_	\\
\midrule
rlo		&‘female goat’		&ortʃəl	&‘poisonous tree’\\
ðərliə́	&‘artery, root’		&ɽrél	&‘wisdom tooth’\\
							&&agól	&‘gap where two lower teeth removed for beauty’\\
							&&táltal	&‘quickly’\\
\midrule
Geminate				\\
\midrule
evəlːa	&‘wild cat’		&lubə́lbə́líə	&‘earlobe’\\
elːeə	&‘wing, feather’		&ə́ld̪ə́máná	&‘bean’\\
lːoá	&‘elbow’		ə́&ltúlé	&‘cheek’\\
ŋalláɾa	&‘yellow locust’		ə́&lt̪ə́miə́	&‘termite mound’\\
kɜvɜ́lːəniə	&‘s/he is boasting’		&ɜlkɜnísɜ	&‘church’\\
						&&alfəɾáɾa	&‘small hatchet’\\
						&&almoʧána	&‘tobacco pipe’\\
\end{supertabular}\label{ex:ch2:36}
\z 

The following are minimal pairs showing single vs. geminate contrast: \textit{kə́ló} ‘chop up!’ vs. \textit{kə́lːó} ‘pull branches from a tree!’.

%add a comment about ll going to  ld in Werria and written moro?

\subsection{/ɽ/}
/ɽ/ is a voiced retroflex flap. It occurs word-initially before /d/, /ɾ/, /r/ as a noun class marker, although the noun class concord marker of ɽ-initial nouns is /l/. It occurs in a few words between vowels, but otherwise ɽ is a relatively rare sound, and is not found in verbs. 
\ea 	
\begin{tabular}[t]{lllp{4cm}}
\#\_\_C	&&	V\_\_V		\\
\midrule
ɽdiə́		&‘dalib fruit’	&éɽo	&‘around’\\
				&&ðoɽár	&‘type of yellow and white snake’\\
ɽɾá		&‘lizard’	&ŋáɽáká	&‘small lizard’\\
ɽrágə́gá	&‘claw’		&ŋəɽə́ŋgé	&‘donkey’\\
ɽwa		&‘long bamboo stick’		&əɽíə	&‘water pot'\\
ə́ɽtú		&‘gazebo	
\end{tabular}\label{ex:ch2:37}
\z 	

Thetogovela has less examples of [ɽ ] than the standard Moro Werria dialect; many [ɽ] have become [g] or [d] in Thetogovela. Examples with [d] are generally surrounded by higher vowels, but there are also [g] with high vowels, so it is not possible to predict why some [ɽ] became [d] and some [g]. 

\ea 	
\begin{tabular}[t]{lll}
	Werria 	&Thetogovela	\\
	dəɽia	&ðɜ́diə́		&‘side’\\
	lëɽwa	&lɜ́dʷɔ́ŋ		&‘back, mid-back’\\
	giɽú	&gidú		&‘s/he fell’\\
	đeɽəm	&ðegə́mé		&‘jaw’\\
	eđapəɽa	&ed̪apəgá	&‘nail’\\
	ŋəɽa	&ŋga		&‘urine’\\
	gabəɽia	&gɜvɜ́giə́		&‘s/he is miscarrying’\\
	garëməɽu	&gɜrɜməgú	&‘s/he got drunk’\\
\end{tabular}\label{ex:ch2:38}
\z


\subsection{/ɾ/}
/ɾ/ is a voiced alveolar flap. It does not occur word-initially, word-finally or following a consonant. It is found between vowels and pre-consonantally before voiceless stops and affricates  (p t̪ t, tʃ, k), ð and m. It cannot occur geminated. It is written r in the orthography.

\ea 
\begin{tabular}[t]{lp{3cm}lp{3cm}}
C\_\_V	&&		V\_\_V	\\
aɾʧə́ŋála	&‘broken piece of gourd’		&ðə́ɾá	‘gourd vine’\\
ðəɾðiə	&‘water lizard with long tail’		&aŋoɾa	&‘elephant’\\
ðəɾmbégʷa	&‘lyre’		&ðageɾe	&‘large pot for sorghum wine’\\
eməɾt̪á	&‘horse’		&ɜmwəɾíní	&‘red-necked cobra’\\
ɜɾpúlɜ	&‘animal skin’			\\
ŋáɾtə́máðá	&‘small lizard’			\\
ɜkúɾkuɾiə	&‘butterlfy’	
\end{tabular}\label{ex:ch2:39}
\z 		

\subsection{/r/}
/r/ is an alveolar trill that occurs word-initially, intervocalically, word-finally and pre- and post-consonantally. It can also occur geminated. It is written r in the orthography, even though the same symbol is used for [ɾ] and the two sounds contrast in intervocalic position and pre-consonantal position. It may be syllabic word-initially or word-internally following a consonant, often alternating with a schwa pronunciation either preceding or following, [ər] or [rə]. When /r/ occurs in the syllable rhyme (either in the nucleus or in the coda), it is tone-bearing. It is not restricted adjacent to vowels. Pre-consonantally, it precedes the stops/affricates [p t̪ t tʃ k g] as well as [m n l w]. Post-consonantally, it follows [b d t g k] as part of a complex onset, and [ɽ] when [ɽ] is syllabic and initial. 

\ea 
\begin{supertabular}[t]{lp{3cm}lp{3cm}}	
\#\_\_C	&&		\#\_\_V		\\
rða		&‘meat’			&rɜ́tᶘíðú	&‘gather together’\\
ərmeə	&‘ribs’			&ró			&‘stab!’\\
rlo		&‘female goat’	&rəmʷɔ		&‘snake, God’\\
						&&rátó		&‘inherit’\\
\midrule
V\_\_C		&&	C\_\_V	\\
\midrule
ʧarbapóða	&‘lung’				&ɜlibŕiɜ	&‘thread’\\
ləvárt̪ə́ŋéə	&‘fig-like fruit’	&at̪ə́ndŕeá	&‘cloven hoof’\\
ðórtóðéa	&‘type of tree’		&ɜ́tŕiə́		&‘gum of mouth’\\
ortʃəl		&‘poisonous tree’	&aləŋgréma	&‘bed’\\
ləmúrgúrgí	&‘type of bird’		&ɜbəlúkriə	&‘dove’\\
gɜmúŕkwɜ	&‘s/he is rolling, sliding’		&ɽréá	&‘upper arm’\\
gakarmó		&‘s/he was found guilty and fined’			\\
gɜvɜ́rniə	&‘s/he is named’			\\
ŋáŕlá		&‘spear’			\\
lə́mbə́rwáðá	&‘religious icon, usually stone’			\\
\midrule
V\_\_\# 	&&		V\_\_V	\\
\midrule
ðəwə́r	&‘spring (water)’		&ara	&‘small animal pen made of dirt’\\
ləmat̪ár	&‘roof post with two prongs’		&gɜrə́giə	&‘s/he is passing under, pushing through’\\
ləpér	&‘tail’		&bətéréká	&‘before the day before yesterday’\\
							&&lokórá &‘throat’\\
							&&ðəbarəla	&‘river, stream’\\
							&&lurumi		&‘chest’\\
							&&gɜriðú	&‘s/he turned over (once)’\\
\midrule
C\_\_C/\# &&			Geminate	\\
\midrule
brlágá	&‘slime’		\\	
pr		&‘very, a lot’\\			
undrt̪u	&‘under’	\\		
tr		&‘policeman’\\
\end{supertabular}\label{ex:ch2:40}			
\z 

%TODO Sharon NEED EXAMPLES OF GEMINATE R


\subsection{/w/}
/w/ is a voiced labio-velar approximant. It can occur word-initially and between vowels. 

\ea 	
\begin{tabular}[t]{lp{3cm}lp{3cm}}
\#\_\_V	&&	V\_\_V		\\
wáðó	&‘poke, pierce, sow!’		&gowá	&‘he is bad at sports’\\
wálá	&‘wool, braids’			\\
wúndú	&‘filter, strain!’		&gawə́t̪á	&‘he is choosing’\\
wə́ndát̪ó	&‘watch, see!’		&suwɜ́jə	&‘kind of dance’\\
wɜ́ndə́ðú	&‘call!’		&ŋədəwén	&‘deciept’\\
wut̪ɜ	&‘low wall of compound’		&ogowélá		&‘monkey’\\
wíjɜ́	&dry dirt, ground’		&ðə́wí	&‘intestines, heart’\\
							&&ɜwíɾə	&‘type of tree’\\
\end{tabular}\label{ex:ch2:41}	
\z 

There are no lexical sequences of [iw] or [ew], but these sequences can arise in verb forms when the 1sg prefix e- abuts a verb root beginning with [w] or when the locative prefix é- attaches to a w-initial noun. 

Geminate /w/ with durative/iterative prefix? 

Geminate wː is written bw in the orthography, and we have previously transcribed it [βw] in Gibbard et al (2009), due to weaker energy in its production. It is certainly a long sound. However, it is not clear that it is a fricative-w sequence. Compare the spectrograms of the words \textit{ŋawa} ‘water’ and \textit{owːa} ‘woman, wife’. In the first case, the [w] has consistent voicing and strong vocalic waveform structure:
 

In the second case, the waveform is weak, but it is still periodic. Furthermore, the duration of the [w:] is longer. 


Therefore, the sound appears to be a long wː that has weaker energy due to its length. 

\subsection{/j/}
/j/ is a voiced palatal approximant. It is written y in the orthography. It is a plural noun class marker, and appears word-initially in some nouns. [j] does not precede the high vowels [i] or [u]. [j] can appear in coda position, which may be analysed as a diphthong. There are no examples of geminate [j]. 

\ea 
\begin{tabular}[t]{lp{3cm}lp{3cm}}
\#\_\_V	&&	V\_\_V		\\
jɜbərt̪ulɜ	&‘type of locusts’	&ajén	&‘mountain\\
jamala		&‘camels’			&ɜ́wíjɜ́	&‘friend’\\
jɜ́ŋguri		&‘chameleons’		&ləgajáŋ	&‘pebble’\\
jomóŋ		&‘big, lazy light-colored rats’		&wojá	&‘type of tree’\\
jwalea 		&‘type of green birds’		&lavajó	&‘they died’\\
\midrule
C\_\_V	&&		V\_\_C	\\
\midrule
lúwjɜ		&‘type of tree’		&ðəlájréa	&‘type of tree’\\
ŋgíljáŋa	&‘loudly’		&réj				&‘hands’\\
							&&ləbəmbɜ́j		&‘yam’
\end{tabular}\label{ex:ch2:42}	
\z



\subsection{Minimal pairs}
Minimal and near minimal pairs illustrating consonant contrasts are given below:

\ea Labial minimal pairs
\begin{tabular}[t]{llp{3cm}ll}
p vs. b	& gapə́gá		&‘s/he is about to weed’			&gabəgá	&‘s/he is strong’\\
p vs. f	&gapːó		&‘s/he carried’					&gafːó	&‘s/he built’\\
b vs. w	&gabə́ðá		&‘s/he played’					&gawə́dá	&‘s/he is burning (it)’\\
b vs. v	&ebea		&‘door made from doleib palm’	&eveá	&‘sand’\\
f vs. v	&ðəgəfiə	&‘tree sp. (green with yellow fruit and thorns’	&ðəviə	&‘tree sp. (large)’\\
v. vs. w	&váðó	&‘shave’							&wáðó	&‘pierce, poke!’
\end{tabular}\label{ex:ch2:43}
\z 


\ea Alveolar/dental minimal pairs
\begin{tabular}[t]{llp{3cm}lp{3cm}}
d vs. ð	&umədí	&‘small biting ant’	&úmə́ðí	&‘sharp spear with cerrations’\\
d̪ vs. d	&d̪oát̪ó	&‘send, forge!’		&doátó	&‘speak!’\\
d̪ vs. ð	&ud̪əmiə	&‘medicine healer’	&úðə́pí	&‘tree sp. with white lowers’\\
t̪ vs. ð	&et̪a	&‘lake, pool’		&eða		&‘meat’\\
t̪ vs. t	&ot̪eleə	&‘spider’			&otéleə	&‘mat woven from palm leaves’\\
t vs. ð				\\
t̪ vs. tʃ&egogóvat̪a	&‘I am about to return’	&egogóvatʃa	&‘I am about to return smthg’\\
t vs. tʃ	\\			
d̪ vs. dʒ		\\		
d vs. dʒ	
\end{tabular}\label{ex:ch2:44}
\z 

\ea Velar minimal pairs
\begin{tabular}[t]{lllll}
k vs. g &	gakad̪ó	&‘s/he planted’	&gagaðó	&‘s/he mixed (food, words)
\end{tabular}\label{ex:ch2:45}
\z

			
\ea Liquid minimal pairs
\begin{tabular}[t]{lllll}
ɾ vs. r	&wárá	&‘animal pen’	&wáɾá	&‘chicken’\\
ɾ vs. ɽ	&ŋgáɾá	&‘salt’			&ŋáɽáká	&‘small lizard’\\
ɽ vs. l	&ɽavə́gá	&‘seed of ardeb tree’	&lavəðá	&‘fruit of evəða tree’\\
r vs. l	&wárá	&‘animal pen’	&wálá	&‘wool, braids’\\
ɾ vs. l	&gaɾága	&‘s/he is crawling’	&galágá	&‘s/he is planting’
\end{tabular}\label{ex:ch2:46}
\z 


\ea Nasal minimal pairs
\begin{tabular}[t]{lllll}
m vs. n	&ome	&‘fish’				&óna			&‘small basket’\\
n vs. ɲ	&nádádá	&‘roofs of mouth’	&ɲadodo		&‘neck glands’\\
ɲ vs. ŋ	&ɲále	&‘flutes’			&ŋále		&‘flute’\\
n vs. ŋ	&nəməníə	&‘olive tree’	&ŋəməneá	&‘olives’
\end{tabular}\label{ex:ch2:47}
\z 

%TODO Sharon CHECK - VOWELS íə or eá?
\ea Nasal minimal pairs
\begin{tabular}[t]{lllll}
	náɾá	& 	‘non-physical hearts, souls’	 	&	ŋáɾá	&	‘thick ropes’\\
\end{tabular}\label{ex:ch2:48}
\z 
