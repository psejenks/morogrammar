\chapter{Nonverbal predicates}\label{chapter:nonverbal}

Nonverbal predicational clauses are predicational clauses which are not headed by a verb. There are four kinds of nonverbal predicates in Moro: adjectives, deictic predicates, and genitive predicates. In certain contexts these predicates can be used for existential or presentational statements, as we will discuss below. 

Nonverbal predicates form a natural class in Moro grammar in three ways. First, adjectives and genitive predicates function as nominal modifiers in which they agree with the noun (Chapter \ref{chapter:nounphrase}). Deictic predicates cannot always occur as nominal modifiers, but demonstratives can be seen as their nominal counterparts. Second, these predicates inflect like verbs in that they agree with the subject and include a clause type vowel. Third, these predicates do not take the Aspect-Mood-Deixis morphology which is diagnostic of verbs (see Chapter \ref{chapter:verbs}), although in some cases there are alternate versions of this morphology.

%TODO Can deictic predicates occur as nominal modifiers distinct from demonstratives? i.e. the difference between 'this newspaper' and 'the newspaper here'?

\section{Adjectives and adjectival predicates}\label{section:adjective}

Most property concept terms in Moro are expressed with adjectives, which constitute a distinct word class from nouns and verbs. This section surveys the basic morphology on adjectives and how they differ from verbs in predicative environments. It also discusses a process of intensive reduplication which is particular to adjectives and comparative constructions.\footnote{This section draws in part from an unpublished manuscript by Laura Kertz, George Gibbard, Farrell Ackerman, and Sharon Rose entitled `Comparison in Thetogovela Moro.'}

\subsection{Basic adjectival morphology}

Adjectives in Moro resemble verbs in that they consist of a root and are conjugated with many of the same kinds of inflectional affixes. In predicate positions, they consist of a preverb --- subject agreement and a clause vowel (Section \ref{section:preverb}) --- and the adjectival predicate. This closely resembles the modifying environments discussed in Section \ref{section:adjsubrelative}. To see how the morphology of adjectival predicates compares to verbal predicates, compare inflected forms of the adjective \textit{daŋ} ‘dirty’ to the verb \textit{ðaŋ} ‘go up’. The initial morpheme in these examples is the morphologically complex preverb, consisting of subject agreement and a clause vowel:

\ea 
\begin{tabular}[t]{llll}
	\multicolumn{2}{c}{Adjective} 	& 	\multicolumn{2}{c}{Verb} \\
	\midrule
	éga-daŋ-á	& ‘I am dirty’ & 	éga-ðaŋ-ó	& ‘I went up’\\
	ága-daŋ-á	& ‘you are dirty’	& ága-ðaŋ-ó	& ‘you went up’\\
	ga-daŋ-á	& ‘(s)he is dirty’	& ga-ðaŋ-ó	& ‘he went up’	\\
	ŋə́ní ŋa-daŋ-á	 & ‘the dog is dirty’	& ŋə́ní ŋa-ðaŋ-ó	& ‘the dog went up’\\
	lopa la-daŋ-á	& ‘the bird is dirty’	& lopa la-ðaŋ-ó	& ‘the bird went up’\\
\end{tabular}
\z 

Like in the example above, most adjectives are suffixed with the final vowel \textit{-á}, which identifies them as a word class (allomorphs [eə] [ɜ] or [iə]). In predicative uses in main clauses, the root is low-toned and the final vowel bears H tone. This is the same tone pattern as the perfective in verbs, but in perfective verbs, the final suffix is \textit{-ó}.  

Below is a list of adjectives with the typical predicative tone pattern. The initial prefix represents the preverb:

%FORMAT: change ..... to (s)he/it for each example
\ea	\begin{tabular}[t]{ll}
	ga-bəg-á		& ‘(s)he is strong (in sport, hunting)’\\
	ga-bətʃ-á	& ‘(s)he is white’	\\ %TODO gabətʃó??
	ga-mənw-ɜ́	& ‘(s)he is black’\\
	g-aɲəl-á	 & ‘(s)he is sweet’\\
	g-avəl-á	& ‘(s)he is sour’\\
	gɜ-tʃ-ɜ́	     & ‘(s)he is bad’\\
	ga-ʧoɲ-á	& ‘(s)he is hungry’\\
	ga-dogw-á	& ‘(s)he is clean’\\
	ga-do-á		& ‘(s)he is fat’\\
	ga-molw-á 	& ‘(s)he is cold’\\
	ga-ŋəɾ-á	& ‘(s)he is good’\\
	g-obəl-á	& ‘(s)he is short’\\
	g-ogən-á	& ‘(s)he is big’ \\
	g-ow-á	& ‘(s)he is soft, poor in sports’ \\  %	BB - təɽe
	ga-t̪-á	& ‘(s)he is small’\\
	ga-daŋ-á	& ‘(s)he is dirty’\\
	g-oaj-á	& ‘(s)he is rough, course’\\
	g-oal-á	& ‘(s)he is long, tall’\\
	gɜ-w:-ɜ́	& ‘(s)he is hot’\\
\end{tabular}
\z

A few adjectives have a final suffix \textit{-é} instead of \textit{-á}:

%FORMAT
\ea	g-eð-é	‘it is green, unripe’\\
 	g-or-é	‘it is red’\\
	g-ov-é	‘(s)he is relaxed’\\
	g-am-é	‘it is delicious, the best’\\
\z 
		
This irregular ending might be connected to the causative suffix discussed below. 

Some adjectives are complex, consisting of an adjectival root together with one of the locative particles, \textit{nano} `by, next to' \textit{ánó} `inside,' or \textit{aŋəno} `body,' as is also the case with verbs. In the case of vowel-initial particles, it is not clear what the final vowel of the adjective is as it is dropped before the particle, but we assume it is [a] [eə] or [iə] in most cases. If there is [j] appearing before the particle, this is probably due to a front diphthong. 

\ea  %provide semantic components of these? I think they would better be presented in two pieces
\begin{tabular}[t]{ll}
g-ɜ-tʃ-íə nano & ‘(s)he  is sad’\\
g-a-ŋəɾá nano & ‘(s)he is happy’\\
g-a-bəgá nano & ‘(s)he is industrious’\\
g-a-dar-ánó & ‘(s)he is flat’\\
g-a-vəɾ-ánó & ‘(s)he is wide’\\
g-ow-ánó & ‘(s)he is weak’\\
g-uɾw-ánó & ‘(s)he is disappointed’\\
g-ov-áŋəno & ‘(s)he is easy’\\
g-a-meðj-áŋəno & ‘(s)he is skinny’\\
g-a-mətːj-áŋəno & ‘(s)he is smooth’\\
g-a-ɲə́l-áŋəno & ‘(s)he is lazy’\\
g-egj-áŋəno & ‘(s)he is bitter’\\
\end{tabular} \z

Some of these adjectives are paired with adjectives without the particle, and they have related meanings, although not predictable ones. Others must always appear with the particle.%include semantic breakdown?

Some adjectives can inflect for aspect. These adjectives distinguish a stative variant, with the final H tone described above, from an inchoative, which indicates that the state described by the adjective is in the process of being attained. The tone pattern of the inchoative is the same as the tone pattern of the proximal imperfective form of verbs (\ref{section:imperfective}). If the root is just a consonant or if it is vowel initial and has a single vowel, no H tone appears \Next[a]. If the root is vowel-initial and has two vowels, H tone appears on the second vowel \Next[b]. If the root is consonant-initial, H tone appears on the root vowel. In some adjectives, the H tone does not extend to the final vowel \Next[c] and in others, it does \Next[d].

\ea 
\begin{tabular}[t]{llll}
a. & L-L & g-ow-a & ‘(s)he is becoming soft, poor (in sport)’\\
b. & LH & g-ogə́n-a & ‘(s)he is becoming big’	 	\\
c. & H-L & g-a-bə́g-a & ‘(s)he is becoming strong (in sport, 
						hunting)’\\
d. & H-H & g-a-bə́tʃ-á & ‘(s)he is becoming white’	\\
\end{tabular} \z

We have not observed any adjectival roots that are consonant initial and have two vowels, while these are common in verbs. 

Adjectives do not take the \textit{v-} progressive prefix that is found for many vowel-initial verbs in the imperfective. This could be because most adjectives that are vowel initial contain a labial element, which prevents the appearance of v- (see section \ref{section:imperfectiveprefix}) or it could be that this prefix is incompatible with adjectives.

Adjectives can also take durative/iterative prefixal reduplication (Section \ref{section:iterative}), which for adjectives conveys a sense of duration or permanence of the state:

\ea 
\begin{tabular}[t]{ll}
g-a-dat-taŋ-á & ‘he is always dirty’\\
g-a-bap-pətʃ-á & ‘he is white (a type of permanent state)’\\
\end{tabular} \z

We see, in summary, that while adjectives are morphologically distinct from verbs in Moro, they take much of the same basic morphology as verbs in predicative positions.

\subsection{Intensification and attenuation}

Moro possesses a number of strategies for modifying adjectives, both intensification and attenuation. These strategies include degree modifiers, a few distinct forms of full and partial reduplication, and suppletion. Like all other instances of modification in Moro, adjectival modifiers follows the adjectives they modify.

The most productive mechanism for intensification is to use the intensive degree modifiers \textit{káɲ} or \texit{pr}, as in \Next. These modifiers are synonymous, and both have an adverbial use as well as the intensifying use with adjectives illustrated below. Similarly, the attenuating expression \textit{at̪en} can both modify adjectives, as in \NNext, or verb phrases:

%TODO peter elyasir CHECK examples below: simple example of intensive modification 

\ea
	\ea \gll ŋen ŋ-a-p-á prr		pr / káɲ\\
			\textsc{cl}ŋ.matter	\textsc{sm}.\textsc{cl}ŋ-easy-\textsc{adj}	very / very\\
		\glt ‘The matter is very easy.’
	\ex	\gll udʒí 		g-a-bəgá nano	 	pr / káɲ	\\
			\textsc{cl}g.person	\textsc{sm}.\textsc{cl}g-industrious on 	very / very\\
		\glt ‘The woman is very industrious.'
\z \z 
	
	%TODO check  can the intensifier change order with the particle? CHECK
	
\ea
	\ea \gll ŋen ŋ-a-p-á 		at̪en\\
			\textsc{cl}ŋ.matter	\textsc{sm}.\textsc{cl}ŋ-easy-\textsc{adj}	a.bit\\
		\glt ‘The matter is pretty easy.’
	\ex	\gll udʒí 		g-a-bəgá nano	 	at̪en	\\
			\textsc{cl}g.person	\textsc{sm}.\textsc{cl}g-industrious by 	a.bit\\
		\glt ‘The woman is a bit industrious.'
\z \z 

Color adjectives can be attenuated by forming compounds with the noun \textit{aŋəno} `body,' which was observed above to be a part of other complex adjectives. The semantic effect is that of approximating the target color, similar to `light' or `almost':

\ea Attenuated complex color terms \\
\begin{tabular}[t]{llll}
goré & `(s)he is red' & goré aŋəno & `(s)he  is orange'\\
geðé & `(s)he is green' & geðé aŋəno & `(s)he is light green’ \\
gamənó  & `(s)he is green' & gamənó aŋəno & `(s)he is dark blue'\\
gabətʃó  & ‘(s)he is white' & gabətʃó aŋəno & ‘(s)he is beige, grey, silver'\\
\end{tabular}
\z


There are several reduplicative patterns for intensive modification. The first is through gerundive nominalization (Section \ref{section:gerund}), formed by the class prefix ð- and a suffix -aŋ. These intensive reduplicated adjectives follow the normal predicative adjective:

%check total reduplication, as cited in Kertz paper
\ea
	\ea \gll umːiə 	g-oal-a		ð-oal-aŋ\\
			\textsc{cl}g.boy	\textsc{sm}.\textsc{cl}g-tall-\textsc{adj}	\textsc{cl}ð-tall-GER\\
		\glt ‘The boy is tall tall (= very tall).’
	\ex	\gll udʒí 		g-aɲəlá-ŋəno 		ð-áɲə́l-áŋ	\\
			\textsc{cl}g.person	\textsc{sm}.\textsc{cl}g-lazy-body	\textsc{cl}ð-lazy-GER\\
		\glt ‘The woman is lazy lazy' (= very lazy).'
\z \z 
	 
For many other adjectives, including color terms, intensification is achieve through a specific reduplicative process which is not necessarily predictable from the phonological shape of the adjective, illustrated in Table \ref{tab:ch10:1}. Not all adjectives have such a reduplicated form. Note that the reduplicant, which again like other intensive modifiers follows the head, often is motivated by phonological template where it obligatorily includes a geminate consonant or heavy syllable. Interestingly, such a template is also part of iterative verbal reduplication (Section \ref{section:iterative}). Some forms include an additional suffix, \textit{-éθ/-íθ}. The reduplicated portion of the adjective must precede the particle in the case of complex adjectives, unlike with the gerundive nominalization. This clearly indicates that the gerundive intensification is a syntactic process while intensive partial reduplication is a morphological process.

\begin{table}
\begin{tabular}{lllll}
VC&	geðé	&	‘green’	&	geðé-ððé		&	`very green' \\
&	gov-ánó	&	‘weak’	&	gová-ff-éθ ánó	&	‘very weak’\\
&	goajá	&	rough’	&	goajá-s-éθ		&	‘very rough’\\
&	gəviə́	&	‘light (weight)’&	gəviə́--ff-íθ	&	‘very light’\\
VCVC&	gabətʃó &	‘white’&	gabətʃə́-bətʃtʃé	&	`very white' \\
%&	&	&	gabətʃó pogél&	`bright shiny light' \\
&	gobəlá	&	‘short’	&	gobəlá-bəll-éθ	&	‘very short’ \\
&	gavəlá	&	‘sour’	&	gavəlá-vəll-éθ	&	‘very sour’\\
&	gogəná	&	‘big’	&	g-o-ogəná		&	‘very big’\\
C&	gamé	&	‘delicious’&	gamə́-mmé		&	‘very delicious’\\
&	gɜwːɜ́	&	‘hot’	&	gɜwwɜ́-pp-iθ		&	‘feverish’\\ %suffix tone?
&	gɜwːɜ́ -aŋəno&	‘lazy’&	gɜwwɜ́-pp-iθ aŋəno	&	‘very lazy’\\
&	gapːá	&	‘light (thin)’&	gabá-pp-éθ	&	‘very light’\\
CVC&	gavəɾ-ánó 	&	‘wide’&	gavəɾa-fərr-éð ánó	&	‘very wide’\\
&	gabəgá	&	‘strong’&	gabəgá-bəgg-éθ	&	‘very strong’\\
&	gatʃoɲá	&	‘hungry’&	gatʃoɲá-tʃoɲɲ-éθ	&	‘very hungry’\\
&	gaɲəlá	&	‘sweet’	&	gaɲəlá-ɲəll-éθ	&	‘very sweet’\\
&	gadowá	&	‘fat’	&	gadow-doww-éθ	&	‘very fat’\\
&	gaməló	&	‘cold’	&	gaməló-moll-é(θ)	&	‘very cold’\\
&	gamətːj-aŋəno &	‘smooth’&	ga-mətə́-mətté aŋəno	&	‘very smooth’\\
&	gadəɲá	&	‘salty’	&	gadəɲá-dəɲɲ-éθ	&	‘very salty’\\
&	gɜtəɲiə́	&	‘thick’	&	gɜtəɲə́-təɲɲ-iθ	&	‘very thick’\\
&	gadəgó	&	‘clean’	&	gadəgó-tega	&	‘very clean’\\
&	gadar-ánó&	‘flat’	&	gadaar-ánó	&	‘very flat’\\
\end{tabular}	
\caption{Intensive partial reduplication}
\label{tab:ch10:1}
\end{table}

Finally, with many common adjectives, intensification can be achieved morphologically, with a irregular intensive suffix which has the general shape \textit{-tV́C}, as illustrated in Table \ref{tab:ch10:2}. These forms are offered in the same elicitation contexts as the partially reduplicated forms in Table \ref{tab:ch10:1}, they they potentially represent alternative realizations of the same grammatical phenomenon. Observe that there are still some traces of phonological dependence for intensive suffixes: the vowel in this suffix appears to undergo height and rounding harmony, while the final consonant seems to be irregular. It may simply be best to think of these forms as suppletive. Intensive suffixes only seem to occur in a restricted number of forms, so irregular realizations of this suffix are likely memorized. Evidence for this conclusion comes from the unpredictable semantic and morphological correspondence between, for example \textit{gat̪á} `small' and \textit{gɜmə-ðə́ŋ }‘tiny’ (people)  vs. 	\textit{gɜmə-tə́ŋ}	‘tiny’ (things).

\begin{table}
\begin{tabular}{lllll}
&	goɾé&	‘red’		&	gor-táŋ			&	`very red'\\
&	gat̪á	&	‘small’		&	gɜmə-ðə́ŋ			&	‘tiny’ (people)\\
&			&			&	gɜmə-tə́ŋ			&	‘tiny’ (things)\\
&	gaŋəɾá	&	‘good’	&	gaŋəɾá-tə́l		&	‘very good’\\
&	gegé	&	‘bitter’&	gegə́--tté		&	‘very bitter’\\
&	gɜtʃiə́	&	‘bad’	&	gɜdʒɜ́--tɜ́ŋ		&	‘very bad’ \\
&	goalá	&	‘tall’	&	goalá-taj		&	`very tall'\\
&	gɜmənwɜ́	&	‘black’	&	gɜmənwɜ́--túθ		&	`very black'\\
&	gadaŋá	&	‘dirty’	&	gadaŋá-táŋ		&	‘very dirty’\\
\end{tabular}	
\caption{Intensive suffixes on adjectives}
\label{tab:ch10:2}
\end{table}



\subsection{Comparative, equative, and superlative forms of adjectives}\label{sec:ch10:comp}

Comparative adjectives are rendered by using the applicative suffix \textit{-(ə)t̪,} which raises vowels, glossed \textsc{\textsc{cmp}} below. The vowel /ə/ is dropped after coronal sonorants. It attaches to the verb root and precedes the final aspect-mood-deixis suffix. Although a standard adjective ends in \textit{-á}, the addition of the applicative suffix causes the adjective to take the perfective final vowel \textit{-ó} instead (raised to [ú] by vowel harmony):

\ea 
	\ea \gll máŋga 	ð-a-ɲəl-á	\\
			\textsc{cl}ð.mango 	\textsc{sm}.\textsc{cl}ð-\textsc{rtc}-sweet-\textsc{adj}\\
		\glt ‘The mango is/mangoes are sweet.’
	\ex	\gll máŋga 	ð-ɜ-ɲəl-t̪-ú 			musi	\\
			\textsc{cl}ð.mango 	\textsc{sm}.\textsc{cl}ð-\textsc{rtc}-sweet-\textsc{cmp}-\textsc{pfv}	\textsc{cl}ð.banana\\
		\glt	‘Mango is sweeter than/tastes better than banana.’
\z 	\z 

In these constructions, Moro only allows phrasal rather than clausal standards of comparison. Thus, there is no Moro equivalent of an English sentence such as \textit{Mangos are sweeter than bananas are}.

Like other applicative objects, standards of comparison behave like regular objects. They take accusative case marking and can occur as object markers on the adjective.

\ea
	\ea \gll 	kúk:u     		g-ɜ-ʧuɲ-t̪-ú                           ŋálo-ŋ\\
 Kuku  	\textsc{cl}g-\textsc{rtc}-hungry-\textsc{cmp}-\textsc{pfv}  	 Ngalo-ACC\\
		\glt  ‘Kuku is hungrier than Ngalo’
	\ex \gll kúk:u  	g-ɜ-ʧuɲ-t̪-ə́-ŋó             \\            	 	
			Kuku  	\textsc{cl}g-\textsc{rtc}-hungry-\textsc{cmp}-\textsc{pfv}-3\textsc{sg.om}  	 \\
		\glt	‘Kuku is hungrier than him’
	\z
\z 

%todo check can comparatives be passivized? 

If no standard of comparison is specified, Moro uses the word \textit{orəwa} meaning ‘sibling, relative’ (Section \ref{section:kinship}), even with inanimate subjects. The use of this word might indicate reference to the subjects `relatives', i.e., its contextually relevant comparison class for the subject.

\ea 
	\ea \gll umːiə́-kːi 		g-ogən-á\\
	\textsc{cl}g.boy-\textsc{cl}g.\textsc{dem}	\textsc{sm}.\textsc{cl}g-big-\textsc{adj}	\\
		\glt ‘this boy is big’
	\ex \gll  umːiə́-kːi 		k-ugən-t̪-ú 			orəwa\\
			\textsc{cl}g.boy-\textsc{cl}g.\textsc{dem}.	\textsc{sm}.\textsc{cl}g-big-\textsc{cmp}-\textsc{pfv}	\textsc{cl}g.brother\\
		\glt ‘this boy is bigger than that one’ or ‘this boy is bigger than most boys’
	\z \z 
	%todo check clausal comparison: Kuku cooked more than I (could) eat
	
Superlative adjectives are formed in the same way as comparatives, but use a plural standard of comparison:

\ea 
	\ea \gll máŋga 	ð-ɜɲəl-t̪-ú 			l-orə-wa	\\
		\textsc{cl}ð.mango 	\textsc{sm}.\textsc{cl}ð-sweet-\textsc{cmp}-\textsc{pfv}	\textsc{cl}l-brother-3.SG.POSS\\
		\glt ‘mango is the sweetest fruit/tastes better than all other fruits’
	\ex	\gll umːiə́-kːi 		k-ugən-t̪-ú 			l-orə-wa\\
		 	\textsc{cl}g.boy-\textsc{cl}g.\textsc{dem}.	\textsc{sm}.\textsc{cl}g-big-\textsc{cmp}-\textsc{pfv}	\textsc{cl}l-brother-3.SG.POSS\\
		\glt ‘this boy is the biggest’
\z \z 

Equative comparisons use the standard form of the adjective. They can, however, be combined with the gerundive intensifier to give the emphasized sense. 

\ea 
	\ea \gll kúk:u 	g-oal-a 		nə́--ŋéɾa \\
			\textsc{cl}g.Kuku	\textsc{sm}.\textsc{cl}g-tall-\textsc{adj}	COMP2-\textsc{cl}ŋ.girl	\\
		\glt ‘Kuku is as tall as the girl’
	\ex \gll kúk:u 	g-oal-á 	ð-oál-áŋ 	nə́--ŋálo	\\
			\textsc{cl}g.Kuku	\textsc{sm}.\textsc{cl}g-tall-\textsc{adj}	\textsc{cl}ð-be tall-GER	COMP2-\textsc{cl}ŋ.Ngalo	\\
		\glt ‘Kuku is as extremely tall as Ngalo’ (both people are very tall)
\z \z 

%todo check: For object in \Last[b], can this take other objects following it? i.e., is this a case of ellipsis?

\subsection{Causative adjectives}\label{sec:ch10:causadj}

Adjectives can occur in a causative form, in which case they take the suffix \textit{-e}. This causative form is different from that which occurs with verbs, which take the suffix \textit{-i} with vowel raising \ref{section:causative}); there is no vowel raising in causative adjectives. Furthermore, while causative verbs show variation between different final vowels, either [eə] or [iə], causative adjectives always surface with final [e].

Semantically, the causative form of adjective converts the stative semantics of the normal adjective to a causative inchoative or caused change of state. So if an adjective means `be X,' the causative will mean `cause someone or something to become X:'

\ea	\begin{tabular}[t]{llll}
	\multicolumn{2}{c}{Adjective} & \multicolumn{2}{c}{Causative adjective} \\
		{g-obəl-á} & ‘(s)he is short’ & {g-obəl-é} & ‘(s)he made s.o. short’\\
		{ga-bet̪-á} & ‘(s)he is satisfied’ &  {ga-betʃ-é} &  ‘(s)he satisfied s.o.’\\
g-avəl-á &	‘it is sour’ & 	g-avəl-é	& ‘(s)he made it sour’\\
\end{tabular}
\z 

If an adjective is used in the causative form, an object can be expressed, who is the affectee and who undergoes the change of state. If an adjective is a complex predicate, the object intervenes between the adjective and the particle. The opposite order is not allowed: 

\ea \gll 	é-g-a-bəg-é 			kúkə-ŋ	nano\\
	1SG\textsc{sm}-\textsc{cl}g-\textsc{rtc}-be strong-\textsc{adj}.CAUS	Kuku-\textsc{acc}	PART\\
	\glt ‘I encouraged, strengthened Kuku.’ 
\z 

This is a general property of complex predicates in Moro, as particles associated with verbs must allow follow the first object.%TODO crossref section?

%What happens with roots that have high vowels?


%TODO peter elyasir Adjectives can also be conjugated in dependent clause forms. SAME AFFIX AS BELOW OCCURS
%
%%		é-g-a-nː-a ɲeXXXXX		‘I am not dirty’ 
%%						‘I want to be tall’

\subsection{Imperative adjectives}

Like verbs, Moro adjectives can appear in the imperative, which has two forms in Moro, a itive and a venitive imperative (Chapter \ref{chapter:imperative}). In both imperative forms, adjectives are distinguished from verbs by the presence of the inchoative suffix \textit{-et̪ }, \textit{-ət̪ }, or \textit{-t̪ }, along with the final vowel \textit{-ó}. This suffix, which occurs in a number of contexts besides the imperative, converts the stative adjective into an inchoative or change of state predicate. Thus, the imperative of ‘clean!’ would be better translated ‘become clean!'  

%TODO peter elyasir Check adjectival infinitives? section on adjectival infinitives/imperfecties? negative adjectives?

Like proximal imperative verbs, proximal imperative adjectives are high-toned. Short roots with a full vowel take \textit{-ət̪,} and the form with no vowel [t̪] appears to be an allomorph of \textit{-ət̪ } after coronal sonorants. Long roots (VCəC) and short roots with no vowel or [ə] take \textit{-et̪.} One could surmise, therefore, that \textit{-ət̪ } is a reduced form of \textit{-et̪.}. %IS IT POSSIBLE FOR LOW-TONED?  %I have the feeling that this might be a regular phonological process and wonder if it should really be in this section.

\ea 
\begin{tabular}[t]{llll}
\multicolumn{2}{c}{Allomorphs \textit{-t̪} and \textit{-ət̪}} & Allomorphs \textit{-et̪}\\
\midrule
wáj-t̪-ó  	& ‘be rough, coarse!’ & áɲə́l-ét̪-ó 	& ‘be sweet’\\
ál-t̪-ó  	& ‘be long!’ & ávə́l-ét̪-ó	& ‘be sour!’\\
óŕ-t̪-ó  	& ‘be red!’ & bə́g-ét̪-ó 	& ‘be strong!’\\
ów-ə́t̪-ó  	& ‘be soft, poor in sports!’ & vɾ-ét̪-ánó 	& ‘be wide!’\\
ám-ə́t̪-ó  	& ‘be best, taste good!’ & ʧ-ít̪-ú	 	& ‘be bad!’\\
dóg-ə́t̪-ó  	& ‘be clean!’ & wː-ít-ú 	& ‘be hot!’\\
dów-ə́t̪-ó 	& ‘be fat’ & ʧóɲ-ét̪-ó 	& ‘be hungry!’\\
éð-ə́t̪-ó 	& ‘be green, unripe!’ & mə́tː-ét̪-áŋəno & ‘be smooth!’\\
dáŋ-ə́t̪-ó 	& ‘be dirty!’ & ŋə́r-ét̪-ó	 & ‘be good!’\\
& & ɲəl-et̪-aŋəno & ‘be lazy’\\
& & ógə́n-ét̪-ó 	& ‘be big!’\\
& & t-ét̪-ó 		& ‘be small!’\\
\end{tabular} 
\z 

Adjectives can also be conjugated in the venitive imperative, also in the inchoative. As with verbs, the imperative is low-toned and ends in \textit{-a}. The sense conveyed is one of venitive, but with a notion of two events: become X and come towards the speaker, as is common with other motion predicates in the venitive. This is the concept of alloying discussed for Tima (Alamin, Schneider-Blum & Dimmendaal 2012, Dimmendaal 2014) %TODO format references, include discussion of this elsewhere? section on space?

\ea  \begin{tabular}[t]{ll}
ŋár-ét̪-ó	& 	‘be good!’\\
ŋar-et̪-a	&	‘be good and come!’ \\
\end{tabular} \\ 
\z

For more on the semantics of the venitive/andative distinction, see Section \ref{section:space}.

Evidence for inchoative nature of \textit{-et} comes from the imperative of causatives, where \textit{-et} disappears. As causative adjectives themselves are inchoative, the inchoative becomes irrelevant:


\ea	\begin{tabular}[t]{llll}
	\multicolumn{2}{c}{Imperative} & \multicolumn{2}{c}{Causative imperative} \\
ávə́l-ét̪-ó	& ‘be sour!’	& ávə́l-é	& ‘make it sour!’\\
óbə́l-ét̪-ó	& ‘be short!’	&  óbə́l-é	& ‘make s.o. short!’\\
\end{tabular}
\z 

In fact, the observation that either \textit{-et} or the causative form is required for adjectives to appear in the imperative, and the fact that once \textit{-et} is attached the adjective inflects like a normal verb in the imperative, suggests that the inchoative and causative are verbalizing suffixes. %TODO peter elyasir check Further evidence for this claim comes from the inability of other non-verbal predicates, such as deictic and possessive predicates, to appear in the imperative or infinitive.?



%\subsection{Summary of adjectives}

%In summary, there are three morphological properties that differentiate adjectives from verbs. First, to convey stative predicative adjectives, the final vowel is \textit{-á} and the root is low-toned. This pattern does not occur with verbs. Second, in the imperative, adjectives must have an inchoative suffix \textit{-et̪ } (or \textit{-ət̪}}). Third, the causative of adjectives is \textit{-e}, whereas verbs take \textit{-i} that triggers vowel raising.  

	 

\section{Deictic and existential predicates}

Deictic copular clauses express spacial relationships, but mark a three-way distinction in the distance of the location relative from the speaker or a contextual supplied origo:
\ea 
	\ea \gll   í-g-ɜ-ní 		ɜ́ni / lɜmú / n-ajén \\ 
				\textsc{1sg}-\textsc{cl}g-\textsc{rtc}-be.1d	here / Khartoum / on-mountain \\
		\glt 	`I'm here / here in Khartoum / here in the mountains.’
	\ex \gll   g-ɜ-tú 		tu / lɜmú / n-ajén \\ 
				\textsc{1sg}-\textsc{cl}g-\textsc{rtc}-be.2d	there / Khartoum / on-mountain \\
		\glt 	`She's there/ there in Khartoum / there in the mountains.’
	\ex \gll   g-ɜ-nnoɜŋ 		nwal:aŋ / lɜmú / n-ajén \\ 
				\textsc{1sg}-\textsc{cl}g-\textsc{rtc}-be.3d	yonder / Khartoum / on-mountain \\
		\glt 	`She's yonder/ in yonder Khartoum / in yonder mountains.’
\z 
\z 

\ea 	\gll nala		n-ɜ́ní-ɲi		ég-ə́tám	\\
			necklace	\textsc{cl}n-\textsc{rtc}-be.loc-\textsc{pfv} LOC-neck\\
		\glt	‘The necklace is here on my neck.’
\z %TODO peter elyasir check THIS SEEMS TO BE SOME KIND OF POSSESSOR RAISING. Is it typically true that when locations relative to one's body are being described that these predicates take a DP complement?

%try the necklace is there on your head
%try the necklace is on her head
%try the necklace is in my house
%try the necklace is in our house

Deictic copula consist of a preverb (Section \ref{section:preverb})and an adverbial predicate. That the central predicate is an adverb can be seen by comparing the adverbial . While they occur with normal subject agreement, deictic copula do not inflect for aspect, they cannot be embedded under auxiliaries, and they do not take any extension suffixes. The locative phrases following the deictic modifiers above are are locative adverbs (Section \ref{section:locativeadverb}), and they are optional, as we will see below.

In order to be negated, deictic constructions make use of the negative copula \textit{-eró}. Because this copula does not make deictic distinctions, these are neutralized in this context unless they are explicit in the complement:

\ea 
	\ea \gll	é-g-eró			ɜ́ni	/ lɜmú / n-ajén \\									
				1sg-clg-not.be	here / Khartoum / on-mountain \\			
		\glt 	‘I’m not here/ in Khartoum/ on the mountain.’
	\ex \gll 	g-eró			ətu			\\							
				clg-not.be		there		\\
		\glt	‘She’s not there.’
\z 
\z 
Like its positive deictic copular counterparts, the negative copula does not mark any of the inflectional distinctions typical of verbs outside of being preceded by a preverb. Thus, these predicates are not verbs. The negative predicate \textit{-eró} also occurs as an negative existential adjective in noun phrases, meaning `no.'

While the locative phrase in a deictic copular clause cannot be extracted, either in a passive (not shown), which shows it is not an argument of the verb, or a cleft (\ref{ex:nonverb:noexa}-\ref{ex:nonverb:noexb}). However, the locative part of the negative predicate can be extracted (\ref{ex:nonverb:noexc}):

\ea 
	\ea[*]{ \gll	ŋgwa		nə́=	 	í-g-ɜ́ni	 \\
		where		\textsc{comp2}	\textsc{1sg}-\textsc{cl}g-be.1d\\
		\glt `Where am I here?’ (intended)}	 \label{ex:nonverb:noexa}
	\ex[*]{ \gll	ŋgwa		 kúkú g-ɜ́-tu	 \\
		where		Kuku  \textsc{cl}g-be.1d\\
		\glt `Where is Kuku there?’ (intended)}		\label{ex:nonverb:noexa}
	\ex  \gll	ŋgwa		nə́-kúk:u	 	g-eró	 \\
		where		\textsc{comp2}-Kuku	\textsc{cl}g-neg.indef\\
		\glt `Where is Kuku not?’\\
	Speaker comment: ``Everywhere you go, you find Kuku.'' \hfill EJ8116
 \label{ex:nonverb:noexc} \z  
\z 
The most likely explanation for the contrast between the negative and positive examples above is pragmatic: because the deictic predicate encodes spacial deixis, it is nonsensical to ask where someone is when the deictic predicate is in use. `Where' questions can instead be formed with the locative copula (Section \ref{section:loccop}).
	
%TODO peter elyasir check example LLast, with 1sg subject; original version of this sentence was with kuku as subject, which might be weird. 

Presentational (`Here's a \ldots') and existential (`There's a \ldots') clauses are identical to deictic copular clauses, however they lack the locative adverb:

\ea 
	\ea \gll	írtə́	g-ɜ́ni\\
				knife 	\textsc{cl}g-here\\
		\glt 	‘Here’s a knife.’ 		
	\ex \gll 	írtə́	g-ɜ́tu\\
				knife 	\textsc{cl}g-there\\
		\glt 	‘There’s a knife.’	\hfill EJ81214	
\z \z 
While bare nouns in subjects position in most contexts are interpreted as definite (Section \ref{section:baren}), the subjects of presentational clauses are typically indefinite. 

%TODO peter elyasir check classes of definite/referential/quantificational elements in this position, e.g. a proper name
%TODO peter elyasir check do these have a demonstrative/definiteness effect when there's no spacial object?
%TODO check with indefinite 'subject'

 
Like deictic copular clauses, existential copular clauses are negated with the negative copula \textit{-eró}:

\ea	
	\ea \gll	írtə́	g-eró\\
				knife 	\textsc{cl}g-not.be\\
		\glt 	‘There isn't a knife.’
	\ex \gll 	ŋáwá ŋ-eró\\
				water \textsc{cl}ŋ-not.be\\
		\glt	`There isn't any water.' \hfill EJ72116
\z \z 

\section{Possessive predicates}\label{section:posspred}

Possessive predicates are formed by adding a preverb (Section \ref{section:preverb}) to a noun or noun phrase. The examples below involve a proper noun and a normal common noun, which can be singular or plural:

\ea 
\ea \gll  	ŋíní ŋ-ɜ-kúk:u\\
			dog \textsc{cl}ŋ-\textsc{rtc}-Kuku\\
	\glt 	`The dog is Kuku's.'
\ex	\gll	ŋíní ŋ-a-ðamala\\
			dog \textsc{cl}ŋ-\textsc{rtc}-camel\\
	\glt	`The dog is the camel's.' \hfill EJ52417
\ex	\gll	ðamala ð-a-ɲerá\\
			camel \textsc{cl}ŋ-\textsc{rtc}-girls\\
	\glt	`The camel is the girls'.' \hfill EJ52417
	\z 
\z 
As the clause vowel (\textsc{rtc}) undergoes harmony with the noun, it is clear that these elements for a single phonological word for the purposes of vowel harmony. Possessive predicates are appropriate in two semantic contexts, as are all possessive predicates: they can express an ownership relationship, i.e., that the dog belongs to Kuku, or a beneficiary one, i.e. the dog is for Kuku.

Because the noun in Moro is always initial within the noun phrase, it is always the noun which the preverb attaches to. The noun can be modified normally, for example by adjectives, indefinite quantifiers and universal quantifiers, and both inalienable and alienable possessors: %thought: what about cases of NP ellipsis? whose mother's is this? It's kuku's? Which man's book is this? This book is that's? This book is these three's (not good in English, interestingly).

\ea 
\ea \gll  	ŋíní ŋ-um:íə g-oala\\
			dog \textsc{cl}ŋ-boy \textsc{cl}g-tall\\
	\glt 	`The dog is the tall boy's.'
\ex	\gll	ŋíní ŋ-um:íə g-ɜn:əŋ\\
			dog \textsc{cl}ŋ-boy \textsc{cl}g-tall\\
	\glt	`The dog is some boy's.'
\ex	\gll	ðamala ð-a-ɲerá ododo\\
			dog \textsc{cl}ŋ-boy \textsc{cl}g-tall\\
	\glt	`The dog is some boy's.'
\ex	\gll	gəlá g-acevaŋ g-ɜ-ləŋg-áɲ \\
			white.gourd \textsc{g}-food \textsc{cl}g-\textsc{rtc}-mother-\textsc{1sg.poss}\\
	\glt	`The plate of food is for my mother.'
\ex	\gll	gəlá g-acevaŋ g-a-ðamala íð-ɜŋəðɜŋ\\ 
			white.gourd \textsc{g}-food \textsc{cl}g-\textsc{rtc}-camel \textsc{scl}ð-\textsc{1sg.poss}\\
	\glt	`The plate of food is for my camel.' \hfill EJ52417
	\z 
\z 

Pronominal possessive predicates consist of a preverb plus a possessive pronoun. In addition to the subject agreement on the preverb, the pronoun itself inflects with the class prefix of the subject, meaning that possessive pronominal predicates essentially agree twice with the subject. When the subject is a first or second person (null) pronoun, the pronominal predicate is inflected the human class \textit{g} (Section \ref{section:classgn}), the same class that accompanies first and second person subject agreement (Section \ref{section:subjectagreement}). The examples below have the pronominal portion of the possessive predicate bracketed to avoid confusion.

\ea
\ea \gll  	ŋíní ŋ-ɜ-[ŋ-ɜ́ŋ]\\
			dog \textsc{cl}ŋ-\textsc{rtc}-[\textsc{cl}ŋ-\textsc{1sg.poss.pro}]\\
	\glt 	`The dog is mine.'
\ex \gll  	ðamala ð-a-[ð-ó]\\
			dog \textsc{cl}ð-\textsc{rtc}-[\textsc{cl}ð-\textsc{2sg.poss.pro}]\\
	\glt 	`The camel is yours.'
\ex \gll  	ɜ́-g-ɜ-[g-ɜ́ŋ] \\
			\textsc{2sg}-\textsc{cl}g-\textsc{rtc}-[\textsc{cl}g-\textsc{1sg.poss.pro}]\\
	\glt 	`You are mine'  \hfill EJ52517
\z \z  
The complete paradigm for possessive pronominal predicates is in \tabref{tab:ch7:3}.

While possessive predicates occur with regular preverbs, possessive predicates are syntactically distinct from verbal predicates in at least two ways. First, in cases of subject extraction, the clause vowel surfaces as \texti{ə́}, which is the dependent clause form of the vowel rather than the expected subject extraction vowel \textit{é} (\textsc{dpc1}, see Section \ref{section:clausevowel} on clause vowels):

\ea
\ea \gll  	ŋwɜ́- ndɜ-k-i 	g-ə́--ŋerá\\
			\textsc{foc}-what-\textsc{scl}g-this \textsc{cl}g-\textsc{dpc2}-\textsc{cl}ŋ-\textsc{1sg.poss.pro}\\
	\glt 	`What is the girl's?'
\ex \gll  	ŋwɜ́- ndɜ-k-i 	g-ə́--ðamala \\
			\textsc{foc}-what-\textsc{scl}g-this \textsc{cl}g-\textsc{dpc2}-\textsc{cl}ŋ-\textsc{1sg.poss.pro}\\
	\glt 	`What is for the camel?'   \hfill EJ52517
	\z 
\z 
Second, like non-predicational copular clauses (Section \ref{section:nonpredcop}), possessive predicates cannot be embedded under normal negation, but instead require the negative auxiliary to occur with a complementizer which embeds a possessive predicate with the main clause vowel:

\ea 
\gll  ðamala ð-ɜ-n:ɜ́  t̪ɜ́	ð-a-ŋerá\\
      camel  \textsc{cl}g-\textsc{rtc}-neg.aux \textsc{comp1} \textsc{cl}g-\textsc{rtc}-girl\\
\glt `The camel is not for the girls.' e%what's going on with vowel harmony here?
\z 
In contrast, verbal predicates occur in an infinitive verb form in the same clause as the negative auxiliary (Section \ref{section:negaux}). 

%check: the dog is not mine
%check: the dog is not yours


%
%Q: 	ŋíní  ŋ-ɜdʒɜ́ (íŋːi)? 	Whose dog is this? 
%A: 	ŋíní ŋa-káka 		It’s Kaka’s dog	(this is a complete sentence)
%	ŋíní ŋɜ-kúk:u		It’s Kuku’s dog	(note the vowel harmony)


%todo peter elyasir check can the adjectives below all occur as main predicates? can numerals?
	
%In addition to the adjectives listed above, there are another set of indefinite adjectives that also agree in noun class, but do not conjugate like predicate adjectives. 
%
%		Cəmat̪aŋ			‘some’
%		Cənəŋ			‘any’
%		CənəCənəŋ		‘one by one’
%		Cunto			‘one, on one’s own, alone’
	
	
	%Black & Black adjectives NOT in our database:
%
%gero		‘be absent’	NOT adj
%gwaiɲa		‘be many’	yes, but is \textsc{adj}?
%gəpiano		‘be useless’	??
%gaminat̪o	‘be more	NOT adj
%g-irəwano	‘be ashamed’	NOT adj → discourage
%guðərwat̪o	‘be straight’	?? NOT adj
%golinan		be deep		oal - ? long, tall?
%gaməlu		not yet		yes, but is \textsc{adj}? aux. verb
%
%gəmat̪an	some
%gwomən	another
%gəmən		another
%gənəŋ		certain
%galdoŋai	light, unimportant
%gɜpi		alright, ok
%gabapeð	light in weight




