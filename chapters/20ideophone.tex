\part{Expressive and social language}

\chapter{Ideophones}\label{chapter:ideophones}

Introduction here

\section{SECTION NAME HERE}

Section intro here.

\subsection{Locative \textit{n-}}

Below is from noun section.

There is another locative prefix n-. In general, Moro employs postpositions rather than prepositions. The two locative prefixes (é- and n-) resemble prepositions in their syntactic usage, but must appear attached to nouns. It is also possible to analyze them as case markers. The general meaning of the locative n- is `on’, but it can also convey other senses such as `off, from, over’.  

\ea 
	\ea \gll é-g-a-daŋ-ó n-deté		\\	
			I-SAt ON-BRANCH\\
		\glt `I sat on the branch'	
	\ex \gll loandra lʌmurkú n-ajn		 \\
			rock rolled on-hill\\
		\glt the rock rolled down the hill
	\ex \gll k-aŋg-at̪-ó n-ʌlbʌ́mbəriə	\\	
			cl-?-loc.appl-pfv on-stool\\
	\glt 	`he moved off the stool'??	6/16/2011 %(SHARON's data)
	\ex		kʌmuɾəʤəʧí n-alét̪a	\\
			rock rolled on-hill\\
			'he passed it over the wall'
	\z
\z 

To determine: is this prefix /n/ or /nə-/?

Allomorphs when attaching to coronal-initial roots in Thetegovela? Clear differences in Werria here.

\subsection{Subsection 2 here}

tions in their syntactic usage, but must appear attached to nouns. It is also possible to analyze them as case markers. The general meaning of the locative n- is `on’, but it can also convey other senses such as `off, from, over’.  

	\ea \gll é-g-a-daŋ-ó n-deté		\\	
			I-SAt ON-BRANCH\\
		\glt `I sat on the branch'	
\z 

To determine: is this prefix /n/ or /nə-/?

Allomorphs when attaching to coronal-initial roots in Thetegovela? Clear differences in Werria here.


\section{Section 2 here}

tions in their syntactic usage, but must appear attached to nouns. It is also possible to analyze them as case markers. The general meaning of the locative n- is `on’, but it can also convey other senses such as `off, from, over’.  

	\ea \gll é-g-a-daŋ-ó n-deté		\\	
			I-SAt ON-BRANCH\\
		\glt `I sat on the branch'	
	\ex \gll loandra lʌmurkú n-ajn		 \\
			rock rolled on-hill\\
		\glt the rock rolled down the hill
\z 

To determine: is this prefix /n/ or /nə-/?

Allomorphs when attaching to coronal-initial roots in Thetegovela? Clear differences in Werria here.

d