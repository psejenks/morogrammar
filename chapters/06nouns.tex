\part{Nouns and noun phrases}

\chapter{Nouns and nominal morphology}\label{chapter:nouns}

This section surveys the grammatical properties of nouns in Moro, including their distribution into noun classes, kinship terms, and nominal morphology. 

Phonologically, most nouns in Moro are two or three syllables long. Single, monosyllabic nouns are infrequent, and are of the form CV, CVC or CCV. There are no single V nouns. Nouns of four or more syllables are also less common than the two or three syllable nouns.

\ea 	\begin{tabular}[t]{llll}
  \multicolumn{2}{l}{Consonant-initial nouns} &   \multicolumn{2}{l}{Vowel-initial nouns} \\
\midrule
ðí	&	‘thorn’			&		&	 \\
lɜmí	&	‘beard’	&	ege		&	‘house’ \\
ŋaməlá	&	‘mark, stain’	&		it̪əlí	&	‘year’\\
ləmakə́ŋé	&	‘type of dance’	&	odəlóŋá	& ‘fox’ \\
			 	\end{tabular} \z


The distribution of tone within the nouns was discussed in section XX on tone. 

\section{Noun classes}

Like other Kordofanian languages, Thetogovela Moro has a rich noun class system. Every noun in Moro is assigned to a noun class, which is typically characterized by a single consonant prefix or vowel prefix. Some vowel-initial nouns, however, had consonant prefixes historically, but these have now disappeared. The noun classes correspond loosely to semantic classes, such as humans, animals or trees. Most nouns have singular and plural forms which differ in noun class prefix marking. For example, the word \textit{ðí} ‘thorn’ has a plural \textit{rí} ‘thorns’ with a different consonant. Some nouns, however, have a single invariant form.

In addition to the prefix that appears on every noun, noun class agreement or concord is observed, in which subject markers on the verb and some words or affixes modifying the noun match the noun class prefix found on the noun itself, as illustrated below:

\ea	Noun class concord
\ea 
\gll ŋ-ə́ní-ŋːí                 ŋ-é-t-á 	ŋ-obəð-ó	\\
\textsc{cl}-dog-\textsc{cl.dem}    \textsc{sm.cl}-\textsc{dpc1}-small-\textsc{adj}    \textsc{sm.cl}-run-\textsc{pfv}\\
\glt ‘this small dog ran away’’
\ex
\gll ð-amalɜ́-ðːí               ð-e-t-á                        ð-obəð-ó \\
\textsc{cl}-dog-\textsc{cl.dem}    \textsc{sm.cl}-\textsc{dpc1}-small-\textsc{adj}    \textsc{sm.cl}-run-\textsc{pfv}\\
\glt ‘this small camel ran away’
\ex
\gll ɲ-ə́ní-ɲːí                 ɲ-e-t-á                          ɲ-obəð-ó \\
\textsc{cl}-dog-\textsc{cl.dem}    \textsc{sm.cl}-\textsc{dpc1}-small-\textsc{adj}    \textsc{sm.cl}-run-\textsc{pfv}\\
\glt ‘these small dogs ran away’
\z
\z 

The noun \textit{ŋə́ní} ‘dog’ belongs to the class ŋ, and this consonant is repeated in the demonstrative, the subject marker on the adjective, and the subject marker on the verb (2a). In (2b), a different noun \textit{ðamala} ‘camel’ belongs to the class ð, and this consonant is used throughout the sentence. In (2c), the plural of (2a) is demonstrated. The word \textit{ɲə́ní} is the plural ‘dogs’, a member of noun class ɲ.

Each singular noun class is paired with a plural noun class. The following tables illustrate the noun classes, both invariant forms and singular/plural class pairings that we have identified (Gibbard et al 2009). There are eight main class pairings, five minor ones, and five single invariant classes. Note that when the initial sound of the noun is a vowel, the concord prefix is always a consonant:

\begin{table}
  \begin{tabular}{lp{1cm}p{1cm}lp{1cm}p{1cm}lll}
    \lsptoprule
\rotatebox{60}{Class}	&  \rotatebox{60}{Initial segment} &	\rotatebox{60}{Concord segment}	& \rotatebox{60}{Singular}	& \rotatebox{60}{Initial segment} &	\rotatebox{60}{Concord segment}	& \rotatebox{60}{Plural}	& \rotatebox{60}{Gloss} \\
\midrule
g/l	& V	&g-/k-	& evaja 	&l-	&l-	& ləvaja & poor person \\
& 	&	&		 uðɜ 		&	&	&ləðʷɜ	& worm \\
l/ŋ	& l-/ɽ-	&	l-	& lavəra
&	ŋ-	& ŋ-	& ŋavəra& 
stick \\
& 	&	&		 ləbú 		&	&	&ŋəbú		& well \\
l/ɲ &	l-/ɽ-&	l-	& laŋwat̪a
&	ɲ-&	ɲ-	& ɲaŋwat̪a &
	water cup \\
& 	&	&		 láwá 		&	&	&ɲáwá	& mosquito \\
ð/r	&ð-	&ð-	& ðaba 
	& r- &	r-	& raba
	& cloud \\
& 	&	&	  ðápá		&	&	& rápá	& friend \\
ð/j	&ð-	&ð-		& ðamala	
	& j-	& j-, s- &	jamala & 
	camel \\
& 	&	&	ðáɾá		&	&	& jáɾá	& rope \\
g/n	&V	&g-/k- &	oʧ:a
	&n-	&n-	& nəʧ:a
& milk pot \\
& 	&	&	 eməɾt̪á		&	&	& nəməɾt̪á		& horse \\
ŋ/ɲ	&	ŋ-	&ŋ-	& ŋeɾá
	& ɲ-	& ɲ-& 	ɲeɾá
	& girl, child \\
& 	&		& ŋusí		&	&	& ɲusí & chick \\
j/j	& low V	 & j-, k-, s- &	ajén  
	& high V	& j-, s- & 	ején 
	& mountain \\
& 	&			& ɜðúní	&	&	& iðúní	& hearthstone \\
\lspbottomrule 
  \end{tabular}
  \caption{Table of eight main noun class pairings}
  \label{tab:ch6:1}
\end{table}

The following table provides the five invariant classes, many of which are mass nouns or abstract nouns. In addition, the infinitive form is included as a noun class since it can serve as a subject and show noun class concord. 

\begin{table}
  \begin{tabular}{lp{1cm}p{1cm}lp{1cm}p{1cm}lll}
    \lsptoprule
\rotatebox{60}{Class}	&  \rotatebox{60}{Initial segment} &	\rotatebox{60}{Concord segment}	& \rotatebox{60}{Singular}	& \rotatebox{60}{Initial segment} &	\rotatebox{60}{Concord segment}	& \rotatebox{60}{Plural}	& \rotatebox{60}{Gloss} \\
\midrule
ŋ	&ŋ-	&ŋ-	& ŋaga &	*&	*	&*	&bottle gum \\
& 	&	&		 ŋɡáɾá		&&	&		& salt \\
& 	&	&		 ŋaðəna		& &	&	& arrogance \\
ð	& b-/p-, m-, ð-	 & ð-	& mogwátá
 &	*	&*	&*	& peanut\\
& 	&	&		 ðəbáɾá		&	&	&		& cotton \\
j	&V/s-	& j-, k-, s-& 	ibugʷɜ́
	& *	&*	&*	&fog \\
& 	&	&		 aveja		&	&	&		& liver \\
g	&V	&g-/k-	& eveá
&	*	&*	&*	& sand\\
& 	&	&		 áŋálá 		&	&	&		& haze \\
 ð	&ð-	&ð-	& ðáwárðáŋ &
	*	&*	&*	&writing \\
& 	&	&		ðə́və́léðáŋ		&	&	&		& milking \\
\lspbottomrule 
  \end{tabular}
  \caption{Table of five invariant single noun class pairings}
  \label{tab:ch6:2}
\end{table}

Finally, there are four minor noun class pairings and a single invariant noun class with one member.

\begin{table}
  \begin{tabular}{lp{1cm}p{1cm}lp{1cm}p{1cm}lll}
    \lsptoprule
\rotatebox{60}{Class}	&  \rotatebox{60}{Initial segment} &	\rotatebox{60}{Concord segment}	& \rotatebox{60}{Singular}	& \rotatebox{60}{Initial segment} &	\rotatebox{60}{Concord segment}	& \rotatebox{60}{Plural}	& \rotatebox{60}{Gloss} \\
\midrule
j/ŋ	&V-	&j-, k-	&úlə́ðí	&ŋ-	&ŋ-	& ŋúlə́ðí	&termite \\
l/j	&l-	&l-	&ləŋáθ	&front V	&j-, s-	&eŋáθ	&tooth \\
r/j	&r-	&r-	&rlo	&front V	&j-, s-	&ego	&female goat\\
ð/ɲ	&ð-	&ɲ-	&ðegə́mé	&ɲ-	&ɲ-	&ɲəgə́mé&	cheek \\
ð/g	&ð-	&ð-	&ðərliə́	&round vowel	&k-, g-	&urliə́	&root \\
l	&l-	&l-	&laja	&*	&*	&*	& honey \\
\lspbottomrule 
  \end{tabular}
  \caption{Table of six minor noun classes/noun class pairings}
  \label{tab:ch6:3}
\end{table}

These noun classes are similar to the ones identified in Black \& Black (1971) in their study of the Umm Dorein dialect of Moro (labeled Werria by the Thetogovela speakers). In both dialects, apart from a handful of irregular nouns, consonant-initial nouns have one of a limited set of class prefixes: /l-, ð-, ŋ-, ɲ-, n-, r-, l-, j-/, all sonorants with the exception of the voiced interdental fricative /ð/. Class concord markers are the same set of prefixes with two additional ones: /g-/, which is used with many vowel-initial nouns as well as a few g-initial irregular nouns and /s-/, which is used with j-initial plurals. 

A summary of the major concord consonant class pairings is shown in (6):

\begin{table} %Draw lines
  \begin{tabular}{ll}
    \lsptoprule
Singular & Plural \\
\midrule
g-k		&	l\\
		&		n\\
	l	&		ŋ\\
		&	\\
	ŋ	&		ɲ\\
		&\\
	ð	&		r\\
		&\\
	j ~ s		&	j ~ s\\
\lspbottomrule 
  \end{tabular}
  \caption{Major noun class pairings}
  \label{tab:ch6:3}
\end{table}

Previous research on Kordofanian noun classes (Stevenson 1956-7, Schadeberg 1981, Guest 1997, Norton 2000) identified a number of different classes, many of which occur in Moro. Stevenson (1956-7) proposed a Bantu-like system of numeral labels for nouns classes for the Kordofanian system as a whole. Our classes do not correspond one-to-one with Stevenson’s, entailing gaps and additions to accommodate our data. Instead of numerals, we will refer to the patterns of paired noun classes by their concord segments.  In the chart in (7) we compare our classification with Guest’s for Umm Dorein Moro, and Stevenson’s (1956-57) for the Koalib-Moro group (equivalent to the Schadeberg’s Heiban group). The semantic properties listed may apply to some members of each class. Some classes include a range of nouns with no clear semantic connection.


\begin{longtable}{lp{2cm}lp{2cm}p{1.5cm}p{2cm}}
    \lsptoprule
  \multicolumn{2}{c}{Thotegovela Moro} &    \multicolumn{2}{c}{Guest} &   \multicolumn{2}{c}{Stevenson} \\
\midrule
Concord & Semantics & Concord & Semantics &Concord & Semantics \\
\midrule
g/l	&	people	&	g/l		&	people		&	1. kw(u)-, gw(u)- ; 2. l(i)- 	&	people \\
n/a	&	n/a		&	n/a		&	n/a		&	3. kw(u)-, gw(u)- ; 4. c-, j-, y- &	 nature \\
l/ŋ	&	round, long things, fruit 	&		l, lɽr, ɽr, ŋ&	long, hollow, deep, round		&	5. l(i)-; 6. ŋw(u)-
	&	 unit/mass \\
See g/n		&	 n/a		&	See g/n	&		n/a		&	 7. k- ; 8. j-, y &	\\
ð/r		&	some animals, long things &		ð/r	&	 long things	&	- &		long things \\
ð/j &	?	&	ð/j&		harmful, large	&	11. t, d- ; 12. c-, j-, y- &	 harmful, large\\
g/n	&	? 	&		g/n	& common things	&	13. k-, g- ; 14. ny-, n-	
&	 hollow, deep \\
ŋ/ɲ	&	 small animals	&	ŋ/ɲ	& small animals	&	 15. ŋ-; 
16. ɲ-	&	 small animals \\
n/a	&	 n/a &		n/a &	n/a		&	15a. t-, tr-		&	 diminutive \\
n/a	&	 n/a		&	n/a		&	n/a		&	17. ŋ		&	augmentative \\
ð	&	 infinitive, abstract, nature		&	ð		&	abstract nouns, emotions		&	19. t(ɨ)-, ð(ɨ)-		&	infinitive \\
ŋ	&	liquids, mass nouns, abstract nouns		&	ŋ		&	liquids, abstract nouns		&	20. ŋ-		&	liquids, abstract \\
r/j	&	 goat, etc.	&		r/j	& goat, etc.		&	21. ŋ- ;
22. y-, j- 	&		goat, etc. \\
l/n	&	 tooth		&	n/a		&	n/a		&	23. l- ;
24. y-, j-		&	eye, etc. \\
j/j	&	 ? 	&		j/j		&	foreign words		&	25. vowel ; 26. y-, j-, i-		&	 miscellaneous\\
l/ɲ	&	 animals, body parts, objects	&	 	l/ɲ		&	 animals and body parts		&	 n/a	&	 	n/a \\
ð/g	&	 trees, derivatives of trees		&	 ð/g		&	 trees, parts of trees		&	 n/a		&	 n/a \\
\lspbottomrule 
  \caption{Noun class semantics and comparison}
  \label{tab:ch6:3}
\end{longtable}

Because the vowel-initial nouns present particular difficulties in determining the prefix nature of the initial vowel, we will first present the consonant-initial nouns and discuss each noun class in turn with examples.

\subsection{l/ŋ class}

The l/ŋ class contains a variety of round or long objects, some pertaining to water, as well as fruit. Examples are given below:

\ea 	l/ŋ nouns
\begin{tabular}[t]{llll}
a.&	laŋwat̪a	&ŋaŋwat̪a	&‘watercup’\\
b.&	lóɾá	&ŋóɾá		&‘creek’\\
c.&	ləbóŋ	&ŋəbóŋ		&‘lake’\\
d.&	ləbú	&ŋəbú		&‘well’\\
e.&	lɜ́dʷɔ́ŋ	&ŋɜ́dʷɔ́ŋ		&‘back’\\
f.&	ləpér	&ŋəpér		&‘tail’\\
g.&	lə́nná	&ŋə́nná		&‘room’\\
h.&	ləðe	&ŋəðe		&‘bone’\\
i.&	ləbɾea	&ŋəbɾea		&‘walking stick’\\
j.&	lavəɾa	&ŋavəɾa		&‘stick’\\
k.&	ləvəgeá	&ŋəvəgeá	&‘anklet’\\
l.&	loandra	&ŋoandra	&‘stone’\\
m.&	lúwjɜ	&ŋúwjɜ		&‘type of tree’ (plural is fruit of tree’)\\
n.&	lavəðá	&ŋavəðá		&‘fruit of evəða tree’\\
o.&	lə́ɾúðí	&ŋə́ɾúðí		&‘grape’\\
p.&	loana	&ŋoana		&‘ear of corn, grain’\\
q.&	ləməneá	&ŋəməneá	&‘fruit like a big grape’\\
\end{tabular} \z

In addition to words beginning with /l/, there are also words in this class that have initial /ɽ/ in the singular:

\ea
\begin{tabular}[t]{llll}
a.&	ɽdjá	&	ŋədjá	&	‘dalib fruit’ \\
b.&	ɽdó		&	ŋədó	&	‘group’\\
c.&	ɽ́ɾéa	&	ŋóɾéa	&	‘earth, ground’\\
d.&	ɽút̪uɜ́	&	ŋúɽt̪uɜ́	&	‘knot’\\
\end{tabular}
\z

These words all have a concord consonant /l/, ex. \textit{ɽdjá-la} ‘with dalib fruit’ or \textit{ɽdjɜ́-li} ‘this dalib fruit’.

The word for ‘head’ is also in this class due to its concord properties, as shown for the instrumental suffix:

\ea	
\begin{tabular}[t]{llll}
a.	&	nda		&	ŋgʷa	&	‘head’\\ 
b.	&	nda-la	&	ŋgʷa-ŋa	&	‘with head’\\
\end{tabular}
\z

The sound /ɽ/ is rarer in Thetogovela Moro than in Umm Dorein, and many /ɽ/ are now realized as /g/ in Thetogovela Moro, such as \textit{ege} ‘house’ which is \textit{eɽe} in Umm Dorein. Given the /g/ in the plural, it may be that the singular had /ɽ/ which is now realized as [d] after the nasal. Another word like this is \textit{nd̪əmana} ‘kidney’, whose plural is \textit{ŋəðəmana}. % TODO CHECK TONE

Finally, there are some words in this class that begin with [ə]. This vowel is epenthetic, as these words have the ability to be pronounced with the first two sounds switched: \textit{lə́d̪ə́máná} / \textit{ə́ld̪ə́máná}. 

\ea	
\begin{tabular}[t]{llll}
a.	&	ə́ld̪ə́máná	&	ŋə́ðə́máná	&	‘bean’ \\
b.	&	ə́lt̪ə́miə́	&	ŋə́t̪ə́miə́	&	‘termite mound’\\
c.	&	əltú	&	ŋətú	&	‘shelter’\\
d.	&	əlná	&	ŋəná	&	‘room’\\
e.	&	ə́ltúlé	&	ŋótúlé	&	‘cheek’\\
\end{tabular}
\z

\subsection{l/ɲ class}
The l/ɲ class consists of animals, insects, body parts and some useful objects. 

\ea
\begin{tabular}[t]{llll}
a.	&	lɜ́ɾí	&	ɲɜ́ɾí	&	‘calf (of leg)’\\
b.	&	lɜmí	&	ɲɜmí	&	‘beard, chin’\\
c.	&	lómóná	&	ɲómóná	&	‘finger, day’\\
d.	&	ɽdóŋ	&	ɲədóŋ	&	‘pointy back of head’\\
e.	&	lloa	&	ɲolloa	&	‘elbow’\\
f.	&	ɽɾá		&	ɲəɾá	&	‘lizard’\\
g.	&	láwá	&	ɲáwá	&	‘mosquito’\\
h.	&	lútí	&	ɲútí	&	‘owl’\\
i.	&	lókógóŋ	&	ɲókógóŋ	&	‘scorpion’\\
j.	&	lɪ́ŋgʷɜ	&	ɲɪ́ŋgʷɜ	&	‘frog’\\
k.	&	ɽɾúma	&	ɲəɾúma	&	‘ram’\\
l.	&	lə́búŋwɜ	&	ɲə́búŋwɜ	&	‘water pot, bottle’\\
m.	&	láʤile	&	ɲáʤile	&	‘bicycle’\\
n.	&	ləvə́ɾəðʲɜ&	ɲəvə́ɾəðʲɜ&	‘blanket’\\
o.	&	logopájá&	ɲogopájá&	‘cup’\\
\end{tabular}
\z

As with the l/ŋ class, some words in this class begin with /ɽ/, but the concord consonant is /l/: \textit{ɽɾá-lá} ‘with lizard’. Also, some words show initial [əl]: \textit{əltúr} (\textsc{sg}.) \textit{ɲətúr} (\textsc{pl}.) ‘umbilical hernia’.
		

\subsection{ð/r class}
The ð/r class is a small class that consists of some animals, body parts, and long things, but is otherwise not well-defined semantically. 

\ea
\begin{tabular}[t]{llll}
a.	&	ðéj		&	réj		&	‘hand’\\
b.	&	ðáðá	&	ráðá		&	‘road, path’\\
c.	&	ðə́lá		&	rə́lá		&	‘grave / horn,trumpet’\\
d.	&	ðegə́mé	&	regə́mé	&	‘jaw’\\
e.	&	ðɜ́viə́	&	rɜ́viə́	&	‘leopard’\\
f.	&	ðugi	&	rugi		&	‘wood plank’\\
g.	&	ðaba	&	raba		&	‘cloud’\\
\end{tabular}
\z

\subsection{ŋ/ɲ class}
The ŋ/ɲ class consists of (generally) small animals as well as young humans. 

\ea	
\begin{tabular}[t]{llll}
a.	&	ŋusí		&	ɲusí	&	‘chick’\\
b.	&	ŋáɾtə́máðá	&	ɲáɾtə́máðá&	‘small lizard’\\
c	&	ŋowːa		&	ɲowːa	&	‘young person (age 15 and up)’\\
d.	&	ŋəməná		&	ɲəməná	&	‘kid (baby goat)’\\
e.	&	ŋgə́m			&	ɲəɖə́m	&	‘squirrel’\\
f.	&	ŋíní		&	ɲə́ní		&	‘dog’\\
g.	&	ŋiðəniə́		&	ɲəðəniə́	&	‘rabbit’\\
h.	&	ŋeɾá		&	ɲeɾá	&	‘child, girl’	\\
\end{tabular}
\z

Note that \textit{ŋowːa} could be related to \textit{ówːá} ‘woman’, although the tones are different. % TODO Are they??


\subsection{ŋ class}
In addition to the noun class pairings, there are also some classes that are unpaired. The noun class ŋ consists of abstract nouns, mass nouns and liquids. 

\ea	
\begin{tabular}[t]{lll}
a.	&	ŋəɲóŋ		&	‘speed’\\
b.	&	ŋə́ɾə́miə́		&	‘darkness’\\
c	&	ŋɜ́və́ní	 	&	‘blood’\\
d.	&	ŋáwá		&	‘water’\\
e.	&	ŋaɲa	 	&	‘grass’\\
f.	&	ŋarléðá		&	‘dirt’\\	
\end{tabular}

\z

\subsection{ð class}

The ð class is another unpaired class. It consists of mass nouns and verbal nouns or gerunds, classes of words that do not have plurals. The tone of the gerunds are either low-toned (Angelo) or high-toned (Elyasir). 

\ea
\begin{tabular}[t]{lll}
a.	&	ðoálá	&	‘money, cattle’\\
b.	&	ðává	&	‘chaff’\\
c.	&	ðəbáɾá	&	‘cotton’\\
d.	&	ðílí	&	‘manure’\\
e.	&	ðəvəleðaŋ&		‘pulling’\\
f.	&	ðəwarðaŋ	&	‘writing’\\
\end{tabular}
\z

The ð-class also includes words that begin with consonants that do not match any of the noun class consonants. Words that begin with stops (p b t̪ k) and the consonant [m] belong to the ð class. There is also one word beginning with [s], although usually s-initial words belong to the j class. 

\ea	
\begin{tabular}[t]{lll}
a.	&	bət̪iə		&	‘butter’\\
b.	&	boʧa		&	‘ashes’\\
c.	&	mogwáta		&	‘peanut’\\
d.	&	músí		&	‘banana’\\
e.	&	paɖólwa		&	‘jute’\\
f.	&	kúɾɜ		&	‘ball’\\
g.	&	t̪ɾəmbílí	&	‘car’\\
h.	&	t̪əbwɜ		&	‘bamboo’\\
i.	&	sɜndúgi		&	‘box’\\	
\end{tabular}

\z

There is one exception. The word \textit{matʃó} ‘guy, man’ is classified as g-class. As humans usually belong to g-class, this means that the ‘human’ classification outweighs the classification based on the initial consonant. The plural of this word is \textit{matʃáńda} rather than a change in the initial consonant. 

Even if some of these words are countable (ball, car, banana), there are no plural forms, not even with a suffix. The noun class membership is determined by concord:

\ea
\begin{tabular}[t]{ll}
	botʃa-ða	&	‘with the ashes’\\
	t̪əbwɜ́-ðːi	&	‘this bamboo’\\
\end{tabular}
\z


\subsection{l class}

The l-class has only three members. The first two are mass nouns, whereas the third is not. Like the word \textit{nda} for ‘head’, it begins with an [nd] cluster but belongs to the l-class. 

\ea
\begin{tabular}[t]{lll}
a.	&	lájá		&	‘honey’\\
b.	&	lugma		&	‘porridge’\\
c.	&	ndogá		&	‘stick inserted under lower lip’\\
\end{tabular}	
\z

\subsection{g/l class}\label{section:glclass}

We now turn to nouns which are vowel-initial. The g/l class pairing is a large group, containing most human nouns as well as a few small animals or insects. The plural is marked with the consonant /l/, but the singular has an initial vowel, one of the five vowels /i e o u ɜ/. There are no a-initial nouns in this class; instead there are stems that begin with the sequence [wa]  Historically, the singular was g- or gw- initial, but the velar consonant [g] has been lost, leaving a vowel initial or w-initial stem in most cases. The single word gwaŋá ‚thing‘ is the only indication of the former structure of this class. 

\ea
\begin{tabular}[t]{llll}	
a.	&	evaja	&	ləvaja		&	‘poor person’\\
b.	&	emaðén	&	ləmaðénanda	&	‘peer of same age’\\
c.	&	íðiə́		&	lə́ðiə́			&	‘son’\\
d.	&	imɜgəniə&	ləmɜgəniə	&	‘excrement’\\
e.	&	ɜ́giə́		&	lɜ́giə́		&	‘mentally ill person’\\ %what is the locative of this? djp
f.	&	ɜ́wíjɜ́	&	lɜ́wíjɜ́		&	‘friend’\\
g.	&	ome		&	ləme			&	‘fish’\\
h.	&	omʷaɾə́ŋá	&	ləmʷaɾə́ŋá	&	‘Moro person’\\
i.	&	uʤí		&	ləʤí		&	‘person’\\
j.	&	umːiə	&	ləmːiə		&	‘boy’\\
k.	&	út̪ɜ́diə́	&	lə́t̪wɜ́diə́		&	‘uncle’\\
l.	&	wáɾá	&	láɾá		&	‘chicken’\\
m.	&	wájá	&	lájá			&	‘fly/bee’ \\
n.	&	gwaŋá	&	laŋá		&	‘thing’\\
\end{tabular}
\z

The word \textit{emaðén} is one of several kinship terms that show inherent possession, so the singular shown here is marked for 3rd person possession \textit{-én}: ‘his/their peer’ whereas the plural has the plural suffix \textit{–énanda} (\sectref{section:kinship}).

The plural marker /l-/ triggers vowel reduction to [ə] for the stem-initial vowels /i e o u/. When the round vowels /o u/ are reduced, rounding surfaces elsewhere in the stem: ex. /l-útɜ́diə́/ → [lə́t̪wɜ́diə́]. See section XX on dissimilation for an explanation. The central vowel /ɜ/ shows no reduction.

The noun class concord properties of the singular are either /g/ or /k/. The /k/ form occurs in the demonstrative and with the locative prefix. The /g/ appears in the instrumental and as a subject agreement marker on the verb.  

\ea	
\begin{tabular}[t]{ll}
umːjɜ́-kːi	&	‘this boy’\\
ík-úmːíə	&	‘in the boy’\\
umːiə-ga	&	‘with the boy’\\
umːiə gasːó	&	‘the boy ate’\\
\end{tabular}
\z

The distribution of [k] and [g] as concord markers parallels those of [s] and [j] for the j-class. The [k] appears in the same forms as the noun class concord marker [s], and the [g] appears in the same forms as the noun class concord marker [j].


\subsection{g/n class}
The g/n class pairing is similar to g/l in that the singular class is vowel initial, while the plural class is marked with a consonant, in this case /n/. Some examples are given below. Unlike the g/l/ class, this class has the full range of vowels in the initial position of the singular stem, including /a/.  In addition, there are a few words that are w-initial; the [w] is retained instead of replaced in the plural. Finally, there are three g-initial words in this class, which have [n] in the plural:

\ea
\begin{tabular}[t]{llll}
a.	&	ád̪ámá	&	nád̪ámá	&	‘book’\\
b.	&	ala		&	nala		&	‘grinding stone’\\
c.	&	ed̪apəgá	&	nd̪apəgá	&	‘nail’\\
d.	&	evəlla	&	nəvəlla	&	‘wild cat’\\
e.	&	id̪əvíní	&	nd̪əvíní	&	‘shoe’\\
f	&	ɜ́dí		&	nɜ́dí	&	‘skin’\\
g.	&	odəlóŋá	&	ndəlóŋá	&	‘fox’\\
h.	&	omágá	&	nəmágá	&	‘snail’\\
i.	&	umədí	&	nəmədí	&	‘small biting ant’\\
j.	&	uɾíθ	&	ndɾíθ	&	‘chain’\\
k.	&	waɾá	&	nwaɾá	&	‘baobab tree’\\
l.	&	wílí	&	nwílí	&	‘dream, picture’\\
m.	&	gálá	&	nálá		&	‘bead’\\
n.	&	gí		&	ní		&	‘farm, field’\\
o.	&	gəla	&	nəla		&	‘bowl’\\	
\end{tabular}

\z

With central vowels a/ɜ, there is no modification of the initial vowel in the plural, as the following examples show:

\ea	
\begin{tabular}[t]{llll}
a.	&	abugwala	&	nabugwala	&	‘paper’\\
b.	&	ándə́mé		&	nándə́mé		&	‘flea’\\
c.	&	ɜ́lúŋ		&	nɜ́lúŋ		&	‘promiscuous person’\\
d.	&	ɜnəŋiə́		&	nɜnəŋiə́		&	‘ear’\\	
\end{tabular}

\z

In contrast, when the singular has an initial front vowel, the front vowel does not appear in the plural. There may be no vowel between the prefix /n-/ and the next consonant (a-e), or a vowel [ə] follows the prefix (f-j):

\ea	
\begin{tabular}[t]{llll}
a.	&	édeə́			&	nédeə́		&	‘dalib tree’\\
b.	&	eɾél		&	ndrél	&	‘side of face’\\
c.	&	eɾéθ		&	ndréθ	&	‘promiscuous person’\\
d.	&	etám		&	ntám		&	‘neck’\\
e.	&	it̪əli		&	nt̪əli	&	‘year’\\
f.	&	ebamba		&	nəbamba	&	‘drum’\\
g.	&	eləŋe		&	nələŋe	&	‘king, leader’\\
h.	&	emert̪á		&	nəmərt̪á	&	‘horse’\\
i.	&	evəla		&	nəvəla	&	‘wild cat’\\	
\end{tabular}
\z

Unlike the forms beginning with central vowels, the front vowels are prone to reduction to [ə] or deletion. The front vowel is deleted between the prefix n- and a following coronal consonant from the set [t̪ t d̪ d n ɾ/r] (a-e), thus creating a nasal-consonant cluster. Note that the n+ɾ combination results in [ndr], where the [d] is a transitional sound. The front vowel is not deleted but reduced to [ə] when it appears before [l] or a labial [b m v] (f-i). Loss of the front vowel would result in an unacceptable cluster: *\textit{nvəða}, so reduction occurs. There is one form with a front vowel that shows no reduction and no deletion of the intial vowel: \textit{ege} ‘house’ (plural \textit{nege} ‘houses’). The [g] derives historically from *ɽ, but this does not explain the lack of reduction/deletion. 
	
The final subgroup within the g/n class has initial back round vowels [o] and [u]. Those with first coronal consonants [d nd ɾ ð] as well as [g] have no vowel in the plural. These [g] derived historically from *ɽ, a coronal consonant; in the plural the alveolar [d] appears after the nasal prefix. The lack of the initial vowel before coronal consonants is parallel to the pattern of front vowels in (a-e). In addition to the lack of the initial round vowel, some plural nouns have labialization of a root consonant (f-g).

\ea
\begin{tabular}[t]{llll}
a.	&	odəlóŋá	&	ndəlóŋá		&	‘fox’\\
b.	&	ogómá	&	ndámá		&	‘thief’\\
c.	&	ogəvélá	&	ndəvélá		&	‘monkey’\\
d.	&	ondəðé	&	ndəðé		&	‘lice’\\
e.	&	uɾíθ	&	ndɾíθ		&	‘chain’\\
f.	&	odəgala	&	ndəgwala		&	‘turtle’\\
g.	&	oða		&	ndwa			&	‘kind of deer, antelope’	\\
\end{tabular}
\z

The second group are those that have reduction of the round vowel to schwa. Almost all of these forms show labialization somewhere in the stem. However, many of them have alternate forms with no reduction, and also no labialization, ex. \textit{nomágá} is attested as well as \textit{nəmʷágá}. 

\ea	
\begin{tabular}[t]{llll}
a.	&	oʧːa	&	nəʧːa	&	‘milk pot’\\
b.	&	ot̪ele	&	nətʷele	&	‘spider’\\
c.	&	ot̪ə́mba	&	nətə́mbʷa	&	‘ostrich’\\
d.	&	uməní	&	nəmʷəní	&	‘type of tree’\\
e.	&	ómə́ʧáðá	&	nə́mʷə́ʧáðá	&	‘afterbirth’\\
f.	&	omágá	&	nəmʷágá	&	‘snail’\\
g.	&	ombətea	&	nəmbʷətea&	‘back of shoulder’\\
h.	&	uwːɜ	&	nəwːɜ	&	‘moon, month’\\
\end{tabular}
\z

There are some nouns with round vowels that show no reduction:

\ea
\begin{tabular}[t]{llll}
a.	&	uməd̪í	&	numəd̪í	&	‘small biting ant’\\
b.	&	úmə́ðní	&	númə́ðní	&	‘bank, silo for grains / pocket’\\
c.	&	wárá	&	nwárá	&	‘animal pen, enclosure’	\\
\end{tabular}
\z

This corresponds to class pairs 13/14 and 7/8 in Stevenson’s classification, which have a velar consonant prefix \textit{k-} or \textit{g-} for singular. It appears that Moro lost the initial velar in the singular nouns, but retained velar consonants as concord. The word ‘ear’ \textit{ɜnəŋiə́ }/ \textit{nɜnəŋiə́ has} initial [g] in related languages: \textit{g-öni} / \textit{ny- öni} (Heiban) and \textit{g- öni} / \textit{n- öni} (Otoro). In addition, there are a few nouns that retain the initial [g] in the singular: \textit{gí} /\textit{ní} ‘field, farm’ or \textit{gálá} / \textit{nálá} ‘bead’. 


\subsection{ð/g class}

The ð/g class contains trees and derivatives of trees. The singular begins with ð and the plural begins with a round vowel, either [o] or [u], depending on vowel harmony. Historically, these nouns began with [g]; in one noun, the [g] is still present. 

\ea
\begin{tabular}[t]{llll}
a.	&	ðə́bó́rwá		&	óbə́rá		&	‘tree sp. with long thin branches’\\
b.	&	ðə́gə́ŋálá		&	ógə́ŋálá	&	‘tree sp.’\\
c.	&	ðet̪eá		&	ot̪eá	&	‘big branches, sticks’\\
d.	&	ðəlwəndrí́	&	uləndrí́	&	‘tree sp.’\\
e.	&	ðə́w:ɜ́nə́ŋ		&	úw:ɜ́nə́ŋ	&	‘tree sp.’\\
f.	&	ðudədilíə	&	gudədilíə&	‘tree sp.’	\\
\end{tabular}
\z

\subsection{j/j class}

In Stevenson’s chart of Koalib-Moro classes (1957:152), the j/j class is listed with an initial vowel in the singular in Stevenson’s Koalib-Moro chart, but with a front vowel, glide or palatal stop for the plural. In Thtogovela Moro, the j/j class is characterized by a central vowel (either [a] or [ɜ]) in the singular and by a front vowel (either [e] or [i]) in the plural. The choice of vowel in each number category is determined by vowel harmony. Examples are shown below:

\ea	
\begin{tabular}[t]{llll}
a.	&	ajén		&	ején		&	‘mountain’\\
b.	&	áɾómá		&	éɾómá	&	‘black biting ant’\\
c.	&	ɜbulúkriə	&	ibulúkriə&	‘dove’\\
d.	&	ɜt̪úmi		&	it̪úmi	&	‘onion’	\\
\end{tabular}
\z

This class contains a number of borrowings, particularly from Arabic, which treat the definite article al as part of the stem, converting it to el in the plural: ex. \textit{aləŋgréma} (sg.) \textit{eləŋgréma} (pl.) ‘bed’ < Sudanese Arabic /al-ʕangareːb/. However, not all words in the j/j class are borrowed. A prefix analysis of these forms is straightforward: prefixes are /a-/ in the singular and /e-/ in the plural; allomorphs [ɜ] and [i] are created by vowel harmony. This explains why the initial vowel of the singular is restricted to being /a/. We conclude that the j/j/ class is characterized by prefixes, \textit{a-} in the singular (with allomorph [ɜ]) and \textit{e-} in the plural (with allomorph [i]). 


\subsection{ð/j class}

The ð/j consists of some animals but is otherwise not well-defined semantically. In this class, the plural prefix is \textit{j-}. It appears before all vowels, but if the first vowel is front, either /e-/ or /i-/, the [j] does not appear.  In the related dialect, Werria, the [j] is often present, ex. \textit{éɾá} is written \textit{yəra}, with reduction of /e/ to [ə]. 

\ea
\begin{tabular}[t]{llll}
a.	&	ðərmbégwa	&	ermbégwa		&	‘lyre’\\
b.	&	ðəbarəla	&	ebarəla		&	‘river, stream’\\
c.	&	ðə́ɾá			&	éɾá			&	‘vine of gourd’\\
d.	&	ðərðiə		&	irðiə		&	‘type of water rat’\\
e.	&	ðəbəgwɜ		&	ibəgwɜ		&	‘thread’\\
f.	&	ðə́ʷlí		&	júlí			&	‘giraffe’\\
g.	&	ðomóŋ		&	jomóŋ		&	‘big, lazy, light-colored rat’\\
h.	&	ðwaleə		&	jwaleə		&	‘green bird sp.’\\
i.	&	ðamala		&	jamala		&	‘camel’\\
j.	&	ðɜbərt̪ulɜ	&	jɜbərt̪ulɜ		&	‘locust sp.’ \\
\end{tabular}
\z

The noun class concord for the plural is either /j/ or /s/ depending on the concord context (see Chapter \ref{chapter:nounphrase}). For example, the proximal demonstrative, which always shows noun class concord, is –\textit{ísːi}, whereas the instrumental (b) and subject concord (c) use /j/:

\ea
\begin{tabular}[t]{lll}
	a.	&	jamalɜ́-sːi	&	‘this camels’\\
	b.	&	jamala-ja	&	‘with camel’\\
	c.	&	jamala jasːó&	‘the camels ate’\\
\end{tabular}
\z

The inessive marker /é-/ (\sectref{sec:ch6:inessive}) is realized [és-] if attached to a vowel-initial stems with j-initial plurals. Note that the attachment of this prefix triggers reduction of the initial vowel to [ə]. Compare this with the j-initial nouns in (c-d). 
\ea
\begin{tabular}[t]{llll}
a.	evəra	&	és-ə́və́rá	&‘inside the line’\\
b.	iɾðiə́	&	ís-ə́ɾðiə́	&‘inside the water lizard with the long tail’\\
c.	jamala	&	é-jámálá	&‘inside the camels’\\
d.	júlí	&	í-júlí	&‘inside the giraffes’	\\
\end{tabular}
\z


\subsection{r/j class}
This class is very small, consisting of only four items. The plural behaves the same way as the other plural \textit{j-} class. It is not clear if there is a prefix \textit{e-} or \textit{i-} that replaces the /r/ of the singular, or if this is the vowel realization of a \textit{j-} prefix. 

\ea
\begin{tabular}[t]{llll}
a.	&	rða		&	eða		&	‘meat’\\
b.	&	rlo		&	ego		&	‘goat’\\
c.	&	rəmʷɔ	&	imʷɔ	&	‘God, snake, sky’\\
d.	&	diə		&	iɾiə	&	‘cow’\\
\end{tabular}
\z

\subsection{l/j class}
This class consists of only one item. The plural behaves the same way as the other plural j-class. The concord for the plural \textit{eŋáθ} is either [j] or [s] depending on context. 

\ea	 
\begin{tabular}[t]{lll}
ləŋáθ	&	eŋáθ	&	‘tooth’ \\
\end{tabular}
\z 


\subsection{j/ŋ class}
This class is also very small, consisting of only one item. 

\ea	 
\begin{tabular}[t]{lll}
úlə́ðí	&	ŋúlə́ðí	&	‘termite’\\
\end{tabular}
\z 

This may be an adaptation from the word \textit{lulúð} (\textsc{sg}.) / \textit{ŋwulúð} (\textsc{pl}), which is a ‘termite species or white ant’. If the initial [l] was lost, the word may have been reassigned to the j noun class. 


\subsection{g class}
There are two unpaired class forms with initial vowels. Like the paired vowel-initial class forms, they are divided between the \textit{g-} and \textit{j-} concord classes. The first of these is the g-class, which is like the singular g-class that occurs in the g/n or g/l classes. All initial vowels are attested, and there are also forms beginning with /w/. These words include mass nouns. It also includes some words that could be countable, but still have no plural counterpart. 

\ea 	
\begin{tabular}[t]{llll}
&	g-class\\
a.	&	áʧə́vá(ŋ)	&	g	&	‘food, sorghum porridge’\\
b.	&	áŋálá	&	g	&	‘sweat’\\
c.	&	ɜndiə	&	g	&	‘leather’\\
d.	&	eveá	&	g 	&	‘sand’\\
e.	&	ókóɾa	&	g	&	‘sap’\\
f.	&	ole		&	g	&	‘sound, voice, words, language’\\
g.	&	ilːiə	&	g	&	‘stranger (other Nubans)’\\
h.	&	ulɜŋgi	&	g	&	‘night’\\
i.	&	wálá	&	g	&	‘wool, braids’\\
j.	&	wíjɜ́	&	g	&	‘dry dirt, ground’\\
\end{tabular}
\z

\subsection{j class}\label{sec:ch6:j}

The j-class also contains words that are mass nouns. All the initial vowels in the j-class are central or front.

\ea	
\begin{tabular}[t]{llll}
&	j-class\\
a.	&	aɾóbá	&	j	&	‘whey’\\
b.	&	aveja	&	j	&	‘liver’\\
c.	&	ɜlbúni	&	j	&	‘coffee’\\
d.	&	ét̪oá	&	j	&	‘dew’\\
e.	&	ibugʷɜ́	&	j	&	‘fog’\\
f.	&	iɾiniə	&	j	&	‘snot, mucous’\\
\end{tabular}
\z

Those words beginning with a central vowel (such as \textit{aveja} ‘liver’) have a locative form with [k] like the singulars of the j/j class pairing (\textit{ékávéja}). On the other hand, those words beginning with a front vowel (such as \textit{ibəgʷɜ} ‘fog’) have a locative form with [s] like the plurals of the j/j class (\textit{ís-ibəgʷɜ}).  Based on these similarities, the \textit{j-}class may actually be two separate classes, the two classes that make up the j/j class pairing. Determining whether these forms have prefixes or root vowels is complicated by the fact that there is no plural pairing; a prefix analysis can only be hypothesized based on the analysis given to the j/j class pairing. 

The j-class also includes words that begin with strident consonants, namely /s ʃ tʃ/, some of which are borrowings from Arabic. Those nouns which begin with [tʃ] sometimes vary between ð-class and j-class. 

\ea	
\begin{tabular}[t]{llll}
&	j-class\\
a.	&	súrá	&	j	&	‘picture’\\
b.	&	ʃorba	&	j	&	‘soup’\\
c.	&	tʃaŋgwə́ɾá&	j	&	‘rhinoceros, huge boulder’\\
d.	&	ʧə́ŋge	&	j	&	‘cobra, slang for S. Sudanese’\\
\end{tabular}
\z

This concludes the description of noun classes in Thetogovela Moro. 

\section{Nominalizing morphology}

Moro has two different kinds of nominalization. Gerundive nominalization and property concept nominalization. Gerundive nominalization derives a mass noun from a verb, resulting in a nominal which describes an event. Property concept nominals...

\subsection{Gerundive nominalization}\label{section:gerund}

Gerunds are formed with a circumfix \textit{ðə-aŋ}, \textit{ð-} before vowel initial roots. Gerunds also are marked with all-H tone:

\begin{table}
\caption{Gerundive nominalization}
\begin{tabular}[t]{lll} %check forms
\lsptoprule
Verb root	&	\multicolumn{2}{l}{Gerund (\textit{ð}-class}  	\\
\midrule
-dəɾw-	&	ðə́-də́ɾw-áŋ	&	‘stopping’\\
-ðəw-	&	ðə́-ðəw-áŋ & 	‘poking’\\
-ndr- 	&	ðə́-ndr-áŋ & ‘sleeping’\\
-noan-	&	ðə́-nóán-áŋ & ‘watching’\\
-erl-		& 	ð-érl-áŋ & `walking' \\
\lspbottomrule
\end{tabular}	
\end{table}

There is no plural form of gerund nouns; they are mass nouns. There is dialectal variation in the form of the gerund. In W\"erria and Written Moro, \textit{-aŋ} is \textit{-a}, and the all H melody is an all L melody.

%what morphemes can go inside these nominalizations?

Gerunds are in the \textit{ð}-class corresponding to the initial segment of their prefixal component. They trigger \textit{ð}-class agreement on verbs in subject position:

\ea \gll 	ðə́-wáðá-ŋ		ð-aŋará\\
		\textsc{cl}ð.\textsc{nom}-poke-\textsc{nom}	 \textsc{cl}ð-good.\textsc{adj}\\
	\glt ‘Poking is good.’	
\z

Gerunds seem to be looser than their verbal counterparts in requiring arguments, as neither objects nor subjects need to be present with gerundive nominals.

Gerunds can take locative case prefixes, and must do so when they are the complement of subject and object control verbs and adjectives:

\ea \gll í-g-ʌ-dər-ú é-ðə́-nóán-áŋ ðamala\\ %dbl check case
		\textsc{1sg-cl}g-\textsc{rtc}-stop-\textsc{pfv} \textsc{loc}-\textsc{cl}ð.\textsc{nom}-watch-\textsc{nom} camel\\
	\glt 	‘I stoped watching the camel.’
\ex	\gll í-g-ʌ-tʃ-ʌ́  nano é-ðə́--nóán-áŋ jamala	\\
		\textsc{1sg}-\textsc{cl}g-\textsc{rtc}-bad-\sc{adj} at \textsc{loc}-\textsc{cl}ð.\textsc{nom}-watch-\textsc{nom} camels \\
	\glt ‘I’m sad to watch the camels.’
\ex \gll é-g-a-mədat-ó kúku-ŋ é-ðá-nóán-áŋ ðamala\\ %dbl check case
		\textsc{1sg-cl}g-\textsc{rtc}-help-\textsc{pfv} Kuku-\textsc{acc} \textsc{loc}-\textsc{cl}ð.\textsc{nom}-watch-\textsc{nom} camel\\
	\glt 	‘I helped Kuku watch the camel.’
	
\z 
Verbs and adjectives which take gerundive complements include implicatives, evaluative adjectives, and aspectual verbs. Most seem to involve obligatory, exhaustive control, and with the exception of negative verbs such as `prevent', introduce existence presuppositions on their complements.

\ea  Verbs and adjectives which take gerundive complements
\ea  {Aspectual verbs:}  {-ŋgitʃ-} `finish', \it{-dúrw-} `stop'
\ex  {Implicative:} {-ámadat̪-} `help', \it{-wʌ́tʃ-} `prevent,' {-lʌ́ləŋədʒətʃən-} `remember'
\ex  {Evaluative adjectives:} \it{-tʃ- nano} `sad' (Adj.), {-tʃ-} `bad,' {-ŋər-} `good'
\z 
\z 
See section \ref{sec:ch15:infinitives} for more on control with infinitive clauses.


\subsection{Property nouns}\label{sec:ch6:pcnominal}

A number of nouns in Moro see an alternation between a \textit{g/l}-noun, which describes a person with a particular property, and a \textit{ŋ-}class noun derived from this noun which describes the property concept possessed by this individual. This alternation is illustrated in \tabref{tab:ch6:property}. Property nouns lack a plural counterpart; they are mass nouns.

\begin{table}
\caption{Property nominals}\label{tab:ch6:property}
\begin{tabular}{lllll} %check with Elyasir
\lsptoprule
 Singular (\textit{g}-class) &  Plural (\textit{l}-class) &  & \multicolumn{2}{l}{Property (\textit{ŋ}-class)}  \\
\midrule 
aməda	&  laməda	& `joker(s)' & ŋaməda & `joke'\\
aðəna	& 	laðəna	& `deceiver(s)'& ŋaðəna &  `deceit' \\
um:ía	& 	lɜm:ía	&  `boy(s)'	& ŋɜm:ía	& `boyhood' \\
ɜdum	& 	lɜdum	&  `attractive person'	& ŋëđǝmwa	& `beauty' \\ 
\lspbottomrule
\end{tabular}
\end{table}

When these concepts occur as the main predicate of a clause, the predicate nominal copula \textit{-d-} takes the human-referring \textit{g/l}-class as its complement (\sectref{sec:ch9:nompred}). The property noun itself cannnot occur in these predicative contexts, but rather is only used in argument positions to refer to the abstract concepts. %example?

%ŋorwaṯǝna	&  `poverty' \\




%Result nominals & \\
%ŋorba		&  `brotherhood' \\
%ŋǝlǝŋǝnia	& `relationship' \\
%ŋǝsa		& `food' \\
%ŋëmiñua		& `jokes' \\
%ŋurriñua	& `riddles' \\
%ŋërria		& `culture' \\
%ŋaləŋa		& `song' (from aləŋ `sing')\\
%ŋəɽaɲ 		& `death, disease'\\
%ŋǝɽwata		& `speech', ŋen ŋǝɽwata `news'\\
%ŋəmëɽria 	& `work'	 \\
%ŋənua		& `in laws' (?) \\
%\end{tabular}
%\z 
%

\section{Case and locative morphology}

Moro nouns mark a six-way distinction between nominative, accusative, genitive, two locative cases, inessive  and adessive, and instrumental. Case paradigms for several nouns are provided in \tabref{tab:ch6:case}. Variation in the form of the different modifiers is discussed in each of the corresponding sections below. The genitive prefix is not included in this table, and is described separately in Section \ref{sec:ch8:genitive} as it patterns with other nominal modifiers in agreeing with the noun it modifies. There is also a vocative form for proper names, which we ignore here but introduce in Section \ref{sec:ch6:names}, and discuss the use of in Section \ref{sec:ch21:vocative}, a discussion of greetings.

\begin{table}
\caption{Nominal case morphology}\label{tab:ch6:case}
\begin{tabular}[t]{llllll}
\lsptoprule
Nominative & Accussative & Inessive & Adessive & Instrumental & \\
\midrule
%ŋə́ðə́mán & ŋə́đə́mán-á & é-ŋə́đə́man & né-ŋə́đə́mán & ŋə́đə́mán-ə́ŋá & `bean'\\ %check tone!
ŋaw & ŋaw-a  & é-ŋáw & ne-ŋaw & ŋaw-əŋa & `water' \\
ðəbéɾ & ðəbéɾ-á & é-ðəbéɾ & ne-ðəbéɾ & ðəbéɾ-ə́ðá & `wind' \\ %recheck vowel in instrumental
ómón & ómón-á & ék-ómón & n-ómón & ómónə́-gá & `leopard'\\ %recheck vowel in instrumental
lámón & lámón-á & é-lámón & ne-lámón & lámón-ə́lá & `leopards' \\%check tone of adessive, check instrumental, check tone of acc
ogovél & ogovél-á & ék-ógovél & n-ogovél & ogovél-ə́gá & `monkey'\\ %recheck vowel in instrumental; plural of monkey?
ndəvél & ndəvél-á & é-ndəvél & né-ndəvél & ndəvel-ə́ná & `monkeys'\\ %check instrumental!
ðamala & ðamala & é-ðamala & nə-ðamala & ðamala-ða & `camel'\\ 
jamala & jamala & é-jamala & nə-jamala & jamala-ja & `camels'\\ 
ŋgon & ŋgón & é-ŋgón & nə-ŋgón & ŋgón-ə́ŋá & `squirrel'\\ %check plural?
ɲəŋgón & ɲəŋgón & e-ɲəŋgón & ɲəŋgón-ə́ɲá & `squirrels' \\
ðəbáɾá	& ðəbáɾá & 	é-ðəbáɾá & nə́-ðəbáɾá & ðəbáɾá-ðá & ‘cotton’\\
\lspbottomrule
\end{tabular}
\end{table}


%get paradigm for `girl', `girls'
%check a simple noun like `monkey' in a predicative position after gado
%\textit{é(k)-} `in'
% \textit{n(ə)} `on, at'
% \textit{-Ca} `with'

Besides the nominative-accusative distinction, is may not be obvious that the other cases in Moro should be described as `case' rather than adpositions. Particularly as many adpositions in Moro are clitics which fuse with the noun (\sectref{sec:ch13:adpositions}). 

There are several arguments that these are case markers rather than adpositions. The clearest argument comes from the locative cases, which cannot occur with the accusative suffix \textit{-a}, but instead have the bare root which is characteristic of the nominative in these cases. In contrast, the enclitic adpositions described in Section \ref{sec:ch13:adpositions} freely occur with accusative marked nouns.

Here the instrumental poses somewhat of a problem, because the form of the instrumental is \textit{-əCa} with many consonant final roots, and an argument could be made that the initial vowel of this suffix is the accusative. Yet the fact that this vowel is absent in forms such as \textit{ŋaw-ŋa} `with the water' indicates that there is no accusative suffix in these cases, and hence that the instrumental too is a bona fide case marker. Instead, the schwa occurs in the instrumental to break up phonotactically prohibited consonant clusters.
 
Syntactic evidence that these are case markers comes from the observation that they always mark locative arguments of verbs rather than simple locative adjuncts, which typically occur with full adpositions. See \sectref{sec:ch12:locobj} for a discussion of locative objects.

Finally, the locative clitic \textit{=u} and the instrumental clitic \textit{=ja} are found when a locative or instrumental argument is passivized, extracted, or pronominalized (\sectref{sec:ch11:clitic}). While these enclitics are similar to stranded prepositions, they clearly differ in form from their counterparts which affix to nouns. Because of this, there is no doubt that the nominal affixes are just that, affixes on the nouns which mark location.

%The distinction between 
%The two locative prefixes in Moro are identifiable as case markers 

\subsection{Nominative case}\label{section:nominative}

Nominative case is found only in subject position of finite clauses. It is an unmarked form, consisting of a bare nominal root. While it is the citation form of proper nouns, the citation form of those common nouns which mark a nominative-accusative distinction is the accusative, as discussed in the following section. 

The boundary between subjects and verbs is an environment which allows schwa-epenthesis (\sectref{epenthesis}). Because of this, sonorant-final nominative nouns such as \textit{ómón} leopard in subject position, e.g. \textit{ómón gogəná} `the leopard is big' is realized as [ómónə́ gogəná], nearly identical to its accusative \textit{ómón-á}. Whistling provides an important clue that the schwa is epenthetic in these cases, inserted rather late: while Mr. Julima clearly whistles three high tones for the accusative \textit{ómón-á} (=HHH), the epenthetic schwa is ignored when whistling, hence [ómónə́  gogəná] is whistled HH LLH. 

\subsection{Accusative case}\label{section:accusative}

There are two distinct accusative markers in Moro. The suffix \textit{-a} occurs on common nouns, while human proper nouns take a distinct case suffix \textit{-ŋ/-o}, the latter forms being conditioned by phonological properties of the stem. The accusative form of common nouns is the citation form, while the nominative form of proper nouns is the citation form.

Close relatives of Moro in the Heiban group such as Ebang and Koalib still have a robust, though complex, system of accusative case marking, including addition of vocalic suffixes. The Moro accusative, by comparison, is somewhat marginal. The accusative suffix \textit{-a} is simply absent on many nouns. First, it simply not marked on some nouns, somewhat unpredictably. For example, while \textit{ómón} vs. \textit{ómón-á} `leopard' marks accusative, \textit{ŋgón} `squirrel' does not, though both end in /ón/. Some nouns have roots ending with /a/, and these too are identical in the nominative and accusative; compare \textit{ðabér} vs. \textit{ðabér-a} `wind' to \textit{ðəbárá} `cotton' above. Hence, accusative nouns are not fully predictable from the nominative nor are nominatives predictable from accusatives.

Names of people (\sectref{sec:ch6:names}) consistently occur with accusative case, taking the suffix \textit{-ŋ} if they end in a vowel and \textit{-o} if they end in a consonant (\tabref{tab:ch6:propacc}). A small number of common nouns, including \textit{matʃo} `man' and \textit{ðap:a} `friend' also fall into this pattern.

\begin{table}
\begin{tabular}[t]{ll}
\lsptoprule
Nominative & Accusative\\
 \midrule 
 Jasir & Jasir-o \\
 Bitəɾ & Bitəɾ-o \\
 Yosev & Yosev-o \\
 Kúk:u & Kúk:u-ŋ \\  
 Kák:a & Kák:a-ŋ \\
 ða:pa & ðap:a-ŋ \\
\lspbottomrule
 \end{tabular}
\caption{Case on proper names}\label{tab:ch6:propacc}	
\end{table}

The distribution of accusative case is discussed in Jenks \& Sande (2017). Accusative case occurs on all nominal objects that mark case, and does so regardless of their semantic role relative to the noun. In the case of multiple objects, accusative case occurs on all of them:

\ea \gll  é-g-a-nac-ó ŋál:o-ŋ kódʒa-ŋ\\
\textsc{1sg-cl}g-\textsc{rtc}-give-\textsc{pfv} Ngallo-\textsc{acc} Koja-\textsc{acc}\\
\glt  `I gave Ngallo to Koja.' / `I gave Koja to Ngallo.'
\z 

Accusative case also occurs in non-object positions. For example, accusative occurs on the second conjunct of coordinated nouns, even in subject position:
\ea
\ea  \gll 	kúk:u na ŋál:o-ŋ l-aŋer-\'a\\
			Kuku {and} Ngalo-\textsc{acc} \textsc{cl}l.\textsc{rtc}-good-\textsc{adj} \\
	\glt 	`Kuku and Ngalo are nice.' \hfill (Jenks \& Sande 2017, (4a))
\ex \gll	ogovél na ómón-á l-aŋer-\'a\\
			monkey and leopard-\textsc{acc} \textsc{cl}l.\textsc{rtc}-good-\textsc{adj}\\
	\glt	`The monkey and the leopard are nice' 
\z 
\z
Accusative case can also occurs on complements of kinship nouns (\sectref{sec:ch6:kinship}) when they do not agree with the noun:
\ea
\ea   \gll l\textipa{@N}ge k\'uk:u-\bf{\textipa{N}}\\
mother Kuku-\textsc{acc}\\
\glt  `mother of Kuku'
\gll  l\textipa{@N}g-en g\textipa{\'9}-k\'uk:u \\
  mother-\textsc{3.poss} \textsc{cl}g.\textsc{gen}-Kuku \\
\glt  `Kuku's mom' \hfill  (Jenks \& Sande 2017, (6a-b))
\z 
\z 
Thus, accusative case in Moro cannot be associated primarily with the syntactic function of objecthood. Jenks \& Sande 2017 suggest it is a dependent case, meaning it occurs when two noun phrases are in a particular structural configuration. 


\subsection{Inessive \textit{é-}}\label{sec:ch6:inessive}

The inessive case \textit{é-} attaches to nouns and conveys general location, as well as a sense of concealed enclosure. It has a range of different forms, particularly in front of vowel-initial nouns. 

When the inessive prefix occurs on vowel-initial nouns, a consonant intervenes between the two vowels. The consonant agrees for noun class, but is different depending on the singular or plural nature of the class, as illustrated in \tabref{tab:ch6:iness}. The form of the prefix is \textit{ék-} with singular vowel initial nouns of either the g- or j- class, and \textit{és-} with plural vowel initial nouns of either the g- or j- class. With unpaired noun classes, those usually reserved for mass nouns or words beginning with other consonants, such as sibilants ([s, ʃ ʧ]) for the j-class, the pattern is \textit{ék-} with g-class and \textit{és-} with j-class. It is of historical relevance that s-initial words are very uncommon in Moro, mostly found in borrowings such as \textit{sura} `picture', \textit{suk} `market', both j-class (\sectref{sec:ch6:j}. This point suggests that \textit{*s} may have been lost word-initially at some point in the relatively recent past, but that it was preserved by the inessive prefix.

\begin{table}
\caption{Allomorphy in inessive prefix}\label{tab:ch6:iness}
\begin{tabular}[t]{lllll}
\lsptoprule
Class	&	Form &	Inessive	\\
\midrule 
Singular& g	&	ék-			&	ékomágá 		&	in the snail\\
Singular& j	&	ék-			&	ékajén		&	in the mountain\\
Plural	& g (of singular ð)	&	és-			&	ésóbʷáðá&	in the gum tree\\
Plural	& j (of singular ð)	&	és-			&	ésə́və́rá	&	in the lines\\
Plural	& j (of singular j)	&	és-			&	ésején	&	in the mountains\\
unpaired& g	&	ék-			&	ékólé		&	in the language\\
unpaired& j	& 	és-  		&	ésə́t̪ó		&	in the dew\\
\lspbottomrule
\end{tabular}	
\end{table}

The inessive prefix \textit{é-} is raised to [í] when attached to nouns with higher vowels /i  ɜ u/. In (a-d), the nouns contain lower vowels and the prefix is [é]. In (e-h), the nouns have higher vowels, and the prefix is raised to [í] 

\ea Vowel harmony
\begin{tabular}{llllll}
a.&	é-lóɾá	&	‘in the creek’		&	e.&	í-nɜ́dí	&‘in the skins’\\
b.&	é-ndeə́	&	‘in the dalib tree’	&	f.&	í-ɽút̪uɜ́	&‘in the knot’\\
c.&	é-ŋáná	&	‘in the milk’		&	g.&	í-lɜ́ɾí	&‘in the calf (of leg)’\\
d.&	é-wáɾá	&	‘in the chicken’		&	h.&	í-lútí	&‘in the owl’\\
\end{tabular}
\z 

The high tone of the \textit{é-} prefix spreads rightwards onto the noun. All of the nouns in (2) are high-toned, so the attachment of the prefix does not affect their tone. There are two basic patterns to the high tone spreading, which appear to be optional: spread high onto the first vowel only or spread high tone to the end of the word. The two patterns are illustrated with words that are low-toned:

\ea High tone spreading	\\
\begin{tabular}[t]{lllll}
a.&	boʧa	&	‘ashes’		&é-bóʧa	&‘in the ashes’\\
 &			&		 	 &	 	é-bóʧá	& \\
b.&	ðəvəra	&	‘line’	 	&é-ðə́vərá	&‘in the line’\\
 &			&		 	 &	 é-ðə́və́rá	& \\
c.&	naba	&	‘holes’		&é-nába	&‘in the holes’\\
 &			&		 	 &	 é-nábá &	\\
\end{tabular}
\z 

If the noun begins with one or more low-toned vowels followed by a high tone, high tone spreading halts one syllable away from the high tone on the noun:

\ea	High tone spread interruption\\
\begin{tabular}[t]{lllll} 
a.&	nəmərt̪á		&	‘horses			&	é-nə́mərt̪á	&	‘in the horses’\\
b.&	ŋombogó		&	‘calf (baby cow)’&	 	é- ŋómbogó	&	‘in the calf’\\
c.&	ʧarbapóða	&	‘lung’			&	é-ʧáŕbapóða	&	‘in the lung’\\
d.&	ed̪apəgá		&	‘nail’			&	ék-ə́d̪ápəgá	&	‘in the nail’\\
\end{tabular}
\z 

The inessive prefix in its [ék] form participates in voicing dissimilation. When it is attached to a vowel-initial noun that has an immediately following voiceless stop or affricate (p t̪ t ʧ k), the prefix is realized as [ég] (6b). If another consonant intervenes between the prefix and the voiceless consonant in the noun stem, then no voicing occurs (6c):

\ea	
	\ea 	
	\begin{tabular}[t]{llll}
			ómóná	& 	‘tiger’	&	ék-ómón		&‘in the tiger’\\
			ogovélá	&	‘monkey’	&	ék-ógovél	&‘in the monkey’\\
	\end{tabular}
	\ex 	
	\begin{tabular}[t]{llll}
			ɜ́t̪ŕíə		&	‘gums’	&	íg-ɜ́t̪ŕíə	&	‘in the gums’	\\
 			etám		&	‘neck’	&	ég-ətám		&	‘in the neck’\\
			aʧóŋgʷárá	&‘bird of prey’	&	ég-aʧóŋgʷár	&‘in the bird of prey’	\\
			ɜ́pwɜ		&	‘stick-fighting	&	íg-ɜ́pwɜ	&	‘in the stick-place,fighting place’	\\
	\end{tabular}
	\ex		
	\begin{tabular}[t]{llll}
			óɾə́pʷá	&‘nest hole’	&	ék-óɾə́pʷá	&‘in the nest hole’\\
			íɾtí	&‘knife’		&	ík-ə́ɾtí		&‘in the knife’\\
	\end{tabular}
	\z
\z

Since there are so few nouns that have [s], there is only one word that has the right configuration to test whether [s] triggers voicing dissimlation, too. In this case, there is variable voicing: \textit{ík-úsílá} or \textit{íg-úsílá} ‘in the spirit’.    

%In some cases, addition of the inessive prefix to a vowel-initial noun triggers reduction of the first vowel of the noun to [ə]:
%
%\ea 
%EXAMPLES HERE %TODO NEED EXAMPLES
%\z 
%
%This is a common process in Moro, and is attested with plural noun class markers, too (Gibbard et al 2009).
%

%\subsubsection{Final vowel changes}
%
%  The addition of the inessive prefix often appears to cause the loss of the final vocalic mora of the noun. This can either be loss of a final simple vowel, or simplification of a diphthong. The consonant preceding the final noun must be one that is permitted word-finally. Some examples:

%It can be hypothesized that the reason so many nouns end in vowels in Moro is due to case endings which have now been lexicalized. Related languages such as Ebang and Koalib still have a robust, though complex, system of case marking, including addition of vocalic suffixes. The ‘loss’ in these inessive cases may not be loss at all, but a representation of the original underlying form that used to end in consonant. When the same word is used as a subject or object, it has the final vowel. Therefore, the distinction between the two nouns in question may have been that they were originally distinguished by whether there was a final vowel or not: \textit{ðəbáɾá} versus \textit{ðəbér}. This form appears in the inessive, but else-where the nouns are used with final vowels. Some evidence in favor of this hypothesis is that for some words, the final vowel may only appear in the object form, ex. \textit{ŋáw} ‘water’ or \textit{ogóm} ‘thief’ when subjects but \textit{ŋáwá} and \textit{ogómá} when objects; the inessive forms are \textit{é-ŋáw} and \textit{ék-ogóm}. %is it really true that the same word has a suffix when a subject or an object? won't the final vowel often disappear in one position or another?


\subsection{Adessive \textit{n-}}

There is another locative prefix \textit{n-}, which we label adessive case for reasons discussed in the introduction to this section. The general meaning of the locative \textit{n-} is `on’, but it can also convey other senses such as `off, from, over’.  

\ea 
	\ea \gll é-g-a-daŋ-ó n-deté		\\	
			I-sat on-branch\\
		\glt `I sat on the branch'	
	\ex \gll loandra lɜmurkú n-ajn		 \\
			rock rolled on-hill\\
		\glt the rock rolled down the hill
	\ex \gll k-aŋg-at̪-ó n-ɜlbɜ́mbəriə	\\	
			\textsc{cl}-?-\textsc{loc.appl}-\textsc{pfv} on-stool\\
	\glt 	`he moved off the stool'??	6/16/2011 %(SHARON's data)
	\ex		kɜmuɾəʤəʧí n-alét̪a	\\
			rock rolled on-hill\\
			'he passed it over the wall'
	\z
\z 

To determine: is this prefix /n/ or /nə-/?

Allomorphs when attaching to coronal-initial roots in Thetegovela? Clear differences in Werria here.


\subsection{Instrumental}

The instrumental or comitative marker is \textit{–(ə)Ca}, with the schwa only occurring after consonant final noun roots. The C in these suffixes stands for noun class concord --- a consonant that agrees in noun class with the noun to which the suffix is attached, one of the eight class markers: /n ɲ ŋ l r j g ð/. The suffix is used to indicate an instrument or tool, or accompaniment. In addition, the object of some verbs are required to have an instrument suffix. 

The following words illustrate noun class agreement with the instrumental suffix:

\ea	Noun class agreement with instrumental\\
\begin{tabular}[t]{llllll}
a.	&	nəbamba-na	&	‘with the drums’	&	e.	&	rəmʷa-ra		&	‘with God’\\
b.	&	ɲogopáj-ə́ɲá	&	‘with the cups’		&	f.	&	ɜɲɔŋ-əja		&	‘with the mouth’\\
c.	&	ŋavəɾa-ŋa	&	‘with sticks’		&	g.	&	wálá-gá		&	‘with braids’\\ %hair?
d.	&	liʤí-lá		&	‘with people’		&	h.	&	ðugi-ða		&	‘with a wooden plank\\
\end{tabular}
\z 

The tone of the instrumental suffix matches the tone of the final vowel of the noun stem. It is high-toned if the preceding vowel is high-toned, as in (1b,d,g); otherwise, it is low-toned.  In some cases, the final vowel can be reduced or deleted: ex. \textit{ðəgívi} ‘bread’ forms its instrumental by dropping the final vowel: \textit{ðəgívə́ðá}. The high tone of the second syllable is copied or spread onto the instrumental suffix. 

The instrumental prefix does not undergo vowel harmony, as illustrated by the examples (1d), (1f) and (1h), which all contain higher vowels. The final vowel remains [a], and is not raised to [ɜ]. 

Instrumental nouns have both instrumental and comitative meanings. The following sentences illustrate uses of the instrumental suffix as an instrument. 

\ea \gll	g-adʒə́v-á 	(k-é-nː-a) 	nɜniŋɜ́-ná	\\
\textsc{sm.cl}g-not.know-\textsc{ipfv} 	(\textsc{sm.cl}g-\textsc{dpc}1-hear-\textsc{ipfv})	\textsc{cl}n.ear-\textsc{cl}n.\textsc{inst}\\
\glt ‘he doesn’t know how to hear with his ears’ = he has a hearing problem
\z 
%add clearer instrumental example here.

The following sentences illustrate use of the suffix as a comitative, indicating accompaniment:

\ea
	\ea	\gll Kúku        l-erl-ó             t̪út̪u-ga	\\
\textsc{cl}g.Kuku \textsc{cl}l-walk-\textsc{pfv} \textsc{cl}g.Tutu-\textsc{inst}		\\
		\glt `Kuku walked with Tutu.'\label{ex:ch6:plcomit}
	\ex \gll	ɲá-g-a-dwat-ó kúkːu-ga	\\
1\textsc{pl}-\textsc{cl}g-RTC-speak-\textsc{pfv} \textsc{cl}g.Kuku-\textsc{cl}g.\textsc{inst}			\\
		\glt ‘We talked with Kuku.’
	\ex \gll 	l-abə́ð-á ŋə́ní-ŋá \\
	\textsc{cl}l-pet/play-\textsc{ipfv} \textsc{cl}ŋ.dog-\textsc{cl}ŋ.\textsc{inst} \\
		\glt `They are petting/playing with the dog.’
	\ex	\gll	bə́té	ɲá-g-anː-á 	ɲá-bolw-a 		kodʒa-ga \\
never	1\textsc{pl}.\textsc{sm}-\textsc{cl}g-not-\textsc{pfv}	1\textsc{pl}.\textsc{sm}-wrestle-\textsc{inf}	Koja-\textsc{cl}g.\textsc{inst}\\
		\glt 	‘We never wrestle with Koja’ 
\z \z 
Proper names such as Kuku, Tutu, and Koja are human class g, and therefore the instrumental suffix is \textit{–ga}. The presence of a comitative which accompanies the subject typically triggers plural agreement on the verb, even if the subject is singular, as \REF{ex:ch6:plcomit} demonstrates. This seems to be a subcase of a more general phenomenon in Moro whereby nouns are able to occur discontinuously from their modifiers, which occur post-verbally. 


%The verb, however, has a 3rd plural marker \textit{l-} (noun class for human plural) in (X) and a 1\textsc{pl} marker in (X). 
%
%\ea
%	\ea \gll lóg-ó 	t̪ét̪ə́m-ə́gá		\\
%			tell-\textsc{imp} 	truth-\textsc{cl}g.\textsc{inst}	\\	
%		\glt tell the truth!
%	\ex \gll kə́və́ð-ó 	rumwa-ra 		é-ŋén \\ 
%		share-\textsc{imp}  God-\textsc{cl}r.\textsc{inst}	\textsc{loc}-word\\		
%		\glt `have fellowship with God!’
%	\z
%\z 
%In [a], the expressions of `truth’ describes the manner of speaking while in [b] the suffix contributes a comitative sense of being `with’ God.

%Some verbs require their object to be marked with the instrumental suffix. Some examples are given here:

%Nǝṯia Lǝmwarǝŋ lëbwa lǝṯurala, oɽoya na larla kañ
%Nǝṯia Lǝmwarǝŋ l-ëbw-a lǝṯura-la, oɽo-ya na lar-la kañ
%so moro.people cll-love.rt-ipfv pigs-cll.with, goat-cly.with and hen-cll.with very
%So the Moro people like pigs, goats and hens more than anything else.

%ŋenŋanṯa ñënŋulu ñafo ñëbwa Kapenaga,
%ŋenŋanṯa ñ-ënŋulu ñ-afo ñ-ëbw-a Kapena-ga,
%becausecl ñ-3pl.pro clñ-past.aux clñ-love.rt-ipfv Kapena-with
%because they loved Kapena,

%Other verbs %meet?
%
%\ea 	
%	\gll g-ɜ́min-iə ŋabəga-ŋa \\
%		\textsc{sm.cl}g-be.boastful-\textsc{ipfv}  \textsc{cl}ŋ.strength-\textsc{cl}ŋ.\textsc{inst}\\
%	\glt `he is boastful about his strength’
%\z

%couldn't this just be an adjunct here? Also, this doesn't really seem to match either of the uses above. 

\section{Names and kin}

This section describes terms used for familiar human reference, including names and kinship terms. It also includes a discussion of the associative plural, which is limited to this class of entities.

\subsection{Names}\label{sec:ch6:names}

Humans typically have two kinds of proper names in Moro used for familiar reference: the first is their given name, often of Christian or Muslim origin, the second is a name which indicates their sex and birth order. The numbering is inclusive of both genders, hence the second girl in a family will be \textit{Nni} regardless of whether she has an older brother or sister. Surnames are patrilineal, typically the given name of one's father, and one typically has as many surnames as patrilineal ancestors can be remembered.

The birth order names are provided in \tabref{tab:ch6:birthnames}. These names typically have a consistent CVC:V template with HL tone. These names all have several nickname variants as well, many of which involve some minimal phonological change, such as dropping the first consonant in the name, e.g. \textit{káka} vs. \textit{áka}, and kál:o ŋálo , tʃál:o ál:o
\begin{table}
\caption{Birth order names in Moro}\label{tab:ch6:birthnames}
\begin{tabular}[t]{lll}
\lsptoprule
		& Boys	& Girls \\
\midrule
1 & Kúk:u & Kák:a \\
2 & Kwʌ́ri & Kʌ́n:i \\
3 & Kál:o & Kwátʃe \\ 
4 & Tút:u & Kʌ́tʃi \\
5 & Kʌ́w:ʌ  & Kʌ́w:ʌ \\
6 & Kója  & Kója \\
\lspbottomrule
\end{tabular}	
\end{table}

A common nickname for mother is \textit{nán:a} and father is \textit{áp:a}, both of which can take the proper name accusative suffix: \textit{nán:a-ŋ} and \textit{áp:a-ŋ}.

\subsection{Associative plural}\label{associative}

In addition to marking plurals inflectionally via noun class prefixes, Moro has an associative plural suffix \textit{-andá}, \textit{-ŋə́nda} after most vowel-final stem, which can attach to proper names, kinship nouns, and certain high animacy nouns. 

%what conditions anda vs. enda? tone??
\ea	
\begin{tabular}[t]{llll}
jasər &  `Elyasir' (name) & jasər-andá & `Elyasir and company'\\
dʒordʒ &  `George' (name) & dʒordʒ-andá & `Elyasir and company'\\
kúkú	& 	`Kuku' (name) & kúk:u-ŋə́nda & `Kuku and company' \\ 
áp:a	& `dad'	& 	áp:a-ŋə́nda & `dads' \\
emað-áɲ & `my peer' & lamað-áɲ-andá & `my peers' \\
umurt-áɲ & `my co-spouse' & ləmurt-aɲ-andá & `my co-spouses' \\
matʃó & `guy'  & 	 matʃ-ánda & `guys'\\	
\end{tabular}
\z

This suffix also been incorporated into the plural agreement paradigm and plural pronouns, most clearly in the marking of first person inclusive plural and second person plural forms (see Chapter \ref{pronouns} and Section \ref{agreement}).

The semantics of the associative plural suffix in the examples above is not always the same. For proper nouns, the associative plural can refer to a group of individuals associated with the named individual. For example, \textit{Kúk:u-ŋə́nda} would refer to Kuku and his family or friends, or a contextually relevant group of individuals associated with Kúk:u. Another meaning available for birth order names like Kúk:u is the group of first born boys, in this case, or the group of people named `Kuku.' With kinship terms and nouns like \textit{matʃó}, which do not mark a plural inflectionally, the associative plural has this `true' plural meaning, referring to multiple men or multiple mothers, which would typically. 

%One example of these, kú:ku-ŋənda, had an all H melody in subject position but all-L in object position. This needs to be checked again...

%add textual example?
% is this a clitic? cf: Ŋenŋanṯa ***eṯe Kakaŋanda*** alǝtođa alǝmeṯa ṯwañ, translated "in order to let the fathers of Kakan move nearer to him,"

\subsection{Kinship and inalienable possession}\label{sec:ch6:kinship}

Kinship terms, nouns describing family relations, form a distinct morphological class of nouns in that they can, and in some cases must, occur with a set of possessive agreement suffixes.  Kinship terms uniformly fall into the human g/l class, and they pattern with proper nouns and a few other human nouns in their ability to take the associative plural suffix. 

We have identified eleven inalienably possessed kinship terms. These nouns actually fall in three different groups grammatically. One group must occur with a possessive suffix, a second group does not need a possessive suffix but must have a possessor internal to the noun, and the third group can occur as an unpossessed noun with a general meaning:

\ea Kinship nouns in Moro
\begin{tabular}[t]{llll}
\multicolumn{4}{l}{\emph{Group 1: Obligatory possessive suffix}}\\
eváŋg-aɲ &	`my husband' & *eváŋga & \\
emað-áɲ	& `my peer (sg.)'	& *emaða & \\
iðjəŋɡ-áɲ	& `my offspring (sg.)'	& *iðʲəŋga & \\
ib-ɜ́ɲ	& `my sibling-in-law (sg.)'	& *ibɜ & \\
umərt-áɲ & `my co-spouse'	& *umurtɜ	\\
\multicolumn{4}{l}{\emph{Group 2: Obligatory possession}}\\
 lɜŋg-aɲ	& `my mother'	& lɜŋgə Kukuŋ & 'Kuku's mother' \\
et̪-áɲ	& `my father' &	 et̪ə́ Kukuŋ & `Kuku's father'\\
ud̪ɜ́r-aɲ &  `my uncle' & ud̪uruwa Kukuŋ & `Kuku's uncle' \\
\multicolumn{4}{l}{\emph{Group 3: Optional suffix and possession}} \\ 
was-áɲ & `my wife &  wasa & `a brother's wife' \\	
or-aɲ & `my sibling/cousin' & orəwa & `a relative' \\ 
un-ɜɲ & `my parent/child in-law & unɜ & `an in-law' \\
 \end{tabular} 
 \z 
 
% Lexicon: new inalienably possessed noun, ‘household’, p. 59
% dʌŋgʌɲ		‘My home’
%dʌŋgʌlʌŋ	‘You and my home.
%dʌŋgen		‘His home.’
%dʌŋgʌndr	‘Our home’
%dʌŋgalo	‘Your home’

While they cannot appear without suffixes, nouns in Group 1 can be nominalized by the nominalizing \textit{ŋ-} prefix (SECTION?) and occur as a nominal predicate, as shown below. Similarly, nouns in all groups can occur without possessive marking when they are plural, illustrated with the subject in the second example below, which must be possessed in the singular:

\ea 
	\ea \gll  alə-g-a-d-o-r ŋ-əmurtu\\
			1\textsc{pl}-\textsc{cl}g-\textsc{rtc}-be.\textsc{pred}-\textsc{pfv}-\textsc{pl} \textsc{cl}ŋ.\textsc{nom}-co.spouse\\
		\glt 	`We are married to the same family.' \label{kina}
	\ex \gll ləṯenia l-a-w:o l-a-d-ó ŋ-əmađa\\ 
	fathers \textsc{cl}l-\textsc{rtc}-{\textsc{past.aux}} \textsc{cl}l-\textsc{rtc}-{be.\textsc{pred}}-\textsc{pfv} \textsc{cl}ŋ.\textsc{nom}-peer\\
\glt 	`The fathers who were peers.' \label{kinb}
		\z
\z

Not all nouns which are notional kinship terms can take a possessive suffix. For example, the nouns \textit{ut̪ɜɽi} `old man, grandfather' and \textit{opa} `old woman, grandmother' cannot take any kinship suffixes.

The meanings of these kinship terms are more inclusive than their English translations. For example, `father’ can refer to a father’s brother, a point which explains the meaning of the plural forms below. Uncle/aunt is used for mother’s brother or sister. The word for sibling, as indicated, can also refer to a cousin. `Mother’ can refer to a married woman of good standing in the community. Additionally, \textit{umərtáɲ} `my co-spouse' could be used either by a woman to refer to her husband's other wives, if he has more than one. Alternately, it can refer to another individual of the  same gender as me who has married into the same family, either my wife's sister's husband if I am a man or my husband's brother's wife if I am a woman. Note that in either case my co-spouse will always be the same gender as I am.

The possessor agreement suffixes are summarized below. Unlike verbal agreement and other pronominal paradigms, possessive suffixes do not distinguish between 2nd and 3rd person singular and plural, and there is a single form used for 1st person singular and 1st person plural exclusive. There are, however, separate suffixes for 1st dual inclusive and 1st plural inclusive, the latter of which is built off the 1st dual inclusive, with the addition of another suffix which is clearly related to the associative plural:

\ea Inalienable possessive suffixes\\
\begin{tabular}[t]{llll}
1(\textsc{ex}) &	-aɲ	&	& 	\\
1\textsc{in.du}	& -ɜləŋ	&  1\textsc{in.pl}	&  -ɜləŋ-ə́ńdr\\
2	& -aló	& & 	\\
3	&  -én & & \\
\end{tabular}
\z
Possessive suffixes are identical to the plural forms of possessive pronouns minus class agreement (Section \ref{section:posspro}), a portion of which is reproduced below The first portion marking possession is the same segmental material (with some tone differences and vowel reduction /o/ → [ə]):

\ea Possessive pronouns (for comparison)\\
\begin{tabular}[t]{llll}
1\textsc{ex.pl}	 & íkː-aɲ-k-aɲ & & \\
1\textsc{in.du} &	íkː-ɜlə́ŋ-ə́ki & 1\textsc{in.pl} &	íkː-əndŕ̩-ki\\
2\textsc{pl} & 	íkː-alə́--k-e	&  & \\
3\textsc{pl}	& íkː-en-k-en & & \\
\end{tabular}
\z 
The 1\textsc{sg}-2-3 possessive suffixes all contain low vowels, whereas the 1\textsc{dual} and 1\textsc{pl} inclusive suffixes contain the high vowel [ɜ]. This difference will play a role in vowel harmony.

Of the eleven inalienably possessed kinship nouns that have been identified, six have vowels belonging to the lower set. The vowels are raised by the two suffixes whih the high vowel [ɜ]. This suffix conditions raising of the vowels of the root (e, a, o → i, ɜ, u). It is the only nominal suffix that triggers raising.  %check these, include list?

\ea Singular kinship nouns with low vowel harmony\\
\begin{tabular}[t]{llll}
\lsptoprule
		& father 	& mother 		&	sibling/cousin	\\
	\midrule
root	& et̪-		&ləŋg-			&or- \\
1ex 		&et̪-áɲ		&ləŋg-áɲ		&or-áɲ		 \\
1in.du	&it̪-ɜlə́ŋ	&ləŋg-ɜlə́ŋ		&ur-ɜlə́ŋ \\
1in.pl	&it̪-ɜlə́ŋ-ə́ńdr&	ləŋg-ɜlə́ŋ-ə́ńdr&	ur-ɜlə́ŋ-ə́ńdr\\
2		&et̪-aló		&	ləŋg-aló	&or-aló\\
3		&et̪-én		&ləŋg-ín		&or-én\\
	\midrule
	&	wife	&	husband &	peer \\
	\midrule
root	&	was-		&	eváŋg-		& emað-	\\	 
1ex		&	was-áɲ		&	eváŋɡ-áɲ	& emað-áɲ\\
1indu	& wɜs-ɜlə́ŋ		&	ivɜ́ŋg-ɜ́lə́ŋ	 & emað-ɜ́lə́ŋ\\
1inpl	& wɜs-ɜlə́ŋ-ə́ńdr	&	ivɜ́ŋg-ɜ́lə́ŋ-ə́ńdr	 & emað-ɜ́lə́ŋ-ə́ńdr	\\
2		&	was-aló		&	eváŋg-áló	 & emað-áló\\
3		&	was-én		&	eváŋg-ín	 & emað-én\\
	\lspbottomrule
\end{tabular}	
\z 

The five other kinship terms have high vowels. In these cases, vowel harmony extends from the root, but only in a limited manner. If the root contains a single vowel and the suffix a single vowel, harmony applies, as seen with \textit{un-ín}. However, if the suffix contains two vowels, harmony does not apply: \textit{un-aló}. If the root contains two vowels, no harmony extends, as seen with \textit{ud̪ɜr-áɲ}. This restriction can be interpreted as harmony operating in the rightward or progressive direction in a non-iterative manner, so only one vowel to the right. If harmony is restricted to apply in this manner, then application to a form like \textit{un-aló} would render the suffix disharmonic (\textit{-ɜló}), and harmony is blocked. 

\begin{table} 
\caption{Singular kinship nouns with high vowel harmony} \label{kintable}
\begin{tabular}[t]{llll}
\lsptoprule
		& Parent-in-law	&	Sibling-in-law	&	Mat. uncles/aunts	\\
	\midrule
root	& un-	&ib-	&ud̪ɜr-	\\
1ex 		& un-ɜ́ɲ	&ib-ɜ́ɲ	&ud̪ɜr-áɲ		 \\
1indu	& un-ɜlə́ŋ &	ib-ɜlə́ŋ &	ud̪ɜr-ɜlə́ŋ	\\
1inpl	& un-ɜlə́ŋ-ə́ńdr	& ib-ɜlə́ŋ-ə́ńdr & ud̪ɜr-ɜlə́ŋ-ə́ńdr	\\
2		& un-aló	& ib-aló	& ud̪ɜr-aló	\\
3		& un-ín		& ib-ín		& ud̪ɜr-én	\\
	\midrule
	&	Offspring	&	Co-spouse & \\
	\midrule
root	&	iðjəŋɡ-			&	uməɾt-			& \\	 
1ex		&	iðjəŋɡ-áɲ		&	umurt-áɲ		& \\
1indu	& iðjəŋɡ-ɜlə́ŋ		&	umurt-ɜlə́ŋ	 	& \\
1inpl	& iðjəŋɡ-ɜləŋ-ə́ńdr	&	umurt-ɜlə́ŋ-ə́ńdr	& \\
2		&	iðjəŋɡ-aló		&	umurt-aló	 	& \\
3		&	iðjəŋɡ-én		&	uməɾt-ín		& \\
	\lspbottomrule
\end{tabular}	
\end{table}
	
Restricting vowel harmony in this manner is not attested in verbs. For example, if a verb root has two or more vowels, the aspect mood suffix always harmonizes: compare \textit{k-a-ʧombəð-ó} `he tickled’ with \textit{k-ɜ-murnin-ú} `he pretended, acted like’. Furthermore, harmony also applies to intervening extension suffixes such as the locative \textit{-at̪} (see Section \ref{vharmony}). %TODO revise: redundant discussions of vowel harmony in possessive suffixes?

The number of first person exclusive, second, and third person possessors can be disambiguated by adding possessive pronouns (Section \ref{posspro}), which do make the relevant distinctions in number marking:

\ea 
\hspace{-12pt} \vspace{-12pt} \begin{tabular}[t]{lllllll}
 a. &  ləŋɡ-áɲ & k-əŋkəŋ	& {\ \ \ } & b.	&	ləŋɡ-áɲ &  k-aɲkaɲ\\
 & mother-\textsc{1ex} & \textsc{scl-1sg.poss}  & & &	mother-\textsc{1ex} & \textsc{scl-1plex.poss} \\
& \multicolumn{2}{l}{‘my motherʼ} & &  & \multicolumn{2}{l}{‘my motherʼ}	\\	
\end{tabular}
\z
This example also shows that although the word for `mother’ begins with [l], it is a class-g noun, conditioning concord with [k] on the possessive, as all the singular kinship terms belong to class-g. Additionally, the observation that possessive suffixes co-occur with possessive pronouns demonstrates that these are agreement suffixes rather than incorporated possessive pronouns.

Furthermore, the genitive construction can be used to refer to a particular person’s kin:

\ea \gll was-én ɡ-↓ə́-↓kúkːú \\
wife-3 \textsc{cl}-\textsc{poss}-1\textsc{sg}.\textsc{poss} \\
\glt ‘Kuku's wifeʼ
\z 

In this case, the [g] concord is used rather than [k], so the strong concord iCː- is not used, presumably because of the presence of the specific inalienable possession marker. On the other hand, when kinship terms occur without any possessive suffix, as in the examples in \tabref{kintable}, the possessor can be bare, without any genitive marking at all.

The number of the kinship term is marked by pluralizing the stem itself and by adding an associative plural suffix, illustrated in \tabref{tab:ch7:plkintable}. In most cases, pluralizing the stem consists of adding the plural class marker for humans \textit{l-} along with the predictable vowel changes for a g/l class noun, for example \textit{ib-} `sibling-in-law' to \textit{ləb-} `siblings-in-law' (Section \ref{section:glclass}). However, in the case of father and mother, there is a suppletive form used. For father, singular \textit{et̪-} is replaced with plural \textit{eɾ-}, and for mother singular \textit{ləŋg-} is replaced with plural \textit{el-}. There are also additional changes for the forms for husband and uncle. The singular form for `husband’ \textit{evaŋg-} corresponds to plural ləvál-, where the final [ŋg] is replaced with [l]. The singular form for `maternal uncle/aunt’ \textit{ud̪ɜr-} corresponds to \textit{ə́ldwɜ́rl-}, also with a final [l]. 
	
\begin{table} 
\caption{Plural kinship nouns: Full paradigms}
\label{tab:ch7:plkintable}
	\begin{tabular}[t]{llll}%check peers with Elyasir, and forms
\lsptoprule
	&	Fathers	&	Mothers	&	Siblings/cousins	\\
\midrule
Root	&	eɾ-	&	el-		&	lor-\\
1ex 	&	eɾ-áɲ-andá	&	el-áɲ-andá	&	lorl-áɲ-andá	\\
1indu & ir-ɜlə́ŋ-andá &	il-ɜlə́ŋ-andá&	lurl-ɜlə́ŋ-andá \\
1inpl& ir-ɜlə́ŋ-ə́ńdr &	il-ɜlə́ŋ-ə́ńdr&	lurl-ɜlə́ŋ-ə́ndr \\ 
2	 &	er-ál-andá	& 	el-ál-andá	&	lorl-ál-anda \\
3	 &	er-én-andá	&	el-én-andá	& 	lorl-énandá  \\
\midrule
		&	Wives	&	Husbands		& Peers	\\
\midrule
Root 	&	lwas-		&	ləvál-		&	lamað- \\
1ex		& lwas-áɲ-andá	& ləvál-áɲ-andá	&	lamað-áɲ-andá \\
1indu	& lwɜs-ɜ́lə́ŋ-andá	&	lɜvɜ́ng-ɜ́lə́ŋ-andá & lamað-ɜ́lə́ŋ-andá	\\
1inpl	& lwɜs-ɜ́lə́ŋ-ə́ndr	&	lɜvɜ́nɡ̤-ɜ́lə́ŋ-ə́ndr & lamað-ɜ́lə́ŋ-ə́ndr	 \\
2		& lwas-ál-andá	&	lavál-ál-andá	& lamað-ál-andá	\\
3		&lwas-éʲn-andá	&	lavál-én-andá & lamað-en-andá	\\
\midrule
	& Parents-in-law	& Sibling-in-law & 	Mat. uncles/aunts	 \\
\midrule	
Root 	&	lɲw-			&	ləb-		&	ld̪wɜ́rl-	\\
1ex 		&  lɲw-áɲ-andá 		& 	ləb-áɲ-andá &	ld̪wɜ́rl-áɲ-ənda	\\
1indu	&	lnʷ-ɜlə́ŋ-andá	&	ləb-ɜlə́ŋ-andá	&	ld̪ʷɜ́rl-ɜlə́ŋ-andá	\\
1inpl	&	lnʷ-ɜlə́ŋ-ə́ndr 	&	ləb-ɜlə́ŋ-ə́ndr	&	ld̪ʷɜ́rl-ɜlə́ŋ-ə́ndr \\
2		&	lnʷ-ál-andá		&	ləb-ál-andá	&	ld̪ʷárl-ál-andá	\\
3		&	lnʷ-ín-andá		&	ləb-ín-andá	&	ld̪ʷárl-én-andá	\\
\midrule
		& Offspring (pl.)	& Co-spouses & 	 \\
\midrule		
Root 	&	lið\super{j}əŋɡ-		&	ləmurt-	&  \\
1ex 	&  líð\super{j}ə́ŋɡ-áɲ-andá 	& 	ləmurt-aɲ-andá & \\
1indu	&	líð\super{j}ə́ŋɡ-ɜ́lə́ŋ-andá 	&	ləmurt-ələŋ-andá & \\
1inpl	&	líð\super{j}ə́ŋɡ-ɜ́lə́ŋ-ə́ndr 	&	ləmurt-ələŋ-andá & \\
2		&	líð\super{j}ə́ŋɡ-ál-andá	& ləmurt-al-andá & \\
3		&	líð\super{j}ə́ŋɡ-én-andá 	& ləmurt-in-andá & \\
\lspbottomrule
	\end{tabular}
\end{table}

- Is the assoc suffix obligatory? can it follow a possessor?
% TODO CHECK the noun class concord.






