\chapter{Pronouns}\label{chap:7:pronouns}

Moro possesses five series of pronouns: independent personal pronouns, object markers, reflexive pronouns, and predicative and attributive possessive pronouns.  This chapter provides a descriptive overview of these pronouns with a focus on their morphological makeup and their syntactic distribution. %demonstrative pronouns?

Moro is a pro-drop language, meaning that pronouns are often omitted in subject position. In addition, non-emphatic object pronouns, which we call object markers following the Bantu tradition, are incorporated into the verb and appear as either prefixes or suffixes on the verb stem (See Section \ref{sec:ch7:om}, Jenks \& Rose 2015). Object markers are restricted to human nouns, nonhuman nouns have no overt pronominal correlate in any of the pronoun series (see below and Section \ref{sec:ch12:objects}).

Moro pronominal and subject agreement paradigms distinguish eight different forms which can be seen as the product of three cross-cutting binary features, $\pm$Speaker, $\pm$Addressee, and $\pm$Plural, as schematized in \tabref{tab:ch7:2}.

\begin{table}
	\begin{tabular}[t]{lll}
\lsptoprule
& Singular/Dual & Plural \\
\midrule
$[+${Speaker},$-${Addressee}$]$ & 1st (exclusive) singular & 1st exclusive plural \\
$[+${Speaker},$+${Addressee}$]$ &1st inclusive dual & 1st inclusive plural\\
$[-${Speaker},$+${Addressee}$]$ & 2nd singular & 2nd plural \\
$[-${Speaker},$-${Addressee}$]$ & 3rd singular & 3rd plural\\
\lspbottomrule	
\end{tabular}
  \caption{Person-number distinctions in pronouns}
  \label{tab:ch7:2}
\end{table}

Moro marks four first person forms, including a distinction in clusivity. “Inclusive” refers to the speaker and the addressee (the person being spoken to), whereas “exclusive” excludes the addressee. When a speaker uses the inclusive dual they are referring to ‘me and you (\textsc{sg}.)’, whereas when a speaker uses the inclusive plural they are referring to ‘me, you and person X’ or `me and all of you.' The exclusive plural references ‘me and person or persons X, but not you.’ 

%check that the claim about gender below is correct. Note that we have Werria pronouns which do seem to agree for gender. An important question is whether Elyasir has these. 
While independent pronouns in Thetogovela Moro do not reflect gender, Moro third person pronouns of all series only occur when they are anaphoric to a particular subset of human nouns: proper names, kinship terms, and \textsc{g-} class human nouns such as \textit{matʃe} and  `man', \textit{ðappa} `friend.' This is the same of nouns which take the associative plural suffix in Moro (Section \ref{associative}). Thus, for example, an object pronoun anaphoric to the noun \textit{ŋera} `child, girl' is null. Compare the following examples:

\ea	
	\ea  \gll kúku g-war-ó ŋalló na nə́ŋ-\underline{ŋú}-bug-i\\
		kuku \textsc{cl}g-insult-\textsc{pfv} Nalo and \textsc{3sg.cons}-\textsc{3sg.om}-punch-\textsc{cons.pfv}\\
		\glt `Kuku yelled at Ngallo$_i$ and then punched him$_i$.'
	\ex \gll kúku g-war-ó ŋera na nə́ŋə́-búg-í\\
			kuku \textsc{cl}g-insult-\textsc{pfv} child and \textsc{3sg.cons}-punch-\textsc{cons.pfv}\\
		\glt `Kuku yelled at the child$_i$ and then punched him$_i$.'
	\z
\z 
As a proper name, the object of the first sentence, \textit{ŋalló}, triggers an anaphoric object marker \textit{ŋú-}. In contrast, when this name is replaced by \textit{ŋerá} `child,' there is no overt object marker or object pronoun in the consecutive clause. Because of this restriction, we will abbreviate 3rd person pronouns in Moro as \textsc{3sg.hum}, where `human' should be understood as a grammatical feature rather than a purely semantic one. %what about the plural pronouns? they do resume extracted pluralities, but what about plural inanimate nouns?

The five series of pronouns in Moro are presented in \tabref{tab:ch7:2} for reference. Possessive forms are given in their Class \textit{g-} form. Possessive pronouns share the 3\textsc{pl} formative with 3rd person inalienable possessive suffixes (\sectref{sec:ch6:kinship}), while the object markers series most closely resemble subject agreement (\sectref{sec:ch11:subjectagreement}).

\begin{table}
  \begin{tabular}{llllll}
    \lsptoprule
				&	Indep. 		& Obj. 			& Refl. 		&  Poss Pred. & Poss. Attr. \\
\midrule
1\textsc{sg}	&		ɲ́ɲí		&	=ɲé			&	ɲé-və́gá		& 	gɜ-k-ɜ́ŋ 	 & íkː-ɜŋ-kɜŋ \\
2\textsc{sg}	&	ŋ́ŋá			& 	=ŋá			& 	ŋá-və́gá		&	ga-k-ó	 & íkː-o-kːe \\
3\textsc{sg.hum}	&	ŋ́ŋúŋ			&	=ŋó			&	ŋó-və́gá		&	ga-k-óŋ	 & íkː-oŋ-koŋ\\
1\textsc{in.du}	&	ndɜ́líŋ		&   =ńdə/ńda	& 	lá-və́gá		&   gɜ-k-ɜlə́ŋ	 & íkː-ɜlə́ŋə́-ki\\
1\textsc{in.pl}	&	ńdr			&	=ńdr		& 	lá-və́gá		&   gɜ-k-əndŕ̩	 & íkː-əndŕ̩-ki \\
1\textsc{expl}	&	ɲ́ɲandá		&	=lánda		&	ɲá-və́gá		&   ga-k-áɲ 	 & íkː-aɲ-kaɲ \\
2\textsc{pl}	&	ɲáŋə́ndá 		&	=ńda 		&	ɲáŋá-və́gá	&	ga-k-aló	 & íkː-alə́--ke	\\
3\textsc{pl}	&	ŋúlwɔ́/ŋúlandá & =lo			& 	ŋúlwɔ́-və́gá 	&	ga-k-én 	 & íkː-en-ken \\
\lspbottomrule 
  \end{tabular}
  \caption{Comparison of pronoun paradigms}
  \label{tab:ch7:2}
\end{table}

%check locative and comitative forms of pronouns!!!

\section{Independent personal pronouns}\label{sec:ch7:indep}

Independent pronouns are pronouns that can occur in the same position as full noun phrases, including as subjects or objects. Moro has the following set of independent pronouns.

\ea 
\begin{tabular}[t]{lll}
		 &	\textsc{sg/du} & \textsc{pl} \\
1\textsc{ex}		&	ɲ́ɲí	& ndr\\
1\textsc{in} 	&	ndɜ́liŋ or lɜ́liŋ & ɲ́ɲandá\\
2		&	ŋ́ŋá	& ɲáŋə́nda \\
3\textsc{hum}	&	ŋ́ŋúŋ & ŋwúlwɔ́ or ŋúlwánda\\
\end{tabular}
\z
%These are completely different in Werria. Do we care about this? Also, check whether 3sg and 3pl pronouns referring to non-humans are possible.

In subject position, independent pronouns are typically used for emphasis to introduce a new or unexpected topic:

\ea Independent pronouns
	\ea \gll ɲ́ɲí e-g-a-v-álə́ŋ-a\\
			1\textsc{sg}.\textsc{pro} 1\textsc{sg}-\textsc{cl}g-\textsc{rtc}-\textsc{prog}-sing-\textsc{ipfv}\\
		\glt `As for me, I can sing.'	 (3/17/2010)
	\ex \gll ɲ́ɲí tʃom\\
			1\textsc{sg}.\textsc{pro} also\\
		\glt	`Me too!' 	(said in response)
	\ex \gll ŋwúlɜ́ al-aləŋ-e		\\
			 3\textsc{sg}.\textsc{hum} 3\textsc{pl}.\textsc{inf}-sing-\textsc{juss}\\
		\glt `They, let it be that only they sing.'
	\z
\z
In non-emphatic contexts subject independent pronouns are not used, and a null subject is used instead.

In object positions, independent pronouns can be used together with affixal object markers to indicate focus on their referent:

\ea 
	\gll	 k-a-bwáɲ-á   g-í-ndə-duɜ́d-ət̪-iə                    ɲáŋə́nda\\
	\textsc{sm}.\textsc{cl}g-\textsc{rtc}-want-\textsc{ipfv}   \textsc{sm}.\textsc{cl}g-\textsc{dpc}1-2\textsc{pl.om}-speak-\textsc{appl}-\textsc{ipfv}  2.\textsc{pl}\\
	\glt	‘He wants to talk to you all.’
\z
Otherwise, the normal realization of object pronouns is as object markers (\sectref{sec:ch7:om}). 
	
Independent pronouns can also occur in cleft constructions (\sectref{clefts}). When an object pronoun is clefted, a resumptive object marker appears on the verb:

\ea	\gll	ŋwə́-ndr            k-é-bwáɲá                    g-í-ndə-duɜ́d-ət̪-ia-r\\
	\textsc{cleft}-1pl.\textsc{exc}  \textsc{sm.cl}g-\textsc{dpc}1-want-\textsc{ipfv}  \textsc{sm.cl}g-\textsc{dpc}1-1\textsc{plexclom}-speak-\textsc{appl}-\textsc{ipfv}-\textsc{pl}\\
	\glt	‘it is us that he wants to talk to’
\z

Independent personal pronouns can be further emphasized with the addition of an \textit{-é} suffix. This suffix is realized as [j] following [a], and does not appear on the 1\textsc{sg} marker. 

\ea 
\begin{tabular}[t]{lll}
		&	\textsc{sg}/\textsc{du} & \textsc{pl} \\
\textsc{1ex}		&	ɲɲí	& ndre\\
\textsc{1in }	&	lɜ́liŋé & ɲɲándé\\
\textsc{2	}	&	ŋŋaj	& ɲáŋə́nde \\
\textsc{3hum}	&	ŋŋúŋé & ŋwúlwáj\\
\end{tabular}
\z

%Need to check transcriptions on these and say more on their use. 

\section{Object markers}\label{sec:ch7:om}

Object markers are object pronouns that have been incorporated into the verb as affixes or clitics. Object markers are are the normal realization of pronominal objects in Moro. Object markers distinguish the same eight person-number combinations as other pronoun and agreement series in Moro. Object markers have a complex distribution, occurring in one of two positions or series. In the first series, object markers are realized as enclitics, which are only loosely attached to the verb:

\ea Suffixal object markers on perfective verb (from Rose 2013)\\ %morpheme breaks don't make sense here -- recheck
\begin{tabular}[t]{lll}
1\textsc{sg} & g-a-ʧombəð-ə́=ɲé	& `(s)he tickled me'\\ 
2\textsc{sg} & g-a-ʧombəð-j=áŋá & `(s)he tickled you (\textsc{sg}.)'\\
3\textsc{sg.hum} & g-a-ʧombəð=óŋó & `(s)he tickled her/him'\\
1\textsc{in}.\textsc{du} & g-a-ʧombəðə́=ńda & `(s)he tickled you and me'\\
1\textsc{in}.\textsc{pl} & g-a-ʧombəð-ə́=ńd-r & `(s)he tickled us (incl.)'\\
1\textsc{ex}.\textsc{pl} & g-a-ʧombəð=álánda& `(s)he tickled us (excl.)'\\
2\textsc{pl} &  g-a-ʧombəð-ə́=ńda & `(s)he tickled you (pl.)'\\
3\textsc{pl} &  g-a-ʧombəð-ó=lo & `(s)he tickled them'\\
\end{tabular} \label{ex:ch7:pfvom}
\z

The underlying form of the perfective final vowel is /-ó/ (\sectref{sec:ch11:perfective}), and the changes in vowel quality are due to peripheral vowel reduction and local rounding harmony.

The second set of object markers are realized either as prefixes or as circumfixes:

\ea Prefixal object markers on proximal imperfective verb (from Rose 2013)\\
\begin{tabular}[t]{lll}
1\textsc{sg} & g-a-ɲə́-ʧombəð-a & `(s)he is tickling me'\\
2\textsc{sg} & g-a-ŋá-ʧombəð-a & `(s)he is tickling you ({sg}.)'\\
3\textsc{sg.hum} & g-a-ŋó-ʧombəð-a & `(s)he is tickling her/him'\\
1\textsc{in}.\textsc{du} & g-á-ńdə-ʧombəð-a & `(s)he is tickling you and me'\\
1\textsc{in}.\textsc{pl} & g-á-ńdə-ʧombəð-a-r & `(s)he is tickling us (incl.)'\\
1\textsc{ex}.\textsc{pl} & g-a-ɲə́-ʧombəðá-lánda & `(s)he is tickling us (excl.)'\\
2\textsc{pl} & g-á-ńdə-ʧombəð-a & `(s)he is tickling you (pl.)'\\
3\textsc{pl} & g-a-ʧómbəð-a-lo & `(s)he is tickling them'\\
\end{tabular}
\z
Both third person plural forms show circumfixal object markers, but certain pieces of the circumfixes show a distinct distribution. The plural \textit{-r} suffix in the 1\textsc{in}.\textsc{pl} is attested in a number of other constructions such as plural imperatives, and note that the prefixal \textit{ɲə́}- component of the 1\textsc{ex}.\textsc{pl} also occurs with the 1\textsc{sg}, both of which are [+Speaker,-Addressee]. 

The 3\textsc{pl} object marker is unique in always occurring as an enclitic in Thetogovela Moro. Written Moro, and hence likely other dialects such as Longorban, which is the basis of written Moro, are different in this regard, with a 3\textsc{pl} prefix \textit{lə́-}. In texts, and for one our consultants who is more influenced by the written standard, the 3\textsc{pl} OM is typically placed in the prefixal position, even though he is a Thetogovela Moro speaker. Different dialects have other differences forms elsewhere in their object marker paradigm, for example in Written Moro and Wërria Moro, the 3\textsc{sg}.\textsc{hum} form is \textit{ma-}/\textit{=ma} (tone unknown).

Whether a particular verb form triggers the prefixal or suffixal object marker is dependent on the morphological category of the verb to which it attaches. The different verb forms triggering prefixal and suffixal object markers are listed in \tabref{tab:ch7:om}. See Chapter \ref{chapter:verbs} for more on these forms.

\begin{table}
	
\begin{tabular}[t]{lll}
\lsptoprule
\multicolumn{3}{l}{Verb forms with suffixal object marker (with 3\textsc{sg.hum} object)}\\
\midrule 
\textsc{pfv} 	 &  ɡ-a-ʧombəð-ə́-ŋó	 &  ‘(s)he tickled her/him’\\
\textsc{ven.ipfv} &  ɡ-á-ʧombəð-ə́-ŋó & ‘(s)he is about to tickle her and come’\\
\textsc{imp} &  ʧómbə́ð-ə́-ŋó & ‘tickle her/him!’ \\
\textsc{ven.imp} &  ʧombəð-á-ŋó & ‘tickle her/him there and come!’\\
\lsptoprule
\multicolumn{3}{l}{Verb forms with prefixal/circumfixal object markers (with 3\textsc{sg.hum} object)}\\
\midrule
\textsc{ipfv}   &	ɡ-a-ŋó-ʧombəð-a &	‘(s)he is about to tickle her/him’\\
\textsc{inf1} 	& (n)-áŋ-ŋó-ʧombəð-e	& ‘her to tickle her/him’\\
\textsc{ven.inf1} 	& (n)-áŋ-ŋó-ʧombəð-a &	‘her to tickles her/him and come’\\
\textsc{ven.inf2}  & (n)-áŋ-ŋó-ʧombəð-ó & ‘her to tickles her/him and come’ \\
\textsc{cons.pfv} & n-ə́ŋ-ŋó-ʧombəð-e & 	‘then (s)he tickled her/him’\\
\textsc{ven.cons.pfv} & n-ə́ŋ-ŋó-ʧombəð-a	& ‘then (s)he tickled her/him and came’\\
%g.&	\textsc{neg} & ɡanːá áŋ-ŋó-ʧombəð-a &‘(s)he doesn’t/ didn’t tickle her/him)’\\
%h.&	\textsc{neg.imp} & ánːá á-ŋó-ʧombəð-a& ‘don’t tickle her/him!’	\\
\lspbottomrule
\end{tabular}
\caption{Verb forms with OM prefixes and suffixes}\label{tab:ch7:om}
\end{table}
The categories which condition the position of object markers seem arbitrary, but they form natural classes based on their tonal behavior: the verb forms which condition prefixal object markers all have the default verb tone pattern of a left-aligned H tone on the verb root (\sectref{sec:ch11:defaulttone}). Jenks \& Rose (2015) argue that grammatical constraints on the distribution of tone predict whether object markers are prefixal or suffixal.

Vowel harmony provides evidence that the suffixal forms of the verb are less tightly incorporated into the stem than the prefixal forms, as is illustrated in the following pair (Jenks \& Rose 2015 ex. 22).

\ea Vowel harmony with object markers
\begin{tabular}[t]{ll}
	é-ɡ-a-veð-ə́-ŋá	& ‘I slapped you (\textsc{sg}.)’ \\
	 é-ɡ-a-ŋá-veð-a  &	‘I am about to slap you \textsc{sg}.’\\
	í-ɡ-ɜ-buɡ-ə́-ŋá 	 &	‘I hit you (\textsc{sg}.)’	\\
	 í-ɡ-ɜ-ŋɜ́-buɡw-ɜ	& ‘I am about to hit you (\textsc{sg}.)’ \\
	\end{tabular}
\z

The verb roots \textit{veð} `slap' and \textit{bug} `hit' differ in triggering low and high vowel harmony respectively. As the examples above show, only the prefixal object marker is subject to high vowel harmony.

In normal declarative clauses with no focus on the object, object markers cannot occur with coreferential object noun phrases:

\ea	\label{ex:ch7:objpro}
\gll	Kúku        ɡ-a-ləvəʧ-ó(*-ŋó)	        ŋeɾá		 \\
			\textsc{cl}g.Kuku   \textsc{cl}g.\textsc{sm}-\textsc{rtc}-hide-\textsc{pfv}   \textsc{cl}ŋ.girl	 \\
	\glt		‘Kuku hid the girl’				
\z
The complementarity of object markers and coreferential object noun phrases  provides evidence for the pronominal status of object markers. However, object markers do occur with focused object pronouns, as we saw in the previous section. Object markers also occur as resumptive pronouns in a number of extraction constructions such as clefts (Chapter \ref{chapter:relative}).

\section{Reflexive pronouns}\label{sec:ch7:reflpro}

Reflexive pronouns are formed by prefixing a pronominal component to \textit{və́gá} ‘self’. The singular prefixes resemble object markers rather than pronouns, whereas the 1st and 2nd plural forms use the first syllable of the independent pronoun. 3\textsc{pl} is the 3\textsc{pl} pronoun, not the object marker \textit{-lo}. %CHECK TONE (March 20, 2012). (The 1\textsc{pl} and 2\textsc{pl} constitute the first part of the pronouns. ?)

\ea Prefixal object markers on proximal imperfective verb (from Rose 2013)\\
\begin{tabular}[t]{lll}
1\textsc{sg} & ɲé-vəgá	 & `myself'\\
2\textsc{sg} &  ŋá-vəgá	 & `yourself'\\
3\textsc{sg} &  ŋó-vəgá	 & `herself/himself'\\
1\textsc{in}.\textsc{du} & lá-vəgá	& `yourself and myself'\\
1\textsc{in}.\textsc{pl} & lá-vəgá  & `ourselves (incl.)'\\
1\textsc{ex}.\textsc{pl} & ɲá-vəgá & `ourselves (excl.)'\\
2\textsc{pl} & ɲáŋá-vəgá	 & `yourselves'\\
3\textsc{pl} & ŋúlwɔ́-vəgá & `themselves'\\
\end{tabular}
\z
%WHAT IF ANTECEDENT IS A NONHUMAN NOUN?

Reflexive pronouns occur in two syntactic contexts. First, they occur along with the reflexive extension suffix  \textit{-en} on the verb (identical to the passive, Section \ref{sec:ch11:antipassive}), although they are optional in this context.

\ea 	\gll ɲɜ-g-ɜ-p-ən-ú ɲá-və́gá \\
			1\textsc{ex}.\textsc{pl}-\textsc{cl}g-\textsc{rtc}-beat.rt-pass-pfv 1\textsc{expl}-self\\
		\glt	`We (excl.) hit ourselves.'
\z 
Because the reflexive suffix always targets the surface subject, reflexive pronouns are subject-oriented.

%MORE EXAMPLES HERE< CHECK RECIPROCAL AND ADD HERE.

%(Second, plural reflexive pronouns can serve as reciprocal pronouns along with a reciprocal suffix \textit{-əð} on the verb (identical to the antipassive, Section \ref{antipassive})

%Moro does not have reciprocal pronouns. Reciprocity is indicated by a suffix and object clitics on the verb. 

%REALLY? Need to do more work on reciprocals: CF ɲɜ-gɜ-p-ə-ðu ɲa-vəga `We hit each other' from p. 94 of class grammar.

Second, reflexive pronouns can serve as emphatic pronouns, which must be bound by the subject.

\ea \ea \gll é-g-a-kal-ó 			ɲé-və́gá	\\
		1\textsc{sg}\textsc{sm}-\textsc{cl}g-\textsc{rtc}-chop-\textsc{pfv}	1\textsc{sg}-self  \\
		\glt ‘I chopped it myself’
	\ex \gll k-a-kal-ó 			ŋó-və́gá				\\
		\textsc{sm.cl}g-\textsc{rtc}-chop-\textsc{pfv}		3\textsc{sg}-self \\
		\glt ‘He chopped it himself’
	\ex \gll l-a-kal-ó 			ŋwúlwá-vəga	\\
			\textsc{sm.cl}g-\textsc{rtc}-chop-\textsc{pfv}		3\textsc{sg}-self \\
		\glt ‘they chopped it themselves’
	\ex	\gll íð-ú 	ŋəməgəniə 	ŋá-və́gá		{Jan. 30, 2013}\\
				do-\sc{imp} \textsc{cl}ŋ.work 	2\textsc{sg}-self {} \\
		\glt ‘do the work by yourself!’
	\z
\z
Like reflexive pronouns which accompany reflexive predicates, emphatic uses of reflexive pronouns are also always subject oriented.

%Are these always subject-oriented?

\section{Possessive pronouns}\label{section:posspro} %To pronoun section

Possessive pronouns take two different forms depending on whether they are in predicative or attributive positions, as summarized in \tabref{tab:ch7:3}. Predicative possessive pronouns are simpler (see Section \ref{section:posspred} for more on possessive predicates), consisting of a verbal prefix, and weak concord with the subject which occurs on the possessor itself.  Attributive uses of possessive pronouns typically occur with strong concord (Section \ref{concord}) and always involve a partial reduplication process. The plural possessive roots in \tabref{tab:ch7:3} are also found as suffixes on inalienably possessed kinship nouns (Section \ref{sec:ch6:kinship}).

\begin{table}
	\begin{tabular}[t]{llll}
	\lsptoprule
						& Pronoun root			&	Predicative &	Attributive\\
	\midrule
1\textsc{sg}			& -ɜ́ŋ		& 	gɜ-k-ɜ́ŋ 	 & íkː-ɜŋ-kɜŋ \\
2\textsc{sg}			& -ó		&	ga-k-ó	 & íkː-o-kːe \\
3\textsc{sg}			& -óŋ		&	ga-k-óŋ	 & íkː-oŋ-koŋ\\
1\textsc{in}.\textsc{du} & -ɜlə́ŋ  	&   gɜ-k-ɜlə́ŋ	 & íkː-ɜlə́ŋə́-ki\\
1\textsc{in}.\textsc{pl} &  -əndŕ  	&   gɜ-k-əndŕ̩	 & íkː-əndŕ̩-ki \\
1\textsc{ex}.\textsc{pl} &   -áɲ 	&   ga-k-áɲ 	 & íkː-aɲ-kaɲ \\
2\textsc{pl}			& -aló    	&	ga-k-aló	 & íkː-alə́--ke	\\
3\textsc{pl}			& -én 		&	ga-k-én 	 & íkː-en-ken \\
	\lspbottomrule
	\end{tabular}
	\caption{Possessive pronoun forms}
	\label{tab:ch7:3}
\end{table} 

The full paradigm of attributive possessive forms is given in Table \ref{tab:ch7:4}, including the full range of noun class agreement. The second row in that table shows the phonological schema for each pronominal form, where C is noun class concord. Schwa epenthesis occurs before sonorants, except when they are identical.

\begin{table}
\begin{tabular}[t]{lllll}
\lsptoprule
	&	1\textsc{sg}			&	1\textsc{indu}		&	2\textsc{sg}			&	3\textsc{sg} \\
	& 	íCː-ɜŋ-C-ɜŋ	&	íCː-ɜlə́ŋ-(ə́)C-i & íCː-o-Cː-e 	&	iCː-oŋ-C-oŋ  \\
\midrule
	g&	íkːɜŋkɜŋ 	&	íkːɜlə́ŋə́ki	&	íkːokːe	&	íkːoŋkoŋ\\
	l&	ílːɜŋəlɜŋ 	&	ílːɜlə́ŋə́li	&	ílːolːe	&	ílːoŋəloŋ\\
	n&	ínːɜŋənɜŋ 	&	ínːɜlə́ŋə́ni	&	ínːonːe	&	ínːoŋənoŋ\\
	ŋ&	íŋːɜŋːɜŋ 	&	íŋːɜlə́ŋŋi	&	íŋːoŋːe	&	íŋːoŋoŋ	\\
	ɲ&	íɲːɜŋəɲɜŋ	&	íɲːɜlə́ŋə́ɲi  	&	íɲːoɲːe	&	íɲːoŋəɲoŋ	\\
	ð&	íðːɜŋəðɜŋ 	&	íðːɜlə́ŋə́ði 	&	íðːoðe	&	íðːoŋəðoŋ	\\
	r&	írːɜŋərɜŋ 	&	írːɜlə́ŋə́ri 	&	írːore	&	írːoŋəroŋ	\\
	j&	ísːɜŋsɜŋ 	&	ísːɜlə́ŋsi	&	ísːose	&	ísːoŋsoŋ\\
\lsptoprule
&		1\textsc{expl}		&1\textsc{inpl}&		\textsc{2pl}&		\textsc{3pl} \\
&	íCː-aɲ-C-aɲ	&íCː-ndŕ̩--Cː-i&	iCː-alə́--C-e&	íCː-en-C-en \\
\midrule
	g&	íkːaɲkaɲ &	íkːəndŕ̩ki	&íkːalə́ke	&íkːenken\\
	l&	ílːaɲlaɲ & 	ílːəndŕ̩li  	&ílːalə́le	&	ílːenlen\\
n&	ínːaɲənaɲ &	ínːdŕ̩ni		&ínːalə́ne	&ínːenːen\\
	ŋ&	íŋːaŋəɲaɲ 	&íŋːndŕ̩ŋi	&íŋːalə́ŋe&	íŋːenəŋen\\
	ɲ&	íɲːaɲːaɲ 	&íɲːəndŕ̩ɲi	&íɲːalə́ɲe&	íɲːenəɲen\\
	ð&	íðːaɲəðaɲ &	íðːəndŕ̩ði	&íðːalə́ðe	&íðːenðen\\
	r&	írːaɲəraɲ &	írːndŕ̩ri	&	írːalə́re&		írːendren\\
	j&	ísːaɲsɑɲ& 	ísːəndŕ̩si	&ísːalə́se	&	ísːensen\\
\lspbottomrule
		\end{tabular}
		  \caption{Attributive possessive pronoun paradigm}
  \label{tab:ch7:4}
\end{table} %THESE SEEM TO BE TOTALLy DIFFERENT FOR ANGELO... SHOULD WE GET THESE?

Although there is no vowel harmony between the strong concord íCː element and the rest of the possessive, vowel harmony does appear to be operable within the rest of the construction, as \textit{ɜlə́ŋə́ki} and \textit{alə́ke} demonstrate. Some identifiable pronominal elements are also contained within some of these forms. The sequence \textit{ndr} in 1\textsc{inpl} is the same as the 1\textsc{inpl} object marker. The sequence \textit{ɜləŋ} in 1\textsc{indu} is connected to the 1\textsc{indu} pronoun \textit{ndɜ́liŋ} or \textit{lɜ́liŋ}.

When they modify nouns, possessive pronouns often fuse with the final vowel of the noun, as do all nominal modifiers with strong concord:

\ea
\begin{tabular}[t]{llll}
a. & [lavəɾa]  		& b. & 	[lavəɾɜ́lːɜŋəlɜŋ]\\
	& lavəɾa 	  	&	& lavəɾa-ílː-ɜŋ-l-ɜŋ\\
	& \textsc{cl}l.stick 	& 	& 	\textsc{cl}l.stick-s\textsc{cl}l-1\textsc{sgposs}-\textsc{cl}l-1\textsc{sgposs}\\
	& ‘the/a stick’ & 	&	‘my stick’	\\
\end{tabular}
\z 
The final vowel of the noun raises from /a/ to [ɜ] due to fusion with the initial /i/ of the strong concord prefix (\sectref{concord}), which also contributes a high tone to the final vowel of the noun.

%
%\ea Verb forms with suffixal object marker (w/3\textsc{sg} object)\\
%\begin{tabular}[t]{lllll}
%	 &  			 & 		no object	 & w/ 3\textsc{sg} object	 &  \\
%a. & pfv 	 &  	ɡ-a-ʧombəð-ó & ɡ-a-ʧombəð-ə́-ŋó	 &  ‘(s)he tickled (her/him)’\\
%b. & dist.ipfv & ɡ-á-ʧombəð-ó& ɡ-á-ʧombəð-ə́-ŋó & ‘(s)he is about to tickle there’\\
%c. & prox.imp & ʧómbə́ð-ó & ʧómbə́ð-ə́-ŋó & ‘tickle!’ \\
%d. & dist.imp & ʧombəð-a & ʧombəð-á-ŋó & ‘tickle there!’\\
%\end{tabular}		
%\z 
%
%\ea Verb forms with prefixal/circumfixal object markers (w/3\textsc{sg} object)\\
%\begin{tabular}[t]{lllll}
%				&& 			no object& w/ 3\textsc{sg} object & \\	
%a.&	prox.ipfv   &	ɡ-a-ʧómbəð-a &  	ɡ-a-ŋó-ʧombəð-a &	‘(s)he is about to tickle (him/her)’\\
%b.&	prox.inf1 	& (n)-áŋ-ʧómbəð-e& (n)-áŋ-ŋó-ʧombəð-e	& ‘that (s)he tickles (him/her)’\\
%c.&	dist.inf1 & (n)-áŋ-ʧómbəð-a	& (n)-áŋ-ŋó-ʧombəð-a &	‘that (s)he tickles (him/her) (there)’\\
%d.&	dist.inf2  &	(n)-áŋ-ʧómbəð-ó & (n)-áŋ-ŋó-ʧombəð-ó & ‘that (s)he tickles (him/her) there’ \\
%e.&	prox.cons.pfv & n-ə́ŋə́-ʧómbəð-e & n-ə́ŋ-ŋó-ʧombəð-e & 	‘and then (s)he tickles (him/her)’\\
%f.&	dist.cons.pfv& n-ə́ŋə́-ʧómbəð-a& n-ə́ŋ-ŋó-ʧombəð-a	& ‘and then (s)he tickles (him/her) there’\\
%g.&	neg & ɡ-anːá áŋ-ʧómbəð-a& ɡanːá áŋ-ŋó-ʧombəð-a &‘(s)he doesn’t/ didn’t tickle (him/her)’\\
%h.&	neg.imp & ánːá á-ʧómbəð-a & ánːá á-ŋó-ʧombəð-a& ‘don’t tickle (him/her)!’	\\
%				\end{tabular}
%\z 

